\chapter{Symmetric Monoidal and Stable \texorpdfstring{$\infty$}{infty}-Categories}\label{chap:Monoidal}
\setcounter{dummy}{-1}
\section{\texorpdfstring{$\IE_1$}{E1}-Monoids and \texorpdfstring{$\IE_1$}{E1}-Groups}
We start things off slowly by only considering (not necessarily symmetric) monoidal $\infty$-categories and $\IE_1$-spaces.
\numpar{Stasheff's Definition of Coherently Associative Monoids}
The naive way to define a monoid in $\An$ would be to have an object $M\in \An$ together with a multiplication map
\begin{equation*}
	\mu\colon M\times M\morphism M\,.
\end{equation*}
Now $\mu$ induces two different maps $M^3\morphism M$ (\enquote{two ways of bracketing}), so there ought to be a homotopy $H\colon \mu\circ (\mu\times \id_M)\simeq \mu\circ(\id_M\times \mu)$ between them. But then there are five maps $M^4\morphism M$ (\enquote{five ways of bracketing}), and we see that $H$ induces a loop in $\Hom_\An(M^4,M)$. Since $M$ should be associative up to \emph{coherent homotopy}, this loop has to be filled. The story goes on: There are a number of maps $M^5\morphism M$ (too lazy to count them) and now there are some $3$-simplices to be filled and so on.

Although this looks horrible, it is possible to turn these considerations into a precise definition, and one obtains Stasheff's \emph{$\IA_\infty$-spaces}. But we can take a simpler route! Indeed, thanks to our efforts so far, we now have the luxury of saying: \enquote{Well, a coherently associative monoid is just an $\infty$-category with only one object.} This leads to the following definition.
\begin{defi}\label{def:CartesianMonoids}
	Let $\Cc$ be an $\infty$-category with finite products (in particular, $\Cc$ has a final object $*\in \Cc$). A \emph{cartesian monoid in $\Cc$} is a functor $X\colon \IDelta^\op\morphism \Cc$ such that:
	\begin{alphanumerate}
		\item $X_0\simeq *$.
		\item It satisfies the Segal condition, i.e., the Segal maps $e_i\colon [1]\morphism {[n]}$ define an equivalence
		\begin{equation*}
			X_n\isomorphism \prod_{i=1}^nX_1
		\end{equation*}
	\end{alphanumerate}
	Let $\cat{Mon}(\Cc)\subseteq\cat{s}\Cc$ be the full sub-$\infty$-category spanned by cartesian monoids. If $\Cc=\An$, these are also called \emph{$\IE_1$-monoids}, \emph{$\IA_\infty$-spaces}, \emph{coherent monoids}, \emph{special $\Delta$-spaces}, \dotso. For $\Cc=\Cat_\infty$ we simply call them \emph{monoidal $\infty$-categories} (note that these can also be encoded as cocartesian fibrations over $\IDelta^\op$).
\end{defi}
To make sense of the condition from \cref{def:CartesianMonoids}\itememph{b}, note that since $X_0\simeq *$ is terminal, we have $X_1\times_{X_0}\times\dotsb\times_{X_0}X_1\simeq \prod_{i=1}^nX_1$, so this condition really \emph{is} the Segal condition. Somewhat weird though is that we don't impose any completeness conditions, and in fact, cartesian monoids in $\An$ are usually \emph{not} complete Segal spaces! This is in some sense fixed by the following proposition.
\begin{prop}\label{prop:CompletionOfMonoidsFullyFaithful}
	The completion functor $\comp\colon\cat{sAn}\morphism\cat{CSAn}$ from Lemma/Defi-nition~\textup{\labelcref{lemdef:Completion}} restricts to a fully faithful functor
	\begin{equation*}
		\comp\colon\cat{Mon}(\An)\morphism */\cat{CSAn}
	\end{equation*}
	with essential image those pointed complete segal spaces $(X,x)$ with $\pi_0X_0=*$.
\end{prop}
In other words, monoids really are categories with one object (up to equivalence, but not up to contractible choice, as Bastiaan pointed out in the lecture), since the target is also pointed categories with connected base.
\begin{proof}[Proof of \cref{prop:CompletionOfMonoidsFullyFaithful}]
	The proof is not hard, but quite lengthy, so we break it down into four major steps. Be aware that Step~\itememph{3} wasn't done in the lecture, so any errors in it are my fault.
	\begin{alphanumerate}
		\item[\itememph{1}] \itshape There exists a reasonable candidate $\decomp\colon {*/\cat{CSAn}}\morphism \cat{Mon}(\An)$ of an inverse functor \embrace{\enquote{decompletion}}.
	\end{alphanumerate}
	%Recall that if $X$ is Segal, then the completion functor does not affect the pullback spaces
	%\begin{equation*}
	%	\begin{tikzcd}
	%		P_{x,y}\rar\dar\drar[pullback] & X_1\dar["{(d_1,d_0)}"]\\
	%		*\rar["{(x,y)}"] & X_0\times X_0
	%	\end{tikzcd}
	%\end{equation*}
	%(this follows essentially from \cref{eq:asscat2}).
	
	If $(X,x)$ is a pointed complete Segal space, and if $Y\in \cat{Mon}(\An)$ is the \enquote{universal way} to make $(X,x)$ connected and still satisfy the Segal condition, then it's reasonable to expect that $Y_n$ sits inside a pullback
	\begin{equation*}
		\begin{tikzcd}[column sep=large]
			Y_n\dar\rar\drar[pullback]& X_n\dar\\
			*\rar["{(x,\dotsc,x)}"] & X_0^{n+1}
		\end{tikzcd}
	\end{equation*}
	for all $[n]\in \IDelta^\op$. We will see that this gives indeed the correct functor.
	
	Let's first address the elephant in the room, i.e.\ how to make the pointwise-defined $Y_n$ into a functor $Y\colon\IDelta^\op\morphism \An$, which in turn should be functorial in $X$. Since taking pullbacks is functorial, we'll have solved both problems at once if we show that the cospan diagram
	$*\morphism X_0^{n+1}\lmorphism X_n$ is functorial in $(X,x)$ and in $[n]$. To this end consider $X$ as a functor $\IDelta\morphism \An^\op$ and colimit-extend it to a functor
	\begin{equation*}
		|\blank|_X\colon \cat{sAn}\morphism \An^\op\,.
	\end{equation*}
	This can be done functorially in $X$, as \cref{thm:ColimitPreservingRepresentable} shows. In more abstract terms, what we do here is to identify  $\cat{sAn}\simeq \Fun(\IDelta,\An^\op)^\op\simeq \Fun^L(\cat{sAn},\An^\op)^\op$. We define two functors $\Delta,\Delta_0\colon \IDelta\morphism\cat{sSet}\subseteq\cat{sAn}$ via
	\begin{equation*}
		\Delta\big([n]\big)=\Delta^n\quad\text{and} \quad\Delta_0\big([n]\big)=\coprod_{i=0}^n\Delta^0\,. 
	\end{equation*}
	Since $\Delta$ and $\Delta_0$ are functors between $1$-categories (before we compose them with the inclusion $\cat{sSet}\subseteq \cat{sAn}$), we don't need to check any higher coherences and immediately obtain that $\Delta$ and $\Delta_0$ are indeed functors. Moreover, the canonical map $\coprod_{i=0}^n\Delta^0\morphism \Delta^n$ sending the $i\ordinalth$ component to the $i\ordinalth$ vertex of $\Delta^n$ is clearly functorial in $[n]\in \IDelta$, hence it induces a natural transformation $\Delta_0\Rightarrow\Delta$ (which exhibits $\Delta_0$ as the \enquote{$0$-skeleton} of $\Delta$, hence the notation). Now consider the two composites
	\begin{equation*}
		\cat{sAn}\simeq \Fun\left(\IDelta,\An^\op\right)^\op\simeq \Fun\left(\cat{sAn},\An^\op\right)^\op\doublemorphism[\Delta^*][\Delta_0^*]\Fun\left(\IDelta,\An^\op\right)^\op\simeq \cat{sAn}\,,
	\end{equation*}
	which we denote $|\Delta|_{(-)}$ and $|\Delta_0|_{(-)}$. The transformation $\Delta_0\Rightarrow \Delta$ induces a transformation $|\Delta|_{(-)}\Rightarrow |\Delta_0|_{(-)}$ going in the other direction, since there has been an $(-)^\op$ in between. Upon closer inspection, we find that $\Delta\colon \IDelta\morphism\cat{sAn}$ is actually nothing else but the Yoneda embedding, hence \cref{thm:ColimitPreservingRepresentable} shows that $|\Delta|_{(-)}\simeq \id$. Moreover, since $|\blank|_X$ preserves colimits, we easily obtain
	\begin{equation*}
		X_0^{n+1}\simeq \left|\coprod_{i=0}^n\Delta^0\right|_X\,,
	\end{equation*}
	hence the $X_0^{n+1}$ can be organized into a simplicial anima $|\Delta_0|_X$. The transformation $\id\simeq |\Delta|_{(-)}\Rightarrow |\Delta_0|_{(-)}$ gives a natural map $X\morphism |\Delta_0|_X$.
	
	This shows that the maps $X_0^{n+1}\lmorphism X_n$ can indeed be made functorial in $X$ and $[n]$. A similar argument applies shows that the maps $*\morphism X_0^{n+1}$, which assemble into a functorial map $\Delta^0\morphism |\Delta_0|_X$. We conclude that $Y\simeq X\times_{|\Delta_0|_X}\Delta^0$ as above is indeed a simplicial anima and functorial in $X$. We have $Y_0\simeq *$ by construction and also $Y_n\simeq Y_1\times_{Y_0}\dotsb\times_{Y_0}Y_1$ follows from the fact that $X$ itself satisfies the Segal conditions together with some abstract pullback nonsense (basically that \enquote{limits commute}, see the dual of \cref{prop:ColimitsCommute}). So $Y\in \cat{Mon}(\An)$ and we can finally write down the functor
	\begin{equation*}
		\decomp\colon {*/\cat{CSAn}}\morphism \cat{Mon}(\An)\,.
	\end{equation*}
	This finishes Step~\itememph{1}.
	\begin{alphanumerate}
		\item[\itememph{2}] \itshape The completion functor $\comp\colon \cat{Mon}(\An)\morphism \cat{CSAn}$ from \cref{lemdef:Completion} factors over $*/\cat{CSAn}\morphism \cat{CSAn}$ and its essential image is contained in those pointed complete Segal spaces $(X,x)$ with $\pi_0X_0=*$.
	\end{alphanumerate}
	
	To get the factorisation, observe that $\cat{Mon}(\An)$ has an initial object, namely $\Delta^0\in\sSet\subseteq\cat{sAn}$ (in the lecture we called it \enquote{$\const*$}, but it's really just $\Delta^0$). Indeed, if $T\in \cat{Mon}(\An)$, then
	\begin{equation*}
		\Hom_{\cat{Mon}(\An)}(\Delta^0,T)\simeq \Hom_{\cat{sAn}}(\Delta^0,T)\simeq T_0\simeq *
	\end{equation*}
	by \cref{def:CartesianMonoids}\itememph{a}. Also $\comp (\Delta^0)\simeq *$, so the completion functor really lifts to a functor $\cat{Mon}(\An)\simeq \Delta^0/\cat{Mon}(\An)\morphism */\cat{CSAn}$, as required.
	
	To check that $\pi_0(\comp T)_0=*$, recall that $(\comp T)_0\simeq \core \asscat T\simeq |T^\times|$ by \cref{eq:asscat1}. Now we have to use the following general fact:
	\begin{alphanumerate}
		\item[\itememph{*}] \itshape For all $X\in\cat{sAn}$ the map $\pi_0X_0\morphism \pi_0|X|$ is surjective.
	\end{alphanumerate}
	This shows what we want since $(T^\times)_0\simeq T_0\simeq *$ holds by assumption. To see where \itememph{*} comes from, observe that $\pi_0$, being a left adjoint (\cref{exm:MyFirstAdjoints}\itememph{a}), commutes with colimits. This shows $\pi_0|X|\simeq \pi_0\colimit_{\IDelta^\op}X\simeq \colimit_{\IDelta^\op}\pi_0X$, where the colimit on the right-hand side is a good old colimit taken in $\sSet$. Now $\{d_0,d_1\colon [1]\shortdoublemorphism [0]\}\subseteq \IDelta^\op$ is $1$-cofinal (but not $\infty$-cofinal!), hence
	\begin{equation*}
		\colimit_{\IDelta^\op}\pi_0X=\Coeq\left(\pi_0X_1\doublemorphism[d_1][d_0] \pi_0X_0\right)\,.
	\end{equation*}
	This immediately implies \itememph{*} and thus Step~\itememph{2} is done.
	\begin{alphanumerate}
		\item[\itememph{3}]\itshape The functors from Step~\itememph{1} and Step~\itememph{2} fit into an adjunction
		\begin{equation*}
			\comp\colon \cat{Mon}(\An)\doublelrmorphism {*/\cat{CSAn}}\noloc \decomp
		\end{equation*}
	\end{alphanumerate}
	
	Consider $T\in \cat{Mon}(\An)$, which is naturally pointed via the unique (up to contractible choice) map $1_T\colon \Delta^0\morphism T$, and let $(X,x)$ be a pointed complete Segal space. Then
	\begin{align*}
		\Hom_{*/\cat{CSAn}}\big(\comp T,(X,x)\big)&\simeq \Hom_{*/\cat{sAn}}\big((T,1_T),(X,x)\big)\\
		&\simeq \Hom_{\cat{sAn}}(T,X)\times_{\Hom_{\cat{sAn}}(\const 1_T,X)}\{\const x\}\\
		&\simeq \Hom_{\cat{sAn}}(T,X)\times_{\Hom_\An(1_T,X_0)}\{x\}\,.
	\end{align*}
	The first equivalence follows from \cref{lemdef:Completion} (and the fact that adjunctions extend to slice categories), the second follows from \cite[Corollary~VIII.6]{HigherCatsII}. The last one follows by inspection. All of them are functorial in $(X,x)$. Moreover, we have
	\begin{align*}
		\Hom_{\cat{Mon}(\An)}\big(T,\decomp (X,x)\big)&\simeq \Hom_{\cat{sAn}}\left(T,X\times_{|\Delta_0|_X}\Delta^0\right)\\
		&\simeq \Hom_{\cat{sAn}}(T,X)\times_{\Hom_{\cat{sAn}}(T,|\Delta_0|_X)}\Hom_{\cat{sAn}}(T,\Delta^0)\,.
	\end{align*}
	Since $\id\Rightarrow |\Delta_0|_{(-)}$ is a natural transformation, one quickly checks that the morphism $\Hom_{\cat{sAn}}(T,X)\morphism \Hom_{\cat{sAn}}(T,|\Delta_0|_X)$ factors over $\Hom_{\cat{sAn}}(|\Delta_0|_T,|\Delta_0|_X)$. But $T_0\simeq *$, hence $|\Delta_0|_T\simeq \Delta^0$. Also $\Hom_{\cat{sAn}}(T,\Delta^0)\simeq *$. Putting everything together, we can rewrite
	\begin{equation*}
		\Hom_{\cat{Mon}(\An)}\big(T,\decomp (X,x)\big)\simeq \Hom_{\cat{sAn}}(T,X)\times_{\Hom_\An(T_0,X_0)}\{x\}\,,
	\end{equation*} 
	and after another inspection, we see that our calculations of $\Hom_{*/\cat{CSAn}}(\comp T,(X,x))$ and $\Hom_{\cat{Mon}(\An)}(T,\decomp (X,x))$ agree, as they are supposed to. This finishes Step~\itememph{3}.
	\begin{alphanumerate}
		\item[\itememph{4}] \itshape The restriction $\decomp\colon (*/\cat{CSAn})_{\pi_0=*}\morphism \cat{Mon}(\An)$ is indeed an inverse to $\comp$, where $(*/\cat{CSAn})_{\pi_0=*}\subseteq */\cat{CSAn}$ denotes the full subcategory spanned by those $(X,x)$ with $\pi_0X_0=*$.
	\end{alphanumerate}
	
	Thanks to Step~\itememph{3}, we already have unit and counit transformations $\id\Rightarrow \decomp\circ \comp$ and $\comp\circ\decomp\Rightarrow \id$, so we only need to show that they are equivalences. Let's start with the counit. We must show
	\begin{equation*}
		\comp\big(\decomp (X,x)\big)\simeq (X,x)
	\end{equation*}
	for all $(X,x)\in (*/\cat{CSAn})_{\pi_0=*}$. Observe that $\comp X\simeq X$ because $X$ is already complete. Hence it suffices to show $\asscat(\decomp(X,x))\simeq \asscat X$ (and check that the chosen base points on both sides correspond, but that's easy) because $\comp\simeq \N^r\circ \asscat$ by \cref{lemdef:Completion} and $\N^r$ is fully faithful by \cref{thmdef:RezkNerve}. Using \cref{eq:asscat1} and the fact that $X^\times$ is a constant simplicial anima with value $X_0$ as $X$ is complete, we see that
	\begin{equation*}
		\pi_0\core(\asscat X)\simeq \pi_0|X^\times|\simeq \pi_0X_0\simeq *\,,
	\end{equation*}
	so $\asscat(\decomp(X,x))\morphism \asscat X$ is essentially surjective for trivial reasons. Moreover, we have $\decomp(X,x)_0\simeq \{x\}$ by construction, and \cref{eq:asscat2} (which is applicable here because $\decomp(X,x)$ is Segal by construction) easily shows
	\begin{equation*}
		\Hom_{\asscat(\decomp(X,x))}(x,x)\simeq P_{x,x}\simeq \Hom_{\asscat X}(x,x)\,,
	\end{equation*}
	so $\asscat(\decomp(X,x))\morphism \asscat(X)$ is fully faithful too. This shows that the counit is an equivalence, as claimed. The argument for the unit is similar: We have to show
	\begin{equation*}
		T\simeq \decomp(\comp T)
	\end{equation*}
	for all $T\in \cat{Mon}(\An)$. Since both sides are functors $\IDelta^\op\morphism \cat{An}$ and a natural transformation between them is already given, we can do this degree-wise by \cref{thm:JoyalEquivalence}\itememph{b}. Since both sides are cartesian monoids, is suffices to check that the map in degree $1$ is an equivalence, i.e.\ that $T_1\simeq \decomp(\comp T)_1$. Write $(X,x)\simeq \comp T$. By construction we have
	\begin{equation*}
		\decomp(X,x)_1\simeq P_{x,x}\quad\text{and}\quad T_1\simeq P_{1_T,1_T}\,,
	\end{equation*}
	where as usual $1_T\colon \Delta^0\morphism T$ is the natural pointing of $T$. As observed at the very beginning of this proof, the spaces $P_{y,z}$ aren't affected by completion, hence $P_{x,x}\simeq P_{1_T,1_T}$ and we're done. 
\end{proof}
\lecture[More on $\IE_1$-monoids and $\IE_1$-groups. Group completion.\newline --- \emph{\enquote{The fact that you can write down a group or a monoid in semester 2 is sort of an accident.}}]{2020-11-26}Under the equivalence $\Cat_\infty\simeq \cat{CSAn}$ from \cref{thmdef:RezkNerve}, the construction of $\decomp$ from the proof of \cref{prop:CompletionOfMonoidsFullyFaithful} corresponds to a functor $*/\Cat_\infty\morphism\cat{Mon}(\An)$ which becomes an equivalence when restricted to the full subcategory $(*/\Cat_\infty)_{\geq 1}\subseteq */\Cat_\infty$ spanned by those $\Cc$ with $\pi_0\core\Cc\simeq *$. In diagrams,
\begin{equation*}
	\begin{tikzcd}
		*/\Cat_\infty\rar & \cat{Mon}(\An)\\
		(*/\Cat_\infty)_{\geq 1}\uar[symbol=\subseteq]\urar[iso] & 
	\end{tikzcd}
\end{equation*}
Explicitly, this functor sends a pointed $\infty$-category $(\{x\}\morphism \Cc)$ to $\Hom_\Cc(x,x)$ in degree $1$ (and then all other degrees are determined by the conditions from \cref{def:CartesianMonoids}). We thus obtain:
\begin{cor}
	If $\Cc$ is an $\infty$-category and $x\in \Cc$, then $\Hom_\Cc(x,x)$ carries a canonical structure of an $\IE_1$-monoid.
\end{cor}
\begin{defi}\label{def:E1Group}
	Let $\Cc$ be an $\infty$-category with finite products. A cartesian monoid $X$ in $\Cc$ is called a \emph{cartesian group} (or \emph{$\IE_1$-group} in the case $\Cc=\An$) if the map
	\begin{align*}
		(\pr_1,\circ)\colon X_1\times X_1&\isomorphism X_1\times X_1\\
		(f,g)&\longmapsto (f,f\circ g)
	\end{align*}
	is an equivalence. Here \enquote{$\circ$} is the composition as defined in \cref{thmdef:RezkNerve}\itememph{c}. We denote by $\Grp(\Cc)\subseteq \cat{Mon}(\Cc)$ the full sub-$\infty$-category of cartesian groups.
\end{defi}
\begin{prop}\label{prop:Grp(An)=(*/An)Connected}
	The equivalences $\cat{Mon}(\An)\simeq (*/\Cat_\infty)\simeq (*/\cat{CSAn})_{\pi_0=*}$ restrict to equivalences
	\begin{equation*}
		\begin{tikzcd}[column sep=small]
			\cat{Mon}(\An)\dar[iso]\rar[symbol=\supseteq] & \Grp(\An)\dar[iso]\\
			(*/\Cat_\infty)_{\geq 1}\dar[iso] \rar[symbol=\supseteq] & (*/\An)_{\geq 1}\dar[iso]\\
			(*/\cat{CSAn})_{\pi_0=*}\rar[symbol=\supseteq] & (*/\cat{sAn}_{\mathrm{const}})_{\pi_0=*}
		\end{tikzcd}
	\end{equation*}
	In the lower right corner we use the notation from \cref{exm:MyFirstRezkNerves}\itememph{c}.
\end{prop}
\begin{proof}[Proof sketch]
	In the lecture we noted only that $X\in \cat{Mon}(\An)$ is an $\IE_1$-group iff the set $\pi_0X_1$ with its induced ordinary monoid structure is an ordinary group.
	
	To elaborate on this a bit more, first note that we may assume $X\simeq \Hom_\Cc(x,x)$ for some $\infty$-category $\Cc$ with $\pi_0\core \Cc\simeq *$, as was argued above. If $X$ is an $\IE_1$-group, then $\pi_0\Hom_\Cc(x,x)$ must be an ordinary group, which implies that all endomorphisms of $x$ are equivalences, hence all morphisms in $\Cc$ are equivalences as $\pi_0\core \Cc\simeq *$. This proves that $\Cc$ is an anima by \cref{thm:JoyalLifting}. Conversely, if $\Cc$ is an anima, either inclusion $[1]\monomorphism J$ into the free-living isomorphism induces a trivial fibration $\Fun(J,\Cc)\isomorphism \Ar(\Cc)$. Choosing a section and composing it with the the other map $\Fun(J,\Cc)\morphism \Ar(\Cc)$ (induced by the other inclusion $[1]\monomorphism J$) yields an \enquote{inversion} map $(-)^{-1}\colon \Hom_\Cc(x,x)\morphism \Hom_\Cc(x,x)$. From this we can construct an inverse of the map in \cref{def:E1Group}
\end{proof}
\numpar{Some Explicit Calculations}\label{par:Grp(An)=(*/An)Connected}
The equivalence $\Grp(\An)\simeq (*/\An)_{\geq 1}$ is explicitly given as follows:
\begin{equation*}
	B=|\blank|\colon \Grp(\An)\doublelrmorphism[\sim][\sim](*/\An)_{\geq 1}\noloc \Omega\,.
\end{equation*}
The use $B$ to denote the functor $|\blank| = \colimit_{\IDelta^\op}$ is common in the literature, as it is short for \enquote{bar construction}. Fabian doesn't like it, since he finds it stupid to name something after the way it was typeset in the pre-\LaTeX\ age (where people had to resort to bars to denote their tensor products). Amusingly though, his preferred notation consists of actual bars \dotso In these notes, however, I'll show my bad taste and consistently use $B$.


Anyway, back to the matters at hand. To make sense of why the above equivalence is indeed given by $B=|\blank|$ and $\Omega$, recall that the composition
\begin{equation*}
	\cat{Mon}(\An)\xrightarrow{\comp}\cat{CSAn}\morphism[\ev_0]\An
\end{equation*}
sends $X\mapsto |X^\times|$, as follows from \cref{eq:asscat1} and some unravelling. If $X$ is an $\IE_1$-group, one easily verifies $X^\times=X$ and thus $X\in \Grp(\An)$ is sent to $BX=|X|$, as claimed. Conversely, the composition
\begin{equation*}
	(*/\An)_{\geq 1}\xrightarrow{\decomp}\Grp(\An)\morphism[\ev_1]\An
\end{equation*}
sends a pointed connected anima $(K,k)$ to $\Hom_K(k,k)$, as argued previously. So to justify why $(*/\An)_{\geq 1}\isomorphism \Grp(\An)$ should be thought of as a \enquote{loop space functor}, we must explain why $\Omega_kK\simeq \Hom_K(k,k)$. This can be done by several arguments and I will present a different one than in the lecture. The loop space is usually defined by the left of the following pullback diagrams in $\An$:
\begin{equation*}
	\begin{tikzcd}
		\Omega_kK\vphantom{\Hom_K^L}\dar\rar\drar[pullback]& *\vphantom{K_{k/}}\dar["k"]\\
		*\rar["k"] & K
	\end{tikzcd}\quad\isomorphism\quad
	\begin{tikzcd}
		\Hom_K^L(k,k)\dar\rar\drar[pullback]& K_{k/}\dar\\
		*\rar["k"]& K
	\end{tikzcd}
\end{equation*}
The map $*\morphism K$ can be factored into the left anodyne map $*\monomorphism K_{k/}$ followed by the left fibration $K_{k/}\morphism K$, which is even a Kan fibration since $K$ is an anima. Replacing one of the $*\morphism K$ accordingly, we obtain the pullback diagram on the right, which is already a pullback in $\sSet$ as one leg is a Kan fibration (see \cref{thm:HomotopyLimits}). Its pullback is thus $\Hom_K^L(k,k)\simeq \Hom_K(k,k)$ as left-$\Hom$ anima coincide with the regular $\Hom$ anima by \cite[Digression~I Corollary~D.2]{HigherCatsII}.

In particular, we have proved another result from algebraic topology (or rather deduced it from the much more general statement of \cref{thmdef:RezkNerve}, which we didn't prove, but never mind that).
\begin{cor}[Recognition Principle for Loop Spaces, Stasheff]
	The loop functor $\Omega\colon {*/\An}\morphism \An$ lifts to an equivalence
	\begin{equation*}
		\Omega\colon (*/\An)_{\geq 1}\isomorphism \Grp(\An)\,.
	\end{equation*}
\end{cor}
Note that $\ev_1\colon \cat{Mon}(\An)\morphism \An$ preserves limits since $\cat{Mon}(\An)$ is closed under limits in $\cat{sAn}$ (the Segal condition is given by a limit and limits commute by the dual of \cref{prop:ColimitsCommute}). Colimits however are bloody complicated, and even though one can show that $\cat{Mon}(\An)$ is cocomplete, colimits in it are \emph{by far not} just colimits of underlying simplicial anima, same as colimits of ordinary monoids are much more complicated than colimits in sets.
\begin{prop}\label{prop:InftyGrp}
	The inclusion $\Grp(\An)\subseteq\cat{Mon}(\An)$ has a left adjoint called \emph{group completion} and denoted
	\begin{equation*}
		(-)^\inftygrp\colon \cat{Mon}(\An)\morphism \Grp(\An)\,.
	\end{equation*}
	It is given by $X^\inftygrp\simeq \Omega BX$, where $B\colon \Mon(\An)\morphism ({*/\An})_{\geq 1}$ denotes a functor extending the $B$ from \cref{par:Grp(An)=(*/An)Connected}.
\end{prop}
\begin{proof}
	We know that $\An\subseteq \Cat_\infty$ has a left-adjoint $|\blank|\colon \Cat_\infty\morphism \An$. The unit and counit are equivalences on $*\in \An$, hence the adjunction passes to slices under it. Moreover the $(-)_{\geq 1}$-parts on both sides are clearly preserved, thus $|\blank|$ descends to a left adjoint
	\begin{equation*}
		|\blank|\colon (*/\Cat_\infty)_{\geq 1}\morphism (*/\An)_{\geq 1}
	\end{equation*}
	of the obvious inclusion in the other direction. Now use \cref{prop:Grp(An)=(*/An)Connected} to get the desired left-adjoint $(-)^\inftygrp\colon \cat{Mon}(\An)\morphism\Grp(\An)$. The explicit description follows from our calculations in \cref{par:Grp(An)=(*/An)Connected}, and \cref{rem:Realisation}, which ensures that the different meanings of $|\blank|$ ($\colimit_{\IDelta^\op}$ or localisation of an $\infty$-category at all its morphisms) are actually compatible.
\end{proof}
A consequence that wasn't explicitly mentioned, but used later in the course, is the following:
\begin{cor*}\label{cor*:BOmegaAdjunction}
	Without restricting them as in \cref{par:Grp(An)=(*/An)Connected}, the functors $B$ and $\Omega$ form an adjunction
	\begin{equation*}
		B\colon \Mon(\An)\doublelrmorphism {*/\An}\colon \Omega\,.
	\end{equation*}
\end{cor*}
\begin{proof*}
	For a pointed anima $(X,x)$, let $X_x$ denote the connected component containing $x$. Then $\Omega_xX\simeq \Omega_xX_x$, so the equivalence $\Omega\colon (*/\An)_{\geq 1}\isomorphism\Grp(\An)$ from \cref{par:Grp(An)=(*/An)Connected} extends to a functor $\Omega\colon {*/\An}\morphism \Grp(\An)$. Now if $M\in \Mon(\An)$ is an $\IE_1$-monoid, we can compute
	\begin{align*}
		\Hom_{\Mon(\An)}(M,\Omega_xX)&\simeq \Hom_{\Grp(\An)}(M^\inftygrp,\Omega_xX_x)\\
		&\simeq \Hom_{\Grp(\An)}\big(\Omega BM,\Omega_xX_x\big)\\
		&\simeq \Hom_{(*/\An)_{\geq 1}}\big(BM,X_x\big)\\
		&\simeq \Hom_{*/\An}\big(BM,X\big)\,,
	\end{align*}
	which establishes the desired adjunction. The second equivalence follows from \cref{prop:InftyGrp}, the third equivalence uses that $\Omega\colon (*/\An)_{\geq 1}\isomorphism\Grp(\An)$ is an equivalence, and the fourth equivalence follows from the fact that $BM$ is connected (by construction), hence any pointed map $BM\morphism X$ only hits the connected component $X_x$.
\end{proof*}
Next we will discuss some examples. We begin with free monoids and groups.
\begin{prop}\label{prop:FreeMonoids}
	The evaluation functor $\ev_1\colon \cat{Mon}(\An)\morphism \An$ has a left adjoint $\operatorname{Free}^{\Mon}\colon \An\morphism \cat{Mon}(\An)$. It is given explicitly by the \enquote{anima of words of arbitrary length}
	\begin{equation*}
		\operatorname{Free}^{\Mon}(K)_1\simeq \coprod_{n\geq 0}K^n\,.
	\end{equation*}
\end{prop}
\begin{proof*}
	We didn't prove this in the lecture, but here's a very simple argument that Fabian explained to me. We call a morphism in $\IDelta$ \emph{inert} if it is the inclusion of an interval, and \emph{active} if it preserves the largest and the smallest element. Note that the Segal maps $e_i\colon [1]\morphism {[n]}$ are precisely the inert maps with source $[1]$, and that every map can be uniquely factored into an inert followed by an active. We let $\IDelta^\op_\mathrm{int}\subseteq \IDelta^\op$ denote the (non-full) subcategory spanned by the inert maps. Let's first verify the following two claims.
	\begin{alphanumerate}
		\item[\itememph{1}] \itshape Consider the full sub-$\infty$-category $\Fun^\mathrm{Seg}(\IDelta^\op_\mathrm{int},\An)$ spanned by the \enquote{reduced Segal functors}, i.e.\ those $X\colon \IDelta^\op_\mathrm{int}\morphism \An$ satisfying $X_0\simeq *$ and $X_n\simeq \prod_{i=1}^nX_1$ via the Segal maps. Then evaluation at $[1]$ induces an equivalence
		\begin{equation*}
			\ev_{1}\colon \Fun^\mathrm{Seg}(\IDelta^\op_\mathrm{int},\An)\isomorphism \An\,.
		\end{equation*}
		\item[\itememph{2}] Consider the left-Kan extension functor $\Lan_i\colon \Fun(\IDelta^\op_\mathrm{int},\An)\morphism \Fun(\IDelta^\op,\An)$ along $i\colon \IDelta^\op_\mathrm{int}\morphism \IDelta^\op$.Then $\Lan_i$ preserves reduced Segal functors.
	\end{alphanumerate}
	To show \itememph{1}, we verify that right-Kan extension along the inclusion $j\colon \{[1]\}\morphism \IDelta_\mathrm{int}^\op$ defines an inverse. Using the pointwise formulas from \cref{thm:KanExtension} together with the fact that the inert maps $[1]\morphism{[n]}$ are precisely the Segal maps, we see that $\Ran_j$ indeed takes values in the reduced Segal functors, and that $\Ran_j\circ \ev_{1}\simeq \id$. To show $\ev_{1}\circ \Ran_j\simeq \id$, observe that $j$ is fully faithful and use \cref{cor:FullyFaithfulKanExtension}. This proves \itememph{1}.
	
	To prove \itememph{2}, let $X\colon \IDelta^\op_\mathrm{int}\morphism \An$ be reduced Segal. By \cref{thm:KanExtension},
	\begin{equation*}
		(\Lan_iX)_n\simeq \colimit_{([k]\morphism{ [n]})\in \IDelta^\op_\mathrm{int}/[n]}X_k\simeq \colimit_{([k]\morphism {[n]})\in \IDelta^\op_\mathrm{int}/[n]} X_1^k\,.
	\end{equation*}
	An object $[k]\morphism {[n]}$ of $\IDelta^\op_\mathrm{int}/[n]$ corresponds to a map $[n]\morphism{} [k]$ in $\IDelta$. Any such map can be uniquely factored into an inert $[n']\morphism{} [k]$ followed by an active $\alpha\colon [n]\morphism{} [n']$. This allows us to decompose $\IDelta^\op_\mathrm{int}/[n]$ into a disjoint union $\coprod_\alpha \Cc_\alpha$, indexed by all active maps $\alpha\colon [n]\morphism {[n']}$ with source $[n]$. Moreover, each $\Cc_\alpha$ has a terminal object, which is given by $\alpha$ itself (or rather the corresponding object $\alpha^\op\colon[n']\morphism{[n]}$ in $\IDelta^\op_\mathrm{int}/[n]$).
	
	For $n=0$ this shows $(\Lan_iX)_n\simeq *$, since $\id_{[0]}\colon [0]\morphism {[0]}$ is the only active map with source $[0]$. For $n=1$, we get a unique active map $[1]\morphism {[n]}$ for all $n$, which shows $(\Lan_iX)_1\simeq \coprod_{k\geq 0}X_1^k$. For general $n$, we get
	\begin{equation*}
		(\Lan_iX)_n\simeq \coprod_{k_1,\dotsc,k_n\geq 0} X_1^{k_1+\dotsb +k_n}\,.
	\end{equation*}
	Upon inspection, this shows that $\Lan_iX$ is again reduced Segal. We've thus proved~\itememph{2}. Combining \itememph{1} and~\itememph{2} shows that $\Free^\Mon\colon \An\morphism \Mon(\An)$ is given by $\Lan_i\circ \Ran_j$. But then the calculations above show $\operatorname{Free}^{\Mon}(K)_1\simeq \coprod_{n\geq 0}K^n$, as claimed.
\end{proof*}
%We postpone the proof of \cref{prop:FreeMonoids} indefinitely (but Fabian's notes give a sketch in the much more general setting of algebras over $\infty$-operads, see \cite[Chapter~II pp.~128--132]{KTheory}). 
By the explicit construction of $\Free^\Mon$, the diagram
\begin{equation*}
	\begin{tikzcd}[column sep=large]
		\Set\dar\rar["\operatorname{Free}"]& \cat{Mon}(\Set)\dar\\
		\An\rar["\operatorname{Free}^{\Mon}"] & \cat{Mon}(\An)
	\end{tikzcd}
\end{equation*}
commutes. But be aware that this is not automatic, and in fact we'll see it very much breaks once we impose commutativity (see pages~\labelcpageref{par:FreeCMon,page:TerrifyingExample}). As some first examples, one has $\operatorname{Free}^{\Mon}(*)\simeq\IN$ and $\operatorname{Free}^{\Mon}(\emptyset)\simeq *$, which is initial in $\cat{Mon}(\An)$.
\begin{prop}\label{prop:FreeGroups}
	The evaluation functor $\ev_1\colon \Grp(\An)\morphism \An$ has a left-adjoint $\Free^\Grp\colon \An\morphism\Grp(\An)$ too. Explicitly, it is given by the composite
	\begin{align*}
		\An&\morphism */\An\morphism[\Sigma]*/\An\morphism[\Omega]\Grp(\An)\\
		X&\longmapsto X\sqcup *\,.
	\end{align*}
	In other words, the free $\IE_1$-group on $X$ is $\Free^\Grp(X)\simeq\Omega\Sigma X_+$, where $X_+=X\sqcup *$. As usual, the suspension functor $\Sigma\colon \An\morphism \An$ is given by the pushout diagram
	\begin{equation*}
		\begin{tikzcd}
			X\dar\rar\drar[pushout] & *\dar\\
			* \rar& \Sigma X
		\end{tikzcd}
	\end{equation*}
\end{prop}
\begin{rem*}\label{rem*:CoLimitsIn(*/An)OrAn}
	For the proof, we'll need a functor $\Sigma\colon {*/\An}\morphism */\An$ instead. It is induced by its unpointed variant, of course, but we can also define it by taking the above pushout in $*/\An$. In fact, it doesn't matter whether the pushout is taken in $*/\An$ or $\An$ since $*/\An\morphism \An$ commutes with weakly contractible colimits (but not with arbitrary ones, which I erroneously claimed in an earlier version of these notes; for example, it doesn't preserve the initial object). Indeed, if $\Ii$ is any diagram shape, then a quick calculation shows $\Fun(\Ii,*/\An)\simeq (\const *)/\Fun(\Ii,\An)$. For the adjunction 
	\begin{equation*}
		\colimit_\Ii\colon \Fun(\Ii,\An)\doublelrmorphism\An\noloc \const
	\end{equation*}
	to descend to slice categories below $\const*$ and $*$, we need that these objects are mapped to each other and that unit/counit are equivalences on them. If $\Ii$ is weakly contractible, then $\colimit_\Ii (\const *)\simeq |\Ii|\simeq *$ by \cref{prop:CoLimitsInCat} and the required conditions are easily checked. Therefore, the adjunction above descends to an adjunction 
	\begin{equation*}
		\colimit_\Ii\colon \Fun(\Ii,*/\An)\simeq (\const *)/\Fun(\Ii,\An)\doublelrmorphism */\An\noloc \const
	\end{equation*}
	in this case. If you think about this for a moment, this is exactly what we need to show that $*/\An\morphism \An$ commutes with colimits over $\Ii$.
	
	Similarly, once we have chosen a basepoint $x\in X$, it doesn't matter whether we take the defining pullback of $\Omega_xX$ in $*/\An$ or in $\An$ (and then equip it with its natural basepoint). This can be seen as above, or simply by the fact that $*/\An\morphism \An$ has a left adjoint, so we can apply \cref{obs:AdjointsPreserveLimits}. However, in contrast to $\Sigma$, we can't define $\Omega$ as a functor on $\An$, since there is no canonical choice of basepoint.
\end{rem*}
\begin{proof}[Proof sketch of \cref{prop:FreeGroups}]
	The existence of a left adjoint would be immediate from Propositions~\labelcref{prop:InftyGrp,prop:FreeMonoids}, but we can give a direct argument that also yields the explicit description. We have seen in \cref{par:Grp(An)=(*/An)Connected} that the diagram
	\begin{equation*}
		\begin{tikzcd}
			\Grp(\An)\dar["B"']\rar["\ev_1"]& \An\\
			(*/\An)_{\geq 1}\urar["\Omega"',""{name=A,sloped}] & \arrow[from=1-1,to=A,phantom,"\scriptscriptstyle/\!/\!/"]
		\end{tikzcd}
	\end{equation*}
	commutes. Observe that $(-)_+=-\sqcup*\colon \An\morphism */\An$ is a left adjoint of the forgetful functor $*/\An\morphism \An$ (this is basically trivial) and that $\Sigma\colon {*/\An}\shortdoublelrmorphism */\An\noloc \Omega$ are adjoints (this we leave as an exercise---just play around with the defining pushouts/pullbacks, which can be taken in $*/\An$ by \cref{rem*:CoLimitsIn(*/An)OrAn}). Finally, $\Omega\colon (*/\An)_{\geq 1}\shortdoublelrmorphism \Grp(\An)\noloc B$ is an adjunction (even an equivalence) by \cref{par:Grp(An)=(*/An)Connected}. The assertion now follows from the fact that left adjoints compose.
\end{proof}
\begin{exm}\label{exm:MyFirstMonoidals}
	\begin{alphanumerate}
		\item The evaluation functor $\ev_1\colon \cat{Mon}(\An)\morphism \An$ factors canonically over $*/\An\morphism \An$, and as it turns out, the resulting functors
		\begin{equation*}
			\ev_1\colon \cat{Mon}(\An)\morphism {*/\An}\quad\text{and}\quad \ev_1\colon \Grp(\An)\morphism {*/\An}
		\end{equation*}
		both have left adjoints. For a pointed anima $(X,x)$, we denote the corresponding left adjoint objects by $\operatorname{Free}^{\Mon}(X,x)$ and $\operatorname{Free}^{\Grp}(X,x)$ and think of them as the \enquote{free $\IE_1$-monoid/group with unit $x$}. We have
		\begin{equation*}
			\Free^\Mon(X,x)^\inftygrp\simeq \Free^\Grp(X,x)\simeq\Omega\Sigma X
		\end{equation*}
		The first equivalence follows from the fact that left adjoints compose. The second equivalence can be seen along the lines of the proof of \cref{prop:FreeGroups}, without us needing to know that $\Free^\Mon(X,x)$ exists.
		
		If $(X,x)$ is connected, one actually has $\operatorname{Free}^{\Mon}(X,x)\simeq \operatorname{Free}^{\Grp}(X,x)$. Indeed, as we've seen in the proof of \cref{prop:Grp(An)=(*/An)Connected}, we only need to check that the ordinary monoid $\pi_0\operatorname{Free}^{\Mon}(X,x)_1$ is already an ordinary group. Playing around with universal properties, we see that $\pi_0\operatorname{Free}^{\Mon}(X,x)_1$ is the free ordinary monoid on the set $\pi_0 X$ with unit $[x]$, hence just $\{[x]\}$ itself since $X$ is connected, hence it already is a group.
		\item By \cref{prop:FreeMonoids}, $\operatorname{Free}^{\Mon}(\IS^1)\simeq \coprod_{n\geq 0}\IT^n$ is an infinite union of tori $\IT^n=\prod_{i=1}^n\IS^1$ and thus a reasonable geometric object. However, \cref{prop:FreeGroups} implies
		\begin{equation*}
			\operatorname{Free}^{\Grp}(\IS^1)\simeq \Omega\Sigma(\IS^1_+)\simeq \Omega(\IS^2\vee\IS^1)\,,
		\end{equation*}
		and the homotopy groups of the right-hand side are not fully known to this day. Similarly, using the pointed variant from \itememph{a} above, we get
		\begin{equation*}
			\operatorname{Free}^{\Mon}(\IS^1,*)\simeq\operatorname{Free}^{\Grp}(\IS^1,*)\simeq \Omega\Sigma(\IS^1,*)\simeq\Omega\IS^2
		\end{equation*}
		\item \lecture[A pushout computation. Monoidal $1$-categories give monoidal $\infty$-categories. Definition of algebraic and hermitian $K$-theory.]{2020-12-01}\hspace{-1ex}It is completely formal to see that $\ev_1\colon \cat{Mon}(\An)\morphism \An$ and $\ev_1\colon \Grp(\An)\morphism \An$ are represented by $\operatorname{Free}^{\Mon}(*)$ and $\operatorname{Free}^{\Grp}(*)\simeq \Free^\Mon(*)^\inftygrp$ respectively. What's not formal, however, is that $\operatorname{Free}^{\Mon}(*)\simeq \IN$ and $\operatorname{Free}^{\Grp}(*)\simeq \Omega \IS^1\simeq\IZ$, regarded as discrete simplicial anima. This really needs Propositions~\labelcref{prop:FreeMonoids,prop:FreeGroups}.
		
		Moreover, we'll see (pages~\labelcpageref{par:FreeCMon,page:TerrifyingExample}) that the analogous functors are no longer represented by $\IN$ and $\IZ$ once we impose commutativity, even though they are the free $\IE_1$-monoid/group on a point and happen to be commutative. In fact they are \enquote{too commutative to be free commutative monoids} (*\emph{sphere spectrum intensifies}*).
		\item Let's for some reason compute the pushout
		\begin{equation*}
			\begin{tikzcd}
				\IN \rar["\cdot2"]\dar\drar[pushout] & \IN\dar\\
				* \rar & \IN /\!\!/ 2
			\end{tikzcd}
		\end{equation*}
		in $\cat{Mon}(\An)$. The answer is $\IN/\!\!/ 2\simeq \Omega \IR\IP^2$. To see this, first observe that the functor $\pi_0\colon \cat{Mon}(\An)
		\morphism\cat{Mon}(\Set)$ commutes with colimits, so $\pi_0(\IN/\!\!/2)=\IZ/2$ since we know how to compute pushouts in $\cat{Mon}(\Set)$. But $\IZ/2$ is a group, so $\IN/\!\!/2\in \Grp(\An)$ by \cref{prop:Grp(An)=(*/An)Connected}. In particular, since the group completion functor $(-)^\inftygrp$ commutes with colimits, we get $\IN/\!\!/2\simeq \IZ/\!\!/2$, i.e., the diagram
		\begin{equation*}
			\begin{tikzcd}
				\IZ \rar["\cdot2"]\dar\drar[pushout] & \IZ\dar\\
				* \rar & \IN /\!\!/ 2
			\end{tikzcd}
		\end{equation*}
		is a pushout in $\Grp(\An)$. Translating to $*/\An$ via \cref{prop:Grp(An)=(*/An)Connected} gives a diagram
		\begin{equation*}
			\begin{tikzcd}
				\IS^1\rar["\cdot 2"]\dar\drar[pushout] & \IS^1\dar\\
				\ID^2\rar & \IR\IP^2
			\end{tikzcd}
		\end{equation*}
		(we have replaced $*$ by the disk $\ID^2$ to make one leg of the pushout into a cofibration so that it can be computed by an ordinary pushout of CW complexes by \cref{thm:HomotopyLimits}) and translating back using \cref{par:Grp(An)=(*/An)Connected} shows $\IN/\!\!/2\simeq \Omega\IR\IP^2$, as claimed.
		\item Every monoidal $1$-category $(\Cc,\otimes)$ gives rise to a monoidal $\infty$-category, i.e.\ an object in $\cat{Mon}(\Cat_1^{(2)})\subseteq \cat{Mon}(\Cat_\infty)$. We will produce this by straightening a suitable cocartesian fibration $p^{\otimes}\colon \Cc^\otimes\morphism \IDelta^\op$ of $1$-categories. Define $\Cc^\otimes\in \Cat_1^{(2)}$ as follows: Its objects are
		\begin{equation*}
			\operatorname{obj}(\Cc^\otimes)=\coprod_{n\geq 0}\operatorname{obj}(\Cc)^n\,,
		\end{equation*}
		i.e.\ given by tuples $(n, x_1,\dotsc,x_n)$ with $[n]\in \IDelta$ and $x_i\in \Cc$. We'll usually abbreviate $x=(x_1,\dotsc,x_n)$ and simply write $(n,x)$. Morphisms are defined as
		\begin{equation*}
			\Hom_{\Cc^\otimes}\big((n,x),(m,y)\big)=\left\{(\alpha,f)\st \begin{tabular}{c}
				$\alpha\colon [m]\rightarrow [n]$ in $\IDelta$ and $f=(f_1,\dotsc,f_m)$\\
				where $f_j\colon x_{\alpha(j-1)+1}\otimes\dotsb\otimes x_{\alpha(j)}\rightarrow y_j$\\
				are morphisms in $\Cc$ for all $j=1\dotsc,m$
			\end{tabular}\right\}\,.
		\end{equation*}
		If $\alpha(j-1)+1>\alpha(j)$ for some $j$, then $x_{\alpha(j-1)+1}\otimes\dotsb\otimes x_{\alpha(j)}$ is the empty tensor product, which is the tensor unit $1_\Cc\in\Cc$ by definition.
		
		The composition of $(\alpha,f)\colon(n,x)\morphism (m,y)$ and $(\beta,g)\colon (m,y)\morphism (k,z)$ is given as follows: Its first component is $\alpha\circ\beta\colon [k]\morphism{[n]}$. For the second component, observe that we can write
		\begin{equation*}
			x_{\alpha\beta(j-1)+1}\otimes \dotsb \otimes x_{\alpha\beta(j)}=\bigotimes_{i=1}^{\beta(j)-\beta(j-1)}\left(x_{\alpha(\beta(j-1)+i-1)+1}\otimes\dotsb\otimes x_{\alpha(\beta(j-1)+i)}\right)\,.
		\end{equation*}
		The $i\ordinalth$ tensor factor on the right maps to $y_{\beta(j-1)+i}$ via $f_{\beta(j-1)+i}$. Tensoring them together and postcomposing with $g_j\colon y_{\beta(j-1)+1}\otimes\dotsb\otimes z_j$ provides the desired composition of $f$ and $g$. This finishes the construction of $\Cc^\otimes$.
		
		The functor $p^\otimes\colon \Cc^\otimes\morphism\IDelta^\op$ just extracts $[n]$ from $(n,x)$ and $\alpha$ from $(\alpha,f)$. The fibres $\Cc_n^\otimes$ of $p^\otimes$ are then obviously given by $\Cc^n$. To show that $p$ is a cocartesian fibration of $1$-categories, we must provide cocartesian lifts of morphisms in $\IDelta^\op$. Given $\alpha\colon [m]\morphism {}[n]$ in $\IDelta$ and $(n,x)\in \Cc^\otimes$, we claim that the identities can be regarded as a morphism
		\begin{equation*}
			\alpha_*\colon (n,x)\morphism \left(m,x_{\alpha(0)+1}\otimes\dotsb\otimes x_{\alpha(1)},\dotsc,x_{\alpha(m-1)+1}\otimes\dotsb\otimes x_{\alpha(m)}\right)\,.
		\end{equation*}
		It's straightforward to check that $\alpha_*$ is a cocartesian lift of $\alpha$ (note that the daunting homotopy pullbacks in \cref{def:WeirdCocartesianDefinition} become just good old pullbacks of sets/discrete anima, since we are dealing with $1$-categories).
		
		It remains to check that $\St (p^\otimes)$ satisfies the conditions from \cref{def:CartesianMonoids}. As seen above, $\St (p^\otimes)_0\simeq \Cc_0^\otimes \simeq \Cc^0\simeq *$ is a point, as desired. To verify the Segal condition, let's organize the world a little bit. Recall that a map in $\IDelta$ is called \emph{inert} if it is the inclusion of an interval and \emph{active} if it preserves the largest and the smallest element. The Segal maps $e_i\colon [1]\morphism {}[n]$ are precisely the inert maps with source $[1]$. We claim that for a general inert map $\alpha\colon [m]\morphism{[n]}$, the induced functor (given by taking cocartesian lifts) 
		\begin{equation*}
			\alpha_*\colon \Cc^n\simeq \Cc_n^\otimes \morphism \Cc_m^\otimes \simeq \Cc^m
		\end{equation*}
		simply sends $(x_1,\dotsc,x_n)\in \Cc^n$ to $(x_{\alpha(1)},\dotsc,x_{\alpha(m)})\in \Cc^m$ with no tensor products occuring. Thus the Segal condition follows immediately.
		
		As a slogan, \enquote{inerts induce forgetful functors on fibres}, whereas \enquote{actives don't forget, they just tensor}. %Generally, any map in $\IDelta$ can be uniquely factored into an inert followed by an active.
		\item The construction from \itememph{e} also works if $\Cc$ is Kan enriched. In this case one can again construct a cocartesian fibration $p^\otimes\colon \Cc^\otimes \morphism \IDelta^\op$, but of Kan-enriched categories this time (where $\IDelta^\op$ receives its discrete enrichment). Then $\N^c(\Cc^\otimes)\morphism \N^c(\IDelta^\op)\simeq \IDelta^\op$ straightens to a monoidal $\infty$-category as well. Fabian wouldn't tell us what a cocartesian fibration of Kan-enriched categories is though, since this would take---not a lot, but too much---time.
	\end{alphanumerate}
\end{exm}
%We're now very close to defining $K$-theory as $K_i(R)\coloneqq \pi_i\Omega (\Proj^\fg(R))^\inftygrp$ (or, if you prefer the Bar notation that Fabian hates, $\pi_i\Omega B(\Proj^\fg(R))$). What's left to do is to investigate $\Proj^\fg(R)$ and to turn it into an $\IE_1$-monoid.
\cref{exm:MyFirstMonoidals}\itememph{e} allows us to construct many interesting examples. All examples will turn out to be commutative (see \cref{exm:CommutativeHorizontalAdjoints}), but that doesn't matter for now.

\numpar{Algebraic $K$-Theory}\label{par:AlgebraicKTheory}
Let $R$ be a ring and consider $\cat{Mod}(R)$ as a monoidal category under $\oplus$. Moreover, let $\Proj(R)$ denote the sub-groupoid of spanned by finite projective $R$-modules (\enquote{vector bundles on $\Spec R$}) and their isomorphisms. This inherits a (symmetric) monoidal structure by \cref{exm:MyFirstMonoidals}\itememph{e}, so we obtain
	\begin{equation*}
		\Proj(R)\in \cat{Mon}\big(\Grpd_1^{(2)}\big)\subseteq \cat{Mon}(\An)\,.
	\end{equation*}
\begin{smalldefi}[Quillen]\label{def:AlgebraicKTheory}
	The \emph{projective class anima} (or \emph{algebraic $K$-theory space}) of a ring $R$ is defined as 
	\begin{equation*}
		k(R)\coloneqq \Proj(R)^\inftygrp\in \Grp(\An)\,,
	\end{equation*}
	and the \emph{higher projective class groups} (or \emph{higher $K$-groups}) of $R$ are given by
	\begin{equation*}
		K_i(R)=\pi_i\big(k(R)_1,*\big)\,.
	\end{equation*}
\end{smalldefi}
The reason we write $k(R)$ is that the notation $K(R)$ is reserved for the $K$-theory \emph{spectrum} which we will eventually define. Note immediately that
\begin{equation*}
	K_0(R)=\pi_0\big(\Proj(R)^\inftygrp\big)=\big(\pi_0\Proj(R)\big)^\grp=\left\{\begin{tabular}{c}
		iso.\ classes of finite\\
		projective $R$-modules
	\end{tabular}\right\}^\grp\,,
\end{equation*}
where $(-)^\grp\colon \cat{Mon}(\Set)\morphism \Grp(\Set)$ denotes the ordinary group completion. This shows that $K_0(R)$ is precisely what we wanted it to be way back in the introduction, \cref{def:K0R}. We will soon(ish, \cref{cor:K1R}) see that also $K_1(R)$ can be described as in \cref{def:K1R}.


\numpar{Hermitian $K$-Theory/Grothendieck--Witt Theory}\label{par:HermitianKTheory}
Let $R$ be a ring, $M$ an $R\otimes R$-module and $\sigma\colon M\morphism M$ an involution that is flip-linear, i.e.\ $\sigma((a\otimes b)m)=(b\otimes a)\sigma(m)$. Moreover, assume that $M$ is finite projective when considered as an $R$-module via the map $R\simeq \IZ\otimes R\morphism R\otimes R$ (one can check that it doesn't matter if we choose this map or $R\simeq R\otimes \IZ\morphism R\otimes R$) and that the map
\begin{align*}
	R&\isomorphism \Hom_R(M,M)\\
	r&\longmapsto \big(m \mapsto (r\otimes 1)m\big)
\end{align*}
is an isomorphism. A pair $(M,\sigma)$ subject to these conditions is called an \emph{invertible module with involution} over $R$ (beware that this is non-standard terminology).

For a finite projective $R$-module $P\in \Proj (R)$ consider the group $\Hom_{R\otimes R}(P\otimes P,M)$ of $M$-valued $R$-bilinear forms on $P$. This is acted upon by $\IZ/2$ via conjugation with the flip of $P\otimes P$ and $\sigma$ on $M$. Define the groups of \emph{symmetric} and \emph{quadratic forms} on $P$ as
\begin{align*}
	\Sym_R(P,M)&\coloneqq \Hom_{R\otimes R}(P\otimes P,M)^{\IZ/2}\\
	\Quad_R(P,M)&\coloneqq \Hom_{R\otimes R}(P\otimes P,M)_{\IZ/2}\,,
\end{align*}
where $(-)^{\IZ/2}$ denotes the invariants of the $\IZ/2$-action and $(-)_{\IZ/2}$ denotes the coinvariants (i.e.\ take the coequalizer rather than the equalizer). Finally, let the group of \emph{even forms} on $P$ be
\begin{equation*}
	\Even_R(P,M)\coloneqq \im\left(\Nm\colon \Quad_R(P,M)\rightarrow \Sym_R(P,M)\right)\,,
\end{equation*}
where $\Nm$ denotes the \emph{norm map} which can be defined on general abelian groups with a $\IZ/2$-action: If $X\in \IZ/2\mhyph\Ab$ is an abelian group with $\IZ/2$ acting via an involution $\tau\colon X\morphism X$,
\begin{align*}
	\Nm\colon X_{\IZ/2}&\morphism X^{\IZ/2}\\
	[x]&\longmapsto x+\tau(x)\,.
\end{align*}
A symmetric form $q\in \Sym_R(P,M)$ is called \emph{unimodular} if the induced map
\begin{equation*}
	q_*\colon P\isomorphism \Hom_R(P,M)\eqqcolon D_M(P)
\end{equation*}
is an isomorphism. The right-hand side should be thought of the \enquote{$M$-valued dual} of $P$. This is reasonable because the assumptions on $M$ are chosen precisely to ensure that $D_M\colon \Proj(R)^\op\isomorphism \Proj(R)$ is an equivalence with inverse $D_M^\op$.

Similarly, an even or quadratic form is called \emph{unimodular} if their image in $\Sym_R(P,M)$ (via the inclusion or via $\Nm$) is unimodular. For $r\in\{\text{sym},\text{quad}, \text{even}\}$ we denote
\begin{equation*}
	\Unimod^r(P,M)=\left\{\begin{tabular}{c}
		groupoid of unimodular $r$-forms, i.e. pairs\\
		$(P,q)$, where $q$ is a unimodular $r$-form on $P$,\\
		and isomorphisms between them
	\end{tabular}\right\}
\end{equation*}
This is a symmetric monoidal $1$-groupoid via $\oplus$, hence induces a symmetric monoidal anima by \cref{exm:MyFirstMonoidals}\itememph{e}.
\begin{smalldefi}[Karoubi]
	For $r\in\{\text{sym},\text{quad}, \text{even}\}$, the \emph{Grothendieck--Witt anima} of $(R,M)$ is given by
	\begin{equation*}
		\gw^r(R,M)\coloneqq \Unimod^r(R,M)^\inftygrp\,.
	\end{equation*}
\end{smalldefi}
\lecture[Topological $K$-theory and its connection to deep theorems in geometric topology. Cartesian commutative monoids and groups.]{2020-12-03}To make the rather technical definitions of symmetric, quadratic, and even forms a bit clearer, we discussed some examples. Choose $R$ a commutative ring, let $M=R$ and make it an $R\otimes R$-module via the multiplication map $\mu\colon R\otimes R\morphism R$. Choosing either $\sigma=\id_R$ or $\sigma=-\id_R$, we we see that $\Sym_R(P,M)$ returns the usual notion of symmetric or skew-symmetric $R$-bilinear forms on $P$.

If $R=\IC$, we can also equip $M=\IC$ with a $\IC\otimes\IC$-module structure via
\begin{equation*}
	\IC\otimes\IC\xrightarrow{\id\otimes \overline{(-)}}\IC\otimes \IC\morphism[\mu]\IC\,,
\end{equation*}
where $\overline{(-)}\colon \IC\morphism\IC$ denotes complex conjugation, as usual. In this case, $\Sym_\IC(P,\IC)$ amounts to hermitian or skew-hermitian forms on $P$, depending on whether $\sigma=\id_\IC$ or $\sigma=-\id_\IC$.

\begin{exc}\label{exc:WhatQuad?}
	Show that $\Quad_R(P,M)$ is in bijection with the set of all pairs $(b,q)\in \Sym_R(P,M)\times \Hom_\Set(P,M_{\IZ/2})$ satisfying
	\begin{equation*}
		b(p,p)=\Nm\big(q(p)\big)\quad\text{and}\quad q(rp)=(r\otimes r)q(p) 
	\end{equation*}
	for all $p\in P$ and $r\in R$. Under this bijection, the map $\Nm\colon \Quad_R(P,M)\morphism\Sym_R(P,M)$ simply forgets $q$.
\end{exc}
What \cref{exc:WhatQuad?} means depends a little on $R$. We always assume that $R$ is commutative.
\begin{alphanumerate}
	\item If $\sigma=\id_M$, then $M_{\IZ/2}=M=M^{\IZ/2}$ and $\Nm$ is simply multiplication by $2$. So if $M$ has no $2$-torsion, then $q$ is already defined by $b$ and the equation $b(p,p)=\Nm(q(p))$. Hence the norm map $\Nm\colon \Quad_R(P,M)\morphism \Sym_R(P,M)$ is an isomorphism onto its image, which means that quadratic forms and even forms are in bijection in this case.
	\item If $\sigma=-\id_M$, then $M_{\IZ/2}=M/2$ and $M^{\IZ/2}=M[2]=2\text{-torsion of }M$. Hence $\Nm\colon M_{\IZ/2}\morphism M^{\IZ/2}$ is the zero map. If $M[2]=0$, then $b(p,p)=0$ for all $b\in \Sym_R(P,M)$ and all $p\in P$, hence $\Nm\colon \Quad_R(P,M)\morphism \Sym_R(P,M)$ is surjective, because under the identification from \cref{exc:WhatQuad?}, $(b,0)$ is always a preimage of $b$. Thus all symmetric forms are even, but in general an even form may have lots of quadratic forms that map to it.
	\item If $2\in R^\times$ is a unit and $\sigma=\id_M$, then $\Nm\colon \Quad_R(P,M)\morphism\Sym_R(P,M)$ is an isomorphism and the notions of symmetric, quadratic, and even forms all agree.
\end{alphanumerate} 

\numpar{Topological $K$-Theory}\label{par:TopologicalKTheory}
Let's define four ordinary categories as follows:
\begin{align*}
	\cat{Vect}_\IR&=\left\{\begin{tabular}{c}
		finite-dim.\ $\IR$-vector spaces,\\
		$\IR$-linear homeomorphisms
	\end{tabular}\right\}\,, & \cat{Eucl}_\IR&=\left\{\begin{tabular}{c}
	finite-dim.\ $\IR$-vector spaces,\\
	arbitrary homeomorphisms
	\end{tabular}\right\}\,,\\
	\cat{Vect}_\IC&=\left\{\begin{tabular}{c}
		finite-dim.\ $\IC$-vector spaces,\\
		$\IC$-linear homeomorphisms
	\end{tabular}\right\}\,, & \cat{Sph}_\IR&=\left\{\begin{tabular}{c}
	finite-dim.\ $\IR$-vector spaces,\\
	proper homotopy equiv's
\end{tabular}\right\}\,.
\end{align*}
We didn't include $\cat{Vect}_\IC$ in the lecture, but in hindsight we really should have. These four categories can be Kan-enriched: For each category $\Cc$ among them, we let $\F_\Cc(U,V)_n$ be the set of continuous maps $|\Delta^n|\times U\morphism V$ such that $\{d\}\times U\morphism V$ is a morphism in $\Cc$ for all points $d\in|\Delta^n|$ of the topological $n$-simplex. Moreover, all of them carry a Kan-enriched (symmetric) monoidal structure via the usual product  $\times$. Hence we may apply the construction from \cref{exm:MyFirstMonoidals}\itememph{f}, to obtain monoidal anima
\begin{equation*}
	\cat{\Vv ect}_\IR\,,\quad\cat{\Vv ect}_\IC\,,\quad\cat{\Ee ucl}_\IR\,,\quad\text{and}\quad\cat{\Ss ph}_\IR\,.
\end{equation*}
Observe that while $\cat{Vect}_\IR$, $\cat{Vect}_\IC$, and $\cat{Eucl}_\IR$ are already groupoids, $\cat{Sph}_\IR$ is not. But it becomes an anima after taking the coherent nerve, since homotopy equivalences have homotopy inverses. Also observe that the one-point compactification $(-)^*$ of topological spaces induces equivalences
\begin{equation*}
	\Hom_{\cat{\Ss ph}_\IR}(U,V)\simeq \Hom_{\core(*/\An)}(U^*,V^*)
\end{equation*}
for all finite-dimensional $\IR$-vector spaces $U,V\in\cat{\Ss ph}_\IR$. Writing $U= \IR^n$, $V=\IR^m$, we get that $U^*=\IS^n$ and $V^*=\IS^m$ are actually spheres. Hence the name $\cat{\Ss ph}_\IR$.
\begin{smalldefi}\label{def:koEtAl}
	Define the following $\IE_1$-groups:
	\begin{equation*}
		k\cat{o}=\cat{\Vv ect}_\IR^\inftygrp\,,\quad k\cat{u}=\cat{\Vv ect}_\IC^\inftygrp\,,\quad k\cat{top}=\cat{\Ee ucl}_\IR^\inftygrp\,,\quad\text{and}\quad k\cat{sph}=\cat{\Ss ph}_\IR^\inftygrp\,.
	\end{equation*}
\end{smalldefi}
Before we indulge ourselves in their properties, I'd like to do two reality checks. The first one is an alternative construction of $\cat{\Vv ect}_\IR$ and $\cat{\Vv ect}_\IC$.
\begin{lem*}\label{lem*:VectBO}
	Let $\cat{O}(n)\subseteq \GL_n(\IR)$ and $\cat{U}(n)\subseteq \GL_n(\IC)$ be the $n\ordinalth$ orthogonal and the $n\ordinalth$ unital group. Then
	\begin{equation*}
		\cat{\Vv ect}_\IR\simeq \coprod_{n\geq 0}B\cat{O}(n)\quad\text{and}\quad\cat{\Vv ect}_\IC\simeq \coprod_{n\geq 0}B\cat{U}(n)\,.
	\end{equation*}
\end{lem*}
The groups $\cat{O}(n)$, $\GL_n(\IR)$, $\cat{U}(n)$ and $\GL_n(\IC)$ all admit CW structures, hence they define $\IE_1$-groups via
\begin{equation*}
	\Grp(\cat{CW})\morphism \Grp\big(\N^c(\cat{CW})\big)\simeq \Grp(\An)\,.
\end{equation*}
Thus it makes sense to talk of $B\cat{O}(n)$, $B\!\GL_n(\IR)$, $B\cat{U}(n)$, and $B\!\GL_n(\IC)$. We can also find similar descriptions of $\cat{\Ee ucl}_\IR$ and $\cat{\Ss ph}_\IR$, but let me not get into that.
\begin{proof*}[Proof of \cref{lem*:VectBO}]
	It's well-known that $\cat{O}(n)$ and $\cat{U}(n)$ are deformation retracts of $\GL_n(\IR)$ and $\GL_n(\IC)$, respectively (see \cite[Section~3.D]{Hatcher} for example). Hence it suffices to prove stuff for the latter.
	
	Let $\cat{\Vv ect}_\IR^n\subseteq \cat{\Vv ect}_\IR$ denote the connected component corresponding to $n$-dimensional vector spaces. By \cref{par:Grp(An)=(*/An)Connected} and \cref{thm:CordierPorter}, constructing an equivalence $B\!\GL_n(\IR)\isomorphism \cat{\Vv ect}_\IR^n$ is the same as constructing an equivalence
	\begin{equation*}
		\GL_n(\IR)\isomorphism \Omega_{\IR^n}\cat{\Vv ect}_\IR^n\simeq \Hom_{\cat{\Vv ect}_\IR}(\IR^n,\IR^n)\simeq \F_{\cat{Vect}_\IR}(\IR^n,\IR^n)\,.
	\end{equation*}
	But we easily get $\F_{\cat{Vect}_{\IR}}(\IR^n,\IR^n)_i=\Hom_{\cat{Top}}(|\Delta^i|,|\GL_n(\IR)|)$ from the definition above (where we write $|\GL_n(\IR)|$ to indicate that we really mean the topological space, not the associated anima), hence
	$\F_{\cat{Vect}_{\IR}}(\IR^n,\IR^n)=\Sing |\GL_n(\IR)|$, hence the unit map 
	\begin{equation*}
		\GL_n(\IR^n)\isomorphism \Sing|\GL_n(\IR)|\,.
	\end{equation*}
	provides the desired equivalence. This is also an equivalence of $\IE_1$-groups since both $\Sing$ and $|\blank|$ preserve finite products, hence the Segal condition. The complex case works completely analogous.
\end{proof*}
\begin{lem*}\label{lem*:BG=BG}
	If $G$ is a topological group with the homotopy type of a CW complex, then $BG$ \embrace{as constructed in \cref{par:Grp(An)=(*/An)Connected}} and $BG$ \embrace{the classifying space for principal $G$-bundles} coincide.
\end{lem*}
\begin{proof*}
	There are probably lots of ways to see this, but what convinced me was that $\Omega BG\simeq G$ also holds on the topological side of things (see \cite[Example~(14.4.7)]{TomDieck}), hence both $BG$s are mapped to $G$ under the equivalence $\Omega\colon (*/\An)_{\geq 1}\isomorphism \Grp(\An)$.
\end{proof*}
From Lemmas~\labelcref{lem*:VectBO}, \labelcref{lem*:BG=BG}, and the classification of principal $G$-bundles we find that
\begin{equation*}
	\pi_0\Hom_\An(X,\cat{\Vv ect}_\IR)=\pi_0\cat{Vect}_\IR(X)\quad\text{and}\quad \pi_0\Hom_\An(X,\cat{\Vv ect}_\IC)=\pi_0\cat{Vect}_\IC(X)
\end{equation*}
are the sets of isomorphism classes of real and complex vector bundles on $X$ whenever $X$ is a CW complex (it doesn't really make sense to talk about vector bundles over anima). Taking this a step further, we can generalise our constructions from above and produce monoidal anima
\begin{equation*}
	\cat{\Vv ect}_\IR(X)\,,\quad \cat{\Vv ect}_\IC(X)\,,\quad \cat{\Ee ucl}_\IR(X)\,,\quad\text{and}\quad\cat{\Ss ph}_\IR(X)\,,
\end{equation*}
of real vector bundles, complex vector bundles, $\IR^n$-fibre bundles, and spherical fibrations over $X$, respectively. These satisfy (either by construction and the argument above, or simply by definition; it's up to you to decide)
\begin{align*}
	\cat{\Vv ect}_\IR(X)&\simeq \Hom_\An(X,\cat{\Vv ect}_\IR)\,, & \cat{\Ee ucl}_\IR(X)\simeq \Hom_\An(X,\cat{\Ee ucl}_\IR)\,,\\
	\cat{\Vv ect}_\IC(X)&\simeq \Hom_\An(X,\cat{\Vv ect}_\IC)\,, & \cat{\Ss ph}_\IR(X)\simeq \Hom_\An(X,\cat{\Ss ph}_\IR)\,.
\end{align*}
Now observe that $\Hom_\An(X,-)\colon \An\morphism \An$ preserves products, hence Segal conditions. Thus it preserves $\IE_1$-monoids, and by an easy check also $\IE_1$-groups (since we only need that $\pi_0\Hom_\An(X,-)$ sends $\IE_1$-groups to ordinary groups). Hence \cref{def:koEtAl} and the universal property of group completion induces comparison maps
\begin{align*}
	\cat{\Vv ect}_\IR(X)^\inftygrp&\morphism \Hom_\An(X,k\cat{o})\,, &\cat{\Ee ucl}_\IR(X)^\inftygrp&\morphism \Hom_\An(X,k\cat{top})\,,\\
	\cat{\Vv ect}_\IC(X)^\inftygrp&\morphism \Hom_\An(X,k\cat{u})\,,&\cat{\Ss ph}_\IR(X)^\inftygrp&\morphism \Hom_\An(X,k\cat{sph})\,.
\end{align*}
Be aware that in general these aren't equivalences in general! This is only true when $X$ is a finite CW complex (or a retract of such), as we will see in \cref{cor:BOBU} once we have the group completion theorem available.

Now put $k\cat{o}^*(X)\coloneqq \pi_{*}\Hom_\An(X,k\cat{o})$ and define $k\cat{u}^*(X)$, $k\cat{top}^*(X)$, and $k\cat{sph}^*(X)$ similarly. These groups are pretty important invariants in geometric topology! For example, $k\cat{o}^*(X)$ and $k\cat{u}^*(X)$ are what's usually called \emph{real} and \emph{complex $K$-theory}. In the special case $X=*$ they are completely known, thanks to the famous \emph{Bott periodicity theorem} (which we'll see again in a refined version in \cref{thm:BottPeriodicity}):
\begin{equation*}
	\pi_n(k\cat{u})\simeq\begin{cases*}
		\IZ & if $n\equiv 0\mod 2$\\
		0 & if $n\equiv 1\mod 2$
	\end{cases*}\quad\text{and}\quad 
	\pi_n(k\cat{o})\simeq \begin{cases*}
		\IZ & if $n\equiv 0\mod 8$\\
		\IZ/2 & if $n\equiv 1\mod 8$\\
		\IZ/2 & if $n\equiv 2\mod 8$\\
		0 & if $n\equiv 3\mod 8$\\
		\IZ & if $n\equiv 4\mod 8$\\
		0 & if $n\equiv 5\mod 8$\\
		0 & if $n\equiv 6\mod 8$\\
		0 & if $n\equiv 7\mod 8$
	\end{cases*}\,.
\end{equation*}
The homotopy groups of $k\cat{sph}$ are also \enquote{known} in that $\pi_n(k\cat{sph})\simeq \pi_n^s$ is the $n\ordinalth$ stable homotopy group of spheres. Finally, we won't give a description of the homotopy groups of $k\cat{top}$, but we can at least their \enquote{differences} to those of $k\cat{o}$ and $k\cat{sph}$: For $n\neq 4$ we have that
\begin{equation*}
	 \pi_n\big(\fib(k\cat{o}\rightarrow k\cat{top})\big)\simeq \Theta^n
\end{equation*}
is the group of exotic spheres in dimension $n$, i.e.\ exotic smooth structures on $\IS^n$. Moreover,
\begin{equation*}
	\pi_n\big(\fib(k\cat{top}\rightarrow k\cat{sph})\big)\simeq L_n^\mathrm{q}(\IZ)\simeq \begin{cases*}
		\IZ & if $n\equiv 0\mod 4$\\
		0 & if $n\equiv 1\mod 4$\\
		\IZ/2 & if $n\equiv 2\mod 4$\\
		0 & if $n\equiv 3\mod 4$
	\end{cases*}\,.
\end{equation*}
yields Wall's quadratic $L$-groups of the integers.

\section{\texorpdfstring{$\IE_\infty$}{Einfty}-Monoids and \texorpdfstring{$\IE_\infty$}{Einfty}-Groups}
The \enquote{idea} how to impose commutativity is to replace $\IDelta^\op$ by some other category $\IGamma^\op$ which has \enquote{flips} in it. Lurie uses $\cat{Fin}_*^\op$ to denote this category, but Fabian decided to honour Segal's original notation.
\begin{defi}\label{def:CartesianCommutativeMonoids}
	We denote by $\IGamma^\op$ the $1$-category whose objects are finite sets and whose morphisms are partially defined maps. Write $\langle n\rangle=\{1,\dotsc,n\}$. There is a functor
	\begin{equation*}
		\Cut\colon\IDelta^\op\morphism\IGamma^\op
	\end{equation*}
	defined on objects via $\Cut([n])=\langle n\rangle$. On a morphism $\alpha\colon [m]\morphism {[n]}$ in $\IDelta$, which then corresponds to a morphism $\alpha^\op$ in $\IDelta^\op$ pointing in the other direction, $\Cut(\alpha^\op)\colon \langle n\rangle\morphism\langle m\rangle $ is defined via
	\begin{equation*}
		\Cut(\alpha^\op)(i)=\begin{cases*}
			\text{undefined} & if $i\leq\alpha(0)$\\
			j & if $\alpha(j-1)<i\leq \alpha(j)$\\
			\text{undefined} & if $\alpha(m)<i$
		\end{cases*}\,.
	\end{equation*}
	In more invariant words, $\Cut$ sends a finite non-empty totally ordered set $I\in \IDelta^\op$ to its set of \emph{Dedekind cuts} (i.e.\ partitions into two non-empty intervals), and a map $\alpha\colon I\morphism J$ in $\IDelta$ is sent to $\Cut(\alpha^\op)=\alpha^*\colon \Cut(J)\morphism \Cut(I)$ which is given by taking preimages (whenever these are non-empty, otherwise it's undefined).
	
	If now $\Cc$ is an $\infty$-category with finite products, then a \emph{cartesian commutative monoid} in $\Cc$ is a functor $X\colon \IGamma^\op\morphism \Cc$ such that $X\circ \Cut\colon \IDelta^\op\morphism \Cc$ is a cartesian monoid in the sense of \cref{def:CartesianMonoids}. If it is even a cartesian group, then $X$ is called a \emph{cartesian commutative group}  in $\Cc$. We let $\CMon(\Cc),\CGrp(\Cc)\subseteq \Fun(\IGamma^\op,\Cc)$ denote the full sub-$\infty$-categories spanned by cartesian commutative monoids/groups.
\end{defi}
\refstepcounter{smallerdummy}
\numpar*{\thesmallerdummy}Time to unwind! The functor $\Cut\colon \IDelta^\op\morphism\IGamma^\op$ takes the Segal maps $e_i\colon [1]\morphism {[n]}$ in $\IDelta$ to the maps $\rho_i\colon \langle n\rangle \morphism\langle 1\rangle$ defined by
\begin{equation*}
	\rho_i(j)=\begin{cases*}
		1 & if $i=j$\\
		\text{undefined} & else
	\end{cases*}\,.
\end{equation*}
Moreover, the face maps $d_0,d_1,d_2\colon [1]\morphism{[2]}$ in $\IDelta$ are sent to $l,m,r\colon \langle 2\rangle \morphism\langle 1\rangle$, the unique maps defined on $\{1\}$, $\{1,2\}$, and $\{2\}$ respectively. So composition in a cartesian commutative monoid becomes
\begin{equation*}
	\begin{tikzcd}[column sep=2em]
		\mu\colon X_1\times X_1& X_2\morphism[m]X_1\lar["\sim", "{(l,r)}"']
	\end{tikzcd}\,.
\end{equation*}
But consider the flip map $\operatorname{flip}\colon \langle 2\rangle \morphism\langle 2\rangle$ that swaps $1$ and $2$. Then $m\circ \operatorname{flip}=m$, $r\circ\operatorname{flip}=l$, and $l\circ \operatorname{flip}=r$ hold in $\IGamma^\op$, so the diagram
\begin{equation*}
	\begin{tikzcd}[row sep=small]
		X_1\times X_1\ar[dd,"\text{flip factors}"']\drar[bend left=20,"\mu"] & \\
		\phantom{X}\rar[phantom, pos=0.4, "\scriptscriptstyle/\!/\!/"] & X_1\\
		X_1\times X_1\urar[bend right=20, "\mu"'] &
	\end{tikzcd}
\end{equation*}
commutes. So it really makes sense to think of $X$ as commutative!
\begin{exc}
	If $\Cc$ is a $1$-category, show that $\Cut^*\colon \Fun(\IGamma^\op,\Cc)\morphism\Fun(\IDelta^\op,\Cc)$ restricts to a fully faithful functor $\CMon(\Cc)\morphism\Mon(\Cc)$. Note that this is false for general $\infty$-categories, even for $\Cc=\Cat_1^{(2)}$!
\end{exc}\refstepcounter{smallerdummy}
\numpar*{\thesmallerdummy}\label{par:EinftyRefinement}
Similar to \cref{exm:MyFirstMonoidals}\itememph{e}, any symmetric monoidal $1$-category $(\Cc,\otimes)$ gives rise to an object in $\CMon(\Cat_1^{(2)})\subseteq \CMon(\Cat_\infty)$. Similar to the monoidal case, this is constructed from a cocartesian fibration $p^\otimes\colon \Cc^\otimes\morphism \IGamma^\op$.

Let's briefly sketch the construction. Objects of $\Cc^\otimes$ are pairs $(n,x_1,\dotsc,x_n)$, where $\langle n\rangle \in \IGamma^\op$ and $x_1,\dotsc,x_n\in \Cc$. Morphisms $(n,x)\morphism (m,y)$ comprise the data of a map $\alpha\colon \langle n\rangle \morphism\langle m\rangle $ in $\IGamma^\op$ together with maps $f_j\colon \bigotimes_{i\in\alpha^{-1}(j)}x_i\morphism y_j$ for all $j=1,\dotsc,m$. Composition is defined in the \enquote{obvious} way, using the braiding $\sigma$, and to see that composition is associative one has to use $\sigma^2=\id$. If you are confused by this braiding stuff: In a nutshell, a \emph{braiding} of a monoidal category is a choice of isomorphisms $\sigma_{x,y}\colon x\otimes y\isomorphism y\otimes x$ subject to some naturality conditions. In this terminology, a symmetric monoidal category is a braided monoidal category for which $\sigma_{x,y}\circ \sigma_{y,x}=\id$.

One can check that the cocartesian fibration $p_\mathrm{mon}^\otimes\colon\Cc_\mathrm{mon}^\otimes\morphism\IDelta^\op$ from \cref{exm:MyFirstMonoidals}\itememph{e} fits into a pullback diagram
\begin{equation*}
	\begin{tikzcd}
		\Cc_\mathrm{mon}^\otimes\dar["p_\mathrm{mon}^\otimes"']\rar\drar[pullback] & \Cc^\otimes\dar["p^\otimes"]\\
		\IDelta^\op\rar["\Cut"] & \IGamma^\op
	\end{tikzcd}
\end{equation*}
It turns out that the diagram
\begin{equation*}
	\begin{tikzcd}
			\CMon\big(\Cat_1^{(2)}\big)\rar["\Cut^*"]\dar[iso] & \Mon\big(\Cat_1^{(2)}\big)\dar[iso]\\
		\cat{SymMonCat}_1^{(2)}\rar["\text{forget}"]& \cat{MonCat}_1^{(2)}
	\end{tikzcd}
\end{equation*}
commutes (we never defined the bottom categories, but if you sit down for a while you will surely come up with suitable a definition of the $2$-category of symmetric/arbitrary monoidal $1$-categories). However, a monoidal category can have many symmetries, so the horizontal functors are not at all fully faithful! For example, let $(\Cc,\otimes)$ be the category of $\IZ$-graded $R$-modules equipped with the graded tensor product. Any unit $u\in R^\times$ defines a braiding $\tau_u$ on $\Cc$ via $\tau_u\colon X\otimes Y\isomorphism Y\otimes X$ given by $\tau_u(x\otimes y)=u^{|x||y|}y\otimes x$ on elementary tensors. If $u^2=1$, then this braiding defines a symmetric monoidal structure. In particular, there can be many of these! So commutativity is a \emph{structure}, not a \emph{property}.
\refstepcounter{smallerdummy}
\numpar*{\thesmallerdummy. Semi-Additive and Additive $\infty$-Categories}
\lecture[Semi-additive and additive $\infty$-categories, monoids and groups in these.]{2020-12-08} An $\infty$-category is called \emph{semi-additive} if it admits finite products and finite coproducts, its initial and terminal object agree, and for any $x,y\in \Cc$ the natural map
\begin{equation*}
	\begin{pmatrix}
		\id_x & 0\\
		0 & \id_y
	\end{pmatrix}\colon x\sqcup y\isomorphism x\times y
\end{equation*}
is an equivalence. Here $0\colon x\morphism 0\morphism y$ denotes the unique (up to contractible choice) map in $\Hom_\Cc(x,y)$ factoring through an initial/terminal object which we also denote $0\in \Cc$. If $\Cc$ is semi-additive, then we'll usually denote its unified product and coproduct by $\oplus$.

We call $\Cc$ \emph{additive} if also the shear map
\begin{equation*}
	\begin{pmatrix}
		\id_x & \id_x\\
		0 & \id_x
	\end{pmatrix}\colon x\oplus x\isomorphism x\oplus x
\end{equation*}
is an equivalence for all $x\in \Cc$. The typical source of examples is everything that has to do with abelian groups, so for example $\Kk(\IZ)$ and the derived category $\Dd(\IZ)$ are additive.
\begin{prop}\label{prop:CMonOfSemiAdditive}
	If $\Cc$ is a semi-additive $\infty$-category, then the forgetful maps
	\begin{equation*}
		\CMon(\Cc)\isomorphism\Mon(\Cc)\isomorphism \Cc
	\end{equation*}
	are equivalences. If $\Cc$ is additive, then the same holds true for
	\begin{equation*}
		\CGrp(\Cc)\isomorphism\Grp(\Cc)\isomorphism \Cc\,.
	\end{equation*}
\end{prop}
Before we get into the proof, let's show a consequence.
\begin{cor}\label{cor:HomSemiAdditveFactorsThroughCGrpAn}
	If $\Cc$ is an additive $\infty$-category, then there is a canonical lift
	\begin{equation*}
		\Hom_\Cc\colon \Cc^\op\times \Cc\morphism \CGrp(\An)\,.
	\end{equation*}
	If $\Cc$ is only semi-additive, one still gets a lift to $\CMon(\An)$.
\end{cor}
\begin{proof*}
	In general, if $F\colon \Cc\morphism\Dd$ is a finite product-preserving functor, then the induced functor $F_*\colon \Fun(\IGamma^\op,\Cc)\morphism \Fun(\IGamma^\op,\Dd)$ preserves the Segal condition and also the condition from \cref{def:E1Group}, hence it restricts to functors $F_*\colon \CMon(\Cc)\morphism\CMon(\Dd)$ and $F_*\colon \CGrp(\Cc)\morphism\CGrp(\Dd)$.
	
	We can't apply this directly to our situation, since $\Hom_\Cc\colon \Cc^\op\times\Cc\morphism\An$ doesn't preserve finite products---or rather it is too good at respecting products, as
	\begin{equation*}
		\Hom_\Cc\left(\bigoplus_{i=1}^nx_i,\bigoplus_{i=1}^ny_i\right)\simeq \prod_{i,j=1}^n\Hom_\Cc(x_i,y_j)\,,
	\end{equation*}
	of which the diagonal $\prod_{i=1}^n\Hom_\Cc(x_i,y_i)$ is only a factor.
	
	At least $\Hom_\Cc(x,-)\colon \Cc\morphism\An$ preserves products for all $x\in \Cc$, whence it induces a functor $\Hom_\Cc(x,-)_*\colon \CGrp(\Cc)\morphism\CGrp(\An)$. But if $\Cc$ is additive, then $\CGrp(\Cc)\simeq \Cc$ by \cref{prop:CMonOfSemiAdditive}, which provides the desired lift
	\begin{equation*}
		\Hom_\Cc(x,-)\colon \Cc\morphism\CGrp(\An)\,.
	\end{equation*}
	To paste these lifts together into a functor $\Hom_\Cc\colon\Cc^\op\times \Cc\morphism \CGrp(\An)$, consider the composite
	\begin{equation*}
		\Fun(\Cc,\An)\morphism\Fun\big(\Fun(\IGamma^\op,\Cc),\Fun(\IGamma^\op,\An)\big)\morphism \Fun\big(\CGrp(\Cc),\Fun(\IGamma^\op,\An)\big)\,.
	\end{equation*}
	If we restrict to the full subcategory $\Fun^\times(\Cc,\An)\subseteq \Fun(\Cc,\An)$ of finite product-preserving functors, it lands in $\Fun(\CGrp(\Cc),\CGrp(\An))$, as argued above. Moreover, if $\Cc$ is additive, then $\CGrp(\Cc)\simeq \Cc$ by \cref{prop:CMonOfSemiAdditive}. Now $\Hom_\Cc\colon \Cc^\op\morphism\Fun(\Cc,\An)$ has image in $\Fun^\times(\Cc,\An)$, hence we may compose it with the above to obtain the desired functor
	\begin{equation*}
		\Hom_\Cc\colon \Cc^\op\morphism\Fun\big(\Cc,\CGrp(\An)\big)\,.
	\end{equation*}
	
	If $\Cc$ is only semi-additive, then the argument still works with $\CMon(-)$ instead of $\CGrp(-)$ everywhere. Moreover, one can check (but we won't do that here) that if we we use the (semi-)additive structure on $\Cc^\op$ instead to do the constructions above, we still end up with the same functor $\Hom_\Cc\colon \Cc^\op\times \Cc\morphism\CGrp(\An)$.
\end{proof*}
\begin{proof}[Proof sketch of \cref{prop:CMonOfSemiAdditive}]
	The key to the proof is the following claim:
	\begin{alphanumerate}
		\item[\itememph{\boxtimes}] \itshape Let $\Ii\in\{\IDelta^\op,\IGamma^\op\}$. Given a functor $X\colon \Ii\morphism\Cc$ satisfying $X_0\simeq 0$, then $X$ satisfies the Segal condition iff it is left-Kan extended from $X_{|\Ii_{\leq 1}}$ along $p\colon \Ii_{\leq 1}\morphism\Ii$.
	\end{alphanumerate}
	To conclude the statement of the proposition from \itememph{\boxtimes}, write $\Fun(\Ii_{\leq 1},\Cc)_0\subseteq \Fun(\Ii_{\leq 1},\Cc)$ for the full subcategory of functors $F$ satisfying $F(0)\simeq 0$. Since $0\in \Cc$ is initial and terminal, one easily checks that $\Cc\isomorphism\Fun(\Ii_{\leq 1},\Cc)_0$, sending $x\in \Cc$ to the unique (up to contractible choice) functor $F_x$ satisfying $F_x(0)\simeq 0$ and $F_x(1)\simeq x$, is an equivalence. Moreover,
	\begin{equation*}
		\Fun(\Ii_{\leq 1},\Cc)_0\subseteq \Fun(\Ii_{\leq 1},\Cc)\xrightarrow{\Lan_p}\Fun(\Ii,\Cc)
	\end{equation*}
	is fully faithful by \cref{cor:FullyFaithfulKanExtension}, and \itememph{\boxtimes} precisely says that its essential image is $\Mon(\Cc)$ for $\Ii=\IDelta^\op$ or $\CMon(\Cc)$ for $\Ii=\IGamma^\op$. This shows $\Mon(\Cc)\simeq \Cc\simeq \CMon(\Cc)$, as required. If $\Cc$ is additive, then the image of $x\in \Cc$ under the above composite is automatically a cartesian group: Indeed, after an easy unravelling, the condition from \cref{def:E1Group} translates into the condition that the shear map on $x$ be an equivalence, which is guaranteed by $\Cc$ being additive.
	
	To prove \itememph{\boxtimes}, first note that we get a natural counit transformation $\Lan_p X_{|\Ii_{\leq 1}}\Rightarrow X$ for free, so whether this is an equivalence can be checked on objects. Recall from \cref{thm:KanExtension} that
	\begin{equation*}
		\Lan_p X_{|\Ii_{\leq 1}}(n)\simeq \colimit_{i\in \Ii_{\leq 1}/n}X_i
	\end{equation*}
	(it will follow from the discussion below that these colimits exist in $\Cc$, so the theorem is indeed applicable). Therefore we have to understand the slice categories
	\begin{equation*}
		\IGamma_{\leq 1}^\op/\langle n\rangle\quad\text{and}\quad\IDelta_{\leq 1}^\op/[n]\simeq ([n]/\IDelta_{\leq 1})^\op\,.
	\end{equation*}
	
	
	\emph{The slice category $[n]/\IDelta_{\leq 1}$.} Let's describe $[n]/\IDelta_{\leq 1}$ first in words and then in a picture (whichever you start with will probably confuse you, and then the other thing will hopefully unconfuse you). Its vertices are given by maps $[n]\morphism {[1]}$ and $[n]\morphism {[0]}$ in $\IDelta$. There's only one of the latter kind and it is denoted $0$ (in the picture it's depicted as a black dot on the right). There are $n+2$ maps $[n]\morphism {[1]}$: One that is constantly $0$, one that is constantly $1$ (depicted on the top and on the bottom), and $n$ surjective maps (depicted on the left). Moreover, there's a bunch of non-degenerate edges. The pink ones are those induced by $0\colon [0]\morphism {[1]}$ and $0\colon [1]\morphism {[1]}$, the purple ones are those induced by $1\colon [0]\morphism{[1]}$ and $1\colon [1]\morphism{[1]}$, and finally the black ones are induced by the unique morphism $[1]\morphism {[0]}$.
	\begin{center}
	\begin{tikzpicture}[x=0.6cm,y=0.6cm, line cap=round]
		\fill (0,0) circle (0.5ex) coordinate (A);
		\fill (0,-1) circle (0.5ex) coordinate (B);
		\fill (0,-3) circle (0.5ex) coordinate (C);
		\fill (3,-4) circle (0.5ex) coordinate (D);
		\fill (3,1) circle (0.5ex) coordinate (E);
		\node[rotate=-90] at (0,-2) {$\dotso$};
		\fill (6,-1.5) circle (0.5ex) coordinate (F);
		\node[left=1ex]  at (A) {$0\dotso01$};
		\node[left=1ex] (001) at (B) {$0\dotso11$};
		\node[left=1ex] (011) at (C) {$01\dotso1$};
		\node[below=0.5ex] (111) at (D) {$1\dotso1$};
		\node[above=0.5ex] (000) at (E) {$0\dotso0$};
		\node[right=1ex] at (F) {$0$};
		\path (001.south) -- (011.north) node[pos=0.5,sloped] {$\dotso$};
		\draw[dotted, rounded corners=2] (1ex,1ex) to[out=0,in=150] (3.6cm+1ex,-0.9cm+1ex) to ++ (0,-2ex) to[out=210,in=0] (1ex,-1.8cm-1ex) to (-1ex,-1.8cm-1ex) to (-1ex,1ex) to (1ex,1ex);
		%\draw[dotted, rounded corners=2, shift={(F)}] (1ex,1ex) -- (1ex,-1ex) -- (-1ex,-1ex) -- (-1ex,1ex) -- cycle;
		% pink
		\draw[FabiansPink,-to,shorten <=1.25ex,shorten >=1.25ex,bend left=20] (A) to (E);
		\draw[FabiansPink,-to,shorten <=1.25ex,shorten >=1.25ex,bend left=10] (B) to (E);
		\draw[FabiansPink,-to,shorten <=1.25ex,shorten >=1.25ex,bend left=5] (C) to (E);
		\draw[FabiansPink,-to,shorten <=1.25ex,shorten >=1.25ex,bend left=10] (D) to (E);
		\draw[FabiansPink,-to,shorten <=1.25ex,shorten >=1.25ex,bend right=45] (F) to (E);
		\draw[FabiansPink,-to,looseness=4] (000.115) to[out=115,in=65] (000.65);
		%purple
		\draw[FabiansPurple!67!FabiansPink,-to,shorten <=1.25ex,shorten >=1.25ex,bend left=5] (A) to (D);
		\draw[FabiansPurple!67!FabiansPink,-to,shorten <=1.25ex,shorten >=1.25ex,bend right=15] (B) to (D);
		\draw[FabiansPurple!67!FabiansPink,-to,shorten <=1.25ex,shorten >=1.25ex,bend right=20] (C) to (D);
		\draw[FabiansPurple!67!FabiansPink,-to,shorten <=1.25ex,shorten >=1.25ex,bend left=15] (E) to (D);
		\draw[FabiansPurple!67!FabiansPink,-to,shorten <=1.25ex,shorten >=1.25ex,bend left=45] (F) to (D);
		\draw[FabiansPurple!67!FabiansPink,-to,looseness=4] (111.-115) to[out=-115,in=-65] (111.-65);
		% black
		\draw[-to,shorten <=1.25ex,shorten >=1.25ex,bend left=15] (A) to (F);
		\draw[-to,shorten <=1.25ex,shorten >=1.25ex,bend right=5] (B) to (F);
		\draw[-to,shorten <=1.25ex,shorten >=1.25ex,bend right=15] (C) to (F);
		\draw[-to,shorten <=1.25ex,shorten >=1.25ex,bend right=20] (D) to (F);
		\draw[-to,shorten <=1.25ex,shorten >=1.25ex,bend left=20] (E) to (F);
	\end{tikzpicture}
	\end{center}
	However, the punchline is that all of this doesn't really matter, since the dotted part, consisting of the vertices on the left and $0$ on the right, turns out to be final (in the lecture we claimed that already the vertices on the left are final, but I think that's not true). To prove this, verify the criterion from \cref{thm:JoyalQuillenThmA}\itememph{b} and use that the slice categories in question (in fact, slice categories between slice categories) contain the respective arrow starting at $0$ as a terminal object. After passing to $\IDelta_{\leq 1}^\op/[n]$ instead of $[n]/\IDelta_{\leq 1}$, the dotted part becomes cofinal, hence the colimit in question can be taken over the dotted part only. But the $0$-vertex on the right is mapped to $X_0\simeq 0$ and there are no edges between the vertices on the left, hence it's easy to see that the colimit over the dotted part is just $X_1^{\oplus n}$. This is precisely what we wanted to show!
	
	\emph{The slice category $\IGamma^\op_{\leq 1}/\langle n\rangle$.} The vertices of $\IGamma^\op_{\leq 1}/\langle n\rangle$ consist of partially defined maps $\langle 0\rangle \morphism\langle n\rangle$ and $\langle 1\rangle \morphism\langle n\rangle$. As $\langle 0\rangle=\emptyset$, there is only one such map of the first kind, depicted by the vertex on the right in the picture below. The vertex in the middle represents the unique nowhere-defined map $\langle 1\rangle\morphism \langle n\rangle$, and the vertices on the right correspond to the maps $\langle 1\rangle\morphism \langle n\rangle$ which are actually defined on $1$.
	\begin{center}
		
		\begin{tikzpicture}[x=0.6cm,y=0.6cm, line cap=round]
			\fill (0,0) circle (0.5ex) coordinate (A);
			\fill (0,-1) circle (0.5ex) coordinate (B);
			\fill (0,-3) circle (0.5ex) coordinate (C);
			\fill (3,-1.5) circle (0.5ex) coordinate (D);
			\node[rotate=-90] at (0,-2.065) {$\dotso$};
			\fill (6,-1.5) circle (0.5ex) coordinate (F);
			\node[left=1ex]  at (A) {$1$};
			\node[left=1ex] (001) at (B) {$2$};
			\node[left=1ex] (011) at (C) {$n$};
			\node[right=0.5ex] (111) at (D) {$\emptyset$};
			\node[right=1ex] at (F) {$\emptyset$};
			\path (001.north) -- (011.south) node[pos=0.5,sloped] {$\dotso$};
			\draw[dotted, rounded corners=2] (-1ex,-1.8cm-1ex) to (-1ex,1ex) to[bend left] (3.6cm+1ex,-0.9cm+1ex) to ++ (0,-2ex) to[bend left]  cycle;
			\draw[dotted, rounded corners=2] (1.5,-1.65) to[out=90,in=90] (5,-1.65) to[out=-90,in=-90] cycle;
			% pink
			\draw[FabiansPink,-to,looseness=4] (111.-115) to[out=-115,in=-65] (111.-65);
			\draw[FabiansPink,-to,shorten <=1.25ex,shorten >=1.25ex] (D) -- (A);
			\draw[FabiansPink,-to,shorten <=1.25ex,shorten >=1.25ex] (D) -- (B);
			\draw[FabiansPink,-to,shorten <=1.25ex,shorten >=1.25ex] (D) -- (C);
			\draw[FabiansPink,-to,shorten <=1.25ex,shorten >=1.25ex,bend right=20] (111.east) to (F);
			\draw[FabiansPink,-to,shorten <=1.25ex,shorten >=1.25ex,bend right=20] (F) to (111.east);
			%purple
			% black
			\draw[-to,shorten <=1.25ex,shorten >=1.25ex,bend right] (F) to (A);
			\draw[-to,shorten <=1.25ex,shorten >=1.25ex,bend right] (F) to (B);
			\draw[-to,shorten <=1.25ex,shorten >=1.25ex,bend left] (F) to (C);
		\end{tikzpicture}
	\end{center}
	Again, the dotted part (i.e.\ everything except the vertex in the middle) is cofinal and the colimit over it evaluates to $X_1^{\oplus n}$ since the vertex on the right is sent to $X_0\simeq 0$.
\end{proof}


Since $\Dd_{\geq 0}(\IZ)$ is additive, a simple consequence of \cref{prop:CMonOfSemiAdditive} is that the Eilenberg--MacLane functor from Very Long Example~\cref{exm:EilenberMacLane} lifts to a functor
\begin{equation*}
	K\colon \Dd_{\geq 0}(\IZ)\morphism\CGrp(\An)\,.
\end{equation*}
Indeed, $K$ is a right-adjoint by construction, hence it preserves products, so we can use the argument from the proof of \cref{cor:HomSemiAdditveFactorsThroughCGrpAn} to produce the desired lift. We will soon (in \cref{exm:MyFirstSpectra}\itememph{c}) extend this even further!

So Eilenberg--MacLane anima give rise to $\IE_\infty$-groups. Another source of examples for $\IE_\infty$-monoids/groups is
\begin{equation*}
	\CMon(\Kan)\morphism \CMon(\An)\quad\text{and}\quad \CGrp(\Kan)\morphism \CGrp(\An)\,,
\end{equation*}
(again note that $\Kan\morphism \An$ preserves products), but these are essentially the same examples: It is a theorem of Dold--Thom, that a connected commutative monoid in Kan complexes has underlying anima a \emph{gem} (\enquote{generalized Eilenberg--MacLane space}; we defined these things in Very Long Example~\cref{exm:EilenberMacLane}). The same then holds for not necessarily connected commutative groups in Kan complexes, since all connected components are equivalent.
\begin{prop}\label{prop:RealisationCommutesWithFiniteProducts}
	The realisation functor $|\blank|\colon \cat{sAn}\morphism\An$ preserves finite products.
\end{prop}
\begin{proof}
	We compute
	\begin{align*}
		|M\times N|\simeq\colimit_{[n]\in\IDelta^\op}\left(M_n\times N_n\right)&\simeq \colimit_{([n],[m])\in \IDelta^\op\times \IDelta^\op}\left(M_n\times N_m\right)\\
		&\simeq \colimit_{[n]\in\IDelta^\op}\left(M_n\times \colimit_{[m]\in\IDelta^\op}N_m\right)\\
		&\simeq \left(\colimit_{[n]\in\IDelta^\op}M_n\right)\times \left(\colimit_{[m]\in\IDelta^\op}N_m\right)\\
		&\simeq |M|\times |N|\,.
	\end{align*}
	For the second equivalence see \cref{exc:IDeltaOpSifted} below. The third equivalence follows from \cref{prop:ColimitsCommute} together with the fact that $M_n\times -$ commutes with arbitrary colimits. Indeed, for any $X\in\An$ the functor $X\times-\colon \An\morphism\An$ is a left adjoint of $\Hom_\An(X,-)$. Applying the same argument to $-\times \colimit_{\IDelta^\op}N_m$ gives the fourth equivalence.
\end{proof}
\begin{exc}\label{exc:IDeltaOpSifted}
	Show that the diagonal $\Delta\colon \IDelta^\op\morphism\IDelta^\op\times \IDelta^\op$ is cofinal as a map of $\infty$-categories. In other words, $\IDelta^\op$ is \emph{sifted}.
\end{exc}
\begin{proof*}
	By \cref{thm:JoyalQuillenThmA}\itememph{b} we need to check that $\IDelta^\op\times_{\IDelta^\op\times \IDelta^\op}([m],[n])/(\IDelta^\op\times \IDelta^\op)$ is weakly contractible for all $[m],[n]\in \IDelta^\op$. Note that we may form the slice category as a $1$-category, since the nerve functor commutes with limits. Also note that we may replace the slice category by its opposite $\Ss_{m,n}\coloneqq\IDelta\times_{\IDelta\times \IDelta}(\IDelta\times \IDelta)/([m],[n])$ for convenience.
	
	Unravelling definitions, we find that objects in $\Ss_{m,n}$ can be thought of as maps of posets $\alpha\colon [i]\morphism {}[m]\times [n]$. Any such map factors uniquely into a composition $[i]\epimorphism{} [j]\monomorphism{} [m]\times [n]$ of a surjective and an injective map. This defines a functor $r\colon \Ss_{m,n}\morphism\Cc_{m,n}$ into the category $\Cc_{m,n}$ of non-empty linearly ordered subposets of $[m]\times [n]$ ($\Cc_{m,n}$ is partially ordered by inclusion). But conversely, any such subposet is given by an injective map $[j]\monomorphism {}[m]\times [n]$ for some $j\leq m+n+1$, hence we get an inclusion $s\colon \Cc_{m,n}\morphism \Ss_{m,n}$ in the reverse direction. Clearly $r\circ s=\id_{\Cc_{m,n}}$, whereas the surjections $[i]\epimorphism{} [j]$ define a natural transformation $\id_{\Ss_{m,n}}\Rightarrow s\circ r$.
	
	This shows that $\Ss_{m,n}$ and $\Cc_{m,n}$ are weakly (and even strongly) homotopy equivalent after taking nerves. Now if we recall \cite[Definition~V.4.7]{HigherCatsI}, we recognize $\Cc_{m,n}$ as the barycentric subdivision $\operatorname{sd}(\Delta^m\times \Delta^n)$, which is weakly contractible because $\Delta^m\times \Delta^n$ is.
\end{proof*}
\refstepcounter{smallerdummy}
\numpar*{\thesmallerdummy}
	Recall from \cref{cor*:BOmegaAdjunction} that we have an adjunction
	\begin{equation*}
		B\colon \Mon(\An)\doublelrmorphism {*/\An}\colon \Omega\,.
	\end{equation*}
	Now $B=|\blank|$ and $\Omega$ both preserve finite products---the former by \cref{prop:RealisationCommutesWithFiniteProducts}, the latter since it is a right-adjoint. So both functors preserve the Segal condition. Therefore, if we apply $\Fun(\IGamma^\op,-)$ to both sides of the adjunction above, then the resulting adjunction restricts to another adjunction
\begin{equation}\label{eq:BOmegaAdjunction}
	B\colon \CMon\big(\Mon(\An)\big)\doublelrmorphism \CMon(*/\An)\noloc \Omega
\end{equation}
Now it is easy to see that the forgetful map $\CMon(*/\Cc)\morphism\CMon(\Cc)$ is an equivalence for any $\infty$-category $\Cc$ with finite products (and thus a terminal object $*\in \Cc$). We claim that also%\setcounter{smallerdummy}{0}
\begin{equation}\label{eq:EckmannHilton}
	\begin{tikzcd}[row sep=small]
		\CMon\big(\Mon(\Cc)\big)\ar[dd,iso]\drar[bend left=20,"\mathrm{forget}_*","\sim"{swap,sloped}]& \\
		\phantom{X}\rar[phantom, pos=0.4, "\scriptscriptstyle/\!/\!/"]&\CMon(\Cc)\\
		\Mon\big(\CMon(\Cc)\big)\urar[bend right=20,iso,"\mathrm{forget}"'] &
	\end{tikzcd}
\end{equation}
commutes (the forgetful functors are given by evaluation in degree $1$ of course). Once we have this, we will obtain the following statement (which comes only later in Fabian's notes, hence the gap in numbering).
\addtocounter{dummy}{2}
\begin{cor}\label{cor:CommutativeHorizontalAdjoints}
	There is a commutative diagram of horizontal adjunctionts
	\begin{equation*}
		\begin{tikzcd}
			\CMon(\An)\rar[shift left=0.45ex, "B"]\dar["\Cut^*"'] & \CMon(\An)\lar[shift left=0.45ex,"\Omega"]\dar["\ev_1"]\\
			\Mon(\An)\rar[shift left=0.45ex, "B"] & */\An\lar[shift left=0.45ex,"\Omega"]
		\end{tikzcd}
	\end{equation*}
	Moreover, we have the following analogues of \cref{par:Grp(An)=(*/An)Connected} and \cref{prop:InftyGrp}:
	\begin{alphanumerate}
		\item Restricting $B$ to the full subcategory $\CGrp(\An)\subseteq\CMon(\An)$ gives a fully faithful functor $B\colon \CGrp(\An)\morphism\CMon(\An)$.
		\item Both functors $B,\Omega\colon \CMon(\An)\morphism\CMon(\An)$ actually take values in $\CGrp(\An)$.
		\item $\Omega B\colon \CMon(\An)\morphism\CGrp(\An)$ is left-adjoint to $\CGrp(\An)\subseteq \CMon(\An)$.
	\end{alphanumerate}
\end{cor}
\begin{proof*}
	We construct the diagram and prove commutativity first. The bottom row is already known. After identifying $\CMon(*/\An)\simeq \CMon(\An)$ and $\CMon(\An)\simeq \CMon(\Mon(\An))$, the upper row becomes \cref{eq:BOmegaAdjunction}. To see that the diagram commutes, we first check that the these identifications transform the forgetful functor $\CMon(\Mon(\An))\morphism\Mon(\An)$ into $\Cut^*\colon \CMon(\An)\morphism \Mon(\An)$. Indeed, this follows easily from \cref{eq:EckmannHilton}. Now both vertical arrows are of the form $\mathrm{forget}\colon \CMon(-)\morphism (-)$, from which commutativity of the diagram is clear.
	
	Next we check \itememph{b}. Let $M\in \CMon(\An)$. To check $\Omega M\in\CGrp(\An)$, it suffices to check $\Cut^*\Omega M\in \Grp(\An)$ (that's just how $\IE_\infty$-groups are defined). But ${\Cut^*}\circ \Omega\simeq \Omega\circ{\ev_1}$ and the right-hand side takes values in $\Grp(\An)$ by \cref{par:Grp(An)=(*/An)Connected}. To show $BM\in\CGrp(\An)$, it similarly suffices to check that the ordinary monoid $\pi_0(BM)_1$ is an ordinary group, as argued in the proof of \cref{prop:Grp(An)=(*/An)Connected}. But in fact $\pi_0(BM)_1\simeq *$, since $B\colon \Mon(\An)\morphism {*/\An}$ takes values in $(*/\An)_{\geq 1}$ and the diagram above commutes. This shows \itememph{b}.
	
	In particular, the top row adjunction restricts to an adjunction
	\begin{equation*}
		B\colon \CGrp(\An)\shortdoublelrmorphism \CGrp(\An)\noloc \Omega\,.
	\end{equation*}
	To show \itememph{a}, it suffices therefore to check that the unit $\id\Rightarrow \Omega B$ is an equivalence. Equivalences of $\IE_\infty$-groups can be detected on underlying $\IE_1$-groups (in fact, even on underlying anima, i.e.\ after applying $\ev_1$, since the Segal condition will do the rest). So it suffices that $\Cut^*\Rightarrow \Cut^*\Omega B$ is an equivalence. But $\Cut^*\Omega B\simeq \Omega\ev_1 B\simeq \Omega B\Cut^*$, and if $G$ is an $\IE_\infty$-group, then
	\begin{equation*}
		\Omega B\Cut^*G\simeq (\Cut^*G)^\inftygrp\simeq \Cut^*G
	\end{equation*}
	by \cref{prop:InftyGrp}. This finishes the proof of \itememph{a}.
	
	Let $\Cc\subseteq \CGrp(\An)$ denote the essential image of $B_{|\CGrp(\An)}$. Then $B\colon \CGrp(\An)\isomorphism\Cc$ is an equivalence, hence its right adjoint $\Omega\colon \Cc\morphism\CGrp(\An)$ is an equivalence too. For $M\in \CMon(\An)$ and $G\in \CGrp(\An)$ we may thus compute
	\begin{align*}
		\Hom_{\CGrp(\An)}(\Omega BM,G)&\simeq \Hom_{\CGrp(\An)}(\Omega BM,\Omega BG)\\
		&\simeq \Hom_\Cc(BM,BG)\\
		&\simeq \Hom_{\CMon(\An)}(M,\Omega BG)\\
		&\simeq \Hom_{\CMon(\An)}(M,G)\,,
	\end{align*}
	where we also used $G\simeq \Omega BG$ twice. This shows \itememph{c}.
\end{proof*}
\begin{smallexm}\label{exm:CommutativeHorizontalAdjoints}
	Some consequences:
	\begin{alphanumerate}
		\item \cref{cor:CommutativeHorizontalAdjoints}\itememph{c} shows that group completion in $\Mon(\An)$ and $\CMon(\An)$ is compatible. Hence all examples from \labelcref{par:AlgebraicKTheory,par:HermitianKTheory,par:TopologicalKTheory}, i.e.
		\begin{equation*}
			k(R)\,,\quad \gw^r(R)\,,\quad k\mathrm{o}\,,\quad k\mathrm{u}\,,\quad k\mathrm{top}\,,\quad \text{and}\quad k\mathrm{sph}\,,
		\end{equation*}
		are canonically $\IE_\infty$-groups. Indeed, they arise as group completions of certain $\IE_1$-monoids, all of which have canonical $\IE_\infty$-monoid refinements by \labelcref{par:EinftyRefinement}.
		\item It follows from \cref{cor:CommutativeHorizontalAdjoints}\itememph{a} that $\Omega BX\simeq X$ for all $X\in\CGrp(\An)$. Since it's well-known that $\Omega$ shifts homotopy groups down, we see that $B$ shifts homotopy groups up. That is, if $e\in X_1$ denotes the identity element, then
		\begin{equation*}
			\pi_i\big((BX)_1\big)\simeq \begin{cases*}
				\pi_{i-1}(X_1,e) & if $i>0$\\
				* & if $i=0$
			\end{cases*}
		\end{equation*}
		(in particular, we don't need to specify a basepoint for $(BX)_1$ since it is connected).
		\item We get a commutative solid diagram
		\begin{equation*}
			\begin{tikzcd}[column sep=large]
				\Dd_{\geq 0}(\IZ)\dar[shift right=0.45ex,"{[1]=\Sigma}"']\rar["K"]\drar[phantom,"\scriptscriptstyle/\!/\!/"] & \CGrp(\An)\dar[shift right=0.45ex,"B"']\\
				\Dd_{\geq 0}(\IZ)\uar[dotted,shift right=0.45ex,"{\Omega}"']\rar["K"] & \CGrp(\An)\uar[dotted,shift right=0.45ex,"\Omega"']
			\end{tikzcd}
		\end{equation*}
		in $\Cat_\infty$. So $B$ preserves Eilenberg--MacLane anima and only shifts the homotopy groups up by $1$.
		
		To see commutativity, first note that the diagram with the dotted vertical arrows commutes (where $\Omega=(\tau_{\geq 1}-)[-1]\colon \Dd_{\geq 0}(\IZ)\morphism \Dd_{\geq 0}(\IZ)$ is defined as in Philosophical Nonsense~\cref{rem:PhilosophicalNonsenseII}). Indeed, $\Omega$ is given by a pullback which is preserved by the right-adjoint functor $K$ (being a right adjoint is still true for the lift $K\colon \Dd_{\geq 0}(\IZ)\morphism \CGrp(\An)$, simply by construction and our \cref{obs:AdjunctionOfFunctorCats} that adjunctions extend to functor categories). By abstract nonsense, commutativity of the dotted diagram induces a natural transformation $BK\Rightarrow K(-[1])$. Whether this is an equivalence can be checked on objects $C\in \Dd_{\geq 0}(\IZ)$. But whether $B(KC)\morphism K(C[1])$ is an equivalence can in turn be checked on homotopy groups of underlying anima. But from \itememph{b} we see that homotopy groups of both $B(KC)$ and $K(C[1])$ are those of $KC$, shifted up by one.
	\end{alphanumerate}
\end{smallexm}

Now towards the proof of \cref{eq:EckmannHilton}. Let's denote by $\Cat_\infty^\times\subseteq \Cat_\infty$ the (non-full) sub-$\infty$-category spanned by $\infty$-categories with finite products and functors preserving these. Let furthermore $\Cat_\infty^\mathrm{add}\subseteq \Cat_\infty^{\mathrm{semi\mhyph add}}\subseteq \Cat_\infty^\times$ denote the full sub-$\infty$-categories spanned by the additive and semi-additive $\infty$-categories respectively.\addtocounter{dummy}{-3}
\begin{thm}\label{thm:CMonCGrpAdjunctions}
	If an $\infty$-category $\Cc$ has products, then $\CMon(\Cc)$ and $\CGrp(\Cc)$ are semi-additive and additive, respectively. Furthermore, the functors
	\begin{align*}
		\CMon\colon \Cat_\infty^\times&\morphism \Cat_\infty^\mathrm{semi\mhyph add}\\
		\CGrp\colon \Cat_\infty^\times &\morphism \Cat_\infty^\mathrm{add}
	\end{align*}
	are right-adjoint to the inclusions $ \Cat_\infty^{\mathrm{semi\mhyph add}}\subseteq \Cat_\infty^\times$ and $\Cat_\infty^\mathrm{add}\subseteq\Cat_\infty^\times$ respectively.
\end{thm}
Using \cref{thm:CMonCGrpAdjunctions} we get an easy proof of \cref{eq:EckmannHilton}. First note that the two maps
\begin{equation*}
	\begin{tikzcd}
		\CMon\big(\CMon(\Cc)\big)\rar[bend left,start anchor=north east, end anchor=north west, "\sim"', "\mathrm{forget}_*"]\rar[bend right,start anchor=south east, end anchor=south west, "\sim", "\mathrm{forget}"'] & \CMon(\Cc)
	\end{tikzcd}	
\end{equation*}
agree and are equivalences. Indeed, $\CMon(-)\morphism (-)$ is the counit of the adjunction from \cref{thm:CMonCGrpAdjunctions}, hence an equivalence since the left adjoint $\Cat_\infty^\mathrm{semi\mhyph add}\subseteq \Cat_\infty^\times$ is fully faithful. This shows that both functors are equivalences. Moreover, by writing down the triangle identities we find that both maps are inverses (once from the right, once from the left) to the unit $\CMon(\Cc)\morphism \CMon(\CMon(\Cc))$, hence they must agree. This almost shows \cref{eq:EckmannHilton}, except that we want to replace one $\CMon$ by a $\Mon$. To do this, note that \cref{prop:CMonOfSemiAdditive} implies $\CMon(\CMon(\Cc))\simeq \Mon(\CMon(\Cc))$. Moreover $\CMon(\Mon(\Cc))\simeq \Mon(\CMon(\Cc))$ follows by straightforward inspection, as both sides can be interpreted as full subcategories of $\Fun(\IDelta^\op\times \IGamma^\op,\Cc)$. Hence we are done.

Note that our argument can not be extended to show that the two forgetful functors $\Mon(\Mon(\Cc))\shortdoublemorphism\Mon(\Cc)$ agree. This is still true, but much harder to show. Moreover, if $\Cc$ is a $1$-category, then $\Mon(\Mon(\Cc))\simeq \CMon(\Cc)$, a fact that is usually called the \emph{Eckmann--Hilton trick}. This is false for $\infty$-categories! What is true instead is
\begin{equation*}
	\CMon(\Cc)\simeq \limit_{n\in \IN^\op}\Mon^{(n)}(\Cc)\,,
\end{equation*}
where $\Mon^{(n)}(-)=\Mon(\Mon(\dotso(\Mon(-))\dotso))$ is the $n$-fold iteration of $\Mon(-)$. The transition morphisms $\Mon^{(n)}(\Cc)\morphism \Mon^{(n-1)}(\Cc)$ are given by forgetting one $\Mon$. There are $n$ choices to do so, but it turns out that they all agree, generalizing our claim about $\Mon(\Mon(\Cc))\shortdoublemorphism\Mon(\Cc)$.



For the proof of \cref{thm:CMonCGrpAdjunctions}, we need a criterion for semi-additivity.
\begin{lem}[compare {\cite[Proposition~\HAthm{2.4.3.19}]{HA}}, which is stronger]\label{lem:SemiAddCriterion}
	Let $\Cc$ be a category with finite products. Then $\Cc$ is semi-additive if the following conditions hold:
	\begin{alphanumerate}
		\item The terminal object $*\in \Cc$ also is initial.
		\item We have a natural transformation $\mu\colon (\Delta\colon x\mapsto x\times x)\Rightarrow\id_\Cc$ such that both compositions
		\begin{align*}
			x&\simeq x\times *\xrightarrow{{\id_x}\times 0} x\times x\morphism[\mu_x]x\\
			 x&\simeq *\times x\xrightarrow{0\times{\id_x}} x\times x\morphism[\mu_x]x
		\end{align*}
		are homotopic to $\id_x$ for all $x\in \Cc$, and the diagram
		\begin{equation*}
			\begin{tikzcd}[column sep=small]
				(x\times x)\times (y\times y)\drar["\mu_x\times \mu_y"']\ar[rr,"\id_x\times\mathrm{flip}\times \id_y"] & & (x\times y)\times (x\times y)\dlar["\mu_{x\times y}"]\\
				& x\times y & 
			\end{tikzcd}
		\end{equation*}
		commutes up to homotopy for all $x,y\in \Cc$.
	\end{alphanumerate}
\end{lem}
\begin{rem*}
	Lurie only requires that the first of the two compositions in \cref{lem:SemiAddCriterion}\itememph{b} is homotopic to $\id_x$, and Fabian may or may not have done the same in the lecture. However, the second one doesn't follow from the first, and we'll really need both to be homotopic to $\id_x$ in the proof. In fact, if only the first composition were to be homotopic to $\id_x$, we could take $\mu_x$ to be $\pr_1\colon x\times x\morphism x$ for all $x\in \Cc$ (it's clear that this also makes the diagram above commute), and then we could show that every $\infty$-categories with finite products and a zero object is semi-additive. This can't be true!
\end{rem*}
\begin{proof}[Proof of \cref{lem:SemiAddCriterion}]
	Let $c,d\in \Cc$. We want to show that the maps
	\begin{equation*}
		c\simeq c\times *\xrightarrow{\id_c\times 0}c\times d\quad\text{and}\quad d\simeq *\times d\xrightarrow{0\times\id_d}c\times d
	\end{equation*}
	exhibit $c\times d$ as a coproduct of $c$ and $d$. That is, we need to show that the natural transformation
	\begin{equation*}
		\phi\colon \Hom_\Cc(c\times d,-)\overset{\sim}{\Longrightarrow}\Hom_\Cc(c,-)\times \Hom_\Cc(d,-)
	\end{equation*}
	induced by these maps is an equivalence. This can be checked on objects. Given $a\in \Cc$, we construct an inverse $\psi_a$ to $\phi_a$ as follows:
	\begin{equation*}
		\psi_a\colon \Hom_\Cc(c,a)\times \Hom_\Cc(d,a)\morphism[\times]\Hom_\Cc(c\times d,a\times a)\morphism[\mu_a]\Hom_\Cc(c\times d,a)
	\end{equation*}
	Note that we do not claim that the $\psi_a$ combine into a natural transformation $\psi$ which is inverse to $\phi$, only that $\psi_a$ is inverse to $\phi_a$, since it suffices to have $\phi$ a pointwise equivalence. Let's prove $\phi_a\circ \psi_a\simeq \id$ first. This follows essentially from the first condition in \itememph{b}, but since this sparked some confusion in the lecture, I'll try to give a much more detailed proof than in the lecture and be as precise as possible (hopefully this doesn't make things worse). We begin by drawing a tiny diagram:
	\begin{equation*}
		\begin{tikzcd}[column sep=1.765ex]
			& \Hom_\Cc(c,a)\times \Hom_\Cc(d,a)\dar[iso]\ar[ldd,bend right,start anchor=west,"\times"']\ar[rddd,bend left,start anchor=east,"\id","\sim"{swap,sloped}] &[-3ex] \\
			\phantom{X}\drar[phantom, start anchor=center, end anchor=center, "\mathrm{(I)}"]& \Hom_\Cc(c\times *,a\times *)\times \Hom_\Cc(*\times d,*\times a)\dar\drar[phantom, start anchor=center, end anchor=center, "\mathrm{(III)}"] &[-3ex] \\
			\Hom_\Cc(c\times d,a\times a)\dar["\mu_a"']\rar\drar[phantom, start anchor=center, end anchor=center,"\mathrm{(II)}"] & \Hom_\Cc(c\times *,a\times a)\times \Hom_\Cc(*\times d,a\times a)\dar["\mu_a\times \mu_a"] &[-3ex] \phantom{X} \\
			\Hom_\Cc(c\times d,a)\rar & \Hom_\Cc(c\times*,a)\times \Hom_\Cc(*\times d,a)\rar[iso] &[-3ex] \Hom_\Cc(c,a)\times \Hom_\Cc(d,a)
		\end{tikzcd}
	\end{equation*}
	If we can show that this diagram commutes up to homotopy, then walking around its perimeter will show $\phi_a\circ \psi_a\simeq \id$. The cell labelled (III) commutes up to homotopy because the composition of the three downward pointing arrows and the rightward pointing arrow is induced by $\mu_a\circ (\id_a\times 0)\colon a\simeq a\times *\morphism a\times a\morphism a$ in the first factor and by $\mu_a\circ (0\times\id_a)\colon a\simeq *\times a\morphism a\times a\morphism a$ in the second factor, both of which are homotopic to $\id_a$ by assumption.
	
	To see that the cell labelled (II) commutes up to homotopy, it suffices to check this after composition with $\pr_1\colon \Hom_\Cc(c\times *,a)\times \Hom_\Cc(*\times d,a)\morphism \Hom_\Cc(c\times *,a)$ and after composition with $\pr_2\colon \Hom_\Cc(c\times *,a)\times \Hom_\Cc(*\times d,a)\morphism \Hom_\Cc(*\times d,a)$. We only give the argument for $\pr_1$, since the argument for $\pr_2$ will be analogous. After composition with $\pr_1$, we end up with two maps $\Hom_\Cc(c\times d,a\times a)\morphism \Hom_\Cc(c\times *,a)$. Both are given by precomposition with $c\times *\morphism c\times d$ and postcomposition with $\mu_a\colon a\times a\morphism a$, but in different orders. However, it doesn't matter (up to homotopy) in which order we postcompose and precompose, which proves commutativity of (II).
	
	Finally, to see that (I) commutes, we look at the diagram (and its obvious analogue where $c\times *$ is replaced by $*\times d$)
	\begin{equation*}
		\begin{tikzcd}
			\Hom_\Cc(c,a)\times \Hom_\Cc(d,a)\dar\rar["\times"] & \Hom_\Cc(c\times d,a\times a)\dar\\
			\Hom_\Cc(c,a)\times \Hom_\Cc(*,a)\rar["\times"] & \Hom_\Cc(c\times *,a\times a)\\
			\Hom_\Cc(c,a)\times \Hom_\Cc(*,*)\uar[iso]\rar["\times"] & \Hom_\Cc(c\times *,a\times *)\uar
		\end{tikzcd}
	\end{equation*}
	This commutes \enquote{by naturality of the $\times$-construction on morphism anima}. The bottom left vertical arrow is an equivalence since $*$ is initial, so $\Hom_\Cc(*,-)\simeq *$. In particular, this shows $\Hom_\Cc(c,a)\times \Hom_\Cc(*,*)\simeq \Hom_\Cc(c,a)$. Under this identification, the bottom row becomes the morphism $\Hom_\Cc(c,a)\morphism \Hom_\Cc(c\times *,a\times *)$ from (I) and the two left vertical arrows compose to the projection $\pr_1\colon \Hom_\Cc(c,a)\times \Hom_\Cc(d,a)\morphism \Hom_\Cc(c,a)$. This shows that (I) commutes indeed and therefore that $\phi_a\circ \psi_a\simeq \id$.
	
	Now we prove $\psi_a\circ \phi_a\simeq \id$. First we check that $\psi_a\circ\phi_a$ is homotopic to the map induced by precomposition with
	\begin{equation*}
		c\times d\simeq (c\times *)\times (*\times d)\morphism (c\times d)\times (c\times d)\xrightarrow{\mu_{c\times d}} c\times d\,.
	\end{equation*}
	To show this, we draw another moderately large diagram:
	\begin{equation*}
		\begin{tikzcd}[column sep=0.5em]
			\Hom_\Cc(c,a)\times \Hom_\Cc(d,a)\dar\drar[phantom, start anchor=center, end anchor=center, "\mathrm{(IV)}"]&[-0.4em] \Hom_\Cc(c\times d,a)\dar["\Delta"]\lar\ar[ddr,bend left,start anchor=east, end anchor=north,"\mu_{c\times d}"]\ar[ddr,phantom, start anchor=center, end anchor=center, "\mathrm{(V)}",pos=0.25] & \\
			\Hom_\Cc(c\times d,a\times a)\dar["\mu_a"']\ar[drr,phantom, start anchor=center, end anchor=center, "\mathrm{(VI)}"] & \Hom_\Cc\big((c\times d)\times (c\times d),a\times a)\big)\lar[dotted]\drar[bend left=20, start anchor=east,"\mu_a"'] &  \\
			\Hom_\Cc(c\times d,a) & \Hom_\Cc\big((c\times *)\times (*\times d),a\big)\lar & \Hom_\Cc\big((c\times d)\times (c\times d),a\big)\lar
		\end{tikzcd}
	\end{equation*}
	The cell labelled (V) commutes up to homotopy because $\mu\colon \Delta\Rightarrow \id$ is a natural transformation (and I believe Fabian is right that we really need this here---if we were to use Lurie's weaker condition, it would only commute on $\pi_0$). If we choose the dotted arrow to be induced by $c\times d\simeq(c\times *)\times (*\times d)\morphism (c\times d)\times (c\times d)$, then also (VI) commutes up to homotopy because it doesn't matter in which order precomposition with said morphism and postcomposition with $\mu_a$ are applied. Finally, (IV) commutes by inspection. Walking around the perimeter of this diagram now proves that $\psi_a\circ \phi_a$ is really induced by precomposition with the map from above.
	
	Now we finally use the second condition from \itememph{b} to obtain a homotopy-commutative diagram
	\begin{equation*}
		\begin{tikzcd}
			c\times d\rar[iso]\drar[iso] & (c\times *)\times (*\times c)\eqar[d]\rar & (c\times d)\times (c\times d)\rar["\mu_{c\times d}"] & c\times d\\
			& (c\times *)\times (*\times d)\rar & (c\times c)\times (d\times d)\uar["\id_c\times\mathrm{flip}\times \id_c","\sim"{swap,sloped}]\urar["\mu_c\times \mu_d"']
		\end{tikzcd}
	\end{equation*}
	in $\Cc$. Using the first condition from \itememph{b} we see that the composition of the lower three arrows is homotopic to $\id_c\times \id_d$. Hence the morphism we precompose with is homotopic to the identity on $c\times d$, which shows $\psi_a\times \phi_a\simeq \id$, as required.
\end{proof}
\begin{proof}[Proof of \cref{thm:CMonCGrpAdjunctions}]
	We first prove that $\CMon(\Cc)$ is semi-additive. Of course we would like to apply \cref{lem:SemiAddCriterion}, so we must produce the corresponding data on $\CMon(\Cc)$. First of all, note that $\const *\colon \IGamma^\op\morphism \Cc$ is both terminal and initial in $\CMon(\Cc)$. It is even terminal in $\Fun(\IGamma^\op,\Cc)$ because $*\in \Cc$ is terminal and limits in functor categories are computed pointwise (\cref{lem:f^*preservesColimits}). To see that it is initial, let $X\in \CMon(\Cc)$ and compute
	\begin{equation*}
		\Nat(\const *,X)\simeq \Hom_\Cc\left(*,\limit_{\langle n\rangle\in \IGamma^\op}X_n\right)\simeq \Hom_\Cc(*,X_0)\simeq \Hom_\Cc(*,*)\simeq *\,,
	\end{equation*}
	where we used that $\langle 0\rangle\in \IGamma^\op$ is an initial object and thus spans a final subset in the sense of \cref{def:cofinal} (and yes, our terminology here isn't optimal, but then again there is no good terminology at all).
	
	Next we need to construct functorial maps $\mu_M\colon M\times M\morphism M$ for all $M\in \CMon(\Cc)$. Consider the map $\times \colon \IGamma^\op\times \IGamma^\op\morphism \IGamma^\op$ given by crossing the sets and crossing the morphisms (note that this wouldn't work with $\IDelta^\op$). It induces a functor $\Fun(\IGamma^\op,\Cc)\morphism \Fun(\IGamma^\op,\Fun(\IGamma^\op,\Cc))$, and one easily checks that the Segal condition is preserved, i.e.\ we get a functor
	\begin{equation*}
		\operatorname{Double}\colon \CMon(\Cc)\morphism\CMon\big(\CMon(\Cc)\big)\,.
	\end{equation*}
	Unravelling definitions, we find that $\operatorname{Double}(-)_1\colon \CMon(\Cc)\morphism\CMon(\Cc)$ is naturally equivalent to the identity and $\operatorname{Double}(-)_2\colon \CMon(\Cc)\morphism\CMon(\Cc)$ is naturally equivalent to the diagonal $\Delta\colon M\mapsto M\times M$. Hence we can take $\mu_M\colon M\times M\morphism M$ to be the multiplication $m\colon \operatorname{Double}(M)_2\morphism \operatorname{Double}(M)_1$, i.e.\ the map induced by $m\colon \langle 2\rangle \morphism\langle 1\rangle$ in $\IGamma^\op$ (see \cref{def:CartesianCommutativeMonoids}). Since the $\mu_M$ are induced by a map in $\IGamma^\op$, it's clear that they assemble into a natural transformation $\mu$. Also the required conditions for $\mu$ are straightforward to check.
	
	This shows that $\CMon(\Cc)$ is semi-additive by \cref{lem:SemiAddCriterion}. Hence the same holds for its full subcategory $\CGrp(\Cc)\subseteq \CMon(\Cc)$ since this is closed under finite products. To show that $\CGrp(\Cc)$ is additive, we need to check that the shear map is an equivalence for all objects $G\in \CGrp(\Cc)$. But in view of the fact that the codiagonal $G\times G\morphism G$ is induced by $\mu_G$ (this follows by unravelling the explicit inverse in the proof of \cref{lem:SemiAddCriterion}), hence by multiplication on $\operatorname{Double}(G)$, this gets translated into the fact that $\operatorname{Double}(G)$ is a cartesian commutative group itself, which is again easy to check.
	
	Finally, we need to show that $\CMon\colon \Cat_\infty^\times\morphism \Cat_\infty^\mathrm{semi\mhyph add}$ and $\CGrp\colon \Cat_\infty^\times \morphism \Cat_\infty^\mathrm{add}$ are right adjoints. If we denote $R=\CMon$ and $\eta\colon R\Rightarrow \id$ the forgetful natural transformation (given by evaluation in degree $1$), then \cref{prop:CMonOfSemiAdditive} and the fact that $\CMon(\Cc)$ is semi-additive show that $\eta R\colon R\circ R\overset{\sim}{\Longrightarrow} R$ and $R\eta\colon R\circ R\overset{\sim}{\Longrightarrow} R$ are natural equivalences. Thus we may apply the dual of Proposition~\labelcref{prop:LLaddendum} to see that $R$ is a right adjoint to the inclusion $\im(R)\subseteq \Cat_\infty^\times$ of its essential image. But $\im(R)=\Cat_\infty^\mathrm{semi\mhyph add}$: Indeed, \enquote{$\subseteq$} was checked above, whereas \enquote{$\supseteq$} follows from \cref{prop:CMonOfSemiAdditive} again. The argument for $\CGrp$ is completely analogous.
\end{proof}
\numpar*{Free $\IE_\infty$-Monoids}\lecture[Spectra! The recognition principle for infinite loop spaces. Stable $\infty$-categories.\newline --- \emph{\enquote{Old-timey topologists hate this proof. I love it.}}]{2020-12-10}\label{par:FreeCMon}
Similar to \cref{prop:FreeMonoids}, the forgetful functor $\CMon(\An)\morphism\An$ as a left adjoint $\Free^\CMon\colon \An\morphism \CMon(\An)$. %Like \cref{prop:FreeMonoids}, we won't prove this, but Fabian gives a sketch in \cite[Chapter~II pp.\:128--132]{KTheory}, after our venture into the land of $\infty$-operads. 
On objects $X\in\An$, it is given by
\begin{equation*}
	\Free^\CMon(X)_1\simeq \coprod_{n\geq 0}X_{h\SS_n}^n\,,
\end{equation*}
i.e.\ we take free words in $X$, but we need to homotopy-quotient out the action of the symmetric group $\SS_n$, which the notation $(-)_{h\SS_n}$ is supposed to do. In general, given a functor $F\colon BG\morphism\Cc$, where $\Cc$ is an $\infty$-category with colimits and $G$ is a discrete group, we put
\begin{equation*}
	F_{hG}\simeq \colimit_{BG}F\,.
\end{equation*}
On $X^n$, the $\SS_n$-action is given by \enquote{permuting factors}. More precisely, we may apply \cref{par:Grp(An)=(*/An)Connected} to see that giving a functor $B\SS_n\morphism\core(\An)$ sending the unique $0$-simplex of $B\SS_n$ to $X^n$ is the same as giving a functor $\SS_n\morphism \Omega_{X^n}\core(\An)\simeq \Hom_{\core(\An)}(X^n,X^n)$. But each factor permutation is a map in $\Hom_{\core(\An)}(X^n,X^n)$, hence there's indeed a canonical map from the discrete anima $\SS_n$ into it.

To prove the formula above, we can proceed as in the proof* of \cref{prop:FreeMonoids}: Call a map in $\IGamma^\op$ \emph{inert} if it is a bijection (where it is defined), and \emph{active} if it is defined everywhere. With this terminology, the proof* of \cref{prop:FreeMonoids} can be copied, except for the following small change: In the decomposition $\IGamma_\mathrm{int}^\op/\langle 1\rangle\simeq \coprod_\alpha \Cc_\alpha$, the categories $\Cc_\alpha$ no longer have the active map $\alpha\colon \langle n\rangle\morphism \langle 1\rangle$ as a terminal object; instead, $\{\alpha\colon \langle n\rangle\morphism \langle 1\rangle\}$ together with its automorphisms spans cofinal subcategory. This subcategory is equivalent to the anima $B\SS_n$, as follows immediately from \cref{par:Grp(An)=(*/An)Connected}. Thus we get indeed $\Free^\CMon(X)_1\simeq \coprod_{n\geq 0}X_{h\SS_n}^n$.

As an example,
\begin{equation*}
	\Free^\CMon(*)_1\simeq \coprod_{n\geq 0}B\SS_n\simeq \big(\{\text{finite sets, bijections}\},\sqcup\big)\eqqcolon \SS,
\end{equation*}
where the right-hand side is the symmetric monoidal $1$-groupoid of finite sets and bjections between them, whose tensor product is given by disjoint union. In particular, even though $\Free^\Mon(*)\simeq \IN$ is the free monoid on a point and commutative, it's not the free commutative monoid on a point.



\section{Spectra and Stable \texorpdfstring{$\infty$}{infinity}-Categories}
Let $\CGrp(\An)_{\geq i}\subseteq \CGrp(\An)$ be the full sub-$\infty$-category spanned by those $X$ with $\pi_jX_1=0$ for $j<i$. We have seen in \cref{cor:CommutativeHorizontalAdjoints} that $B\colon \CGrp(\An)\morphism\CGrp(\An)$ is fully faithful with right adjoint $\Omega$. Moreover, we've seen during the proof that $B$ actually has essential image in $\CGrp(\An)_{\geq 1}$. In fact, its essential image \emph{is} $\CGrp(\An)_{\geq 1}$! To see this, it suffices to check that the counit $c\colon B\Omega X\isomorphism X$ is an equivalence for all $X\in \CGrp(\An)_{\geq 1}$. This can be checked on underlying anima, i.e.\ after $\ev_1$, and then again on homotopy groups. Since $\Omega$ shifts homotopy groups down and $B$ shifts them up, we get that $c_*\colon \pi_i(B\Omega X)_1\isomorphism \pi_i X_1$ is an isomorphism for $i>0$. For $i=0$ it is one as well as both sides are connected.

Iterating this argument, we find that the $i$-fold application $B^{(i)}=B\circ\dotsb\circ B$ is an equivalence
\begin{equation*}
	B^{(i)}\colon \CGrp(\An)\isomorphism\CGrp(\An)_{\geq i}\,.
\end{equation*}
Moreover, since $\Omega B\simeq \id$ by \cref{cor:CommutativeHorizontalAdjoints}\itememph{a}, we get $\Omega B^{(i+1)}\simeq B^{(i)}$. This yields a fully faithful functor
\begin{equation*}
	B^\infty\colon \CGrp(\An)\morphism \limit_{\IN^\op}\left(\dotso\morphism[\Omega]\CGrp(\An)\morphism[\Omega]\CGrp(\An)\morphism[\Omega]\CGrp(\An)\right)
\end{equation*}
(the limit being taken in $\Cat_\infty$ of course) with essential image
\begin{equation*}
	\limit_{\IN^\op}\left(\dotso\morphism[\Omega]\CGrp(\An)_{\geq 2}\morphism[\Omega]\CGrp(\An)_{\geq 1}\morphism[\Omega]\CGrp(\An)\right)\,.
\end{equation*}\stepcounter{dummy}

\begin{defi}\label{def:spectra}
	We say that an $\infty$-category $\Cc$ has \emph{finite limits} if it has finite products and pullbacks (one can check that this is equivalent to having limits over all finite simplicial sets). If this is the case, then a suspension functor $\Omega\colon {*/\Cc}\morphism */\Cc$ can be defined and we put
	\begin{equation*}
		\Sp(\Cc)\coloneqq \limit_{\IN^\op}\left(\dotso\morphism[\Omega]*/\Cc\morphism[\Omega]*/\Cc\right)\,
	\end{equation*}
	called the $\infty$-category of \emph{spectrum objects} in $\Cc$. If $\Cc=\An$, we obtain $\Sp=\Sp(\An)$, the $\infty$-category of \emph{spectra}.
\end{defi}
\begin{prop}\label{prop:SpCGrpIsSpAgain}
	Suppose $\Cc$ has finite limits. Then the forgetful functor
	\begin{equation*}
		\Sp\big(\CGrp(\Cc)\big)\isomorphism\Sp(\Cc)
	\end{equation*}
	is an equivalence.
\end{prop}
\begin{proof}[Proof sketch]
	Topologists hate this proof. First observe that $\Sp(\CGrp(\Cc))\simeq \CGrp(\Sp(\Cc))$. This is because
	\begin{equation*}
		\limit_{\IN^\op}\left(\dotso\morphism[\Omega]\Fun(\IGamma^\op,*/\Cc)\morphism[\Omega]\Fun(\IGamma^\op,*/\Cc)\right)\simeq \Fun\left(\IGamma^\op,\limit_{\IN^\op}\left(\dotso\morphism[\Omega]*/\Cc\morphism[\Omega]*/\Cc\right)\right)
	\end{equation*}
	(here we use that $\Fun(\Dd,-)\colon \Cat_\infty\morphism\Cat_\infty$ commutes with limits since it is right-adjoint to $\Dd\times -$ and also that $\Omega$ in functor categories can be computed pointwise by \cref{lem:f^*preservesColimits}) and one checks that the Segal condition and the group condition are preserved.
	
	Therefore, we will be done by \cref{prop:CMonOfSemiAdditive} once we show that $\Sp(\Cc)$ is an additive category. Of course we will employ \cref{lem:SemiAddCriterion} again. Clearly $(*,*,\dotsc)\in \Sp(\Cc)$ is both initial and terminal. Also $\Sp(\Cc)$ again has finite limits, inherited from $*/\Cc$ because limits in limits can be computed degreewise (combine the fact that $\Hom$ anima in limits are limits of $\Hom$ anima with \cref{cor:HomPreservesColimits} and \cref{prop:ColimitsCommute}). Moreover, we know that $\Omega\colon \Sp(\Cc)\isomorphism \Sp(\Cc)$ is an equivalence by construction and factors as
	\begin{equation*}
		\Omega\colon \Sp(\Cc)\morphism \Grp\big(\Sp(\Cc)\big)\morphism[\ev_1]\Sp(\Cc)
	\end{equation*}
	(I didn't find this quite obvious, so I decided to sketch a proof in \cref{rem*:OmegaFactorsOverGrp} below). Finally, to apply \cref{lem:SemiAddCriterion}, we must produce functorial morphisms $\mu_X\colon X\times X\morphism X$ with the properties stated there. For this, we use $X\simeq \Omega(\Omega^{-1}X)_1$ and take $\mu_X$ to be induced by the multiplication on $\Omega(\Omega^{-1}X)\in \Grp(\Sp(\Cc))$.
	
	Up to checking the various conditions, this shows that $\Sp(\Cc)$ is semi-additive. For additivity we must check that the shear map is an equivalence on every $X\in \Sp(\Cc)$. Similar to the proof of \cref{thm:CMonCGrpAdjunctions} this follows from the fact that $\Omega(\Omega^{-1}X)$ is a cartesian group rather than just a monoid.
\end{proof}
\begin{rem*}\label{rem*:OmegaFactorsOverGrp}
	For any $\infty$-category $\Dd$ with finite limits and a zero object $*\in \Dd$ (i.e.\ an object that is both initial and terminal) the functor $\Omega\colon \Dd\morphism\Dd$ factors as
	\begin{equation*}
		\Omega\colon \Dd\morphism\Grp(\Dd)\morphism[\ev_1]\Dd
	\end{equation*}
	This can be bootstrapped from the case $\Dd=*/\An$ which we already know from \cref{par:Grp(An)=(*/An)Connected}. The first step is to enhance the Yoneda embedding to a fully faithful functor $\Dd\morphism \Fun(\Dd^\op,*/\An)$. To this end we consider $\Fun(\Dd^\op,\An)\morphism\Fun(\Ar(\Dd^\op),\Ar(\An))\morphism\Fun(*/\Dd^\op,\Ar(\An))$ and check that its image lands in the \enquote{correct fibre}, i.e.\ in $\Fun(*/\Dd^\op,*/\An)$ if we restrict the source to finite limits-preserving functors (where the Yoneda embedding lands).
	
	On $\Fun(\Dd^\op,*/\An)$ we have loop functor $\Omega\colon \Fun(\Dd^\op,*/\An)\morphism \Fun(\Dd^\op,*/\An)$ as well, and it is simply given by postcomposition with $\Omega\colon {*/\An}\morphism*/\An$. Hence it factors as 
	\begin{equation*}
		\begin{tikzcd}
			& \Fun\big(\Dd^\op,\Grp(*/\An)\big)\dar["\ev_1"]\rar[iso] & \Grp\big(\Fun(\Dd^\op,*/\An)\big)\\
			\Fun(\Dd^\op,*/\An)\rar["\Omega"]\urar[dashed,"\Omega"]&\Fun(\Dd^\op,*/\An)
		\end{tikzcd}
	\end{equation*}
	Since the Yoneda embedding preserves limits, we get a lift $\Omega\colon \Dd\morphism \Grp(\Fun(\Dd^\op,*/\An))$ and one checks that it lands in the full sub-$\infty$-category $\Grp(\Dd)$, as required.
\end{rem*}
\refstepcounter{smallerdummy}
\numpar*{\thesmallerdummy. Homotopy Groups of Spectra}
Because  $\IN^\op\subseteq \IZ^\op$ is final in the sense of \cref{def:cofinal}, we can also write $\smash{\Sp(\Cc)\simeq\limit_{\IZ^\op}(\ActsOn{*/\Cc}{\Omega})}$. This gives evaluation functors
\begin{equation*}
	\Omega^{\infty-i}\colon \Sp(\Cc)\morphism\Cc\,.
\end{equation*}
for all $i\in \IZ$. Note $\Omega\Omega^{\infty-i}X\simeq \Omega^{\infty-(i-1)}X\simeq\Omega^{\infty-i}\Omega X$, so this somewhat weird notation actually makes sense. In the case $\Cc=\An$, we can now define homotopy groups for $X\in \Sp$ as
\begin{equation*}
	\pi_i(X)=\pi_0(\Omega^{\infty+i}X)\,.
\end{equation*}
If we think of a spectrum $X$ as a sequence $X=(X_0,X_1,X_2,\dotsc)$, where $X_i=\Omega^{\infty-i}X$, together with equivalences $X_i\simeq \Omega X_{i+1}$, then we see that $\pi_i(X)\simeq \pi_0(X_{-i})\simeq \pi_j(X_{j-i})$ for all $j\geq 0$ (we don't need to specify any base points, since the $X_i$ are pointed anima) because $\Omega\colon {*/\An}\morphism */\An$ shifts homotopy groups down.

It's probably clear to all of you, but let me also mention that equivalences of spectra can be detected on homotopy groups:
\begin{lem*}\label{lem*:WhiteheadForSpectra}
	A map $f\colon X\morphism Y$ of spectra is an equivalence in $\Sp$ iff it induces isomorphisms $f_*\colon \pi_*(X)\isomorphism \pi_*(Y)$ on homotpy groups.
\end{lem*}
\begin{proof*}
	Write $X_i=\Omega^{\infty-i}$ and $Y_i=\Omega^{\infty-i}$ as above. Then $f$ is an equivalence in $\Sp$ iff each $\Omega^{\infty-i}f\colon X_i\morphism Y_i$ is an equivalence in $\An$. Indeed, the inverses of $\Omega^{\infty-i}f$ will again assemble into a compatible system of morphisms and hence into an inverse of $f$. Whether $\Omega^{\infty-i}f$ is an equivalence in $\An$ can be detected on homotopy groups. In fact, both $X_i$ is an element of $\CGrp(\An)$, so all its connected components are equivalent, and the same is true for $Y_i$. Hence, instead of considering homotopy groups with arbitrary base points, it suffices to check that $\pi_0(X)\morphism \pi_0(Y)$ is a bijection and $\pi_j(X_i)\morphism \pi_j(Y_i)$ (with the canonical choice of base points) are group isomorphisms for all $j>0$. 
	
	Now $\pi_j(X_i)\simeq \pi_{j-i}(X)$ for all $j\geq 0$, and the same holds for $Y$, hence it suffices to have isomorphisms $f_*\colon \pi_*(X)\isomorphism \pi_*(Y)$, as claimed.
\end{proof*}
\refstepcounter{smallerdummy}
\numpar*{\thesmallerdummy. Summary of What We Know So Far}
Let's explicitly record the following facts we already learned in the proof of \cref{prop:SpCGrpIsSpAgain} and afterwards:
\begin{alphanumerate}
	\item $\Sp(\Cc)$ has finite limits (because $*/\Cc$ has) and limits in $\Sp(\Cc)$ are computed degreewise, i.e.\ $\Omega^{\infty-i}\colon \Sp(\Cc)\morphism\Cc$ preserves limits.
	\item[\itememph{b^*}] The only reason this doesn't work for colimits as well is that the degreewise colimits might not be compatible with the $\Omega$-maps any more, so they don't necessarily define an element of $\Sp(\Cc)$. If that happens to be the case, then the degreewise colimit is in fact the colimit in $\Sp$, by the same argument as for limits (combine the fact that $\Hom$ anima in limits are limits of $\Hom$ anima with \cref{cor:HomPreservesColimits} and \cref{prop:ColimitsCommute}).\stepcounter{enumi}
	\item $\Omega\colon \Sp(\Cc)\isomorphism\Sp(\Cc)$ is an equivalence.
	\item $\Sp(\Cc)$ is additive.
\end{alphanumerate}
\begin{cor}[Recognition principle for infinite loop spaces, Boardman--Vogt, May, Segal]\label{cor:RecognitionForInfiniteLoopSpaces}
	The functor
	\begin{equation*}
		B^\infty\colon\CGrp(\An)\morphism\Sp
	\end{equation*}
	is fully faithful with essential image those $X\in \Sp$ with $\pi_iX=0$ for $i<0$ \embrace{as a slogan, $\IE_\infty$-groups are spectra with no negative homotopy groups}. Also $\pi_i(B^\infty X)=\pi_i(X_1,*)$ for all $\IE_\infty$-groups $X\in\CGrp(\An)$ and all $i\geq 0$.
\end{cor}
\begin{proof*}
	The target of $B^\infty$ is $\Sp(\CGrp(\An))\simeq \Sp$ by \cref{prop:SpCGrpIsSpAgain}. We have seen that $B^\infty$ is fully faithful and identified its essential image before \cref{def:spectra}. Clearly, any element $X$ of the limit characterising the essential image satisfies $\pi_iX=0$ for $i<0$. Conversely, if $X$ satisfies this condition, then $\pi_j\Omega^{\infty-i}X\simeq \pi_{j-i}X$, which vanishes for $j<i$. Remembering $\Sp\simeq \Sp(\CGrp(\An))$, this shows $\Omega^{\infty-i}X\in\CGrp(\An)_{\geq i}$, so $X$ lies in the essential image of $B^\infty$. The final assertion is clear since $\pi_i(B^\infty X)\simeq \pi_i(\Omega^\infty B^\infty X)\simeq \pi_i(X_1,*)$ holds by definition.
\end{proof*}
\begin{exm}\label{exm:MyFirstSpectra}
	Let $\Cc$ always have finite limits. Here are \enquote{my first spectra}:
	\begin{alphanumerate}
		\item If $\Omega\colon {*/\Cc}\isomorphism */\Cc$ is an equivalence, then $\Sp(\Cc)\simeq \Cc$ (because in that case we're taking a constant limit over $\IN^\op$, which is weakly contractible, so that limit is $\Cc$ by \cref{prop:CoLimitsInCat}). For example, $\Sp(\Dd(\IZ))\simeq \Dd(\IZ)$ or, even more silly, $\Sp(\Sp)\simeq \Sp$.
		\item Let's compute $\Sp(\Dd_{\geq 0}(R))$ for a ring $R$. Recall that the loop functor on $\Dd_{\geq 0}(R)$ is defined as $\Omega C\simeq \tau_{\geq 0}(C[-1])\in \Dd_{\geq 0}(R)$. Then the functor
		\begin{align*}
			\Dd(R)&\isomorphism \Sp\big(\Dd_{\geq 0}(R)\big)\\
			C&\longmapsto \big(\tau_{\geq 0}C,(\tau_{\geq -1}C)[1],(\tau_{\geq -2}C)[2],\dotsc\big)
		\end{align*}
		is an equivalence. The inverse functor sends a sequence $(C_0,C_1,\dotsc)$ with equivalences $\sigma_i\colon C_i\simeq (\tau_{\geq 1}C_{i+1})[-1]$ to $\colimit_\IN C_i[-i]$. The structure maps $C_i[-i]\morphism C_{i+1}[-(i+1)]$ in this colimit are constructed from the $\sigma_i$ in the obvious way.
		\item Consider the Eilenberg--MacLane functor $K\colon \Dd_{\geq 0}(\IZ)\morphism \An$. Being a right adjoint, it preserves finite limits, and in particular loop spaces. Hence it upgrades canonically to a functor
		\begin{equation*}
			\begin{tikzcd}
				\Sp\big(\Dd_{\geq 0}(\IZ)\big)\eqar[d]\rar["\Sp(K)"]& \Sp(\An)\eqar[d]\\
				\Dd(\IZ)\rar["H"] & \Sp
			\end{tikzcd}
		\end{equation*}
		the \emph{Eilenberg--MacLane spectrum functor}. By construction, the diagram
		\begin{equation*}
			\begin{tikzcd}[column sep=small]
				\Dd(\IZ)\drar["{K(\tau_{\geq -i}-)[i]}"']\ar[rr,"H"] & & \Sp\dlar["\Omega^{\infty-i}"]\\
				& \CGrp(\An)
			\end{tikzcd}
		\end{equation*}
		commutes. If $C\in \Dd_{\geq 0}(\IZ)$, then $HC=(K(C),K(C[1]),K(C[2]),\dotsc)$ and we have $\pi_i (HC)=H_i(C)$. If $C$ is an abelian group $A$ concentrated in degree $0$, we obtain the \emph{Eilenberg--MacLane spectrum} $HA=(K(A,0),K(A,1),K(A,2),\dotsc)$.
		\item We have $\Sp(\An^\op)\simeq *$, because it is a sequential limit over $(\emptyset/\An^\op)\simeq (\An/\emptyset)^\op\simeq *$ (since there is no map $X\morphism\emptyset$ in $\An$ except $\id_\emptyset\colon \emptyset\morphism \emptyset$). Also
		\begin{align*}
			\Sp\big((*/\An)^\op\big)&\simeq \limit_{\IN^\op}\left(\dotso\morphism[\Omega](*/\An)^\op\morphism[\Omega](*/\An)^\op\right)\\
			&\simeq \limit_{\IN^\op}\left(\dotso\morphism[\Sigma]*/\An\morphism[\Sigma]*/\An\right)^\op\simeq *\,,
		\end{align*}
		since only contractible spaces can be an infinite suspension (as all homotopy groups must be vanish).
	\end{alphanumerate}
\end{exm}
\numpar{Homology and Cohomology of Spectra}\label{par:HomologyOfSpectra}
Via the Eilenberg--MacLane spectrum functor $H$, the adjunction $\snake{C}_\bullet\colon {*/\An}\shortdoublelrmorphism \Dd_{\geq0}(\IZ)\noloc K$ from Very Long Example~\cref{exm:EilenberMacLane} can be upgraded to another adjunction
\begin{equation*}
	C_\bullet\colon \Sp\doublelrmorphism \Dd(\IZ)\noloc H\,.
\end{equation*}
The new $C_\bullet$ is defined by the formula $C_\bullet(X)\simeq \colimit_{i\in \IN}\snake{C}_\bullet(\Omega^{\infty-i}X)[-i]$ on objects. To see where the transition maps in this colimit come from, recall that $\Omega=[-1]$ is the loop functor in $\Dd(\IZ)$. Together with $\snake{C}(*)\simeq 0$, this provides a natural transformation $\snake{C}_\bullet(\Omega\,-)\Rightarrow \snake{C}_\bullet(-)[-1]$, which yields the required transition maps.

To check that $C_\bullet$ is a left adjoint, it suffices to do so on objects by \cref{cor:AdjointsPointwise} (in particular, this will show that $C_\bullet$ is a functor at all), where we can compute
\begin{align*}
	\Hom_{\Dd(\IZ)}\big(C_\bullet(X),D\big)&\simeq \limit_{i\in \IN^\op}\Hom_{\Dd(\IZ)}\big(\snake{C}_\bullet(\Omega^{\infty-i}X),D[i]\big)\\
	&\simeq \limit_{i\in \IN^\op}\Hom_{*/\An}\big(\Omega^{\infty-i}X,K(D[i])\big)\\
	&\simeq \limit_{i\in \IN^\op}\Hom_{*/\An}(\Omega^{\infty-i}X,\Omega^{\infty-i}HD)\,,
\end{align*}
and the last term is $\Hom_\Sp(X,HD)$ since $\Hom$ anima in $\Sp$ (in fact, in arbitrary limits of $\infty$-categories) can be computed as a limit over the $\Hom$ anima in each degree. Also all equivalences are clearly functorial in $D\in \Dd(\IZ)$.

Now define the \emph{homology} and the \emph{cohomology of $X\in \Sp$ with coefficients in $A\in \Ab$} to be
\begin{gather*}
	H_*(X,A)\coloneqq H_*\big(C_\bullet (X)\otimes_\IZ^LA\big)=\colimit_{i\in \IN}H_*(\Omega^{\infty-i}X,A)\,,\\
	H^*(X,A)\coloneqq H_{*}\big(R\!\Hom_\IZ(C_\bullet(X),A)\big)\,.
\end{gather*}
In the special case $X\simeq\IS[Y]$ for some anima $Y$, it turns out that $H_*(\IS[Y],A)=H_*(Y,A)$ and $H^*(\IS[Y],A)=H^{-*}(Y,A)$ coincide with singular homology and (up to a sign swap) singular cohomology. See \cref{lem:HomologyOfS} below. In \cref{cor*:SpectraCohomologyTheory}  we'll also find out that $H_*(X,A)=\Tor_*^\IS(X,HA)$ and $H^*(X,A)\simeq \Ext_\IS^*(X,HA)$ can be interpreted as \enquote{$\Tor$ and $\Ext$ over the sphere spectrum}.

\numpar*{\smash{\Attention} Warning}
Don't confuse the homology or cohomology \emph{of} a spectrum with the homology or cohomology theory \emph{associated} to a spectrum, which we'll introduce in \cref{def:SpectraCohomologyTheory} below. Also, neither $H_*(HC,A)$ for a complex $C\in \Dd(\IZ)$ nor $H_*(B^\infty M,A)$ for some $M\in\CGrp(\An)$ are easy to compute in general!

\begin{propdef}\label{propdef:StableInftyCategory}
	Suppose $\Cc$ is an $\infty$-category with a zero object \embrace{i.e.\ a simultaneously initial and terminal object $0\in \Cc$}. Then the following conditions are equivalent:
	\begin{alphanumerate}
		\item $\Cc$ has finite limits and $\Omega\colon \Cc\isomorphism\Cc$ is an equivalence.
		\item $\Cc$ has finite colimits and $\Sigma\colon \Cc\isomorphism \Cc$ is an equivalence.
		\item $\Cc$ has finite limits and finite colimits and a commutative square in $\Cc$ is a pushout square iff it is a pullback square.
	\end{alphanumerate}
	Such $\infty$-categories are called \emph{stable $\infty$-categories}.
\end{propdef}
\begin{proof}
	The implications \itememph{c} $\Rightarrow $ \itememph{a}, \itememph{b} are clear: Applying the pushout-pullback condition to the pushout square defining $\Sigma$ and the pullback square defining $\Omega$ shows that the unit $\id\overset{\sim}{\Longrightarrow}\Omega\Sigma$ and counit $\Sigma\Omega\overset{\sim}{\Longrightarrow}\id$ are natural equivalences, so $\Sigma$ and $\Omega$ must be equivalences of $\infty$-categories.
	
	 For \itememph{a} $\Rightarrow$ \itememph{c}, the same argument as in the proof of \cref{prop:SpCGrpIsSpAgain} shows that $\Cc$ is additive  (write $X$ as $\Omega Y$ in $\Grp(\Cc)$ and apply \cref{lem:SemiAddCriterion}). So we only need to check that pushouts exist and coincide with pullbacks. Let $P\subseteq \Fun(\Delta^1\times\Delta^1,\Cc)$ be the full subcategory spanned by pullback squares. We claim:
	\begin{alphanumerate}
		\item[\itememph{\boxtimes}] \itshape The restriction $P\isomorphism \Fun(\Lambda_0^2,\Cc)$ is an equivalence.
	\end{alphanumerate}
	To prove \itememph{\boxtimes}, we construct an inverse functor. Given a diagram $F\colon \Lambda_0\morphism \Cc$, which we can view as a span $c\lmorphism a\morphism b$ in $\Cc$, we construct the following moderately large diagram:
	\begin{equation*}
		\begin{tikzcd}
			\Omega a\rar\dar\drar[pullback] & \Omega c\rar\dar\drar[pullback] & 0\dar & \\
			\Omega b\rar\dar\drar[pullback] & x \rar\dar\drar[pullback] & g \rar\dar\drar[pullback] & 0\dar\\
			0\rar & f\rar\dar\drar[pullback] & a\dar\rar & b\\
			& 0\rar & c & 
		\end{tikzcd}
	\end{equation*}
	All squares are pullbacks as indicated. The fact that $\Omega a$, $\Omega b$, and $\Omega c$ appear in the top left corner follows by combining suitable pullback squares into larger pullback rectangles.
	
	The inverse functor now sends 
	\begin{equation*}
		F=\left(\begin{tikzcd}
			a\dar\rar & b\\
			c
		\end{tikzcd}\right)\longmapsto
		\Omega^{-1}\left(\begin{tikzcd}
			\Omega a\dar\rar\drar[pullback] & \Omega b\dar\\
			\Omega c\rar & x
		\end{tikzcd}\right)
	\end{equation*}
	(technically we have only defined the inverse on objects, but its clear how to make it functorial since limits are functorial).
	
	To construct pushouts using \itememph{\boxtimes}, let $F\colon \Lambda_0^2\morphism\Cc$ be a span $c\lmorphism a\morphism b$ as above. We know that it can be uniquely (up to contractible choice) extended to a pullback square, where the bottom right corner is some object $d\in \Cc$. The same goes for the trivial span consisting of identities $x=x=x$ for some $x\in \Cc$, and in this case the object we have to add is $x$ again of course. Hence
	\begin{equation*}
		\Nat\left(\begin{tikzcd}
			a\dar\rar & b\\
			c
		\end{tikzcd},\begin{tikzcd}
		x\eqar[r]\eqar[d] & x\\
		x
	\end{tikzcd}\right)\simeq \Nat\left(\begin{tikzcd}
	 a\dar\rar\drar[pullback] &  b\dar\\
	 c\rar & d
	\end{tikzcd},\begin{tikzcd}
	x\eqar[r]\eqar[d]\drar[pullback] & x\eqar[d]\\
	x\eqar[r] & x
	\end{tikzcd}\right)\simeq \Hom_\Cc(d,x)
	\end{equation*}
	because $\Delta^1\times \Delta^1$ has terminal vertex $(1,1)$. This computation shows that $d$ is a pushout of the given span. Moreover, we get the property that pushouts agree with pullbacks for free. This finishes the proof of \itememph{a} $\Rightarrow$ \itememph{c}. For \itememph{b} $\Rightarrow$ \itememph{c} just dualize everything.
\end{proof}
An immediate consequence of \cref{propdef:StableInftyCategory} is that $\Sp(\Cc)$ is stable for all $\infty$-categories $\Cc$ with finite limits. In particular, it holds for $\Sp$ and $\Dd(R)$ for any ring $R$ by \cref{exm:MyFirstSpectra}\itememph{b} (but it's also true for $\Kk(R)$). Moreover, $\Fun(\Ii,\Cc)$ is stable whenever $\Cc$ is.
%\addtocounter{dummy}{1}
\begin{cor}\label{cor:ExactFunctors}
	\upshape\lecture[Spectra and prespectra, spectrification. $\Sigma^\infty$ and the Baratt--Priddy-Quillen theorem. Another model for spectra. Towards a symmetric monoidal structure on $\Sp$.]{2020-12-15}\itshape
	For a functor $F\colon \Cc\morphism\Dd$ between stable $\infty$-categories the following are equivalent:
	\begin{alphanumerate}
		\item $F$ preserves finite limits.
		\item $F$ preserves finite colimits.
	\end{alphanumerate}
\end{cor}
\begin{proof}
	Since $\Cc$ and $\Dd$ are additive, $F$ preserves finite coproducts iff it preserves finite products. Since also pushout squares and pullback squares coincide, $F$ preserves the former iff it preserves the latter. This suffices to get all finite colimits and limits.
\end{proof}
\begin{defi}
	\begin{alphanumerate}
		\item Let $F\colon \Cc\morphism\Dd$ be a functor between $\infty$-categories with finite limits. We call $F$ \emph{left exact} if it preserves finite limits.
		\item We denote by $\Cat_\infty^\mathrm{lex}\subseteq \Cat_\infty$ the (non-full) sub-$\infty$-category  spanned by these. Dually, $\Cat_\infty^\mathrm{rex}\subseteq \Cat_\infty$ is the full sub-$\infty$-category spanned by \emph{right-exact} functors, i.e.\ finite colimit-preserving functors between $\infty$-categories with finite colimits.
		\item Finally, we denote by $\Cat_\infty^\mathrm{st}$ the full subcategory of both $\Cat_\infty^\mathrm{lex}$ and $\Cat_\infty^\mathrm{rex}$ spanned by the stable $\infty$-categories.
	\end{alphanumerate}
\end{defi}
For example, the functor $\Omega^\infty\colon \Sp(\Cc)\morphism \Cc$ is left exact for every $\Cc$ with finite limits. Also note that $\Omega^\infty$ factors canonically over $\CGrp(\Cc)$. This needs doesn't need an argument as in \cref{rem*:OmegaFactorsOverGrp}, simply recall that $\Sp(\Cc)\simeq \Sp(\CGrp(\Cc))$ by \cref{prop:SpCGrpIsSpAgain}.
\begin{rem*}
	Before we move on, I would like to point out a subtlety that's perhaps easy to overlook: In \cref{thm:SpLeftAdjoint} below, we'll investigate the functor
	\begin{equation*}
		\Sp(-)\colon \Cat_\infty^\mathrm{lex}\morphism\Cat_\infty^\mathrm{st}\,.
	\end{equation*}
	But why is it a functor though? My own knee-jerk reaction was \enquote{well, limits are functorial}, but that's not the problem. The problem is that $\Omega\colon {*/\Cc}\morphism */\Cc$ is also supposed to be functorial in $\Cc$! Essentially this leads to the following question:
	\begin{alphanumerate}
		\item[\itememph{\boxtimes}] \itshape Let $\Cat_\infty^{\Ii\mhyph\mathrm{ex}}\subseteq \Cat_\infty$ be the \embrace{non-full} subcategory spanned by $\infty$-categories with $\Ii$-shaped limits and functors preserving these. Why do the maps $\limit_\Ii\colon \Fun(\Ii,\Cc)\morphism \Cc$ for $\Cc\in \Cat_\infty^{\Ii\mhyph\mathrm{ex}}$ assemble into a natural transformation
		\begin{equation*}
			\limit_\Ii\colon \Fun(\Ii,-)\Longrightarrow (-)
		\end{equation*}
		of functors $\Cat_\infty^{\Ii\mhyph\mathrm{ex}}\morphism \Cat_\infty$?
	\end{alphanumerate}
	There are a few more steps involved to show that $\Omega$ is natural too. Namely one has to produce natural transformations $(-)\Rightarrow */(-)\Rightarrow \Fun(\Lambda_2^2,*/(-))$, but I'll leave that to you.
	
	To show \itememph{\boxtimes}, let's first produce the \enquote{adjoint} transformation $\const\colon (-)\Rightarrow \Fun(\Ii,-)$. Since the adjunction
	\begin{equation*}
		-\times\Ii\colon \Cat_\infty\shortdoublelrmorphism \Cat_\infty\noloc \Fun(\Ii,-)
	\end{equation*}
	extends to $\Fun(\Cat_\infty^{\Ii\mhyph\mathrm{ex}},\Cat_\infty)$ by \cref{obs:AdjunctionOfFunctorCats}, we may equivalently give a transformation $-\times \Ii\Rightarrow (-)$. Here the projection to the first factor is an obvious candidate and it is indeed the correct one.
	
	We may view the natural transformation $\const$ as a functor $\Cat_\infty^{\Ii\mhyph\mathrm{ex}}\morphism\Ar (\Cat_\infty)$. Its image lies in the full subcategory spanned by those $[1]\morphism\Cat_\infty$ that represent left-adjoint functors (because $\limit_\Ii$ is a pointwise right adjoint). But in fact, more is true (and that's the crucial point). Consider objects $(L\colon \Cc\morphism\Dd)$ and $(L'\colon \Cc'\morphism\Dd')$ in $\Ar(\Cat_\infty)$ as well as a morphism between them, which we view as a solid diagram
	\begin{equation*}
		\begin{tikzcd}
			\Cc\dar["L"']\rar& \Cc'\dar["L'"']\\
			\Dd\uar[dotted, bend right, "R"{swap,anchor=west}]\rar  & \Dd'\uar[dotted, bend right,"R'"{swap, anchor=west}]
		\end{tikzcd}
	\end{equation*}
	If $L$ and $L'$ are left adjoints, then the dotted arrows exist, but the resulting new square only commutes up to a natural transformation, not up to natural equivalence in general. We let $\Ar^L(\Cat_\infty)$ be the non-full sub-$\infty$-category spanned by those squares for which the adjoint square does in fact commute. Then $\Cat_\infty^{\Ii\mhyph\mathrm{ex}}\morphism\Ar (\Cat_\infty)$ already factors over $\Ar^L(\Cat_\infty)$ since morphisms in $\Cat_\infty^{\Ii\mhyph\mathrm{ex}}$ preserve $\Ii$-shaped limits.
	
	Now note that under the equivalence $\Un^\cocart\colon\Ar(\Cat_\infty)\isomorphism \Cocart([1])$, the non-full sub-$\infty$-category $\Ar^L(\Cat_\infty)$ corresponds to $\cat{Bicart}([1])$, the $\infty$-category of bicartesian fibrations over $\Delta^1$ with morphisms that preserve both cocartesian and cartesian edges (we know from \cite[Proposition~XI.10]{HigherCatsII} that bicartesian fibrations correspond to adjunctions, and the additional condition unravels to the condition that cartesian edges are preserved as well). Now we can apply the cartesian straightening $\St^\cart\colon \cat{Bicart}([1])\morphism \Fun([1]^\op,\Cat_\infty)$ to get a functor $\Cat_\infty^{\Ii\mhyph\mathrm{ex}}\morphism \Fun([1]^\op,\Cat_\infty)$. After currying around, this gives a map $[1]^\op\morphism \Fun(\Cat_\infty^{\Ii\mhyph\mathrm{ex}},\Cat_\infty)$ which finally provides the desired transformation.
\end{rem*}
\begin{thm}\label{thm:SpLeftAdjoint}
	The functor $\Sp(-)\colon \Cat_\infty^\mathrm{lex}\morphism\Cat_\infty^\mathrm{st}$ is right-adjoint to the inclusion $\Cat_\infty^\mathrm{st}\subseteq \Cat_\infty^\mathrm{lex}$, the counit being given by $\Omega^\infty\colon \Sp(\Cc)\morphism \Cc$. In particular, if $\Cc$ is a stable $\infty$-category, we get a canonical lift
	\begin{equation*}
		\begin{tikzcd}
			& \Sp\dar["\Omega^\infty"]\\
			\Cc^\op\times \Cc \rar["\Hom_\Cc"']\urar[dashed,"\hom_\Cc"] & \An
		\end{tikzcd}
	\end{equation*}
	of the $\Hom_\Cc$ functor.
\end{thm}
Fabian remarks that $\Cat_\infty^\mathrm{st}\subseteq \Cat_\infty^\mathrm{lex}$ also has a left adjoint given by $\colimit (\ActsOn{*/\Cc}{\Omega})$
but we probably won't ever use this. More interestingly, we can apply \cref{thm:SpLeftAdjoint} to the stable $\infty$-categories $\Dd(R)$ and $\Kk(R)$ for any ring $R$:\refstepcounter{smallerdummy}
\numpar*{\thesmallerdummy. Corollary}\label{cor:homDRIsRHom}\itshape
The $\Hom$ functors on $\Dd(R)$ and $\Kk(R)$ have canonical refinements over $\Sp$, given by
\begin{equation*}
	\hom_{\Dd(R)}(C,D)\simeq H\big(R\!\Hom_R(C,D)\big)\quad\text{and}\quad\hom_{\Kk(R)}(C,D)\simeq H\big(\Hhom_R(C,D)\big)\,.
\end{equation*}\upshape
\begin{proof*}[Proof sketch]
	To prove the left equivalence, it suffices to check that we get $\Hom_{\Dd(R)}(C,D)$ after applying $\Omega^\infty$ (and that everything is functorial in $C$ and $D$, but this will be clear). So we compute
	\begin{align*}
		\Omega^\infty H\big(R\!\Hom_R(C,D)\big)\simeq K\big(R\!\Hom_R(C,D)\big)&\simeq \Hom_\An\big(*,K(R\!\Hom_R(C,D))\big)\\
		&\simeq \Hom_{\Dd(\IZ)}\big(\IZ[0],R\!\Hom_R(C,D)\big)\\
		&\simeq \Hom_{\Dd(R)}\left(\IZ[0]\otimes_\IZ^LC,D\right)\\
		&\simeq	\Hom_{\Dd(R)}(C,D)\,,
	\end{align*}
	using $C_\bullet(*)\simeq \IZ$ and the derived tensor-$\Hom$ adjunction. The right equivalence can be proved analogously, but one should first check that $C_\bullet\colon \An\morphism \Dd_{\geq 0}(\IZ)$ already factors over $\Kk_{\geq 0}(\IZ)$ and gives rise to the same Eilenberg--MacLane functor $K\colon \Kk_{\geq 0}(\IZ)\morphism \An$. We haven't done this, but one can simply modify the construction in Very Long Example~\cref{exm:EilenberMacLane} (details left to the reader).
\end{proof*}

\begin{proof}[Proof of \cref{thm:SpLeftAdjoint}]
	By the dual of Proposition~\labelcref{prop:LLaddendum}, all we need to show is that the two functors $\Omega^\infty\colon \Sp(\Sp(\Cc))\isomorphism\Sp(\Cc)$ and $\Omega^\infty_*\colon \Sp(\Sp(\Cc))\morphism\Sp(\Cc)$ are equivalences. But we know that $\Omega^\infty$ is one by \cref{exm:MyFirstSpectra}\itememph{a}. Also $\Omega_*^\infty$ is an equivalence too because it is a \enquote{coordinate flip} of $\Omega^\infty$. By that, we mean the following: Plugging in the definitions, we can write $\Sp(\Sp(\Cc))$ as a limit indexed by $\IN^\op\times\IN^\op$. Using the dual of \cref{prop:ColimitsCommute}, we can flip that diagram around, swapping $\Omega^\infty$ and $\Omega^\infty_*$, which proves the latter is an equivalence as well.
	
	For the \enquote{in particular}, we need another small argument similar to the proof* of \cref{cor:HomSemiAdditveFactorsThroughCGrpAn}, since $\Hom_\Cc\colon \Cc^\op\times\Cc\morphism \An$ isn't left exact. Every $\Hom_\Cc(x,-)\colon \Cc\morphism\An$ is, however, hence the adjunction provides us with canonical lifts $\hom_\Cc(x,-)\morphism\Sp$ for all $x\in \Cc$ when $\Cc$ is a stable $\infty$-category. To show that these combine into a functor $\hom_\Cc\colon \Cc^\op\times \Cc\morphism \An$, consider the full subcategory $\Fun^{\mathrm{lex}}(\Cc,\An)\subseteq \Fun(\Cc,\An)$ spanned by left exact functors. We claim that
	\begin{equation*}
		\Omega_*^\infty\colon\Fun^\mathrm{ex}(\Cc,\Sp)\isomorphism \Fun^\mathrm{lex}(\Cc,\An)
	\end{equation*}
	is an equivalence. Indeed, it becomes one after applying $\core(-)$ by the adjunction we've just proved. To show that it is an equivalence after $\core\Ar(-)$, we can use the same argument once we convince ourselves that $\Ar\Fun^\mathrm{ex}(\Cc,\Sp)\simeq \Fun^\mathrm{ex}(\Cc,\Ar(\Sp))$ (use that limits in arrow categories are pointwise by \cref{lem:f^*preservesColimits}) and that $\Ar(\Sp)\simeq \Sp(\Ar(\An))$ (use that $\Ar(-)$ commutes with limits).
	
	Now the Yoneda embedding $\Cc^\op\morphism \Fun(\Cc,\An)$ has image in $\Fun^\mathrm{lex}(\Cc,\An)\simeq \Fun^\mathrm{ex}(\Cc,\Sp)$ and we get our desired $\hom_\Cc$ after currying.
\end{proof}

\numpar{Spectra and Prespectra}\label{par:Prespectra}
We already know that $\Sp$ is complete and that the functor $\Omega^{\infty-i}\colon \Sp\morphism\An$ preserves limits for all $i$, see the summary before \cref{cor:RecognitionForInfiniteLoopSpaces}. We also know from \cref{propdef:StableInftyCategory} that $\Sp$ has finite colimits because it is stable. But in fact, $\Sp$ is even cocomplete! To see this, we will write $\Sp$ as a Bousfield localisation of a suitable cocomplete $\infty$-category of \emph{prespectra} and apply \cref{cor:CoLimitsInBousfield}.

Prespectra can be introduced in general. So let $\Cc$ be an $\infty$-category with finite limits and (thus) a terminal object $*\in \Cc$. We let
\begin{equation*}
	\cat{PSp}(\Cc)\subseteq \Fun(\IN^2,*/\Cc)
\end{equation*}
be the full subcategory of functors that \enquote{vanish away from the diagonal}. So the objects of $\cat{PSp}(\Cc)$ can be pictured as lattice diagrams
\begin{equation*}
	\begin{tikzcd}[row sep=1.2em,column sep=1.2em]
		\vdots & \vdots & \vdots & \\
		*\rar\uar & *\rar\uar & X_2\rar\uar & \cdots \\
		*\rar\uar & X_1\rar\uar & *\rar\uar & \cdots\\
		X_0\rar\uar & *\rar\uar & *\rar\uar &\cdots
	\end{tikzcd}	
\end{equation*}
The diagonal squares induce morphisms $\Sigma X_i\morphism X_{i+1}$ (provided $\Cc$ has pushouts) or equivalently $X_i\morphism \Omega X_{i+1}$. If the latter were to be isomorphisms, we would recover $\Sp(\Cc)$, as we'll see in a moment.


To construct a fully faithful embedding $\Sp(\Cc)\monomorphism \cat{PSp}(\Cc)$, write $\IN^2=\colimit_{n\in \IN}\IN^2_{\leq (n,n)}$ as an ascending union of finite pieces $\IN^2_{\leq (n,n)}$ and consider the diagram
\begin{equation*}
	\begin{tikzcd}
		\cat{PSp}(\Cc)\rar[symbol=\subseteq] & \limit\limits_{n\in \IN^\op}\Fun\big(\IN^2_{\leq (n,n)},*/\Cc\big)\\
		\Sp(\Cc)\uar[dashed]\rar[iso] & \limit\limits_{n\in\IN^\op}*/\Cc\uar["\eta=\limit \eta_n"']
	\end{tikzcd}
\end{equation*}
We need to explain where the upward-pointing arrow $\eta=\lim\eta_n$ comes from: The morphisms
\begin{equation*}
	\eta_n\colon {*/\Cc}\morphism \Fun\big(\IN^2_{\leq (n,n)},*/\Cc\big)
\end{equation*}
send an object $X\in */\Cc$ to the right Kan extension of the following functor:
\begin{center}
	\begin{tikzpicture}[x=3em,y=3em,line cap=round]
		\node (00) at (0,0) {$\phantom{W}$};
		\node (10) at (1,0) {$*$};
		\node (20) at (2,0) {$\cdots$};
		\node (30) at (3,0) {$*$};
		\node (01) at (0,1) {$*$};
		\node (11) at (1,1) {$\phantom{W}$};
		\node (21) at (2,1) {$*$};
		\node[inner sep=0] (31) at (3,1) {$\raisebox{1.25ex}{\vdots}$};
		\node[inner sep=0] (02) at (0,2) {$\raisebox{1.25ex}{\vdots}$};
		\node (12) at (1,2) {$*$};
		\node (22) at (2,2) {$\phantom{W}$};
		\node (32) at (3,2) {$*$};
		\node (03) at (0,3) {$*$};
		\node (13) at (1,3) {$\cdots$};
		\node (23) at (2,3) {$*$};
		\node (33) at (3,3) {$X$};
		\draw[-to] (00) -- (01);
		\draw[-to] (01) -- (02);
		\draw[-to] (02) -- (03);
		\draw[-to] (10) -- (11);
		\draw[-to] (11) -- (12);
		\draw[-to] (12) -- (13);
		\draw[-to] (20) -- (21);
		\draw[-to] (21) -- (22);
		\draw[-to] (22) -- (23);
		\draw[-to] (30) -- (31);
		\draw[-to] (31) -- (32);
		\draw[-to] (32) -- (33);
		\draw[-to] (00) -- (10);
		\draw[-to] (10) -- (20);
		\draw[-to] (20) -- (30);
		\draw[-to] (01) -- (11);
		\draw[-to] (11) -- (21);
		\draw[-to,shorten >=1ex] (21) -- (31);
		\draw[-to,shorten <=1ex] (02) -- (12);
		\draw[-to] (12) -- (22);
		\draw[-to] (22) -- (32);
		\draw[-to] (03) -- (13);
		\draw[-to] (13) -- (23);
		\draw[-to] (23) -- (33);
		\draw[FabiansPurple!67!FabiansPink,dotted, rounded corners=2] (00) ++ (-1ex,-1ex) -- ++ (2ex,0) -- ++ (0,2ex) -- ++(-2ex,0) -- cycle;
		\draw[FabiansPurple!67!FabiansPink,dotted, rounded corners=2] (11) ++ (-1ex,-1ex) -- ++ (2ex,0) -- ++ (0,2ex) -- ++(-2ex,0) -- cycle;
		\draw[FabiansPurple!67!FabiansPink,dotted, rounded corners=2] (22) ++ (-1ex,-1ex) -- ++ (2ex,0) -- ++ (0,2ex) -- ++(-2ex,0) -- cycle;
		\path (0,-0.5) to node {$\underbrace{\hspace{10em}}$} node[below=0.5ex] {$\scriptstyle n+1$} (3,-0.5);
		\path (3.5,0) to node[sloped] {$\underbrace{\hspace{10em}}$} node[right=0.5ex] {$\scriptstyle n+1$} (3.5,3);
		\node[left] at (-0.25,1.5) {$\IN^2_{\leq(n,n)}$};
	\end{tikzpicture}
\end{center}
Note the blank spots along the diagonal, except for the $(n,n)\ordinalth$ entry! So we right-Kan extend along the inclusion $\IN^2_{\leq (n,n)}\smallsetminus \{(0,0),\dotsc,(n-1,n-1)\}\subseteq \IN^2_{\leq (n,n)}$. Since this inclusion is fully faithful, \cref{cor:FullyFaithfulKanExtension} (combined with \cref{propdef:BousfieldLocalisation}\itememph{a}) shows that $\eta_n$ is fully faithful, and thus also the limit $\eta$ over all $\eta_n$ gives a fully faithful map between the limits. One easily checks that the image of $\eta$ lands in the full subcategory $\cat{PSp}(\Cc)$ and thus we get the desired fully faithful functor $\Sp(\Cc)\monomorphism \cat{PSp}(\Cc)$.

As another consequence of the maps $\eta_n$ being defined by right Kan extension, we obtain canonical equivalences
\begin{equation*}
	\Hom_{\cat{PSp}(\Cc)}(X,Y)\simeq \limit_{n\in \IN^\op}\Hom_{*/\Cc}(X_n,Y_n)
\end{equation*}
for all $X\in \cat{PSp}(\Cc)$ and $Y\in \Sp(\Cc)$ (just compute the left-hand side as a $\Hom$ anima in $\limit_{n\in \IN^\op}\Fun(\IN_{\leq (n,n)}^2,*/\Cc)$ and use the universal property of right Kan extension).

Note that $\cat{PSp}(\Cc)$ inherits all limits and colimits from $*/\Cc$. Also $\Sp(\Cc)\subseteq \cat{PSp}(\Cc)$ is closed under limits, but not colimits. To show that $\Sp(\Cc)$ is cocomplete (under appropriate conditions), we need to show that it is in fact a Bousfield localisation of $\cat{PSp}(\Cc)$. This is done by the following proposition.
\begin{smallprop}\label{prop:Spectrification}
	If $*/\Cc$ has sequential colimits and $\Omega\colon {*/\Cc}\morphism*/\Cc$ commutes with them, then the inclusion $\Sp(\Cc)\subseteq\cat{PSp}(\Cc)$ has a left adjoint $(-)^\mathrm{sp}\colon \cat{PSp}(\Cc)\morphism\Sp(\Cc)$. On objects it is given by
	\begin{equation*}
		\Omega^{\infty-i}(X^\mathrm{sp})\simeq \colimit_{n\in \IN}\Omega^nX_{n+i}\,,
	\end{equation*}
	where the transition maps are induced by the morphisms $X_i\morphism\Omega X_{i+1}$ that where constructed on the previous page.
\end{smallprop}
Before we prove \cref{prop:Spectrification}, we'll discuss some consequences. First, to be able to apply it in the case $\Cc=\An$, we show:
\begin{smalllem}\label{lem:piPreservesSequentialColimits}
	The loop space functor $\Omega\colon {*/\An}\morphism*/\An$ for pointed anima commutes with sequential colimits. In particular, $\Sp$ is a Bousfield localisation of $\cat{PSp}$ and therefore cocomplete by \cref{cor:CoLimitsInBousfield}.
	
	Since we are already at it, here are two more useful assertions that weren't mentioned in the lecture.
	\begin{alphanumerate}
		\item[\itememph{a^*}] The functors $\Omega^{\infty-i}\colon \Sp\morphism */\An$ commute with sequential colimits for all $i\in \IZ$.
		\item[\itememph{b^*}] The functors $\pi_i\colon \Sp\morphism\cat{Ab}$ for $i\in\IZ$ and $\pi_i\colon {*/\An}\morphism \cat{Set}$ for $i\geq 0$ commute with sequential colimits. In the latter case, also recall that sequential colimits in $\Grp$ \embrace{for $i=1$} or $\Ab$ \embrace{for $i\geq 2$} are computed on underlying sets.
	\end{alphanumerate}
\end{smalllem}
\begin{proof*}
	First of, $\IN$ is weakly contractible, hence the colimits can be formed in $\An$ instead (\cref{rem*:CoLimitsIn(*/An)OrAn}). We can replace any sequential diagram $(X_1\morphism X_2\morphism \dotso)$ in $\An$ by a diagram $(X_1'\monomorphism X_2'\monomorphism \dotso)$ $\Kan$ such that all transition maps are cofibrations; then \cref{thm:HomotopyLimits} allows us to compute the colimit in $\Kan$ instead. Now write $\Omega\simeq \F_*((\IS^1,*),-)$ as the pointed mapping space out of $\IS^1$ (or any simplicial model for it, it doesn't even need to be Kan) and observe that the right-hand side commutes with filtered colimits in $\Kan$ because $\IS^1$ is compact. This shows the first assertion.
	
	Now that we know $\Omega\colon {*/\An}\morphism*/\An$ commutes with filtered colimits, we may apply the summary before \cref{cor:RecognitionForInfiniteLoopSpaces} to see that sequential colimits in $\Sp$ can be computed degreewise, proving \itememph{a^*}. Hence in \itememph{b^*} it suffices to show the assertion for $*/\An$. We can write $\pi_i\simeq \pi_0\colon \F_*((\IS^i,*),-)$. As in the first part, $\F_*((\IS^i,*),-)$ commutes with filtered colimits in $\Kan$, and $\pi_0\colon \Kan\morphism\Set$ preserves arbitrary colimits since it is a left adjoint.
\end{proof*}
\begin{rem*}\label{rem*:piPreservesFilteredColimits}
	Both assertions of \cref{lem:piPreservesSequentialColimits} remain true if we replace \enquote{sequential colimits} by \enquote{filtered colimits}, where an indexing $\infty$-category $ \Ii$ is \emph{filtered} if every map $K\morphism \Ii$ from a finite simplicial set $K$ extends to a map $K^\triangleright \morphism \Ii$ from its cone.
	
	In fact, the proof can more or less be carried over: Filtered $\infty$-categories are weakly contractible (one can use the filteredness condition to show $\pi_i(\Ex^\infty\Ii,*)=0$ for all $i$ and all basepoints), we can again replace the given diagram by some cofibrant diagram in the correct model structure on $\Fun(\CC[\Ii],\sSet)$, and then use that $F_*((\IS^i,*),-)$ preserves filtered colimits by compactness.
\end{rem*}

Similarly to our construction of the embedding $\Sp(\Cc)\subseteq \cat{PSp}(\Cc)$, one can show that we have an adjunction
\begin{equation*}
	\snake{\Sigma}^\infty\colon {*/\Cc}\doublelrmorphism \cat{PSp}(\Cc)\noloc \ev_{(0,0)}\,.
\end{equation*}
if $\Cc$ has finite colimits. On objects $c\in */\Cc$ (which is all we need by \cref{cor:AdjointsPointwise}), we define $\snake{\Sigma}^\infty c$ as follows: Recall
\begin{equation*}
	\cat{PSp}(\Cc)\subseteq\limit_{n\in\IN^\op}\Fun\big(\IN^2_{\leq(n,n)},*/\Cc\big)\,,
\end{equation*}
so we may imagine $\snake{\Sigma}^\infty c$ as a sequence $((\snake{\Sigma}^\infty c)_0,(\snake{\Sigma}^\infty c)_1,(\snake{\Sigma}^\infty c)_2,\dotsc)$. The $n\ordinalth$ piece $(\snake{\Sigma}^\infty c)_n$ is then defined to be the left Kan extension of the of the following picture:
\begin{center}
	\begin{tikzpicture}[x=3em,y=3em,line cap=round]
		\node (00) at (0,0) {$c$};
		\node (10) at (1,0) {$*$};
		\node (20) at (2,0) {$\cdots$};
		\node (30) at (3,0) {$*$};
		\node (01) at (0,1) {$*$};
		\node (11) at (1,1) {$\phantom{W}$};
		\node (21) at (2,1) {$*$};
		\node[inner sep=0] (31) at (3,1) {$\raisebox{1.25ex}{\vdots}$};
		\node[inner sep=0] (02) at (0,2) {$\raisebox{1.25ex}{\vdots}$};
		\node (12) at (1,2) {$*$};
		\node (22) at (2,2) {$\phantom{W}$};
		\node (32) at (3,2) {$*$};
		\node (03) at (0,3) {$*$};
		\node (13) at (1,3) {$\cdots$};
		\node (23) at (2,3) {$*$};
		\node (33) at (3,3) {$\phantom{W}$};
		\draw[-to] (00) -- (01);
		\draw[-to] (01) -- (02);
		\draw[-to] (02) -- (03);
		\draw[-to] (10) -- (11);
		\draw[-to] (11) -- (12);
		\draw[-to] (12) -- (13);
		\draw[-to] (20) -- (21);
		\draw[-to] (21) -- (22);
		\draw[-to] (22) -- (23);
		\draw[-to] (30) -- (31);
		\draw[-to] (31) -- (32);
		\draw[-to] (32) -- (33);
		\draw[-to] (00) -- (10);
		\draw[-to] (10) -- (20);
		\draw[-to] (20) -- (30);
		\draw[-to] (01) -- (11);
		\draw[-to] (11) -- (21);
		\draw[-to,shorten >=1ex] (21) -- (31);
		\draw[-to,shorten <=1ex] (02) -- (12);
		\draw[-to] (12) -- (22);
		\draw[-to] (22) -- (32);
		\draw[-to] (03) -- (13);
		\draw[-to] (13) -- (23);
		\draw[-to] (23) -- (33);
		\draw[FabiansPurple!67!FabiansPink,dotted, rounded corners=2] (33) ++ (-1ex,-1ex) -- ++ (2ex,0) -- ++ (0,2ex) -- ++(-2ex,0) -- cycle;
		\draw[FabiansPurple!67!FabiansPink,dotted, rounded corners=2] (11) ++ (-1ex,-1ex) -- ++ (2ex,0) -- ++ (0,2ex) -- ++(-2ex,0) -- cycle;
		\draw[FabiansPurple!67!FabiansPink,dotted, rounded corners=2] (22) ++ (-1ex,-1ex) -- ++ (2ex,0) -- ++ (0,2ex) -- ++(-2ex,0) -- cycle;
		\path (0,-0.5) to node {$\underbrace{\hspace{10em}}$} node[below=0.5ex] {$\scriptstyle n+1$} (3,-0.5);
		\path (3.5,0) to node[sloped] {$\underbrace{\hspace{10em}}$} node[right=0.5ex] {$\scriptstyle n+1$} (3.5,3);
		\node[left] at (-0.25,1.5) {$\IN^2_{\leq(n,n)}$};
	\end{tikzpicture}
\end{center}
Again, note the blank spots along the diagonal, so the Kan extension is along the inclusion $\IN^2_{\leq (n,n)}\smallsetminus \{(1,1),\dotsc,(n,n)\}\subseteq \IN^2_{\leq (n,n)}$ this time.

Combining this with \cref{prop:Spectrification}, we see that $\Omega^\infty\colon\Sp\morphism {*/\An}$ has a left adjoint $\Sigma^\infty=(\snake{\Sigma}^\infty)^\mathrm{sp}\colon {*/\An}\morphism\Sp$. Plugging in the explicit formulas for Kan extension (\cref{thm:KanExtension}) to compute $\snake{\Sigma}^\infty$, we get
\begin{equation*}
	\Omega^\infty\Sigma^\infty(X,x)\simeq \colimit_{n\in \IN}\Omega^n\Sigma^n(X,x)
\end{equation*}
for all $(X,x)\in */\An$. The left-hand side is sometimes denoted $Q(X,x)$, but that has nothing to do with the Quillen $Q$-construction from page~\labelcpageref{par:QuillenQConstruction}).

We will denote the composite
\begin{equation*}
	\IS[-]\colon \An\xrightarrow{(-)_+}*/\An\morphism[\Sigma^\infty]\Sp\,.
\end{equation*}
In particular, we obtain $\IS\coloneqq\IS[*]\simeq \Sigma^\infty(\IS^0,*)$, the legendary \emph{sphere spectrum}. Plugging in the adjunctions we know gives $\Hom_\Sp(\IS,-)\simeq \Hom_\An(*,\mathop{\Omega^\infty}-)\simeq \Omega^\infty$, so $\IS$ represents the functor $\Omega^\infty\colon \Sp\morphism\An$. Using \cref{lem:piPreservesSequentialColimits}\itememph{b}, the homotopy groups of $\IS$ are given by
\begin{equation*}
	\pi_i(\IS)=\colimit_{n\in\IN} \pi_i\Omega^n\Sigma^n(\IS^0,*)=\colimit_{n\in\IN}\pi_{i+n}(\IS^n,*)\,,
\end{equation*}
i.e.\ the \emph{stable homotopy groups of spheres}. In contrast to the vastly complicated problem of computing these, the homology of $\IS$ (defined as in \cref{par:HomologyOfSpectra}) is rather boring:
\begin{smalllem}\label{lem:HomologyOfS}
	For any abelian group $A$, the homology of $\IS$ with coefficients in $A$ is given by
	\begin{equation*}
		H_i(\IS,A)=\begin{cases*}
			A & if $i=0$\\
			0 & else
		\end{cases*}\,.
	\end{equation*}
	More generally, we have $C_\bullet(\IS[X])\simeq C_\bullet(X)$ for all $X\in \An$ \embrace{where the $C_\bullet(-)$ on the left-hand side is the one from \cref{par:HomologyOfSpectra}, and the $C_\bullet(-)$ on the right-hand side was defined in Very Long Example~\textup{\labelcref{exm:EilenberMacLane}}}, so in particular,
	\begin{equation*}
		H_*\big(\IS[X],A\big)= H_*(X,A)\quad\text{and}\quad H^*\big(\IS[X],A\big)= H^{-*}(X,A)\,.
	\end{equation*}
\end{smalllem}
\begin{proof*}
	Fabian claimed the proof needs a bit of homotopy theory to compute $H_*(\Omega^n\IS^n,A)$ in low degrees, but I believe there is also a straightforward way that doesn't need anything. By construction (see \cref{par:HomologyOfSpectra}) we have
	\begin{equation*}
		C_\bullet\IS\simeq \colimit_{i\in\IN}\snake{C}_\bullet(\Omega^{\infty-i}\IS)[-i]\simeq \colimit_{i\in\IN}\colimit_{n\in\IN_{\geq i}}\snake{C}_\bullet(\Omega^{n-i}\IS^n)[-i]\,,
	\end{equation*}
	where we used that $\snake{C}_\bullet\colon */\An\morphism \Dd(\IZ)$ commutes with colimits. Moreover, the homology functors $H_*\colon \Dd(\IZ)\morphism \cat{Ab}$ commute with filtered colimits, hence
	\begin{align*}
		H_*(\IS,A)\simeq H_*\big(C_\bullet(\IS)\otimes_\IZ^LA\big)&\simeq \colimit_{i\in\IN}\colimit_{n\in\IN_{\geq i}}\snake{H}_{*+i}(\Omega^{n-i}\IS^n,A)\\
		&\simeq \colimit_{n\in\IN}\colimit_{i\in\IN_{\leq n}}\snake{H}_{*+i}(\Omega^{n-i}\IS^n,A)\\
		&\simeq \colimit_{n\in\IN}\snake{H}_{*+n}(\IS^n,A)\,,
	\end{align*}
	where we used that $i=n$ is terminal in $\IN_{\leq n}$ to get the third isomorphism. The bottom term clearly vanishes for $*\neq 0$ and equals $A$ for $*=0$, as we wished to show. The additional assertion follows from the fact that $C_\bullet(-)\colon \An\morphism \Dd(\IZ)$ and $C_\bullet(\IS[-])\colon \An\morphism \Dd(\IZ)$ both preserve colimits and agree on $*\in\An$ (as we just checked), hence they must be equal by \cref{thm:ColimitPreservingRepresentable}.
\end{proof*}

\begin{cor}[Baratt--Priddy--Quillen]\label{cor:BarattPriddyQuillen}
	The functor $\ev_1\colon \CGrp(\An)\morphism */\An$ has a left adjoint $\Free^\CGrp\colon {*/\An}\morphism \CGrp(\An)$ given by
	\begin{equation*}
		\Free^\CMon(X,x)^\inftygrp\simeq\Free^\CGrp(X,x)\simeq \Omega^\infty\Sigma^\infty(X,x)\,.
	\end{equation*}
	Here $\Omega^\infty\colon \Sp_{\geq 0}\isomorphism \CGrp(\An)$ denotes the inverse of $B^\infty\colon \CGrp(\An)\isomorphism \Sp_{\geq 0}$, which we recall is an equivalence by \cref{cor:RecognitionForInfiniteLoopSpaces}.
\end{cor}
\begin{proof*}
	We didn't spell this out in the lecture since there isn't much to do: It's evident that $\Sigma^\infty\colon {*/\An}\morphism \Sp$ lands in the full subcategory $\Sp_{\geq 0}$, hence it's also a left-adjoint to $\Omega^\infty\colon \Sp_{\geq 0}\morphism */\An$. After applying the other $\Omega^\infty$ (i.e.\ the inverse of $B^\infty$) we obtain the second equivalence. For the first equivalence, recall that \cref{cor:CommutativeHorizontalAdjoints} actually shows that the $\IE_1$-group completion of an $\IE_\infty$-monoid has a canonical $\IE_\infty$-group structure.
\end{proof*}
\label{page:TerrifyingExample}
Applying \cref{cor:BarattPriddyQuillen} to the unnumbered example on page~\labelcpageref{par:FreeCMon}, we see that
\begin{equation*}
	\SS^\inftygrp=\big(\{\text{finite sets, bijections}\},\sqcup\big)^\inftygrp\simeq \Free^\CMon(*)^\inftygrp\simeq \Omega^\infty\IS\,,
\end{equation*}
an example that Fabian already terrified us with in the introduction. In particular, we see that even though $\Free^\Grp(*)\simeq \IZ$ is a free group on a point and commutative, it fails rather drastically to be the free commutative group on a point.


\begin{proof}[Proof of \cref{prop:Spectrification}]
	By \cref{cor:AdjointsPointwise}, it suffices to check that $X^\mathrm{sp}$ as defined there is really a spectrum and a left-adjoint object to $X$ for all $X\in \cat{PSp}(\Cc)$. To check that $X^\mathrm{sp}$ is a spectrum, we compute
	\begin{equation*}
		\Omega X_{i+1}^\mathrm{sp}\simeq \Omega\colimit_{n\in \IN}\Omega^nX_{n+i+1}\simeq \colimit_{n\in \IN}\Omega^{n+1}X_{n+i+1}\simeq \colimit_{n\in\IN}\Omega^nX_{n+i}\simeq X_i^\mathrm{sp}
	\end{equation*}
	(and all of this is functorial) since $\Omega$ commutes with sequential colimits. To check that $X^\mathrm{sp}$ is really a left-adjoint object of $X$, we compute
	\begin{align*}
		\Hom_{\Sp(\Cc)}(X^\mathrm{sp},Y)&\simeq \limit_{i\in \IN^\op}\Hom_{*/\Cc}\left(X_i^\mathrm{sp},Y_i\right)\\
		&\simeq\limit_{i\in\IN^\op}\left(\colimit_{n\in \IN}\Omega^nX_{n+i},Y_i\right)\\
		&\simeq \limit_{i\in\IN^\op}\Hom_{*/\Cc}\bigg(\colimit_{n\in\IN_{\geq i}}\Omega^{n-i}X_n,\Omega^{n-i}Y_n\bigg)\\
		&\simeq \limit_{i\in\IN^\op}\limit_{n\in \IN^\op_{\geq i}}\Hom_{*/\Cc}\left(\Omega^{n-i} X_n,\Omega^{n-i}Y_n\right)\\
		\hphantom{\Hom_{\Sp(\Cc)}(X^\mathrm{sp},Y)}&\simeq \rlap{$\displaystyle\limit_{n\in \IN^\op}\limit_{j\in \IN_{\leq n}}\Hom_{*/\Cc}\left(\Omega^jX_n,\Omega^jY_n\right)$}\hphantom{\limit_{i\in\IN^\op}\Hom_{*/\Cc}\bigg(\colimit_{n\in\IN_{\geq i}}\Omega^{n-i}X_n,\Omega^{n-i}Y_n\bigg)}\\
		&\simeq \limit_{n\in \IN^\op}\Hom_{*/\Cc}(X_n,Y_n)\\
		&\simeq \Hom_{\cat{PSp}(\Cc)}(X,Y)\,.
	\end{align*}
	The first two equivalences are just plugging in definitions. For the third equivalence, we do an index shift by $-i$ and use $\Omega^{n-i}Y_n\simeq Y_i$ since $Y$ is a spectrum. The fourth equivalence is \cref{cor:HomPreservesColimits}. For the fifth equivalence, we substitute $j=n-i$ and use that limits commute (by the dual of \cref{prop:ColimitsCommute}). Since $j$ has \enquote{negative slope in $i$}, we really take the limit over $j\in \IN_{\leq n}$; the missing $(-)^\op$ is not a typo (and crucial for the next argument). The sixth equivalence now follows from the fact that $j=0$ is initial in $0\in \IN_{\leq n}$. Finally, the seventh equivalence is the formula for $\Hom_{\cat{PSp}(\Cc)}(X,Y)$ that we deduced just before the formulation of \cref{prop:Spectrification}. 
\end{proof}
Our next goal is to define a more \enquote{Segal style} model for $\Sp(\Cc)$. To do this, we construct another $\infty$-category $\IDigamma^\op$. Fabian remarks that this symbol is not the \texttt{\textbackslash mathbb}-version of the Roman letter F, but of the archaic Greek letter Digamma, which was the letter after Gamma and Delta in the ancient Greek alphabet, so $\IGamma^\op$, $\IDelta^\op$, and $\IDigamma^\op$ fit together nicely. \href{https://en.wikipedia.org/wiki/Digamma}{Wikipedia} claims, however, that Digamma was only the sixth letter and that Epsilon came between it and Delta. Nevertheless, I'll obediently stick to Fabian's notation.%I took the liberty though to typeset that symbol using \verb|\mathbb{F}| (I'm already going to \LaTeX\ hell for my $\IGamma$ being a \verb|\mathbb{L}| turned upside down \dotso).
\begin{defi}\label{def:OtherModelForSpectra}
	Let $\IDigamma^\op\subseteq */\An$ be the smallest sub-$\infty$-category containing $\IS^0$ and closed under finite colimits. There's a functor $\IGamma^\op\morphism\IDigamma^\op$ sending $\langle n\rangle$ to $\langle n\rangle_+$ (i.e.\ we add a basepoint to the discrete anima $\langle n\rangle$) and any partially defined map $\alpha\colon\langle m\rangle \morphism\langle n\rangle$ to the map $\alpha_+\colon \langle m\rangle_+\morphism\langle n\rangle_+$ which is $\alpha$ wherever it is defined and sends everything else to the basepoint.
	
	Finally, if $\Cc$ is an $\infty$-category with finite limits and (thus) a terminal object $*\in\Cc$, we define $\widetilde{\Sp}(\Cc)\subseteq \Fun(\IDigamma^\op,*/\Cc)$ to be the full subcategory of \emph{excisive} and \emph{reduced} functors, i.e.\ those taking pushouts to pullbacks and $*=\langle 0\rangle_+$ to $*\in */\Cc$. We might as well define $\widetilde{\Sp}(\Cc)$ as the full sub-$\infty$-category of $\Fun(\IDigamma^\op,\Cc)$ (rather than $\Fun(\IDigamma^\op,*/\Cc)$) spanned by reduced and excisive functors, since every reduced functor $F\colon \IDigamma^\op\morphism\Cc$ lifts canonically to $*/\Cc$.
\end{defi}
Note that $\widetilde{\Sp}(\Cc)$ is stable: It has finite limits inherited from $*/\Cc$ and $\Omega$ is an equivalence with inverse induced by precomposition with $\Sigma\colon \IDigamma^\op\morphism \IDigamma^\op$. Hence the criterion from \cref{propdef:StableInftyCategory}\itememph{a} is fulfilled.
\begin{prop}\label{prop:OtherModelForSpectra}
	With assumptions as in \cref{def:OtherModelForSpectra}, there is a natural equivalence of $\infty$-categories
	\begin{align*}
		\widetilde{\Sp}(\Cc)&\isomorphism \Sp(\Cc)\\
		F&\longmapsto \big(F(\IS^0),F(\IS^1),\dotsc\big)\,.
	\end{align*}
\end{prop}
\begin{proof}[Proof sketch]
	There are multiple ways to prove this. One can directly show that $\widetilde{\Sp}(\Cc)$ satisfies the universal property from \cref{thm:SpLeftAdjoint}. Or one can simply check that both
	\begin{equation*}
		\ev_{\IS^0}\colon \widetilde{\Sp}\big(\widetilde{\Sp}(\Cc)\big)\isomorphism \widetilde{\Sp}(\Cc)\quad\text{and}\quad\ev_{\IS^0,*}\colon \widetilde{\Sp}\big(\widetilde{\Sp}(\Cc)\big)\isomorphism\widetilde{\Sp}(\Cc) 
	\end{equation*}
	are equivalences and then apply the same reasoning (in particular, use Proposition~\labelcref{prop:LLaddendum}) as in the proof of that theorem.
	
	We will sketch a third way by constructing an inverse functor. For this, we use the following general construction: Let $\An^\mathrm{fin}\subseteq \An$ denote the smallest sub-$\infty$-category containing $*$ and closed under finite colimits. Let $\Dd$ be any $\infty$-category with finite limits and finite colimits. For $d\in \Dd$ we put
	\begin{equation*}
		X\otimes d\coloneqq \colimit_X\const d\quad\text{and}\quad d^X\coloneqq \limit_X\const d\,.
	\end{equation*}
	These exist by assumption and are functorial in $X$ and $d$. If $(X,x)\in \IDigamma^\op$ and $\Dd$ is \emph{pointed}, i.e.\ has a zero object, then we put
	\begin{equation*}
		(X,x)\otimes d\coloneqq \cofib\left(\colimit_{\{x\}}\const d\rightarrow \colimit_Xd\right)\,.
	\end{equation*}
	The inverse functor $\Sp(\Cc)\morphism\widetilde{\Sp}(\Cc)$ now sends a spectrum $E\in \Sp(\Cc)$ to the functor $\Omega^\infty(-\otimes E)\colon \IDigamma^\op\morphism */\Cc$. Note that $\Sp(\Cc)$ has finite colimits since it is stable, so the functor $-\otimes E\colon \IDigamma^\op\morphism \Sp(\Cc)$ is well-defined.
\end{proof}
Here's a proposition that was added in the $16\ordinalth$ lecture.
\refstepcounter{smallerdummy}
\numpar*{\thesmallerdummy. Proposition}\label{prop:OtherModelSpectrification}\itshape
In addition to the assumptions from \cref{def:OtherModelForSpectra}, suppose that $*/\Cc$ has sequential colimits and that $\Omega\colon {*/\Cc}\morphism*/\Cc$ commutes with them. Then the inclusion $\widetilde{\Sp}(\Cc)\subseteq\Fun_*(\IDigamma^\op,*/\Cc)$ into the reduced functors has a left adjoint, sending a reduced functor $F\colon \IDigamma^\op\morphism */\Cc$ to
\begin{equation*}
	F^\mathrm{sp}\simeq\colimit_{n\in\IN}\Omega^nF(\mathop{\Sigma^n}-)\,.
\end{equation*}
The transition maps in this colimit are induced by the natural transformations $F\Rightarrow \Omega F(\Sigma\,-)$, which exist since $F$ is reduced.
\upshape
\begin{proof*}[Proof sketch]
	This essential argument is copied from Fabian's notes, \cite[Chapter~II p.\:71]{KTheory}. As usual, by \cref{cor:AdjointsPointwise} all we need to do is to verify that the formula above indeed defines a left-adjoint object of $F$. If $E\colon \IDigamma^\op\morphism */\Cc$ is already reduced and excisive, then $E\simeq \Omega E(\Sigma\,-)$. Using this and a calculation as in the proof of \cref{prop:Spectrification}, one easily shows $\Nat(F,E)\simeq \Nat(F^\mathrm{sp},E)$.
	
	It remains to see, however, that $F^\mathrm{sp}$ is indeed excisive (and that's the hard part of the proof). Since $\Omega$ commutes with filtered colimits, we get $F^\mathrm{sp}\simeq \Omega F^\mathrm{sp}(\Sigma\,-)$. Now consider an arbitrary pushout square in $\IDigamma^\op$ and extend it to a diagram
	\begin{equation*}
		\begin{tikzcd}
			A\rar\dar\drar[pushout] & C\rar\dar\drar[pushout] & *\dar & \\
			B\rar\dar\drar[pushout] & D \rar\dar\drar[pushout] & Q \rar\dar\drar[pushout] & *\dar\\
			*\rar & P\rar\dar\drar[pushout] & \Sigma A\dar\rar\drar[pushout] & \Sigma B\dar\\
			& *\rar & \Sigma C\rar & \Sigma D
		\end{tikzcd}
	\end{equation*}
	The upper left $2\times 2$-square induces a map $F^\mathrm{sp}(B)\times_{F^\mathrm{sp}(D)}F^\mathrm{sp}(C)\morphism \Omega F^\mathrm{sp}(\Sigma A)\simeq F^\mathrm{sp}(A)$ and one can check that this is an inverse to the canonical map in the other direction. This shows that $F^\mathrm{sp}$ turns pushouts into pullbacks, as required.
\end{proof*}

The functors $-\otimes E$ and $E^{(-)}$ for spectra that were constructed in the proof of \cref{prop:OtherModelForSpectra} are interesting in their own right. Observe for example that
\begin{equation*}
	X\otimes HA\simeq H\big(C_\bullet(X)\otimes_\IZ^LA\big)
\end{equation*}
since one can (and we will) show both sides are colimit-preserving functors $\An\morphism\Sp$ with value $HA$ on $*\in \An$, hence they are equal by \cref{thm:ColimitPreservingRepresentable} and the fact that $\Sp$ is cocomplete. This motivates the following definition.
\begin{defi}\label{def:SpectraCohomologyTheory}
	Let $E\in \Sp$ and $X\in \An$. The \emph{homology} and \emph{cohomology theory} associated to $E$ are defined by
	\begin{equation*}
		E_*(X)=\pi_*(X\otimes E)=\pi_*\left(\colimit_XE\right)\quad\text{and}\quad
		E^*(X)=\pi_{*}(E^X)=\pi_{*}\left(\limit_XE\right)\,.
	\end{equation*}
\end{defi}
We'll check in \cref{cor*:SpectraCohomologyTheory} that $E_*(-)$ and $E^*(-)$ satisfy the Eilenberg--Steenrod axioms. It's also a common convention to define the cohomology theory associated to $E$ as $E^*(X)=\pi_{-*}(E^X)$ instead. However, we will see that Fabian's sign convention is superior.


\numpar*{Some Foreshadowing}
We can also extend the tensor product (or \emph{smash product}) $-\otimes -$ to all of $\Sp$ by defining
\begin{equation*}
	E\otimes E'\coloneqq\colimit_{i\in \IN}\big((\Omega^{\infty-i}E')\otimes E\big)[-i]
\end{equation*}
(where $[-i]=\Omega^i$ is the intrinsic loop functor on $\Sp$). It turns out that this defines a symmetric monoidal structure on $\Sp$ behaving like $-\otimes_\IZ^L-$ on $\Dd(\IZ)$, although not even symmetry is in any way obvious from the definition. Proving this will be the content of the next few lectures. Along the way (see \cref{par:SpTensorProduct}) we will also see that $X\otimes E$ and $E^X$ can be equivalently described as $\IS[X]\otimes E$ and $\hom_\Sp(\IS[X],E)$, respectively.

If $M$ is an $\IE_\infty$-monoid, it's usually very hard to understand its group completion $M^\inftygrp$. But we can understand $\IS[M^\inftygrp]\in \Sp$. This is still good enough to compute $H_*(M^\inftygrp,A)$ or even $E_*(M^\inftygrp)$, which will be the  content of the \enquote{group completion theorem}, \cref{thm:GroupCompletion}. Once we have this, we're in business to compute some $K$-groups!

\section{An Interlude on \texorpdfstring{$\infty$}{Infinity}-Operads}
\subsection{\texorpdfstring{$\infty$}{Infinity}-Operads and Symmetric Monoidal \texorpdfstring{$\infty$}{Infinity}-Categories}
\setcounter{dummy}{35}
\lecture[$\infty$-operads. Symmetric monoidal $\infty$-category as $\infty$-operads, strict vs.\ lax monoidal functors. Some examples. A technical lemma about presheaves.]{2020-12-17}
Recall that a morphism in $\IGamma^\op$ is \emph{inert} if it is a bijection (where it is defined) and \emph{active} if it is defined everywhere. This yields two subcategories $\IGamma^\op_{\mathrm{act}}$ and $\IGamma^\op_{\mathrm{int}}$. Recall also that symmetric monoidal $\infty$-categories are precisely the objects of $\CMon(\Cat_\infty)$, which is a full subcategory of $\Cocart(\IGamma^\op)$ via cocartesian unstraightening.
\begin{defi}\label{def:Operad}
	A \emph{\embrace{multicoloured and symmetric} $\infty$-operad \embrace{in anima}} is a functor $p\colon \Oo\morphism \IGamma^\op$ of $\infty$-categories such that the following conditions hold:
	\begin{alphanumerate}
		\item Every inert in $\IGamma^\op$ has cocartesian lifts with arbitrary sources (in the invariant sense of \cref{def:WeirdCocartesianDefinition}\itememph{b}).
		\item The functor $\Oo\colon\IGamma_\mathrm{int}^\op\morphism \Cat_\infty$ arising by cocartesian straightening satisfies the Segal condition. That is, $\Oo_0\simeq *$ and $\Oo_n\simeq \prod_{i=1}^n\Oo_1$ via the Segal maps $\rho_1,\dotsc,\rho_n\colon \langle n\rangle \morphism \langle 1\rangle$.
		\item Let $x,y\in \Oo$ with $p(x)=\langle m\rangle$ and $p(y)=\langle n\rangle$. By \itememph{b}, we can write $y\simeq (y_1,\dotsc,y_n)$ for some $y_i\in \Oo_1$. Then we require that the diagram
		\begin{equation*}
			\begin{tikzcd}[column sep=large,row sep=scriptsize]
				\Hom_\Oo(x,y)\rar["{(\rho_1,\dotsc,\rho_n)}"]\drar[pullback]\dar["p"'] & \displaystyle\prod_{i=1}^n\Hom_\Oo(x,y_i)\dar[shorten <=-2.5ex,shorten >=-2ex,"p"] \\
				\Hom_{\IGamma^\op}\big(\langle m\rangle,\langle n\rangle\big)\rar["{(\rho_1,\dotsc,\rho_n)}"]& \displaystyle\prod_{i=1}^n\Hom_{\IGamma^\op}\big(\langle m\rangle,\langle 1\rangle\big)
			\end{tikzcd}
		\end{equation*}
		is a pullback. Note that ince the bottom arrow is an injective map of discrete sets, the top arrow must be an inclusion of path components!
	\end{alphanumerate}
	We call a morphism $f\colon x\morphism y$ in $\Oo$ \emph{inert} if it is a $p$-cocartesian lift of an inert morphism in $\IGamma$. We call $f$ \emph{active} if it is a (not necessarily $p$-cocartesian) lift of an active morphism. We also call $\Oo_1$ the \emph{underlying $\infty$-category} of $\Oo$.
	
	Finally, the $\infty$-category $\cat{Op}_\infty$ of $\infty$-operads is the (non-full) sub-$\infty$-category of $\Cat_\infty/\IGamma^\op$ spanned by the $\infty$-operads and by functors which preserve inert morphisms.
\end{defi}
\numpar{First Steps with $\infty$-Operads}
Some trivial examples of $\infty$-operads are given by
\begin{equation*}
	\IGamma^\op_\mathrm{int}\morphism\IGamma^\op\,,\quad *\xrightarrow{\langle 0\rangle}\IGamma^\op\,,\quad\text{and}\quad \id\colon \IGamma^\op\morphism \IGamma^\op\,.
\end{equation*}
In general, a rich source of further examples is given by symmetric monoidal $\infty$-categories.
\refstepcounter{smallerdummy}
\numpar*{\thesmallerdummy. Symmetric Monoidal $\infty$-Categories as $\infty$-Operads}\label{par:SymmetricMonoidalAsOperads}
The cocartesian unstraightening induces a functor
\begin{equation*}
	(-)^\otimes=\Un^\cocart\colon \CMon(\Cat_\infty)\morphism \cat{Op}_\infty\,.
\end{equation*}
To see this, note that conditions \cref{def:Operad}\itememph{a} and \itememph{b} are trivially satisfied, so we only have to check \itememph{c}. We've already remarked that the bottom row of the diagram in question is an injection of discrete anima, hence we must show that the top row is an inclusion of the corresponding path components. So let $\alpha\colon \langle m\rangle \morphism\langle n\rangle $ be some morphism and denote by $\Hom_\Oo^\alpha(x,y)$ the path component over $\alpha$. Let $x\morphism \alpha_*(x)$ be a $p$-cocartesian lift of $\alpha$ and consider the diagram
\begin{equation*}
	\begin{tikzcd}[row sep=scriptsize]
		\Hom_{\Oo_n}(\alpha_*x,y)\rar[iso]\dar[iso] & \displaystyle\prod_{i=1}^n\Hom_{\Oo_1}\big((\alpha_*x)_i,y_i\big)\dar[shorten <=-2.5ex,shorten >=-2ex,iso] \\
		\Hom_{\Oo}^\alpha(x,y)\rar & \displaystyle\prod_{i=1}^n\Hom_{\Oo}^{\rho_i\circ\alpha}(x,y)
	\end{tikzcd}
\end{equation*}
That the vertical arrows are equivalences follows easily from the fact that $x\morphism \alpha_*x$ is $p$-cocartesian and the pullback diagram in \cref{def:WeirdCocartesianDefinition}\itememph{a}. The top arrow is an equivalence by the Segal condition. Hence the bottom arrow is an equivalence as well, which is what we wanted to show. Thus $(-)^\otimes$ has its image in contained in $\cat{Op}_\infty\subseteq \Cat_\infty/\IGamma^\op$, as claimed.

However, $(-)^\otimes$ is not fully faithful, and it's a good thing it isn't, since it allows us to define \emph{lax symmetric monoidal functors}. Morphisms in $\CMon(\Cat_\infty)$ are \emph{strongly} symmetric monoidal functor, i.e.\ they have to preserve the tensor product. In contrast to that, we define a \emph{lax symmetric monoidal functor} $\Cc\morphism\Dd$ to be a map of $\infty$-operads $\Cc^\otimes\morphism\Dd^\otimes$. We can now define $\infty$-categories $\cat{SymMonCat}_\infty\simeq \CMon(\Cat_\infty)$ and $\cat{SymMonCat}_\infty^\mathrm{lax}\subseteq \cat{Op}_\infty$ to be the essential image of $(-)^\otimes$ (with the convention that essential images are always full sub-$\infty$-categories).

A quick reality check: As the terminology suggests, a lax monoidal functor $F\colon \Cc^\otimes\morphism \Dd^\otimes$ comes with a natural transformation
\begin{equation*}
	F(-)\otimes_\Dd F(-)\Longrightarrow F(-\otimes_\Cc -)
\end{equation*}
in $\Fun(\Cc\times \Cc,\Dd)$. This is a consequence of the following random lemma* (which wasn't in the lecture).

\begin{lem*}\label{lem*:NotStraightening}
	Let $p\colon \Cc\morphism{} \Delta^1$ and $q\colon \Dd\morphism{} \Delta^1$ be cocartesian fibrations. Their straightenings correspond to functors $\St^\cocart(p)\simeq \alpha\colon \Cc_0\morphism \Cc_1$ and $\St^\cocart(q)\simeq \beta\colon \Dd_0\morphism \Dd_1$ in $\Cat_\infty$. Let furthermore functors $F_0\colon \Cc_0\morphism \Dd_0$ and $F_1\colon \Cc_1\morphism \Dd_1$ be given. Then there is a pullback diagram
	\begin{equation*}
		\begin{tikzcd}
			\Nat(\beta\circ F_0,F_1\circ \alpha)\rar\dar\drar[pullback] &\Hom_{\Cat_\infty/\Delta^1}(\Cc,\Dd)\dar\\
			*\rar["{(F_0,F_1)}"] &  \Hom_{\Cat_\infty}(\Cc_0,\Dd_0)\times \Hom_{\Cat_\infty}(\Cc_1,\Dd_1)
		\end{tikzcd}
	\end{equation*}
\end{lem*}

\begin{proof*}[Proof sketch]
	This should be a consequence of \cite{LurieGoodwillieCalculus}, but I think I can also give a direct (but terribly uninvariant) proof. A map $\Cc\morphism \Dd$ in $\Cat_\infty/\Delta^1$ is the same as a map $\Cc^\flat\morphism \Dd^\natural$ in $\sSet^+/(\Delta^1)^\sharp$, or more precisely,
	\begin{equation*}
		\Hom_{\Cat_\infty/\Delta^1}(\Cc,\Dd)\simeq \core \F^m_{(\Delta^1)^\sharp}(\Cc^\flat,\Dd^\natural)
	\end{equation*}
	is the core of the simplicial set of marked maps. The idea is now to construct a nice marked simplicial set which is marked equivalent to $\Cc^\flat$ in the cocartesian model structure on $\sSet^+/(\Delta^1)^\sharp$. Let
	\begin{equation*}
		\operatorname{Cyl}^\flat(\alpha)=\Cc_0^\flat\times(\Delta^1)^\flat\sqcup_{\Cc_0^\flat\times\{1\}}\Cc_1^\flat\times\{1\}
	\end{equation*}
	be the mapping cylinder of $\alpha$, and let $\operatorname{Cyl}^\natural(\alpha)$ be defined analogously, but with $(\Delta^1)^\sharp$ instead of $(\Delta^1)^\flat$. Then $\operatorname{Cyl}^\natural(\alpha)$ is marked equivalent to $\Cc^\natural$ (but not fibrant in the cocartesian model structure). Indeed, the straightening functor
	\begin{equation*}
		\St^+\colon \sSet^+/(\Delta^1)^\sharp\morphism \Fun^{\sSet}(\CC[\Delta^1],\sSet^+)
	\end{equation*}
	is a left adjoint, hence commutes with colimits, so $\St^+(\operatorname{Cyl}^\natural(\alpha))$ is the pushout (and homotopy pushout) of $(\emptyset\morphism \Cc_0)\Rightarrow (\id\colon \Cc_0\morphism \Cc_0)$ along $(\emptyset\morphism \Cc_0)\Rightarrow (\emptyset\morphism \Cc_1)$, and this pushout clearly agrees with $\St^+(\Cc^\natural)\simeq\alpha\colon \Cc_0\morphism \Cc_1$. Since $\St^+$ is a Quillen equivalence, this shows that $\operatorname{Cyl}^\natural(\alpha)$ and $\Cc^\natural$ are indeed marked equivalent. 
	
	This implies that $\operatorname{Cyl}^\flat(\alpha)$ and $\Cc^\flat$ are marked equivalent too. To make this precise, either use that the underlying simplicial set of $\St^+(X)$ only depends on that of $X$, and then analyse the markings on $\St^+(\Cc^\flat)$ and $\St^+(\operatorname{Cyl}^\flat(\alpha))$. Or use that $\operatorname{Cyl}^\natural(\alpha)\morphism \Cc^\natural$ is the inclusion of a fibrewise left deformation retraction (up to replacing it by a cofibration, see \cite[Lemma~X.37]{HigherCatsII}) and observe that in this particular case this stays true if we remove the markings of the non-equivalence edges.
	
	So we may replace $\Cc^\flat$ by $\operatorname{Cyl}^\flat(\alpha)$. Now consider the following inclusions of elements in $\sSet^+/(\Delta^1)^\sharp$, where marked edges are highlighted in yellow:
	\begin{center}
		\begin{tikzpicture}[x=0.8cm,y=1cm,line cap=round,decoration={markings,mark=at position 0.5 with {\arrow{to}}}]
			\begin{scope}
				\fill (0,0) circle (0.5ex) node[below left] {\scriptsize $\Cc_0$};
				\fill (1.5,0) circle (0.5ex) node[below right] {\scriptsize $\Cc_1$};
				\draw[postaction={decorate}] (0,0) to node[pos=0.5,below] {\scriptsize $\operatorname{Cyl}^\flat(\alpha)$} (1.5,0);
				\fill (0,-1.5) circle (0.5ex) node[below left] {\scriptsize $0$};
				\fill (1.5,-1.5) circle (0.5ex) node[below right] {\scriptsize $1$};
				\draw[postaction={decorate}] (0,-1.5) to node[pos=0.5,below] {\scriptsize $(\Delta^1)^\natural$} (1.5,-1.5);
				\draw[-to] (0.75,-0.65) to (0.75,-1.15);
			\end{scope}
			\node at (2.35cm,0) {$\subseteq$};
			\begin{scope}[xshift=3.5cm]
				\draw[line width=2.5ex,yellow!42!white] (0,0) to (1.5,1);
				\fill (0,0) circle (0.5ex) node[below left] {\scriptsize $\Cc_0$};
				\fill (1.5,0) circle (0.5ex) node[below right] {\scriptsize $\Cc_1$};
				\fill(1.5,1) circle (0.5ex) node[above right] {\scriptsize $\Cc_0$};
				\draw[postaction={decorate}] (0,0) to node[pos=0.5,below] {\scriptsize $\operatorname{Cyl}^\flat(\alpha)$} (1.5,0);
				\fill (0,-1.5) circle (0.5ex);
				\draw[postaction={decorate}] (0,0) to node[pos=0.5,above left] {\scriptsize $\Cc_0\times(\Delta^1)^\sharp$} (1.5,1);
				\fill (0,-1.5) circle (0.5ex) node[below left] {\scriptsize $0$};
				\fill (1.5,-1.5) circle (0.5ex) node[below right] {\scriptsize $1$};
				\draw[postaction={decorate}] (0,-1.5) to node[pos=0.5,below] {\scriptsize $(\Delta^1)^\natural$} (1.5,-1.5);
				\draw[-to] (0.75,-0.65) to (0.75,-1.15);
			\end{scope}
			\node at (5.85cm,0) {$\subseteq$};
			\begin{scope}[xshift=7cm]
				\draw[line width=2.5ex,yellow!42!white] (0,0) to (1.5,1);
				\fill (0,0) circle (0.5ex) node[below left] {\scriptsize $\Cc_0$};
				\fill (1.5,0) circle (0.5ex) node[below right] {\scriptsize $\Cc_1$};
				\fill(1.5,1) circle (0.5ex) node[above right] {\scriptsize $\Cc_0$};
				\draw[postaction={decorate}] (0,0) to node[pos=0.5,below] {\scriptsize $\operatorname{Cyl}^\flat(\alpha)$} (1.5,0);
				\draw[postaction={decorate}] (1.5,1) to node[pos=0.5,right] {\scriptsize $\operatorname{Cyl}^\flat(\alpha)$} (1.5,0);
				\fill (0,-1.5) circle (0.5ex);
				\draw[postaction={decorate}] (0,0) to node[pos=0.5,above left] {\scriptsize $\Cc_0\times(\Delta^1)^\sharp$} (1.5,1);
				\fill (0,-1.5) circle (0.5ex) node[below left] {\scriptsize $0$};
				\fill (1.5,-1.5) circle (0.5ex) node[below right] {\scriptsize $1$};
				\draw[postaction={decorate}] (0,-1.5) to node[pos=0.5,below] {\scriptsize $(\Delta^1)^\natural$} (1.5,-1.5);
				\draw[-to] (0.75,-0.65) to (0.75,-1.15);
				\node at (1.1,0.3) {$\scriptscriptstyle /\!/\!/$};
			\end{scope}
			\node at (9.35cm,0) {$\supseteq$};
			\begin{scope}[xshift=10.5cm]
				\draw[line width=2.5ex,yellow!42!white] (0,0) to (1.5,1);
				\fill (0,0) circle (0.5ex) node[below left] {\scriptsize 	$\Cc_0$};
				\fill (1.5,0) circle (0.5ex) node[below right] {\scriptsize 	$\Cc_1$};
				\fill(1.5,1) circle (0.5ex) node[above right] {\scriptsize 	$\Cc_0$};
				\draw[postaction={decorate}] (1.5,1) to node[pos=0.5,right] 	{\scriptsize $\operatorname{Cyl}^\flat(\alpha)$} (1.5,0);
				\fill (0,-1.5) circle (0.5ex);
				\draw[postaction={decorate}] (0,0) to node[pos=0.5,above left] 	{\scriptsize $\Cc_0\times(\Delta^1)^\sharp$} (1.5,1);
				\fill (0,-1.5) circle (0.5ex) node[below left] {\scriptsize $0$};
				\fill (1.5,-1.5) circle (0.5ex) node[below right] {\scriptsize 	$1$};
				\draw[postaction={decorate}] (0,-1.5) to node[pos=0.5,below] {\scriptsize $(\Delta^1)^\natural$} (1.5,-1.5);
				\draw[-to] (0.75,-0.65) to (0.75,-1.15);
			\end{scope}
		\end{tikzpicture}
	\end{center}
	All of these are marked equivalences. Hence we may further replace $\operatorname{Cyl}^\flat(\alpha)$ by the marked simplicial set on the right, which we denote $X$. The given map $F_0\colon \Cc_0\morphism \Dd_0$ extends uniquely (up to contractible choice) to a marked map $F\colon \Cc_0\times (\Delta^1)^\sharp\morphism \Dd^\natural$, which automatically satisfies $F_{|\Cc_0\times\{1\}}\simeq\beta\circ F_0$. Hence maps $\Cc^\flat\morphism \Dd^\natural$ with given values on $\Cc_0$ and $\Cc_1$ correspond to maps $X\morphism \Dd^\natural$ with given values on $\Cc_0\times(\Delta^1)^\sharp$ and $\Cc_1$, which in turn correspond to maps $\operatorname{Cyl}^\flat(\alpha)\morphism \Dd_1$ with given values on $\Cc_0\times\{0\}$ and $\Cc_1\times\{1\}$. If you think about this and make it a little more precise, you get the desired pullback diagram.
\end{proof*}
\refstepcounter{smallerdummy}
\numpar*{\thesmallerdummy.\ A Recognition Criterion}\label{par:RecognitionCriterion}
How can we decide whether a given $\infty$-operad is a symmetric monoidal $\infty$-category, i.e.\, in the essential image of $(-)^\otimes\colon \CMon(\Cat_\infty)\morphism\Op_\infty$?  We must check that $p\colon \Oo\morphism \IGamma^\op$ has cocartesian lifts of active maps. By the Segal condition and the fact that has $p$-cocartesian lifts for all inert morphisms, it suffices to check that it has $p$-cocartesian lifts for the unique active maps $f_n\colon \langle n\rangle \morphism \langle 1\rangle$ (we leave it as an exercise to work out a proper argument). This can in turn be reformulated as follows: For elements $(a_1,\dotsc,a_n)\in \Oo_n$ and $b\in \Oo_1$ define the \emph{multi-morphism space}
\begin{equation*}
	\begin{tikzcd}
		\Hom_\Oo^\mathrm{act}\big((a_1,\dotsc,a_n),b\big)\dar\drar[pullback]\rar & \Hom_\Oo\big((a_1,\dotsc,a_n),b\big)\dar\\
		*\rar["f_n"] & \Hom_{\IGamma^\op}\big(\langle n\rangle,\langle 1\rangle\big)
	\end{tikzcd}
\end{equation*}
If a morphism $g\colon (a_1,\dotsc,a_n)\morphism a$ over $f_n$ is $p$-cocartesian, then the precomposition $g^*\colon \Hom_{\Oo_1}(a,-)\overset{\sim}{\Longrightarrow} \Hom_\Oo^\mathrm{act}((a_1,\dotsc,a_n),-)$ is an equivalence of functors $\Oo_1\morphism \An$. The converse, however, is \emph{not quite} true in the sense that if $g^*$ is an equivalence, then $g$ is only \emph{locally} $p$-cocartesian by the criterion from \cite[Chapter~IX p.\:23]{HigherCatsII}. The way to think about this is that \enquote{$a\simeq a_1\otimes\dotsb\otimes a_n$}. So if $p$ is a locally cocartesian fibration, then these tensor products exist. But for them to be associative, we need (after some unravelling) that locally $p$-cocartesian lifts compose. By \cite[Proposition~IX.13]{HigherCatsII}, this is precisely the condition we need for the locally cocartesian fibration $p$ to be a cocartesian fibration! 

Summarising, we obtain the following criterion: 
\begin{alphanumerate}
	\item[\itememph{\boxtimes}] \itshape An $\infty$-operad $p\colon \Oo\morphism \IGamma^\op$ is a symmetric monoidal $\infty$-category iff the following two conditions hold:
	\begin{alphanumerate}
		\item For all $\langle n\rangle \in\IGamma^\op$ and all $(a_1,\dotsc,a_n)\in \Oo_n$ there exists an \enquote{$a\simeq a_1\otimes\dotsb\otimes a_n$} in $\Oo_1$ representing the multi-morphism space, i.e.\ there's a map $g\colon (a_1,\dotsc,a_n)\morphism a$ over $f_n\colon \langle n\rangle\morphism \langle 1\rangle$ such that
		\begin{equation*}
			g\colon \Hom_{\Oo_1}(a,-)\overset{\sim}{\Longrightarrow} \Hom_\Oo^\mathrm{act}\big((a_1,\dotsc,a_n),-\big)
		\end{equation*}
		is an equivalence.
		\item The \enquote{tensor product} from \itememph{a} is associative, i.e.
		\begin{equation*}
			a_1\otimes a_2\otimes a_3\simeq (a_1\otimes a_2)\otimes a_3\simeq a_1\otimes (a_2\otimes a_3)\,.
		\end{equation*}
	\end{alphanumerate}
\end{alphanumerate}

The moral of this story is that, intuitively, an $\infty$-operad $\Oo$ is the data you need to make an \enquote{$\infty$-category in which you specify what a map out of a tensor product should be}. Whether such a category exists should equivalent to whether $\Oo\simeq \Cc^\otimes$ for some $\Cc\in \CMon(\Cat_\infty)$.
\refstepcounter{smallerdummy}
\numpar*{\thesmallerdummy. Dual $\infty$-Operads}
There is an entirely dual theory of \emph{dual $\infty$-operads} (be aware that's not the same thing as $\infty$-cooperads; in fact, all four combinations of $(\emptyset/\text{dual})$ $\infty$-$(\emptyset/\text{co})$operads exist and are distinct things). A dual $\infty$-operad is a functor to $\IGamma=(\IGamma^\op)^\op$ satisfying \cref{def:Operad}\itememph{a}, \itememph{b}, and \itememph{c} for cartesian lifts. The resulting $\infty$-category $\cat{dOp}_\infty$ is equivalent to $\cat{Op}_\infty$ via $(-)^\op\colon \cat{Op}_\infty\isomorphism \cat{dOp}_\infty$. Note, however, that this doesn't preserve underlying $\infty$-categories. Instead it fits into a diagram
\begin{equation*}
	\begin{tikzcd}
		\cat{Op}_\infty\rar[iso]\dar["{(-)_1}"'] & \cat{dOp}_\infty\dar["{(-)_1}"]\\
		\Cat_\infty\rar["{(-)^\op}"] & \Cat_\infty
	\end{tikzcd}
\end{equation*}
The image of the cartesian unstraightening $\Un^\cart\colon \CMon(\Cat_\infty)\morphism\cat{dOp}_\infty$ is denoted $\cat{SymMonCat}_\infty^{\mathrm{oplax}}$, thought of as symmetric monoidal $\infty$-categories with oplax monoidal functors. Just as in \labelcref{par:SymmetricMonoidalAsOperads}, an oplax symmetric monoidal functor $F\colon \Cc^\otimes\morphism \Dd^\otimes$ comes with a natural transformation $F(-\otimes_\Cc -)\Rightarrow F(-)\otimes_\Dd F(-)$.
\refstepcounter{smallerdummy}
\numpar*{\thesmallerdummy. Sub-$\infty$-Operads}\label{par:LaxMonoidalAdjoints}
Let $\Oo\morphism\IGamma^\op$ be an $\infty$-operad and $\Dd\subseteq \Oo_1$ a full sub-$\infty$-category. Define $\Dd^\otimes\subseteq \Oo$ to be the full sub-$\infty$-category spanned by the $d\simeq(d_1,\dotsc,d_n)$ with $d_i\in \Dd$. Then $\Dd^\otimes\morphism\IGamma^\op$ is an $\infty$-operad with underlying $\infty$-category $\Dd$.

Since I didn't find this completely obvious that $\Dd^\otimes$ is an $\infty$-operad, here's a full argument. The conditions from \cref{def:Operad}\itememph{b} and \itememph{c} are immediately inherited from $\Oo$. To check the remaining condition \itememph{a}, we must check that whenever $d\morphism d'$ is a cocartesian lift of an inert morphism in $\IGamma^\op$ with $d\in \Dd^\otimes$, then also $d'\in \Dd^\otimes$. Say $d\in \Dd_n^\otimes$ and let $d\simeq (d_1,\dotsc,d_n)$. Then the components $d_i\in \Dd_1^\otimes$ are precisely the endpoints of cocartesian lifts $d\morphism d_i$ of $\rho_i\colon \langle n\rangle \morphism\langle 1\rangle$. Since cocartesian lifts compose (see \cite[Proposition~IX.5]{HigherCatsII}), we see that the components of $d'$ are a subset of the components of $d$, hence $d'\in \Dd^\otimes$ as well. This shows that $\Dd^\otimes$ is indeed an $\infty$-operad.

In the case where $\Oo\simeq \Cc^\otimes$ is a symetric monoidal $\infty$-category, it's naturally to ask whether $\Dd^\otimes$ is one as well. Fabian gave two criteria for this in the lecture:
\begin{alphanumerate}\itshape 
	\item Suppose that $d\otimes d'\in \Dd$ for all $d,d'\in \Dd$ and $1_\Cc\in \Dd$. Then $\Dd^\otimes$ is a symmetric monoidal $\infty$-category and the inclusion $\Dd\subseteq \Cc$ is strongly monoidal. If it has, moreover, a right adjoint $R\colon \Cc\morphism\Dd$, then $R$ can be extended to a map of $\infty$-operads $R^\otimes\colon \Cc^\otimes\morphism\Dd^\otimes$ such that $R^\otimes$ is right-adjoint to the inclusion $\Dd^\otimes\subseteq \Cc^\otimes$. In particular, $R$ has a canonical lax symmetric monoidal refinement.
	\item Suppose that $\Dd\subseteq \Cc$ has a left adjoint $L\colon \Cc\morphism\Dd$ with the property that whenever $L(f)\colon L(x)\isomorphism L(y)$ is an equivalence, then $L(f\otimes \id_z)\colon L(x\otimes z)\isomorphism L(y\otimes z)$ is an equivalence too for all $z\in\Cc$. Then $\Dd^\otimes$ is symmetric monoidal and $L$ extends to a map $L^\otimes\colon \Cc^\otimes\morphism\Dd^\otimes$ of $\infty$-operads such that $L^\otimes$ is left-adjoint to $\Dd^\otimes\subseteq \Cc^\otimes$. Moreover, $L$ is strongly symmetric monoidal and $\Dd\subseteq \Cc$ is lax symmetric monoidal.
\end{alphanumerate}
I'd like to add the following closely related assertion, which wasn't mentioned in the lecture, although we'll need it later. It roughly says that whether a left adjoint is strongly monoidal can be checked on objects.
\begin{alphanumerate}
	\item[\itememph{c^*}]\itshape Let $R^\otimes\colon \Dd^\otimes\morphism\Cc^\otimes$ be a map of $\infty$-operads . If the underlying functor $R\colon \Dd\morphism\Cc$ has a left adjoint $L\colon \Cc\morphism\Dd$ satisfying
	\begin{equation*}
		L(1_\Cc)\simeq 1_\Dd\quad\text{and}\quad L(c_1\otimes_\Cc c_2)\simeq L(c_1)\otimes_\Dd L(c_2)
	\end{equation*}
	for all $c_1,c_2\in \Cc$, then $L$ extends to a map $L^\otimes\colon \Cc^\otimes\morphism \Dd^\otimes$ of $\infty$-operads, which is left-adjoint to $R^\otimes$ and strongly symmetric monoidal.
\end{alphanumerate}
The proofs of \itememph{a}, \itememph{b}, and \itememph{c^*} are pretty straightforward, but a bit tedious.
\begin{proof*}[Proof of \itememph{a}]
	The fact that $\Dd^\otimes$ is a symmetric monoidal $\infty$-category and that $\Dd\subseteq\Cc$ is a strongly monoidal functor (i.e.\ $\Dd^\otimes\subseteq \Cc^\otimes$ preserves all cocartesian edges) follows from the recognition criterion sketched in \labelcref{par:RecognitionCriterion}. To construct $R^\otimes$, we employ \cref{cor:AdjointsPointwise} as usual to see that $R^\otimes$ can be constructed pointwise. So let $c\in \Cc^\otimes$ and write $c=(c_1,\dotsc,c_n)$. We put $R^\otimes (c)=(R(c_1),\dotsc,R(c_n))$ and claim that the natural map $\eta_c\colon c\morphism R^ \otimes (c)$ (induced by the unit maps $\eta_{c_i}\colon c_i\morphism R(c_i)$ for $R$) witnesses it as a right-adjoint object of $c$. That is, we must show that 
	\begin{equation*}
		\eta_{c,*}\colon \Hom_{\Cc^\otimes}(d,c)\isomorphism \Hom_{\Dd^\otimes}\big(d,R^\otimes(c)\big)
	\end{equation*}
	is an equivalence for all $d\in \Dd^\otimes$. By \cref{def:Operad}\itememph{c} it suffices to show the same for $\eta_{c_i,*}\colon \Hom_{\Cc^\otimes}(d,c_i)\isomorphism\Hom_{\Dd^\otimes}(d,R(c_i))$. But whether this is an equivalence of anima can be checked on (derived) fibres over $\Hom_{\IGamma^\op}(\langle n\rangle,\langle 1\rangle)$. Given $\alpha\colon \langle n\rangle \morphism\langle 1\rangle$ in $\IGamma^\op$, choose a cocartesian lift $d\morphism \alpha_*d$ to $\Dd^\otimes$, which is also cocartesian in $\Cc^\otimes$ by construction. Then the fibres over $\alpha$ are given by $\Hom_{\Cc^\otimes}^\alpha(d,c_i)\simeq \Hom_\Cc(\alpha_*d,c_i)$ and $\Hom_{\Dd^\otimes}^\alpha(d,R(c_i))\simeq \Hom_\Dd(\alpha_*d,c_i)$, hence $\eta_{c_i,*}$ is indeed an equivalence since $R\colon \Cc\morphism\Dd$ is a right adjoint of the inclusion $\Dd\subseteq\Cc$. This shows that $R^\otimes$ exist as a map in $\Cat_\infty$. It's clear from the construction that it's also a map in $\Cat_\infty/\IGamma^\op$. Finally, since cocartesian lifts of inert morphisms just \enquote{forget} some of the factors, they are preserved by $R^\otimes$, whence it is a map of $\infty$-operads. This shows \itememph{a}.
\end{proof*}
\begin{proof*}[Proof of \itememph{b}]
	First observe that the condition on $L$ can be strengthened as follows: If we are given equivalences $L(f_i)\colon L(x_i)\isomorphism L(y_i)$ in $\Dd$ for all $i=1,\dotsc,n$, then also
	\begin{equation*}
		L(f_1\otimes\dotsb\otimes f_n)\colon L(x_1\otimes\dotsb\otimes x_n)\isomorphism L(y_1\otimes\dotsb\otimes y_n)
	\end{equation*}
	is an equivalence. To show that $\Dd^\otimes$ is a symmetric monoidal $\infty$-category, we use the recognition criterion from \labelcref{par:RecognitionCriterion}: Given $(d_1,\dotsc,d_n)\in\Dd_n^\otimes$ and any $d\in \Dd$, we have
	\begin{align*}
		\Hom_{\Dd^\otimes}^\mathrm{act}\big((d_1,\dotsc,d_n),d\big)\simeq \Hom_{\Cc^\otimes}^\mathrm{act}\big((d_1,\dotsc,d_n),d\big)
		&\simeq \Hom_\Cc(d_1\otimes\dotsb\otimes d_n,d)\\
		&\simeq \Hom_\Dd\big(L(d_1\otimes\dotsb\otimes d_n),d\big)\,,
	\end{align*}
	hence composing $(d_1,\dotsc,d_n)\morphism d_1\otimes\dotsb\otimes d_n$ with the unit map $ d_1\otimes\dotsb\otimes d_n\morphism L(d_1\otimes\dotsb\otimes d_n)$ in $\Cc^\otimes$ gives a locally cocartesian lift of $f_n\colon \langle n\rangle \morphism\langle 1\rangle$ to $\Dd^\otimes$, which verifies the first condition of the recognition criterion. For the second condition, we must check $L(L(x\otimes y)\otimes z\big)\simeq L(x\otimes y\otimes z)$. But $x\otimes y\morphism L(x\otimes y)$ becomes an equivalence after applying $L$, as $L^2\simeq L$, hence this follows from the assumption on $L$. This proves that $\Dd^\otimes$ is symmetric monoidal.
	
	To construct $L^\otimes$, we proceed as in \itememph{a} and define it pointwise via $L^\otimes(c_1,\dotsc,c_n)=(L(c_1),\dotsc,L(c_n))$. By the same tricks as above, showing that this is indeed a right-adjoint object of $(c_1,\dotsc,c_n)$ reduces to showing
	\begin{equation*}
		\Hom_{\Cc^\otimes}^\alpha\big((c_1,\dotsc,c_n),d\big)\simeq \Hom_{\Dd^\otimes}^\alpha\big((L(c_1),\dotsc,L(c_n)),d\big)
	\end{equation*}
	for all $d\in \Dd$ and all $\alpha\colon \langle n\rangle \morphism\langle 1\rangle$. Let $(c_1,\dotsc,c_n)\morphism c$ be a cocartesian lift of $\alpha$ to $\Cc^\otimes$. Then $\Hom_{\Cc^\otimes}^\alpha((c_1,\dotsc,c_n),d)\simeq\Hom_\Cc(c,d)$ and also $c\simeq c_{i_1}\otimes\dotsb\otimes c_{i_m}$, where $i_1,\dotsc,i_m\in\langle n\rangle$ are the elements where $\alpha$ is defined. The assumption on $L$ then implies
	\begin{equation*}
		L(c)\simeq L\big(L(c_{i_1})\otimes\dotsb\otimes L(c_{i_m})\big)\,,
	\end{equation*}
	because each $c_{i_j}\morphism L(c_{i_j})$ becomes an equivalence after applying $L$. Hence a cocartesian lift of $\alpha$ to $\Dd^\otimes$ is given by $(L(c_1),\dotsc,L(c_n))\morphism L(c)$ and $\Hom_{\Dd^\otimes}^\alpha((L(c_1),\dotsc,L(c_n)),d)\simeq \Hom_{\Dd}(L(c),d)$. Now $\Hom_\Cc(c,d)\simeq \Hom_\Dd(L(c),d)$ because $L\colon \Cc\morphism\Dd$ is left-adjoint to the inclusion $\Dd\subseteq\Cc$, which finally shows the desired equivalence. It's clear from the construction that $L^\otimes$ preserves cocartesian edges, whence $L$ is indeed strongly symmetric monoidal.
\end{proof*}
\begin{proof}[Proof of \itememph{c^*}]
	The condition on $L$ implies $L(c_1\otimes_\Cc\dotsb\otimes_\Cc c_n)\simeq L(c_1)\otimes_\Dd\dotsb\otimes_\Dd L(c_n)$ for all $c_1,\dotsc,c_n\in\Cc$. We define $L^\otimes$ pointwise as in \itememph{b}. By the same arguments as given there, all we need to check is that
	\begin{equation*}
		\Hom_{\Cc^\otimes}^\alpha\big((c_1,\dotsc,c_n),R(d)\big)\simeq \Hom_{\Dd^\otimes}^\alpha\big((L(c_1),\dotsc,L(c_n)),d\big)
	\end{equation*}
	for all $c_1,\dotsc,c_n\in\Cc$, $d\in \Dd$ and all $\alpha\colon \langle n\rangle \morphism\langle 1\rangle$. If $\alpha$ is defined at $i_1,\dotsc,i_m\in\langle n\rangle$, then $(c_1,\dotsc,c_n)\morphism c_{i_1}\otimes_\Cc\dotsb\otimes_\Cc c_{i_m}$ is a cocartesian lift of $\alpha$ to $\Cc^\otimes$, hence
	\begin{equation*}
		\Hom_{\Cc^\otimes}^\alpha\big((c_1,\dotsc,c_n),R(d)\big)\simeq \Hom_\Cc\big(c_{i_1}\otimes_\Cc\dotsb\otimes_\Cc c_{i_m},R(d)\big)\,.
	\end{equation*}
	The same argument for $\Dd^\otimes$ together with the condition on $L$  implies
	\begin{equation*}
		\Hom_{\Dd^\otimes}^\alpha\big((L(c_1),\dotsc,L(c_n)),d\big)\simeq \Hom_\Dd\big(L(c_1\otimes_\Cc\dotsb\otimes_\Cc c_n),d\big)\,,
	\end{equation*}
	whence we are done by the fact that $L$ and $R$ are adjoints. This shows that $L^\otimes$ is an adjoint of $R^\otimes$, as desired. It's straightforward to check that $L^\otimes$ is a map over $\IGamma^\op$ and preserves cocartesian edges, hence $L$ is indeed strongly symmetric monoidal.
\end{proof}
Fabian also mentioned that in general, there is an equivalence of $\infty$-categories
\begin{equation*}
	\left\{\text{oplax monoidal left adjoints $\Cc\rightarrow\Dd$}\right\}\simeq \left\{\text{lax monoidal right adjoints $\Dd\rightarrow\Cc$}\right\}\,.
\end{equation*}
This is not trivial since lax and oplax structures are defined in different fibration pictures. The first written proofs appeared in November 2020 by Rune Haugseng \cite{HaugsengLaxOplax} and independently by Fabian, Sil, and Joost Nuiten \cite{FabianSilLaxOplax}.
\refstepcounter{smallerdummy}
\numpar*{\thesmallerdummy. Derived Tensor Products}\label{par:DerivedTensorProduct}
Let's apply \labelcref{par:LaxMonoidalAdjoints}\itememph{a} to get a symmetric monoidal structure on the derived category $\Dd(R)$ of a commutative ring $R$. One can check that the Kan-enriched category $\Ch(R)$ from \cref{exm:MyFirstInftyCats}\itememph{e} is symmetric monoidal under $-\otimes_R-$ as a Kan-enriched category. Hence its coherent nerve $\Kk(R)$ is a symmetric monoidal $\infty$-category. Now recall from \cref{exm:MyFirstBousfield}\itememph{c} that $\Dd(R)$ is a Bousfield localisation of $\Kk(R)$. More precisely, $\Dd(R)$ can be embedded fully faithfully into $\Kk(R)$ as the full sub-$\infty$-category $\Dd(R) ^{K\mhyph\mathrm{proj}}\subseteq \Kk(R)$ of $K$-projective complexes, i.e.\ those $C$ such that $\Hom_{\Kk(R)}(C,-)$ inverts quasi-isomorphisms. Clearly the tensor unit $R[0]$ is $K$-projective (as is any bounded below degree-wise projective complex) and if $C$ and $D$ are $K$-projective, then so is $C\otimes_RD$ as $\Hom_{\Kk(R)}(C\otimes_RD,-)\simeq \Hom_{\Kk(R)}(C,\Hhom_R(D,-))$ inverts quasi-isomorphisms too.

We may thus apply \labelcref{par:LaxMonoidalAdjoints}\itememph{a} to obtain a symmetric monoidal structure $-\otimes_R^L-$ on $\Dd(R)$. The localisation functor $\Kk(R)\morphism\Dd(R)$ (\enquote{taking $K$-projective resolutions}) is lax symmetric monoidal in the sense explained in \labelcref{par:SymmetricMonoidalAsOperads}. In particular, there are canonical maps $C\otimes_R^LD\morphism C\otimes_RD$ for all $C,D\in \Kk(R)$.

		% which is a symmetric monoidal category iff $d\otimes d'\in \Dd$ for all $d,d'\in \Dd$ and $1_\Cc\in \Dd$. In this case $\Dd\subseteq \Cc$ is strongly monoidal. Moreover, if it has a right adjoint $R\colon \Cc\morphism\Dd$, then this extends to a map $R^\otimes \colon \Cc^\otimes\morphism\Dd^\otimes$ of operads (i.e.\ a lax monoidal functor) which is right-adjoint to $\Dd^\otimes\subseteq\Cc^\otimes$.
		
		%The further claim about when it is a symmetric monoidal category follows from the recognition criterion sketched in \itememph{a}. Finally, the fact that $R^\otimes$ is right adjoint to the inclusion (if it exists) is clear since the unit and counit transformations along with the triangle identities are simply inherited via $(-)^\otimes$.
		
		%
		%\item There's another situation in which $\Dd^\otimes$ from \itememph{d} happens to be a symmetric monoidal $\infty$-category itself. Suppose $\Cc$ is a symmetric monoidal $\infty$-category and $\Dd\subseteq \Cc$ a full sub-$\infty$-category such that the inclusion has a left adjoint $L\colon \Cc\morphism\Dd$ (i.e.\ $\Dd$ is a Bousfield localisation of $\Cc$). If $L$ has the property that whenever $L(f)\colon L(x)\isomorphism L(y)$ is an equivalence, then $L(f\otimes \id_z)\colon L(x\otimes z)\isomorphism L(y\otimes z)$ is an equivalence too for all $z\in\Cc$, then the operad $\Dd^\otimes$ from \itememph{d} is a symmetric monoidal $\infty$-category and the inclusion $\Dd\subseteq\Cc$ is lax symmetric monoidal, whereas $L\colon \Cc\morphism\Dd$ is strongly symmetric monoidal.
		%
		%Indeed, we can argue as in \itememph{d} to deduce that $L^\otimes\colon \Cc^\otimes\morphism \Dd^\otimes$ is again a left adjoint of $\Dd^\otimes\subseteq\Cc^\otimes$. This implies $\Hom_{\Dd^\otimes}^\mathrm{act}(L^{\otimes}(c_1,\dotsc,c_n),d)\simeq \Hom_{\Cc^\otimes}^\mathrm{act}((c_1,\dotsc,c_n),d)$, so if $g\colon (c_1,\dotsc,c_n)\morphism c$ is locally $p$-cocartesian, then so is $L^\otimes (c_1,\dotsc,c_n)\morphism L(c)$. This implies that $\Dd^\otimes\morphism\IGamma^\op$ is locally $p$-cocartesian as well. To see that locally $p$-cocartesian edges compose, one uses that this is true for $\Cc^\otimes\morphism \IGamma^\op$ as well as the assumptions on $L$.
\subsection{The Cocartesian and Cartesian Symmetric Monoidal Structures}
If an $\infty$-category $\Cc$ has finite products (and hence, in particular, a terminal object), then taking products should define a symmetric monoidal structure on $\Cc$, called the \emph{cartesian monoidal structure}. Similarly, on an $\infty$-category with finite coproducts there should be a \emph{cocartesian monoidal structure}. Constructing these guys will be our next immediate goal. Of course, all of this is part of our greater plan to construct the symmetric monoidal structure on $\Sp$.

While Lurie writes down explicit simplicial sets to construct the desired $\infty$-operads, we will cheat a bit and use complete Segal spaces instead. That this works is based on the following lemma.
\begin{lem}\label{lem:TautologicalPresheafLemma}
	Let $\Ee$ be an $\infty$-category. For a presheaf $X\in \Pp(\Ee)$, consider the slice category $\Ee/X$ with respect to the Yoneda embedding $\Yo^\Ee\colon \Ee\morphism P(\Ee)$ and $\{X\}\morphism \Pp(\Ee)$ \embrace{in the notation of \cref{lem*:PresheafColimitOfRepresentables} this would have been $\Yo^\Ee/X$ instead}. The canonical functor
	\begin{equation*}
		\Pp(\Ee/X)\isomorphism \Pp(\Ee)/X\,,
	\end{equation*}
	constructed as the unique colimit-preserving extension of $\Yo^\Ee\colon\Ee/X\morphism \Pp(\Ee)/X$ via \cref{thm:ColimitPreservingRepresentable}, is an equivalence of $\infty$-categories.
\end{lem}
\begin{proof}%[Proof of \cref{lem:TautologicalPresheafLemma}]
	Before we start, observe that $\Pp(\Ee)/X$ is indeed cocomplete since one immediately checks that it inherits colimits from $\Pp(\Ee)$. We will use the following criterion:
	\begin{alphanumerate}
		\item[\itememph{\boxtimes}]\itshape\label{claim:EquivalenceCondition} Let $F\colon \Cc\morphism\Dd$ be a functor from a small $\infty$-category to a cocomplete $\infty$-category. Let $|\blank|_F\colon \Pp(\Cc)\shortdoublelrmorphism \Dd\noloc \Sing_F$ denote the adjunction resulting from \cref{thm:ColimitPreservingRepresentable}. Then $|\blank|_F$ is an equivalence if the following three conditions hold:
		\begin{alphanumerate}
			\item $F$ is fully faithful.
			\item $\Hom_\Dd(F(c),-)\colon\Dd\morphism\An$ commutes with colimits for all $c\in \Cc$.
			\item $\Sing_F\colon \Dd\morphism \Pp(\Cc)$ is conservative.
		\end{alphanumerate}
	\end{alphanumerate}
	We prove \itememph{\boxtimes} first. Condition~\itememph{b} implies that $\Sing_F$ commutes with colimits. Indeed, this can be checked pointwise by \cref{thm:JoyalEquivalence}\itememph{b} since there always is a natural transformation $\colimit_\Ii\Sing_F\Rightarrow\Sing_F\colimit_\Ii$. Moreover, colimits in $\Pp(\Cc)$ are computed pointwise. Plugging in the explicit description of $\Sing_F$ that \cref{cor:ExplicitAdjoint} offers us, we must thus check that $\colimit_\Ii\Hom_\Dd(F(c),d_i)\simeq \Hom_\Dd(F(c),\colimit_\Ii d_i)$ for all $c\in \Cc$ and $d_i\in \Dd$. This is \itememph{b}.
	
	Now since $F$ is fully faithful, the unit $\id_{\Pp(\Cc)}\Rightarrow\Sing_F|\blank|_F$ is an equivalence on representables. Indeed, the right-hand side sends $c\in \Cc$ to $\Hom_\Dd(F\,-,F(c))\simeq \Hom_\Cc(-,c)$ by \cref{cor:ExplicitAdjoint} again. Now both sides preserve colimits and every presheaf on $\Cc$ can be written as a colimit of representatives by \cref{lem*:PresheafColimitOfRepresentables}, hence the unit is an equivalence everywhere. By the triangle identities and the fact that $\Sing_F$ is conservative by \itememph{c}, this implies that the counit $|\Sing_F-|_F\Rightarrow \id_\Dd$ is an equivalence as well. This finishes the proof of \itememph{\boxtimes}.
	
	Now to check that \itememph{\boxtimes} is applicable for $\Cc=\Ee/X$ and $\Dd=\Pp(\Ee)/X$. We already argued that $\Pp(\Ee)/X$ is cocomplete. It's easily checked that $\Ee/X\morphism\Pp(\Ee)/X$ is fully faithful since it's induced by the fully faithful Yoneda embedding, so \itememph{a} holds. To check \itememph{c}, let $\alpha\colon F\morphism F'$ be a morphism in $\Pp(\Ee)/X$ such that $\alpha_*\colon \Hom_{\Pp(\Ee/X)}(-,F)\morphism \Hom_{\Pp(\Ee/X)}(-,F')$ is an equivalence on representable presheaves. We must check that $\alpha$ is an equivalence itself. But if $\alpha_*$ is an equivalence on representables, then the Yoneda lemma implies that $\alpha\colon F(e)\morphism F'(e)$ is an equivalence for all $e\in \Ee$ (except those $e\in \Ee$ for which $X(e)=\emptyset$ since these aren't contained in the image of $\Ee/X\morphism \Ee$, but in that case we have $F(e)=\emptyset=F'(e)$ too since $F$ and $F'$ come with a map to $X$), hence $\alpha$ is an equivalence of presheaves.
	
	Finally, we have to check \itememph{b}. Given a representable object $\Hom_\Ee(e,-)\morphism X$ and an arbitrary object $F\morphism X$ of $\Pp(\Ee)/X$, the Yoneda lemma and \cite[Corollary~VIII.6]{HigherCatsII} provide a pullback square
	\begin{equation*}
		\begin{tikzcd}
			\Hom_{\Pp(\Ee)/X}\big(\Hom_\Ee(e,-),F\big)\rar\dar\drar[pullback]& \Hom_{\Pp(\Ee)}\big(\Hom_\Ee(e,-),F\big)\rar[iso] \dar & F(e)\dar\\
			*\rar & \Hom_{\Pp(\Ee)}\big(\Hom_\Ee(e,-),X\big)\rar[iso]& X(e)
		\end{tikzcd}
	\end{equation*}
	Clearly $F(e)$ commutes with colimits in $F$ because colimits in functor categories are computed pointwise (\cref{lem:f^*preservesColimits}). Hence it suffices to check that colimits in $\An$ commute with pullbacks. In other words, if $f\colon K\morphism L$ is any morphism of anima, then $f^*\colon \An/L\morphism \An/K$ preserves colimits. But $\Right(K)\simeq \An/K$ since every morphism to $K$ can be factored into a right anodyne and a right fibration, and likewise for $L$, so we are done since we verified in \cref{lem*:SmoothBaseChange} that $f^*\colon \Right(L)\morphism \Right(K)$ has a right adjoint $f_*$.
\end{proof}\refstepcounter{smallerdummy}
\numpar*{\thesmallerdummy}\label{par:PresheafConstruction}
Applying \cref{lem:TautologicalPresheafLemma} to $\Ee=\IDelta$ and $X=\N^r(\Cc)\in \cat{sAn}$ for some $\infty$-category $\Cc$, we obtain $\Pp(\IDelta/\N^r(\Cc))\simeq \cat{sAn}/\N^r(\Cc)$. We will usually abbreviate the left-hand side as $\Pp(\IDelta/\Cc)$.  If you think about it, this makes a lot of sense: Objects of $\IDelta/\N^r(\Cc)$ are maps $\Delta^n\morphism \N^r(\Cc)$ of simplicial anima. But $\N^r\colon \Cat_\infty\morphism\cat{sAn}$ is fully faithful by \cref{thmdef:RezkNerve}, so we may equivalently think of maps $[n]\morphism \Cc$. 

We can now specify objects of (or functors to) $\Cat_\infty/\Cc$ as objects of (or functors to) $\Pp(\IDelta/\Cc)$ and then compose with 
\begin{equation*}
	\Pp(\IDelta/\Cc)\simeq \cat{sAn}/\N^r(\Cc)\xrightarrow{\asscat}\Cat_\infty/\Cc\,.
\end{equation*}
This will be done quite a number of times on the next few pages.\refstepcounter{smallerdummy}

\numpar*{\thesmallerdummy}
\lecture[Cocartesian and cartesian monoidal structures. Algebras over $\infty$-operads. Day convolution. Operadic limits and colimits. The tensor product on spectra.\newline
--- \emph{\enquote{I didn't feel like presenting this as a gift to you. It's more like a punishment for Christmas.}}]{2020-12-22}
To check that an object on the left is complete Segal, we will check the \emph{relative Segal condition}. A map $Y\morphism X$ in $\cat{sAn}$ is called \emph{relative Segal} if the diagram
\begin{equation*}
	\begin{tikzcd}
		\Hom_{\cat{sAn}}(\Delta^n,Y)\rar\dar\drar[pullback] & \Hom_{\cat{sAn}}(I^n,Y)\dar\\
		\Hom_{\cat{sAn}}(\Delta^n,X)\rar & \Hom_{\cat{sAn}}(I^n,X)
	\end{tikzcd}
\end{equation*}
is a pullback in $\An$. If $X$ is a Segal anima, then the lower row is an equivalence. Thus $Y$ is a Segal anima too if $Y\morphism X$ is relative Segal.

Likewise, a morphism $Y\morphism X$ is called \emph{relatively complete Segal} if it is relative Segal and the diagram
\begin{equation*}
	\begin{tikzcd}
		\Hom_{\cat{sAn}}(\Delta^0,Y)\rar\dar\drar[pullback] & \Hom_{\cat{sAn}}\big(\Delta^3/(\Delta^{\{0,2\}}\cup \Delta^{\{1,3\}}),Y\big)\dar\\
		\Hom_{\cat{sAn}}(\Delta^0,X)\rar & \Hom_{\cat{sAn}}\big(\Delta^3/(\Delta^{\{0,2\}}\cup \Delta^{\{1,3\}}),X\big)
	\end{tikzcd}
\end{equation*}
Again, if this holds and $X$ is a complete Segal anima, then so is $Y$ (here we use a slight reformulation of \cref{thmdef:RezkNerve}\itememph{b}).


With that let's dive into cocartesian monoidal structure:
\begin{con}\label{con:CocartesianMonoidalStructure}
	Let $\IGamma_\sqcup^\op\subseteq \langle 1\rangle/\IGamma^\op$ be the full subcategory spanned by the \enquote{defined maps} (for every $n$ there's a unique nowhere-defined map $\langle 1\rangle\morphism \langle n\rangle$; this is the only one we exclude). For any $\infty$-category $\Cc$ define $p^\sqcup\colon\Cc^\sqcup\morphism \IGamma^\op$ via \labelcref{par:PresheafConstruction} applied to the presheaf $F_{\Cc^\sqcup}\colon \IDelta/\IGamma^\op\morphism \An$, which is given by
	\begin{equation*}
		F_{\Cc^\sqcup}\big([n]\rightarrow\IGamma^\op\big)\simeq \Hom_{\Cat_\infty}\big([n]\times_{\IGamma^\op}\IGamma^\op_\sqcup,\Cc\big)
	\end{equation*}
	on objects (and turned into a functor of $\infty$-categories in the obvious way).
\end{con}
\begin{prop}\label{prop:CocartesianMonoidalStructure}
	With notation as in \cref{con:CocartesianMonoidalStructure}:
	\begin{alphanumerate}
		\item The presheaf $F_{\Cc^\sqcup}$ defines a complete Segal anima over $\N^r(\IGamma^\op)$ via \cref{lem:TautologicalPresheafLemma}.
		\item $p^\sqcup\colon\Cc^\sqcup\morphism \IGamma^\op$ is always an $\infty$-operad with underlying category $\Cc$.
		\item An active morphism $(a_1,\dotsc,a_n)\morphism b$ in $\Cc^{\sqcup}$ is locally $p^\sqcup$-cocartesian iff the maps $a_i\morphism b$ exhibit $b$ as the coproduct of the $a_i$.
		\item In particular, $p^\sqcup\colon\Cc^\sqcup\morphism\IGamma^\op$ is a symmetric monoidal category iff $\Cc$ has finite coproducts.
	\end{alphanumerate}
\end{prop}
\begin{proof}[Proof sketch]
	The proof of \itememph{a} was left as an exercise, so here's what I figured out. To show that $F_{\Cc^\sqcup}$ is Segal, consider the diagram
	\begin{equation*}
		\begin{tikzcd}
			\Hom_{\cat{sAn}}(\Delta^n,F_{\Cc^\sqcup})\rar\dar & \Hom_{\cat{sAn}}(I^n,F_{\Cc^\sqcup})\dar\\
			\Hom_{\cat{sAn}}\big(\Delta^n,\N^r(\IGamma^\op)\big)\rar[iso] & \Hom_{\cat{sAn}}\big(I^n,\N^r(\IGamma^\op)\big)
		\end{tikzcd}
	\end{equation*}
	The bottom row is an equivalence since $\N^r(\IGamma^\op)$ is Segal. To show that the top row is an equivalence as well, it suffices to check that it induces equivalences on all (derived) fibres. Unravelling what that means (using the computation of $\Hom$ anima in slice categories from \cite[Corollary~VIII.6]{HigherCatsII}), we need to check that
	\begin{equation*}
		\Hom_{\cat{sAn}/\N^r(\IGamma^\op)}(\Delta^n,F_{\Cc^\sqcup})\isomorphism \Hom_{\cat{sAn}/\N^r(\IGamma^\op)}(I^n,F_{\Cc^\sqcup})\simeq\prod_{i=1}^n\Hom_{\cat{sAn}/\N^r(\IGamma^\op)}(\Delta^1,F_{\Cc^\sqcup})
	\end{equation*}
	is an equivalence for all choices of $\Delta^n\morphism \N^r(\IGamma^\op)$. Recall that $\cat{sAn}/\N^r(\IGamma^\op)\simeq \Pp(\IDelta/\IGamma^\op)$ by \cref{lem:TautologicalPresheafLemma}. Using this together with the Yoneda lemma and the definition of $F_{\Cc^\sqcup}$, our assertion translates into 
	\begin{equation*}
		\Hom_{\Cat_\infty}\big([n]\times_{\IGamma^\op}\IGamma^\op_\sqcup,\Cc\big)\isomorphism\prod_{i=1}^n\Hom_{\Cat_\infty}\big([1]\times_{\IGamma^\op}\IGamma^\op_\sqcup,\Cc\big)\,.
	\end{equation*}
	In other words, we must prove that $[n]\times_{\IGamma^\op}\IGamma^\op_\sqcup$ is an iterated pushout of $[1]\times_{\IGamma^\op}\IGamma^\op_\sqcup$ in $\Cat_\infty$. One way to do this is to simply check by hand that 
	\begin{equation*}
		I^n\times_{\N(\IGamma^\op)}\N(\IGamma^\op_\sqcup)\monomorphism\Delta^n\times_{\N(\IGamma^\op)}\N(\IGamma^\op_\sqcup)
	\end{equation*}
	is an inner anodyne map of simplicial sets. This shows that $F_{\Cc^\sqcup}$ is Segal. Completeness can be shown analogously, whence we have proved \itememph{a}.
	
	To prove \itememph{b}, we must investigate the fibres of $p^\sqcup\colon\Cc^\sqcup\morphism \IGamma^\op$. So fix some $\langle n\rangle \in \N^r(\IGamma^\op)$. Since $F_{\Cc^\sqcup}$ is complete Segal by \itememph{a}, we get $F_{\Cc^\sqcup}\simeq \N^r(\asscat F_{\Cc^\sqcup})\simeq \N^r(\Cc^\sqcup)$. Because $\N^r$ preserves limits, we may as well compute the corresponding fibre of $F_{\Cc^\sqcup}\morphism \N^r(\IGamma^\op)$ and then take $\asscat(-)$ again. To do this, we claim that the following diagram commutes:
	\begin{equation*}
		\begin{tikzcd}[column sep=small]
			\Pp(\IDelta/\IGamma^\op)\ar[rr,iso]\drar["\substack{\text{evaluation at}\\
			\text{constant maps}}"'] & & \cat{sAn}/\N^r(\IGamma^\op)\dlar["-\times_{\N^r(\IGamma^\op)}\{\langle n\rangle\}"]\\
			& \cat{sAn} &
		\end{tikzcd}
	\end{equation*}
	The right diagonal arrow sends a simplicial anima $X$ over $\N^r(\IGamma^\op)$ to $X\times_{\N^r(\IGamma^\op)}\{\langle n\rangle\}$ (i.e.\ to the fibre we're interested in). The left diagonal arrow sends a presheaf $F\colon \IDelta/\IGamma^\op\morphism \An$ to a simplicial anima $Y$ defined by $Y_k\simeq F(\const {\langle n\rangle}\colon [k]\morphism \IGamma^\op$). To show commutativity, we may use \cref{thm:ColimitPreservingRepresentable}, whence it suffices to check that both ways around the diagram agree on representable presheaves and preserve colimits. The former is straightforward to check (although it took me some time to wrap my head around this). For the latter, it's clear that the left diagonal arrow preserves colimits, and for the right diagonal arrow we can use that colimits in anima commute with pullbacks (as seen in the proof of \cref{lem:TautologicalPresheafLemma}) and that colimits and pullbacks in $\cat{sAn}$ are computed pointwise.
	
	Let's apply our newfound knowledge to $p^\sqcup\colon\Cc^\sqcup\morphism\IGamma^\op$. It's straightforward to check that the diagram
	\begin{equation*}
		\begin{tikzcd}
			{[k]}\times\langle n\rangle \dar["\pr_1"']\rar\drar[pullback] & \IGamma_\sqcup^\op\dar\\
			{[k]}\rar["\const {\langle n\rangle}"] & \IGamma^\op
		\end{tikzcd}
	\end{equation*}
	is a pullback diagram. In particular, the fibre of $p^\sqcup\colon\Cc^\sqcup\morphism\IGamma^\op$ over $\langle n\rangle$ is the associated category of a simplicial anima $Y$ given by
	\begin{align*}
		Y_k\simeq F_{\Cc^\sqcup}\big(\const {\langle n\rangle}\colon [k]\rightarrow \IGamma^\op\big)&\simeq\Hom_{\Cat_\infty}\big([k]\times_{\IGamma^\op}\IGamma_\sqcup^\op,\Cc\big)\\
		&\simeq \Hom_{\Cat_\infty}\big([k]\times\langle n\rangle,\Cc\big)\\
		&\simeq \Hom_{\Cat_\infty}\big([k],\Cc^n\big)
	\end{align*}
	Up to checking some functoriality stuff, this shows $Y\simeq \N^r(\Cc^n)$, so the fibre of $p^\sqcup\colon\Cc^\sqcup\morphism\IGamma^\op$ over $\langle n\rangle$ is $\Cc^n$. This shows that $p^\sqcup$ satisfies \cref{def:Operad}\itememph{b}.
	
	Two more conditions are to check. Let $x,y\in \Cc^\sqcup$ and let $\alpha\colon p^\sqcup(x)\morphism p^\sqcup(y)$ be some morphism in $\IGamma^\op$. Then the fibre $\Hom_{\Cc^\sqcup}^\alpha(x,y)\coloneqq \Hom_{\Cc^\sqcup}(x,y)\times_{\Hom_{\IGamma^\op}(p^\sqcup(x),p^\sqcup(y))}\{\alpha\}$ of $\Hom_{\Cc^\sqcup}^\alpha(x,y)$ over $\alpha$ has a particularly simple description: If we think of $\alpha$ as a functor $\alpha\colon [1]\morphism \IGamma^\op$, then $\Hom_{\Cc^\sqcup}^\alpha(x,y)$ fits into a pullback
	\begin{equation*}
		\begin{tikzcd}
			\Hom_{\Cc^\sqcup}^\alpha(x,y)\rar\dar\drar[pullback] & \Hom_{\Cat_\infty}\big([1]\times_{\IGamma^\op}\IGamma_\sqcup^\op,\Cc\big)\dar["{(d_1,d_0)}"]\\
			*\rar["{(x,y)}"] & \Hom_{\Cat_\infty}\big([0]\times_{\IGamma^\op}\IGamma_\sqcup^\op,\Cc\big)\times \Hom_{\Cat_\infty}\big([0]\times_{\IGamma^\op}\IGamma_\sqcup^\op,\Cc\big)
		\end{tikzcd}
	\end{equation*}
	To prove this, use Yoneda's lemma, the computation of $\Hom$ anima in slice categories from \cite[Corollary~VIII.6]{HigherCatsII}, and the fact that $\N^r\colon \Cat_\infty\morphism\cat{sAn}$ is fully faithful to compute
	\begin{align*}
		\Hom_{\Cat_\infty}\big([1]\times_{\IGamma^\op}\IGamma_\sqcup^\op,\Cc\big)&\simeq \rlap{$F_{\Cc^\sqcup}(\alpha\colon[1]\rightarrow \IGamma^\op)$}\hphantom{\Hom_{\cat{sAn}}(\Delta^1,F_{\Cc^\sqcup})\times_{\Hom_{\cat{sAn}}(\Delta^1,\N^r(\IGamma^\op))}\{\N^r(\alpha)\}}\\
		&\simeq \Hom_{\Pp(\IDelta/\IGamma^\op)}\big(\alpha\colon [1]\morphism \IGamma^\op,F_{\Cc^\sqcup}\big)\\
		&\simeq \Hom_{\cat{sAn}/\N^r(\IGamma^\op)}\big(\N^r(\alpha)\colon \Delta^1\rightarrow\N^r(\IGamma^\op),F_{\Cc^\sqcup}\big)\\
		\hphantom{\Hom_{\Cat_\infty}\big([1]\times_{\IGamma^\op}\IGamma_\sqcup^\op,\Cc\big)}&\simeq \Hom_{\cat{sAn}}(\Delta^1,F_{\Cc^\sqcup})\times_{\Hom_{\cat{sAn}}(\Delta^1,\N^r(\IGamma^\op))}\{\N^r(\alpha)\}\\
		&\simeq \Hom_{\Cat_\infty}\big([1],\Cc^\sqcup\big)\times_{\Hom_{\Cat_\infty}([1],\IGamma^\op)} \{\alpha\}\,.
	\end{align*}
	With this and an analogous computation for $\Hom_{\Cat_\infty}([0]\times_{\IGamma^\op}\IGamma_\sqcup^\op,\Cc)$, the above pullback square is straightforward to verify. Let's use this to check that $p^\sqcup\colon \Cc^\sqcup\morphism\IGamma^\op$ has cocartesian lifts for inert morphisms. So let $\alpha\colon \langle m\rangle\morphism\langle n\rangle$ be inert, and $x\in \Cc^\sqcup$ a lift of $\langle m\rangle$. Then $x$ corresponds to a map $[0]\times_{\IGamma^\op}\IGamma^\op_\sqcup\morphism\Cc$ and a lift of $\alpha$ corresponds to a map $[1]\times_{\IGamma^\op}\IGamma^\op_\sqcup\morphism \Cc$, where the pullback is formed using $\alpha\colon [1]\morphism\IGamma^\op$. So we need to solve a lifting problem
	\begin{equation*}
		\begin{tikzcd}
			{[0]}\times_{\IGamma^\op}\IGamma^\op_\sqcup\dar[mono,"d_1"']\rar &\Cc\\
			{[1]}\times_{\IGamma^\op}\IGamma^\op_\sqcup\urar[dashed]& 
		\end{tikzcd}
	\end{equation*}
	in such a way that the solution is a $p^\sqcup$-cocartesian lift of $\alpha$. Since $\alpha$ is inert, i.e.\ an isomorphism where it is defined, $[1]\times_{\IGamma^\op}\IGamma^\op_\sqcup$ consists of a disjoint union of copies of $[1]$ (their number corresponds to the number of elements of $\langle m\rangle$ where $\alpha$ is defined) and $[0]$ (corresponding to the elements of $\langle m\rangle$ where $\alpha$ isn't). We define the desired lift $[1]\times_{\IGamma^\op}\IGamma^\op_\sqcup\morphism \Cc$ to be degenerate on every copy of $[1]$, noting that it is already defined on their starting points and on the additional copies of $[0]$.
	
	To show that this gives a cocartesian lift, we must check the condition from \cref{def:WeirdCocartesianDefinition}\itememph{a}, i.e.\ that 
	\begin{equation*}
		\Hom_{\Cc^\sqcup}(y,z)\isomorphism \Hom_{\Cc^\sqcup}(x,z)\times_{\Hom_{\IGamma^\op}(\langle n\rangle,\langle k\rangle)}\Hom_{\IGamma^\op}\big(\langle m\rangle,\langle k\rangle\big)
	\end{equation*}
	is an isomorphism for all $z\in \Cc^\sqcup$ and $p^\sqcup(z)=\langle k\rangle$. This can be checked on (derived) fibres over any $\beta\in \Hom_{\IGamma^\op}(\langle m\rangle,\langle k\rangle)$. The fibre of the left-hand side is $\Hom_{\Cc^\sqcup}^\beta(y,z)$. The fibre of the right-hand side is the same as that of $\Hom_{\Cc^\sqcup}(x,z)\morphism\Hom_{\IGamma^\op}(\langle n\rangle,\langle k\rangle)$ over $\beta\circ\alpha$, which is $\Hom_{\Cc^\sqcup}^{\beta\circ \alpha}(x,z)$. That these two are equivalent is straightforward to check from the constructions, using the pullback diagram above. This proves that $p^\sqcup\colon \Cc^\sqcup\morphism\IGamma^\op$ satisfies \cref{def:Operad}\itememph{a}. The condition from \cref{def:Operad}\itememph{c} can be checked in the same way. This proves \itememph{b}.
	
	For \itememph{c}, it suffices to check the recognition criterion from \labelcref{par:RecognitionCriterion}. By definition, we have 
	\begin{equation*}
		\Hom_{\Cc^\sqcup}^\mathrm{act}\big((a_1,\dotsc,a_n),-\big)\simeq \Hom_{\Cc^\sqcup}^{f_n}\big((a_1,\dotsc,a_n),-\big)\,. 
	\end{equation*}
	The pullback $[1]\times_{\IGamma^\op}\IGamma^\op_\sqcup$ formed using $f_n\colon [1]\morphism \IGamma^\op$ is precisely the cone over a discrete set with $n$ elements. Using this and the pullback diagram above, one easily verifies
	\begin{equation*}
		\Hom_{\Cc^\sqcup}^{f_n}\big((a_1,\dotsc,a_n),-\big)\simeq \prod_{i=1}^n\Hom_\Cc(a_i,-)\,.
	\end{equation*}
	Hence any representing object of $\Hom_{\Cc^\sqcup}^\mathrm{act}((a_1,\dotsc,a_n),-)$ is a coproduct of the $a_i$, whence \itememph{c} follows. Part~\itememph{d} is an immediate consequence of \itememph{c} and the discussion in \labelcref{par:RecognitionCriterion}, as coproducts are clearly associative.
\end{proof}
\cref{prop:CocartesianMonoidalStructure} allows us to define the cocartesian monoidal structure on an $\infty$-category with finite coproducts. To construct the cartesian monoidal structure, once can consider $((\Cc^\op)^\sqcup)^\op\in \cat{dOp}_\infty$. If $\Cc$ has finite products, so $\Cc^\op$ has finite coproducts, then the dual $\infty$-operad $((\Cc^\op)^\sqcup)^\op$ defines a symmetric monoidal $\infty$-category and we can define $p^\times\colon \Cc^\times\morphism \IGamma^\op$ via
\begin{equation*}
	\Cc^\times=\Un^\cocart\big(\St^\cart ((\Cc^\op)^\sqcup)^\op\big)\,.
\end{equation*}
\begin{con}\label{con:CartesianMonoidalStructure}
	There is also a direct construction of $p^\times\colon \Cc^\times\morphism\IGamma^\op$, but this is a bit tricky, which perhaps isn't very surprising since $\infty$-operads would like to specify maps out of a tensor product, whereas products would like to have maps into them specified. We define a $1$-category $\IGamma_\times^\op$ as follows: Its objects are pairs $(n,S)$, where $S\subseteq \langle n\rangle$ is a subset, and morphisms $(n,S)\morphism (m,T)$ are given by morphisms $\alpha\colon \langle n\rangle \morphism\langle m\rangle$ in $\IGamma^\op$ such that $\alpha^{-1}(T)=S$. Let $\snake{p}^\times\colon \snake{\Cc}^\times\morphism\IGamma^\op$ be given by \labelcref{par:PresheafConstruction} applied to the presheaf $F_{\Cc^\times}\IDelta/\IGamma^\op\morphism\An$, which is in turn given by
	\begin{equation*}
		F_{\Cc^\times}\big([n]\rightarrow \IGamma^\op\big)\simeq \Hom_{\Cat_\infty}\big([n]\times_{\IGamma^\op}\IGamma_\times^\op,\Cc\big)\,.
	\end{equation*}
	The same analysis as in the proof of \cref{prop:CocartesianMonoidalStructure}\itememph{a} shows that the fibre of $\snake{p}^\times\colon \snake{\Cc}^\times\morphism\IGamma^\op$ over $\langle n\rangle$ is given by
	\begin{equation*}
		\snake{\Cc}_n^\times\simeq \Fun\big((\text{power set of }\langle n\rangle)^\op,\Cc\big)\,.
	\end{equation*}
	Now let $\Cc^\times$ be the full subcategory of $\snake{\Cc}^\times$ which is fibrewise spanned by those functors $F\colon (\text{power set of }\langle n\rangle)^\op\morphism\Cc$ satisfying $F(S)\simeq \prod_{i\in S}F(\{i\})$. 
\end{con}
\begin{prop}\label{prop:CartesianMonoidalStructure}
	With notation as in \cref{con:CocartesianMonoidalStructure}:
	\begin{alphanumerate}
		\item The presheaf $F_{\Cc^\times}$ defines a complete Segal anima via \cref{lem:TautologicalPresheafLemma}.
		\item $p^\times\colon\Cc^\times\morphism \IGamma^\op$ is always a cocartesian fibration with $\Cc_1^\times\simeq \Cc$ \embrace{but not necessarily an $\infty$-operad}.
		\item If $\Cc$ has finite products, then $p^\times\colon\Cc^\times\morphism\IGamma^\op$ satisfies the Segal condition, hence it is a symmetric monoidal $\infty$-category with underlying $\infty$-category $\Cc$. Moreover, in this case we have
		\begin{equation*}
			\Hom_{\Cc^\times}^\mathrm{act}\big((a_1,\dotsc,a_n),-\big)\simeq \Hom_{\Cc^\times}(a_1\times\dotsb\times a_n,-)\,.
		\end{equation*}
	\end{alphanumerate}
\end{prop}
\begin{proof}
	Similar to \cref{prop:CocartesianMonoidalStructure} and omitted. Fabian's script \cite[Proposition~II.42]{KTheory} has some details.
\end{proof}

\numpar*{Monoids over an $\infty$-Operad}
If $p\colon\Oo\morphism\IGamma^\op$ and $p'\colon\Oo'\morphism \IGamma^\op$ are $\infty$-operads, let $\Fun^{\cat{Op}_\infty}(\Oo,\Oo')$ be the full sub-$\infty$-category of $\Fun_{\IGamma^\op}(\Oo,\Oo')$ spanned by the $\infty$-operad maps. Here $\Fun_{\IGamma^\op}(\Oo,\Oo')$ is given by the pullback
\begin{equation*}
	\begin{tikzcd}
		\Fun_{\IGamma^\op}(\Oo,\Oo')\rar\dar\drar[pullback] & \Fun(\Oo,\Oo')\dar["p'_*"]\\
		*\rar["p"] & \Fun(\Oo,\IGamma^\op)
	\end{tikzcd}
\end{equation*}
as usual. If $\Cc$ has products, then $\Fun^{\cat{Op}_\infty}(\IGamma^\op,\Cc^\times)\simeq \CMon(\Cc)$. In fact, this can be stated in much greater generality, which we'll do in \cref{thm:OMon} below: Define the \emph{$\infty$-category of $\Oo$-monoids} to be the full subcateroy $\Oo\cat{Mon}(\Cc)\subseteq \Fun(\Oo,\Cc)$ spanned by those functors $F\colon \Oo\morphism\Cc$ satisfying the Segal condition. That is, the maps $(a_1,\dotsc,a_n)\morphism a_i$ in $\Oo$ are supposed to induce equivalences $F(a_1,\dotsc,a_n)\isomorphism \prod_{i=1}^n F(a_i)$.

Also note that there is a section $\IGamma^\op\morphism \IGamma_\times^\op$ sending $\langle n\rangle \mapsto (n,\langle n\rangle)$. It induces canonical maps $\Hom_{\Cat_\infty}([n]\times_{\IGamma^\op}\IGamma^\op_\times,\Cc)\morphism\Hom_{\Cat_\infty}([n]\times_{\IGamma^\op}\IGamma^\op,\Cc)\simeq \Hom_{\Cat_\infty}([n],\Cc)$, hence a map $F_{\Cc^\times}\morphism\N^r(\Cc)$ of simplicial anima and therefore a map $\Cc^\times\morphism\Cc$.
\begin{thm}\label{thm:OMon}
	If $\Cc$ has finite products, then the map $\Cc^\times\morphism\Cc$ constructed above induces an equivalence
	\begin{equation*}
		\Fun^{\cat{Op}_\infty}(\Oo,\Cc^\times)\isomorphism \Oo\cat{Mon}(\Cc)		
	\end{equation*}
	is an equivalence. If $\Cc$ has finite coproducts and $\Hom_\Oo^\mathrm{act}(\emptyset,x)\simeq *$ for all $x\in \Oo$ \embrace{so that $\Oo$ is \emph{unital}, see Lemma/Definition~\textup{\labelcref{lemdef:UnitInAlg}} below} then the restriction
	\begin{equation*}
		\Fun^{\cat{Op}_\infty}(\Oo,\Cc^\sqcup)\isomorphism \Fun(\Oo_1,\Cc)
	\end{equation*}
	is an equivalence as well.
\end{thm}
\begin{proof}
	See \cite{HA}: The first part follows from Proposition~\HAthm{2.4.2.5}, the second part from Propositions~\HAthm{2.4.3.16} and~\HAthm{2.3.1.11}.
\end{proof}
\subsection{Algebras over \texorpdfstring{$\infty$}{Infinity}-Operads}
\cref{thm:OMon} is supposed to motivate the following general definition.
\begin{defi}\label{def:AlgO}
	\begin{alphanumerate}
		\item If $\Oo$ is an $\infty$-operad and $\Cc^\otimes$ is a symmetric monoidal structure on an $\infty$-category $\Cc$, then we put
		\begin{equation*}
			\cat{Alg}_\Oo(\Cc^\otimes)\coloneqq \Fun^{\cat{Op}_\infty}(\Oo,\Cc^\otimes)
		\end{equation*}
		and call this the \emph{$\infty$-category of $\Oo$-algebras in $\Cc$}. If the symmetric monoidal structure on $\Cc$ is clear from the context, we'll often write $\cat{Alg}_\Oo(\Cc)$ instead.
		
		\item More generally, let $\Oo'$ be another another $\infty$-operad with a map $\alpha\colon \Oo'\morphism \Oo$, and let $\Cc$ be an \emph{$\Oo$-monoidal $\infty$-category}, i.e.\ an object of $\Oo\cat{Mon}(\Cat_\infty)$. Again, we denote by $(-)^\otimes\colon \Oo\cat{Mon}(\Cat_\infty)\morphism \cat{Op}_\infty/\Oo$ the functor induced by cocartesian unstraightening. Then $\cat{Alg}_{\Oo'/\Oo}(\Cc^\otimes)$ is defined as the pullback
		\begin{equation*}
			\begin{tikzcd}
				\cat{Alg}_{\Oo'/\Oo}(\Cc^\otimes)\rar\dar\drar[pullback] & \Fun^{\cat{Op}_\infty}(\Oo',\Cc^\otimes)\dar\\
				*\rar["\alpha"]& \Fun^{\cat{Op}_\infty}(\Oo',\Oo)
			\end{tikzcd}
		\end{equation*}
		This recovers part~\itememph{a} as the special case$\cat{Alg}_\Oo(\Cc^\otimes)\simeq \cat{Alg}_{\Oo/\IComm}(\Cc^\otimes)$, where we denote $\IComm= \id\colon \IGamma^\op\morphism\IGamma^\op$ (more on that in  \cref{exm:MyFirstAlgebrasOverOperads}\itememph{c}).
	\end{alphanumerate}
\end{defi}
\begin{exm}\label{exm:MyFirstAlgebrasOverOperads}
	\enquote{My first algebras over $\infty$-operads}:
	\begin{alphanumerate}
		\item If $\Oo$ is the $\infty$-operad $*\xrightarrow{\langle 0\rangle}\IGamma^\op$, then $\cat{Alg}_\Oo(\Cc^\otimes)$ is easily identified with the fibre of $\Cc^\otimes$ over $\langle 0\rangle$, hence $\cat{Alg}_\Oo(\Cc^\otimes)\simeq *$.
		\item If $\Oo$ is $\IGamma_\mathrm{int}^\op\morphism\IGamma^\op$, then $\cat{Alg}_\Oo(\Cc^\otimes)\simeq \Cc$. 
		Hence this $\infty$-operad will be denoted $\Oo=\cat{\IT riv}$. We'll give a sketch of why this is true. One easily finds
		\begin{equation*}
			\Fun_{\IGamma^\op}\big(\IGamma^\op_\mathrm{int},\Cc^\otimes\big)\simeq \Fun_{\IGamma^\op_\mathrm{int}}\big(\IGamma^\op_\mathrm{int},\Cc^\otimes\times_{\IGamma^\op}\IGamma^\op_\mathrm{int}\big)\,,
		\end{equation*}
		and this restricts to an equivalence
		\begin{equation*}
			\Fun^{\cat{Op}_\infty}\big(\cat{\IT riv},\Cc^\otimes\big)\simeq\Gamma_\cocart\big(\Cc^\otimes\times_{\IGamma^\op}\IGamma^\op_\mathrm{int}\big)\,,
		\end{equation*}
		where $\Gamma_\cocart$ is defined as in \cref{prop:CoLimitsInCat}. In particular, the right hand side is equivalent to $\limit(\St^\cocart(\Cc^\otimes\times_{\IGamma^\op}\IGamma^\op_\mathrm{int})\colon \IGamma^\op_\mathrm{int}\morphism\Cat_\infty)$, so we are to show that this limit is $\Cc$.
		
		To see this, note that by definition of $\IGamma_\mathrm{int}^\op$, its opoosite category $\IGamma_\mathrm{int}$ can be viewed as the category of finite sets $\langle n\rangle$ with injective maps between them. Hence there is a $\IGamma_\mathrm{int}\morphism\Cat_\infty$ sending $\langle n\rangle$ to itself, considered as a discrete $\infty$-category. Let $U\morphism\IGamma_\mathrm{int}^\op$ be its cartesian unstraightening, so that the fibre $U_n\coloneqq U \times_{\IGamma_\mathrm{int}^\op}\{\langle n\rangle\}$ is isomorphic to $\langle n\rangle$. Then $\Cc_n^\otimes\simeq \Cc^n\simeq \limit_{U_n}\Cc$. Hence we may apply the dual of \cref{prop:ColimitsCommute} to see that the limit we're looking for is given as $\limit_U\Cc$. But now it's straightforward to check that the unique point of the fibre $U_1$ is an initial object of $U$, hence $\limit_U\Cc\simeq \Cc$, as required.
		
		\item If $\Oo$ is $\id\colon \IGamma^\op\morphism\IGamma^\op$, then we define $\cat{Alg}_\Oo(\Cc^\otimes)\eqqcolon \cat{CMon}(\Cc^\otimes)$. If $\Cc$ has finite products and $\Cc^\otimes\simeq \Cc^\times$ is the $\infty$-operad giving the cartesian monoidal structure, then this notation is consistent with our previous definition of $\CMon(\Cc)$ by \cref{thm:OMon}. Consequently, the $\infty$-operad $\id\colon\IGamma^\op\morphism\IGamma^\op$ will be denoted $\Oo=\cat{\IC omm}$ (or sometimes $\IE_\infty$).
		
		
		\item In view of \itememph{c}, a natural question is whether there exists an $\infty$-operad $\cat{\IA ssoc}$ satisfying $\cat{Alg}_{\cat{\IA ssoc}}(\Cc^\times)\simeq \Mon(\Cc)$. This is indeed true and we will construct $\cat{\IA ssoc}$ soon (to be precise, in a future version of Example~\hyperref[exm:MyFirstAlgebrasOverOperadsII]{\labelcref*{exm:MyFirstAlgebrasOverOperads}}\itememph{d}).
		\item Similarly, there is an $\infty$-operad $\cat{\IL Mod}$ encoding pairs $(A,M)$ of associative monoids $A$ and an object $M$ on which $A$ acts from the left. $\ILMod$ will also be constructed in a future version of \hyperref[exm:MyFirstAlgebrasOverOperadsII]{\labelcref*{exm:MyFirstAlgebrasOverOperads}}\itememph{e}.
		\item  If $p\colon\Cc^\otimes\morphism \IGamma^\op$ is any symmetric monoidal $\infty$-category, then we would expect that the tensor unit $1_\Cc\in \Cc$ gives rise to an object in $\CMon(\Cc^\otimes)$. This is indeed true and works in greater generality. To give this result the space it deserves, we outsource it to Lemma/Definition~\labelcref{lemdef:UnitInAlg} below.
	\end{alphanumerate}
\end{exm}\refstepcounter{smallerdummy}
\numpar*{\thesmallerdummy. Lemma/Definition}\label{lemdef:UnitInAlg}\itshape For any $\infty$-operad $\Oo$, the unique object $*_\Oo\in \Oo_0$ is terminal in $\Oo$. Moreover, the following conditions are equivalent:
\begin{alphanumerate}
	\item $\Hom_\Oo^\mathrm{act}(\emptyset,x)\simeq *$ for all $x\in \Oo_1$.
	\item The unique object $*_\Oo\in\Oo_0$ is initial in $\Oo$.
\end{alphanumerate}		
Such $\infty$-Operads are called \emph{unital}. If $\Oo$ is unital and $p\colon \Cc^\otimes\morphism \Oo$ an $\Oo$-monoidal $\infty$-category \embrace{see \cref{def:AlgO}\itememph{b}}, then the tensor unit of $\Cc^\otimes$, i.e.\ the unique  element of $\St^\cocart(p)(*_\Oo)$, canonically gives rise to an initial object $\IOne\in \Alg_{\Oo/\Oo}(\Cc^\otimes)$. In the case $\Oo\simeq \IComm$, the evaluation of $\IOne$ at $\langle 1\rangle\in\IComm$ is indeed the usual tensor unit $1_\Cc\in\Cc$.\upshape
\begin{proof}
	The construction of $\IOne\in \Alg_{\Oo/\Oo}(\Cc^\otimes)$ was done in the lecture (in the special case $\Oo\simeq \IComm$), but the rest wasn't.
	
	The fact that $*_\Oo$ is terminal follows immediately from \cref{def:Operad}\itememph{c}. For the equivalence, observe that there is only one map $\langle 0\rangle\morphism\langle 1\rangle$ in $\IGamma^\op$, hence $\Hom_\Oo^\mathrm{act}(\emptyset,x)\simeq \Hom_\Oo(*_\Oo,x)$ for all $x\in \Oo_1$. This immediately implies \itememph{b} $\Rightarrow$ \itememph{a}. To see \itememph{a} $\Rightarrow$ \itememph{b}, note that $ \Hom_\Oo(*_\Oo,x)\simeq *$ for all $x\in \Oo_1$ already suffices to show the same for all $x\in \Oo$, by \cref{def:Operad}\itememph{c}.
	
	Now let $\Oo$ unital $\infty$-operad and $p\colon \Cc^\otimes\morphism\Oo$ an $\Oo$-monoidal $\infty$-category. To construct $\IOne\in \Alg_{\Oo/\Oo}(\Cc^\otimes)\subseteq \Fun_\Oo(\Oo,\Cc^\otimes)$, our goal is to produce a canonical $\infty$-operad map
	\begin{equation*}
		\begin{tikzcd}[column sep=small]
			\Oo\drar["\id"']\ar[rr] & & \Cc^\otimes\dlar["p"]\\
			& \Oo &
		\end{tikzcd}
	\end{equation*}
	We will choose it to be a map between cocartesian fibrations even. After straightening, a map between cocartesian fibrations over $\Oo$ corresponds to a natural transformation in $\Fun(\Oo,\Cat_\infty)$ between $\const *$ and the functor $\St^\cocart(p)\colon\Oo\morphism \Cat_\infty$. Since $*_\Oo\in\Oo$ is initial, we get a natural transformation $\const {*_\Oo}\Rightarrow \id_\Oo$ in $\Fun(\Oo,\Oo)$, hence a natural transformation 
	\begin{equation*}
		\const *\simeq \const {\St^\cocart(p)(*_\Oo)}\Rightarrow \St^\cocart(p)
	\end{equation*}
	in $\Fun(\Oo,\Cat_\infty)$, as required. In the special case $\Oo\simeq \IComm$ (to which our considerations apply since $\langle 0\rangle\in\IComm_0$ is initial in $\IComm$), the evaluation of $\IOne$ at $\langle 1\rangle$ is the endpoint of a cocartesian lift of $\langle 0\rangle\morphism\langle 1\rangle$, hence indeed $1_\Cc\in\Cc$.
	
	For simplicity, let me only prove that $\IOne$ is initial the special case $\Oo\simeq \IComm$, since the general case works just the same, but would require some new notation. So consider another commutative algebra $A\in\Alg_\IComm(\Cc^\otimes)$, corresponding to a functor $A\colon \IGamma^\op\morphism \Cc^\otimes$ that preserves inerts and satisfies $p\circ A\simeq \id_{\IGamma^\op}$. Let's also write $A_n=A(\langle n\rangle)$. Since $A$ preserves inerts, we have $A_n\simeq(A_1,\dotsc,A_1)$ via the Segal maps. Also $\IOne_n\simeq (1_\Cc,\dotsc,1_\Cc)$ by construction. We compute
	\begin{align*}
		\Hom_{\Alg_\IComm(\Cc^\otimes)}(\IOne,A)&\simeq \Hom_{\Fun_{\IGamma^\op}(\IComm,\Cc^\otimes)}(\IOne,A)\\
		&\simeq \Nat(\IOne,A)\times_{\Nat(\id_{\IGamma^\op},\id_{\IGamma^\op})}\{\id\}\,,
	\end{align*}
	using that $\Alg_\IComm(\Cc^\otimes)\subseteq \Fun_{\IGamma^\op}(\IComm,\Cc^\otimes)$ is a full sub-$\infty$-category
	and that taking $\Hom$ anima commutes with limits of the $\infty$-categories they are taken in. Now plug \cref{cor:HomInFunctorCats} into the right-hand side and use the fact that limits commute (i.e.\ the dual of \cref{prop:ColimitsCommute}) to obtain
	\begin{align*}
		\Hom_{\Alg_\IComm(\Cc^\otimes)}(\IOne,A)&\simeq\limit_{(\alpha\colon\langle m\rangle\morphism \langle n\rangle)\in \TwAr(\IGamma^\op)}\Hom_{\Cc^\otimes}(\IOne_m,A_n)\times_{\Hom_{\IGamma^\op}(\langle m\rangle,\langle n\rangle)}\{\alpha\}\\
		&\simeq \limit_{(\alpha\colon\langle m\rangle\morphism \langle n\rangle)\in \TwAr(\IGamma^\op)}\Hom_{\Cc^\otimes}^\alpha(\IOne_m,A_n)\\
		&\simeq \limit_{(\alpha\colon\langle m\rangle\morphism \langle n\rangle)\in \TwAr(\IGamma^\op)}\Hom_{\Cc^\otimes}^\alpha(\IOne_m,A_n)\\
		&\simeq \limit_{(\alpha\colon\langle m\rangle\morphism \langle n\rangle)\in \TwAr(\IGamma^\op)}\Hom_{\Cc^n}^\alpha(\IOne_n,A_n) \\
		&\simeq \limit_{(\alpha\colon\langle m\rangle\morphism \langle n\rangle)\in \TwAr(\IGamma^\op)}\Hom_{\Cc}(1_\Cc,A_1)^n\,.
	\end{align*}
	In the second-last step we used that $\IOne_m\morphism \alpha_*\IOne_m\simeq \IOne_n$ is a $p$-cocartesian lift of $\alpha$. Now the target projection $t\colon \TwAr(\IGamma^\op)\morphism \IGamma^\op$ is final. Indeed, this can be shown as in the proof of \cref{cor:HomPreservesColimits} (where we considered the source projection $s\colon \TwAr(\Ii)\morphism\Ii^\op$, but we'll see that things dualize in the right way): $t$ is cocartesian, since $(s,t)\colon \TwAr(\IGamma^\op)\morphism\IGamma\times\IGamma^\op$ is a left fibration and $\pr_2\colon \IGamma\times\IGamma^\op\morphism \IGamma^\op$ is cocartesian. Hence we may apply \cref{exc:cocartesianFibrationLeftAdjoint} to get $|t/\langle n\rangle|\simeq |t^{-1}\{\langle n\rangle\}|$. But the fibres of $t$ are given by $t^{-1}\{\langle n\rangle\}\simeq \IGamma/\langle n\rangle$, hence weakly contractible, as required by \cref{def:cofinal}. So $t$ is indeed final.
	
	The upshot is that we may now write
	\begin{equation*}
		\Hom_{\Alg_\IComm(\Cc^\otimes)}(\IOne,A)\simeq \limit_{\langle n\rangle \in\IGamma^\op}\Hom_{\Cc}(1_\Cc,A_1)^n\simeq \Hom_{\Cc}(1_\Cc,A_1)^0\simeq *\,,
	\end{equation*}
	since $\langle 0\rangle\in\IGamma^\op$ is initial. We're done.
\end{proof}
\subsection{Day Convolution}
We'll use the following construction due to Saul Glasman \cite{GlasmanDayConvolution}.
\begin{con}\label{con:DayConvolution}
	Let $\Cc^\otimes$ be a symmetric monoidal $\infty$-category and $\Oo$ an $\infty$-operad. Define $\F(\Cc^\otimes,\Oo)\morphism \IGamma^\op$ via \labelcref{par:PresheafConstruction} applied to the presheaf $F\colon \IDelta/\IGamma^\op\morphism\An$ which is given by
	\begin{equation*}
		F \big([n]\rightarrow \IGamma^\op\big)\simeq \Hom_{\Cat_\infty/\IGamma^\op}\big([n]\times_{\IGamma^\op}\Cc^\otimes,\Oo\big)
	\end{equation*}	
	Then $F$ defines a complete Segal anima over $\N^r(\IGamma^\op)$. Indeed, this can be shown as in the proof of \cref{prop:CartesianMonoidalStructure}, the only difference being that one can no longer show by hand that $I^n\times_{\N^r(\IGamma^\op)}\Cc^\otimes\morphism \Delta^n\times_{\N^r(\IGamma^\op)}\Cc^\otimes$ is a Joyal equivalence; instead, one can for example use that pullback along cocartesian fibrations preserves Joyal equivalences by \cite[Proposition~\HTTthm{3.3.1.3}]{HTT} or a stronger version of \cite[Theorem~IX.17]{HigherCatsII}).
	
	Once we know $F$ is complete Segal, we may apply the same method as in the proof of \cref{prop:CocartesianMonoidalStructure} to compute
	\begin{equation*}
		\F(\Cc^\otimes,\Oo)_n\simeq \Fun(\Cc^\otimes_n,\Oo_n)\simeq \Fun(\Cc^n,\Oo_1^n)\,.
	\end{equation*}
	Let the \emph{day convolution} $\operatorname{Day}(\Cc^\otimes,\Oo)\subseteq \F(\Cc^\otimes,\Oo)$ be the (non-full!) sub-$\infty$-category spanned by those functors $F\colon \Cc_n^\otimes\morphism \Oo_n$ which split up to equivalence as products $F\simeq F_1\times \dotsb \times F_n$ and those morphisms that \enquote{split similarly}. To make it more precise which morphisms we allow, let $\alpha\colon\langle m\rangle\morphism\langle n\rangle$ be a morphism in $\IGamma^\op$. Regarding $\alpha$ as a map $\alpha\colon [1]\morphism \IGamma^\op$, we form the pullbacks $\Cc_\alpha^\otimes\coloneqq [1]\times_{\IGamma^\op}\Cc^\otimes$ and $\Oo_\alpha\coloneqq [1]\times_{\IGamma^\op}\Oo$. Let, finally, $\alpha_i$ denote the active map $\alpha^{-1}(i)\morphism\langle 1\rangle$ for all $i\in\langle n\rangle$. Then there are decompositions
	\begin{equation*}
		\Cc_\alpha^\otimes\simeq \Cc^S\times_{[1]}\Cc_{\alpha_1}^\otimes\times_{[1]}\dotsb\times_{[1]}\Cc_{\alpha_n}^\otimes\quad\text{and}\quad \Oo_\alpha\simeq \Oo_1^S\times_{[1]}\Oo_{\alpha_1}\times_{[1]}\dotsb\times_{[1]}\Oo_{\alpha_n}\,,
	\end{equation*}
	where $S\subseteq\langle m\rangle$ is the subset where $\alpha$ is undefined. By definition, a morphism in $\F(\Cc^\otimes,\Oo)$ covering $\alpha$ is a map $\Cc_\alpha^\otimes\morphism\Oo$ in $\Cat_\infty/\IGamma^\op$, or equivalently, a map $\Cc_\alpha^\otimes\morphism \Oo_\alpha$ in $\Cat_\infty/[1]$. Such a morphism is permitted into $\operatorname{Day}(\Cc^\otimes,\Oo)$ iff it respects the above decompositions. Note that this condition is always satisfied if $\alpha$ is active.
\end{con}
\refstepcounter{smallerdummy}
\numpar*[?]{\thesmallerdummy. An Error in the Construction*}\label{par:DayConvolutionError}
I believe I might've found an error in \cref{con:DayConvolution}. But before I explain what I think goes wrong, let me already tell you that even if there is indeed an error in \cite{GlasmanDayConvolution}, it won't have any consequences, neither for our lecture nor for mathematics as a whole, as we'll see in \labelcref{par:DontWorry} below.

The issue is that I don't see why $\operatorname{Day}(\Cc^\otimes,\Oo)$ would satisfy the Segal condition, and I even think that I have an argument why $\operatorname{Day}(\Cc^\otimes,\Oo)_n\simeq \Fun(\Cc,\Oo_1)^n$ is wrong in general. By definition, $\operatorname{Day}(\Cc^\otimes,\Oo)_n\subseteq\Fun(\Cc^n,\Oo_1^n)$ is spanned by those functors $F$, which are up to equivalence of the form $F_1\times\dotsb\times F_n$, along with those transformations $\eta\colon F\Rightarrow G$, which are up to equivalence of the form $\eta_1\times\dotsb\times\eta_n$ for some $\eta_i\colon F_i\Rightarrow G_i$. We can write
\begin{equation*}
	\Fun(\Cc^n,\Oo_1^n)\simeq \prod_{i=1}^n\Fun(\Cc^n,\Oo_1)\,,
\end{equation*}
which induces a similar decomposition
\begin{equation*}
	\operatorname{Day}(\Cc^\otimes,\Oo)_n\simeq \prod_{i=1}^n\operatorname{Day}(\Cc^\otimes,\Oo)_{n,i}\,,
\end{equation*}
where $\operatorname{Day}(\Cc^\otimes,\Oo)_{n,i}\subseteq \Fun(\Cc^n,\Oo_1)$ is spanned by those functors and those transformations, which factor up to equivalence over the projection $\pr_i\colon \Cc^n\morphism \Cc$. For the Segal condition to hold, we would like that $\pr_i^*\colon \Fun(\Cc,\Oo_1)\morphism \Fun(\Cc^n,\Oo_1)$ is an equivalence onto $\operatorname{Day}(\Cc^\otimes,\Oo)_{n,i}$. Now the inclusion $\operatorname{Day}(\Cc^\otimes,\Oo)_{n,i}\subseteq \Fun(\Cc^n,\Oo_1)$ of simplicial sets is an isofibration. Indeed, lifting against $\Lambda_1^2\morphism \Delta^2$ holds since $\operatorname{Day}(\Cc^\otimes,\Oo)_{n,i}$ is closed under compositions in $\Fun(\Cc^n,\Oo_1)$, lifting against higher horn inclusions is trivial, and lifting of equivalences is due to the fact that $\operatorname{Day}(\Cc^\otimes,\Oo)_{n,i}$ is closed under equivalences in $\Fun(\Cc^n,\Oo_1)$. Moreover, since $\operatorname{Day}(\Cc^\otimes,\Oo)_{n,i}\subseteq\Fun(\Cc^n,\Oo_1)$ is injective as a map of simplicial sets, we get a pullback diagram
\begin{equation*}
	\begin{tikzcd}
		\operatorname{Day}(\Cc^\otimes,\Oo)_{n,i}\eqar[d]\eqar[r]\drar[pullback] & \operatorname{Day}(\Cc^\otimes,\Oo)_{n,i}\dar\\
		\operatorname{Day}(\Cc^\otimes,\Oo)_{n,i}\rar & \Fun(\Cc^n,\Oo_1)
	\end{tikzcd}
\end{equation*}
of simplicial sets. But its legs are isofibrations, hence this is also a pullback diagram in $\Cat_\infty$. Thus, for $\pr_i^*\colon \Fun(\Cc,\Oo_1)\morphism \Fun(\Cc^n,\Oo_1)$ to be an equivalence onto $\operatorname{Day}(\Cc^\otimes,\Oo)_{n,i}$, we would need that
\begin{equation*}
	\begin{tikzcd}
		\Fun(\Cc,\Oo_1)\eqar[d]\eqar[r]\drar[pullback] & \Fun(\Cc,\Oo_1)\dar["\pr_i^*"]\\
		\Fun(\Cc,\Oo_1)\rar["\pr_i^*"] & \Fun(\Cc^n,\Oo_1)
	\end{tikzcd}
\end{equation*}
is a pullback diagram in $\Cat_\infty$ as well. But this can't be true in general. For example, if the above diagram is a pullback for all $\infty$-operads $\Oo$, then \cref{cor:HomPreservesColimits} together with Yoneda's lemma imply that
\begin{equation*}
	\begin{tikzcd}
		\Cc^n\dar["\pr_i"']\rar["\pr_i"]\drar[pushout] & \Cc\eqar[d]\\
		\Cc\eqar[r] & \Cc
	\end{tikzcd}
\end{equation*}
is a pushout (I believe not every $\infty$-category $\Dd$ is of the form $\Dd\simeq \Oo_1$ for some $\infty$-operad $\Oo$, but for the Yoneda argument to work it suffices to have a fully faithful functor $\Dd\morphism \Oo_1$ for every $\infty$-category $\Dd$, which we always have; just equip $\Pp(\Dd)$ with the cocartesian monoidal structure from \cref{prop:CocartesianMonoidalStructure}). This pushout condition can't be true in general: Take $n=2$ and let $\Cc$ be your favourite non-discrete symmetric monoidal anima. Then the pushout condition would imply $\Sigma(\Cc)\simeq *$.

In a nutshell: $\operatorname{Day}(\Cc^\otimes,\Oo)_{n,i}\subseteq \Fun(\Cc^n,\Oo_1)$ is spanned by those functors and transformations, which factor up to equivalence over $\pr_i\colon \Cc^n\morphism \Cc$, whereas $\Fun(\Cc,\Oo_1)\subseteq \Fun(\Cc^n,\Oo_1)$ is spanned by those which do so on the nose. It seems obvious that these two (non-full) sub-$\infty$-categories are equivalent, but I think it's not true.

\refstepcounter{smallerdummy}
\numpar*{\thesmallerdummy. Why we don't need to worry*}\label{par:DontWorry}
First of all, Lurie \cite[Subsection~2.2.6]{HA} has his own construction of Day convolution. So $\infty$-operad theory is not in jeopardy. Second, Bastiaan has pointed out that the issues from \labelcref{par:DayConvolutionError} don't occur in the case where $\Cc$ is weakly contractible. Indeed, we claim:
\begin{alphanumerate}
 	\item[\itememph{\boxtimes}] \itshape If $\Cc$ is weakly contractible, then $\const\colon \Dd\morphism \Fun(\Cc,\Dd)$ is fully faithful for all $\infty$-categories $\Dd$.
\end{alphanumerate}
Applying \itememph{\boxtimes} inductively to $\Dd\simeq \Fun(\Cc^i,\Oo_1)$ for $i=1,\dotsc,n-1$ shows that $\Fun(\Cc,\Oo_1)\subseteq \Fun(\Cc^n,\Oo_1)$ is fully faithful and then everything works. To show \itememph{\boxtimes}, we may assume that $\Dd$ has small colimits since we can always replace $\Dd$ by $\Pp(\Dd)$. Then $\const$ has a left adjoint $\colimit_\Cc\colon \Fun(\Cc,\Dd)\morphism \Dd$ and the counit $\colimit_\Cc\const d\isomorphism d$ is an equivalence for all $d\in \Dd$ because the indexing $\infty$-category $\Cc$ is weakly contractible. Thus $\const$ is fully faithful by \cref{propdef:BousfieldLocalisation}\itememph{a}.

Now for all symmetric monoidal $\infty$-categories $\Cc^\otimes$ to which we are ever going to apply the Day convolution construction in this lecture, $\Cc$ has an initial or terminal object, so we won't get into trouble.

\begin{prop}\label{prop:DayConvolution}
	Let $\Cc^\otimes$ be a symmetric monoidal $\infty$-category such that $\Cc$ is weakly contractible \embrace{of course this condition is only added because of \textup{\labelcref{par:DontWorry}}}, let $\Oo$ be an $\infty$-operad and let $\operatorname{Day}(\Cc^\otimes,\Oo)$ be constructed as in \cref{con:DayConvolution}. Then:
	\begin{alphanumerate}
		\item $\operatorname{Day}(\Cc^\otimes,\Oo)\morphism \IGamma^\op$ is an $\infty$-operad with underlying $\infty$-category $\Fun(\Cc,\Oo_1)$.
		\item If $q\colon \Cc^\otimes\morphism \Cc'^\otimes$ is an $\infty$-operad map, or $p\colon \Oo\morphism\Oo'$ is one, so are
		\begin{equation*}
			q^*\colon \operatorname{Day}(\Cc'^\otimes,\Oo)\morphism\operatorname{Day}(\Cc^\otimes,\Oo)\quad\text{and}\quad p_*\colon \operatorname{Day}(\Cc^\otimes,\Oo)\morphism \operatorname{Day}(\Cc^\otimes,\Oo')\,.
		\end{equation*}
		\item If $\Oo\simeq \Dd^\otimes$, where $\Dd$ is a cocomplete symmetric monoidal category such that $-\otimes_\Dd -$ commutes with colimits in each variable, then $\operatorname{Day}(\Cc^\otimes,\Dd^\otimes)$ represents a symmetric monoidal structure on $\Fun(\Cc,\Dd)$ with tensor product
		\begin{equation*}
			(F_1\otimes_{\operatorname{Day}} F_2)(c)\simeq \colimit_{d\otimes_\Cc d'\morphism c} \big(F_1(d)\otimes_\Dd F_2(d')\big)\,.
		\end{equation*}
		To be really precise, the colimit on the right-hand side is taken over the slice category $(-\otimes_\Cc -)/c$, where $-\otimes_\Cc-\colon \Cc\times\Cc\morphism\Cc$ denotes the tensor product on $\Cc$. Moreover, the underlying $\infty$-category of $\operatorname{Day}(\Cc^\otimes,\Dd^\otimes)$ is again cocomplete and $-\otimes_{\operatorname{Day}}-$ commutes with colimits in either variable.
		
		\item In the situation of \itememph{c}, let $q\colon \Cc\morphism \Cc'$ be a strongly symmetric monoidal functor. Then $q^*$ has a left adjoint \embrace{left Kan extension} which is strongly symmetric monoidal.
	\end{alphanumerate}
\end{prop}
\begin{proof*}[Proof sketch]
	Let $\alpha\colon \langle m\rangle \morphism\langle n\rangle$ and define $\Cc_\alpha^\otimes$, $\Oo_\alpha$ as above. Then $\Hom_{\Cat_\infty/\IGamma^\op}(\Cc_\alpha^\otimes,\Oo)\simeq \Hom_{\Cat_\infty/[1]}(\Cc_\alpha^\otimes,\Oo)$, and likewise $\Hom_{\Cat_\infty/\IGamma^\op}(\Cc_m^\otimes,\Oo)\simeq \Hom_{\Cat_\infty}(\Cc_m^\otimes,\Oo_m)$. Using this and adapting the arguments from the proof sketch of \cref{prop:CocartesianMonoidalStructure}, one shows that we have pullback squares
	\begin{equation*}
		\begin{tikzcd}[column sep=scriptsize]
			\Hom_{\F(\Cc^\otimes,\Oo)}^\alpha(F,G)\rar\dar\drar[pullback] & \Hom_{\Cat_\infty/[1]}(\Cc_\alpha^\otimes,\Oo_\alpha)\dar["{(d_1,d_0)}"]\\
			*\rar["{(F,G)}"] & \Hom_{\Cat_\infty}(\Cc_m^\otimes,\Oo_m)\times \Hom_{\Cat_\infty}(\Cc_n^\otimes,\Oo_n)
		\end{tikzcd}
	\end{equation*}
	for all $F,G\in \F(\Cc^\otimes,\Oo)$ with images $\langle m\rangle$ and $\langle n\rangle$ respectively (but we do not claim that $\F(\Cc^\otimes,\Oo)$ is an $\infty$-operad). If we restrict each factor to the path components spanned by those functors that meet the requirements from \cref{con:DayConvolution}, we obtain a similar pullback diagram for $\Hom_{\operatorname{Day}(\Cc^\otimes,\Oo)}^\alpha(F,G)$. Also note that we can plug in and pull out the decompositions of $\Oo_\alpha$, $\Oo_m$ and $\Oo_n$ to make this pullback more confusing.
	
	To prove \itememph{a}, we must provide cocartesian lifts of inert morphisms. If $\alpha$ is inert, then lifting $\alpha$ to $\F(\Cc^\otimes,\Oo)$ means (by unravelling the pullback diagram above) we need to solve a lifting problem
	\begin{equation*}
		\begin{tikzcd}
			\Cc_m^\otimes\dar[mono,"d_1"']\rar & \Oo_m\dar[mono]\\
			\Cc_\alpha^\otimes\rar[dashed]& \Oo_\alpha
		\end{tikzcd}
	\end{equation*}
	Additionally, we want that the solution lies in $\operatorname{Day}(\Cc^\otimes,\Oo)$ and is cocartesian. Since $\alpha$ is inert, we have $\Cc_\alpha^\otimes\simeq \Cc^S\times \Cc^n\times[1]$ and $\Oo_\alpha\simeq \Oo_1^S\times\Oo_1^n\times[1]$, where again $S\subseteq\langle m\rangle$ is the subset where $\alpha$ isn't defined. Moreover, the top arrow in the diagram above represents an element of $\operatorname{Day}(\Cc^\otimes,\Oo)_m$, hence a functor $F\colon \Cc^m\morphism \Oo_1^m$ which decomposes as $F\simeq F_1\times\dotsb\times F_m$. Thus we can choose the dashed arrow in the diagram above to be $F_i$ on those factors with $i\in S$ and to be $F_i\times\id_{[1]}$ on those factors with $i\notin S$. This clearly defines a map in $\operatorname{Day}(\Cc^\otimes,\Oo)$, and to show that it's cocartesian one can adapt the arguments from the proof sketch of \cref{prop:CocartesianMonoidalStructure} and use the description of $\Hom_{\operatorname{Day}(\Cc^\otimes,\Oo)}^\alpha(-,-)$ given above. This shows that $\operatorname{Day}(\Cc^\otimes,\Oo)\morphism\IGamma^\op$ satisfies \cref{def:Operad}\itememph{a}. For whether or not the Segal condition from \cref{def:Operad}\itememph{b} holds, see our discussion in \labelcref{par:DayConvolutionError} and \labelcref{par:DontWorry}. Finally, the condition from \cref{def:Operad}\itememph{c} can again be checked using our description of $\Hom_{\operatorname{Day}(\Cc^\otimes,\Oo)}^\alpha(-,-)$. This finishes part~\itememph{a}.
	
	
	For \itememph{b}, we get functors $q^*\colon \F(\Cc'^\otimes,\Oo)\morphism\F(\Cc^\otimes,\Oo)$ and $p_*\colon \F(\Cc^\otimes,\Oo)\morphism \F(\Cc^\otimes,\Oo')$ since the formation of the presheaf $F$ in \cref{con:DayConvolution} is clearly functorial in $\Cc^\otimes$ and $\Oo$. So all we need to check is that $q^*$ and $p_*$ preserve the decomposition conditions from \cref{con:DayConvolution} as well as cocartesian lifts of inert morphisms. But both things are clear from our constructions.
	
	Now let's prove \itememph{c}. Let $F,G\in \F(\Cc^\otimes,\Oo)$ lie over $\langle m\rangle$ and $\langle n\rangle$. Then $F$ and $G$ define functors $F\colon \Cc^m\morphism\Dd^m$ and $G\colon \Cc^n\morphism\Dd^n$. The following claim will be used several times during the proof:
	\begin{alphanumerate}
		\item[\itememph{\boxtimes}] \itshape Let $\alpha\colon \langle m\rangle\morphism\langle n\rangle$ be an active morphism. By cocartesian unstraightening, $\alpha$ induces morphisms $\alpha_\Cc\colon \Cc^m\morphism\Cc^n$ and $\alpha_\Dd\colon \Dd^m\morphism\Dd^n$. Then
		\begin{equation*}
			\Hom_{\F(\Cc^\otimes,\Oo)}^\alpha(F,G)\simeq \Nat(\alpha_\Dd\circ F,G\circ \alpha_\Cc)\,.
		\end{equation*}
	\end{alphanumerate}
	Claim~\itememph{\boxtimes} follows easily from \cref{lem*:NotStraightening} and the pullback square at the beginning of the proof. Also note that if $F$ and $G$ belong to $\operatorname{Day}(\Cc^\otimes,\Dd^\otimes)$, then
	\begin{equation*}
		\Hom_{\F(\Cc^\otimes,\Dd^\otimes)}^\alpha(F,G)\simeq\Hom_{\operatorname{Day}(\Cc^\otimes,\Dd^\otimes)}^\alpha(F,G)
	\end{equation*}
	since $\operatorname{Day}(\Cc^\otimes,\Oo)\subseteq \F(\Cc^\otimes,\Oo)$ contains all lifts of the active morphism $\alpha$, as noted at the end of \cref{con:DayConvolution}. So \itememph{\boxtimes} gives a way of computing $\Hom_{\operatorname{Day}(\Cc^\otimes,\Dd^\otimes)}^\alpha(F,G)$.
	
	We'll use the recognition criterion from \labelcref{par:RecognitionCriterion} to show that $\operatorname{Day}(\Cc^\otimes,\Dd^\otimes)$ is symmetric monoidal. As usual, let $f_n\colon \langle n\rangle \morphism\langle 1\rangle$ denote the unique active morphism. Let $F\colon \Cc^n\morphism\Dd^n$ such that $F\simeq F_1\times\dotsb\times F_n$ and let $G\colon \Cc\morphism \Dd$. The maps $(f_n)_\Cc\colon \Cc^n\morphism \Cc$ and $(f_n)_\Dd\colon \Dd^n\morphism \Dd$ are given by the $n$-fold tensor products $\otimes_\Cc^n\colon \Cc^n\morphism\Cc$ and $\otimes_\Dd^n\colon \Dd^n\morphism\Dd$. Thus \itememph{\boxtimes} shows
	\begin{equation*}
		\Hom_{\operatorname{Day}(\Cc^\otimes,\Oo)}^\mathrm{act}\big((F_1,\dotsc,F_n),G\big)\simeq \Hom_{\operatorname{Day}(\Cc^\otimes,\Oo)}^{f_n}(F,G)\simeq \Nat(\otimes_\Dd^n\circ F,G\circ \otimes_\Cc^n)\,,
	\end{equation*}
	Hence a lift $F\morphism F'$ of $f_n$ is locally cocartesian iff $F'\colon \Cc\morphism\Dd$ is the left Kan extension of $\otimes_\Dd^n\circ F\colon \Cc^n\morphism\Dd$ along $\otimes_\Cc^n\colon \Cc^n\morphism\Cc$. Since $\Dd$ is cocomplete, these Kan extensions exist by \cref{thm:KanExtension}. In particular, for $F_1,F_2\colon \Cc\morphism\Dd$ we have that $F_1\otimes_{\operatorname{Day}}F_2$ is the left Kan extension of $F_1\otimes_\Dd F_2\colon \Cc^2\morphism \Dd$ along $\otimes_\Cc^2\colon \Cc^2\morphism\Cc$, which shows that the desired formula holds true (but we don't know yet that $-\otimes_{\operatorname{Day}}-$ defines a symmetric monoidal structure).
	
	So far we haven't used that $-\otimes_\Dd-$ commutes with colimits in either variable, but we'll need it to show that the locally cocartesian edges compose, so that $\operatorname{Day}(\Cc^\otimes,\Oo)\morphism\IGamma^\op$ is not only locally cocartesian, but honestly cocartesian by \cite[Proposition~IX.13]{HigherCatsII}. To show this, it will suffice to check that $-\otimes_{\operatorname{Day}}-$ is associative and unital. For associativity, it suffices to check that $F_1\otimes_\Dd (F_2\otimes_{\operatorname{Day}} F_3)\colon \Cc^2\morphism \Dd$ is the left Kan extension of $F_1\otimes_\Dd F_2\otimes_\Dd F_3\colon \Cc^3\morphism \Dd$ along $\id_\Cc\times \otimes_\Cc^2\colon \Cc^3\morphism\Cc^2$ (and likewise for $(F_1\otimes_{\operatorname{Day}} F_2)\otimes_\Dd F_3$), since then the fact that Kan extension composes will do the rest. Given $(c,c')\in \Cc^2$, we have $(\id_\Cc\times \otimes_\Cc^2)/(c,c')\simeq \id_\Cc/c\times \otimes_\Cc^2/c'$. Now $\{c\}$ is cofinal in $\id_\Cc/c$, hence $\{c\}\times \otimes_\Cc^2/c'$ is cofinal in $\id_\Cc/c\times \otimes_\Cc^2/c'$. The value at $(c,c')$ of the left Kan extension in question can therefore be computed as
	\begin{align*}
		\colimit_{(c,d\otimes_\Cc d')\morphism (c,c')}\big( F_1(c)\otimes_\Dd F_2(d)\otimes_\Dd F_3(d')\big)&\simeq F_1(c)\otimes_\Dd \colimit_{d\otimes_\Cc d'\morphism c'}\big(F_2(d)\otimes_\Dd F_3(d')\big)\\
		&\simeq F_1(c)\otimes_\Dd (F_2\otimes_{\operatorname{Day}}F_3)(c')\,,
	\end{align*}
	as desired. Here where we finally used that $-\otimes_\Dd-$ commutes with colimits. In the exact same way one shows $F\otimes_{\operatorname{Day}}1_{\operatorname{Day}}\simeq F$, where $1_{\operatorname{Day}}\colon \Cc\morphism\Dd$ is given by the left Kan extension of $1_\Dd\colon *\morphism \Dd$ along $1_\Cc\colon *\morphism\Cc$. This concludes the proof that $\operatorname{Day}(\Cc^\otimes,\Dd^\otimes)$ is symmetric monoidal. The additional assertion is clear since left Kan extension preserves colimits and colimits in functor categories are pointwise. This proves \itememph{c}.
	
	For \itememph{d}, note that $q^*\colon \Fun(\Cc',\Dd)\morphism \Fun(\Cc,\Dd)$ has a left adjoint $q_!$ given by left Kan extension. Using the definition of $-\otimes_{\operatorname{Day}}-$ via left Kan extension and the fact that left Kan extension composes, it's straightforward to check $q_!(F_1\otimes_{\operatorname{Day}}F_2)\simeq q_!(F_1)\otimes_{\operatorname{Day}} q_!(F_2)$ and $q_!(1_{\operatorname{Day}})\simeq 1_{\operatorname{Day}}$. Hence \labelcref{par:LaxMonoidalAdjoints}\itememph{c^*} can be applied to finish the proof.
\end{proof*}
\subsection{Stabilisation of \texorpdfstring{$\infty$}{Infinity}-Operads and the Tensor Product on Spectra}
\begin{defi}\label{def:OperadicLimits}
	Let $\Oo$ be an $\infty$-operad and $F\colon \Ii\morphism \Oo_1$ be a diagram. Then a colimit/limit of $F$ is called \emph{operadic} if for every $x_1,\dotsc,x_n,y\in \Oo_1$ we have
	\begin{align*}
		\Hom_\Oo^\mathrm{act}\left(\Big(\colimit_\Ii F,x_2,\dotsc,x_n\Big),y\right)&\simeq \limit_{\Ii^\op}\Hom_\Oo^\mathrm{act}\big((F(-),x_2,\dotsc,x_n),y\big)\\
		\Hom_\Oo^\mathrm{act}\left((x_1,\dotsc,x_n),\limit_\Ii F\right)&\simeq \limit_\Ii\Hom_\Oo^\mathrm{act}\big((x_1,\dotsc,x_n),F(-)\big)
	\end{align*}
	An $\infty$-operad is \emph{pointed/semi-additive/additive/stable} if $\Oo_1$ is and the required limits and colimits are operadic. Spelling this out explicitly, we obtain:
	\begin{alphanumerate}
		\item $\Oo$ is \emph{pointed} if $\Oo_1$ has a zero object which is also both operadic initial and operadic terminal. We define $\cat{Op}_\infty^*\subseteq \cat{Op}_\infty$ as the (non-full) sub-$\infty$-category spanned by $\infty$-operads with an operadic terminal object and maps preserving these. We denote by $\cat{Op}_\infty^\mathrm{pt}\subseteq \cat{Op}_\infty^*$ the full sub-$\infty$-category spanned by pointed $\infty$-operads.
		\item $\Oo$ is \emph{semi-additive/additive} if $\Oo_1$ is and all finite products and coproducts are operadic. We define $\cat{Op}_\infty^\times\subseteq \cat{Op}_\infty$ as the (non-full) sub-$\infty$-category spanned by $\infty$-operads with finite operadic products and maps preserving these. We denote by $\cat{Op}_\infty^\mathrm{semi\mhyph add},\cat{Op}_\infty^\mathrm{add}\subseteq \cat{Op}_\infty^*$ the full sub-$\infty$-categories spanned by semi-additive and additive $\infty$-operads, respectively.
		\item $\Oo$ is \emph{stable} if $\Oo_1$ is and all finite limits and colimits are operadic. We define $\cat{Op}_\infty^\mathrm{lex}\subseteq \cat{Op}_\infty$ as the (non-full) sub-$\infty$-category spanned by $\infty$-operads with finite operadic limits and maps preserving these. We denote by $\cat{Op}_\infty^\mathrm{st}\subseteq \cat{Op}_\infty^*$ the full sub-$\infty$-category spanned by stable $\infty$-operads.
	\end{alphanumerate}
\end{defi}
\begin{rem*}\label{rem*:OperadicLimits}
	If $\Oo\simeq \Cc^\otimes$ is a symmetric monoidal $\infty$-category, then all limits are operadic because $\Hom_{\Cc^\otimes}^\mathrm{act}((x_1,\dotsc,x_n),-)\simeq \Hom_\Cc(x_1\otimes\dotsb\otimes x_n,-)$, and a colimit is operadic iff $\colimit_\Ii(F(-)\otimes x)\simeq (\colimit_\Ii F)\otimes x$ for all $x\in\Cc$.
	
	In general, a limit in $\Oo_1$ is operadic iff it is also a limit in $\Oo$. Indeed, the limit condition in $\Oo$ means that
	\begin{equation*}
		\limit_{\Ii^\op}\Hom_\Oo^\alpha\left((x_1,\dotsc,x_n),\limit_\Ii F\right)\simeq \limit_\Ii\Hom_\Oo^\alpha\big((x_1,\dotsc,x_n),F(-)\big)
	\end{equation*}
	for every $\alpha\colon \langle n\rangle\morphism \langle 1\rangle$ in $\IGamma^\op$. The special case $\alpha=f_n$ shows that if the limit over $F\colon \Ii\morphism \Oo_1\subseteq\Oo$ happens to lie inside $\Oo_1$, then it is operadic. Conversely, any $\alpha$ can be factored into $\alpha=f_m\circ \iota$, where $\iota\colon \langle n\rangle\morphism\langle m\rangle$ is inert. If $x\morphism \iota_*x$ is a cocartesian lift of $\iota$ starting at $x=(x_1,\dotsc,x_n)$, then $\Hom_\Oo^\alpha(x,-)\simeq \Hom_\Oo^\mathrm{act}(\iota_*x,-)$, hence operadic limits are also limits in $\Oo$.
	
	Given the limit case, one might expect that a colimit is operadic iff $(\colimit_\Ii F,x_2,\dotsc,x_n)$ is a colimit in $\Oo$ over $(F(-),x_2,\dotsc,x_n)\colon \Ii\morphism \Oo$ for all $x_2,\dotsc,x_n\in\Oo_1$. But this is false unless $\Ii$ is weakly contractible! The reason why this fails is rather stupid: Similar to above, we must check the colimit condition on $\Hom_\Oo^\alpha$ for any $\alpha\colon \langle n\rangle \morphism\langle 1\rangle$. If we factor $\alpha=f_m\circ \iota$, then it may happen that $\iota$ isn't defined at $1\in\langle n\rangle$, and in this case the colimit condition reads $\Hom_\Oo^\mathrm{act}(\iota_*x,y)\simeq \limit_{\Ii^\op}\Hom_\Oo^\mathrm{act}(\iota_*x,y)$, which only holds in general if $\Ii$ is weakly contractible.
\end{rem*}


\begin{con}\label{con:Operads*CMonCGrpSp}
Given an $\infty$-operad $\Oo$ with sufficient operadic limits and colimits, our goal is to construct new $\infty$-operads
\begin{equation*}
	\Oo_*^{\Op_\infty}\,,\quad \CMon^{\Op_\infty}(\Oo)\,,\quad \CGrp^{\Op_\infty}(\Oo)\,,\quad\text{and}\quad \Sp^{\Op_\infty}(\Oo)
\end{equation*}
with underlying $\infty$-categories $*/\Oo_1$, $\CMon(\Oo_1)$, $\CGrp(\Oo_1)$, and $\Sp(\Oo_1)$ respectively.
	We will tackle all four constructions at once! Consider the following $\infty$-operads:
	\begin{equation*}
		[1]^{\min}\morphism\IGamma^\op\,,\quad (\IGamma^\op)^\times\morphism \IGamma^\op\,,\quad\text{and}\quad (\IDigamma^\op)^\wedge\morphism \IGamma^\op\,.
	\end{equation*}
	The first one is given by the symmetric monoidal structure on $[1]$ obtained by taking the minimum (in particular, $1\in [1]$ is the tensor unit). The second one is the cartesian symmetric monoidal structure (as introduced in \cref{prop:CartesianMonoidalStructure}) on $\IGamma^\op$. The third one is given by the smash product. We didn't construct this yet, but it will arise as follows: Starting from the cartesian symmetric monoidal structure $\An^\times$ on $\An$, the smash product will be defined as its \enquote{pointification} $(*/\An)^\wedge\coloneqq (\An^\times)_*^{\Op_\infty}$. Then $(\IDigamma^\op)^\wedge\subseteq(*/\An)^\wedge$ can be defined via \labelcref{par:LaxMonoidalAdjoints}\itememph{a}. So to be super precise, we would need to prove the assertions about $(-)_*^{\Op_\infty}$ in \cref{thm:NikolausStabilisationOfOperads} below before we could even formulate anything about $\Sp^{\Op_\infty}(-)$, but obviously that's not what we're going to do. We will also check below \cref{rem*:PropII.51ApplicableForP(D)} that the smash product coincides with the one you know from topology. 
	
	If $\Oo\in \Op_\infty^*$, we let
	\begin{equation*}
		\Oo_*^{\Op_\infty}\subseteq\operatorname{Day}\big([1]^{\min},\Oo\big)
	\end{equation*}
	denote the $\infty$-operad obtained via \labelcref{par:LaxMonoidalAdjoints} from the full sub-$\infty$-category $(*/\Oo_1)\subseteq \Fun([1],\Oo_1)$ spanned by those functors $F\colon [1]\morphism\Oo_1$ with $F(0)=*$. Similarly, if $\Oo\in\Op_\infty^\times$, we let
	\begin{equation*}
		\CGrp^{\Op_\infty}(\Oo)\subseteq\CMon^{\Op_\infty}(\Oo)\subseteq\operatorname{Day}\big((\IGamma^\op)^\times,\Oo\big)
	\end{equation*}
	be induced by $\CGrp(\Oo_1)\subseteq\CMon(\Oo_1)\subseteq\Fun(\IGamma^\op,\Oo_1)$, i.e.\ the full sub-$\infty$-categories of functors satisfying the Segal condition and---in the case of $\CGrp(\Oo_1)$---additionally the condition from \cref{def:E1Group}. Finally, for $\Oo\in\Op_\infty^\mathrm{lex}$, we let
	\begin{equation*}
		\Sp^{\Op_\infty}(\Oo)\subseteq \operatorname{Day}\big((\IDigamma^\op)^\wedge,\Oo\big)
	\end{equation*}
	be given by the full sub-$\infty$-category $\Sp(\Oo_1)\subseteq\Fun(\IDigamma^\op,\Oo_1)$ spanned by reduced and excisive functors (see \cref{def:OtherModelForSpectra}).
	
	These $\infty$-operads come equipped with canonical maps
	\begin{equation*}
		\Oo\lmorphism\Oo_*^{\Op_\infty}\lmorphism\CMon^{\Op_\infty}(\Oo)\lmorphism\CGrp^{\Op_\infty}(\Oo)\lmorphism[\Omega^\infty] \Sp^{\Op_\infty}(\Oo)\,.
	\end{equation*}
	Except for the fourth one, which comes from the inclusion $\CMon(\Oo_1)\supseteq \CGrp(\Oo_1)$, these maps are induced via functoriality of Day convolution (see \cref{prop:DayConvolution}\itememph{b}) by the $\infty$-operad maps
	\begin{equation*}
		\cat{\IC omm}\morphism{} [1]^{\min}\morphism (\IGamma^\op)^\times\morphism (\IDigamma^\op)^\wedge\,.
	\end{equation*}
	All of these are induced by strongly monoidal functors between the underlying categories: The first one comes from $1\colon *\morphism{} [1]$, the second one from $[1]\morphism\IGamma^\op$ sending $i\mapsto \langle i\rangle$ for $i=0,1$, and the third one comes from the functor $(-)_+\colon \IGamma^\op\morphism \IDigamma^\op$ introduced in \cref{def:OtherModelForSpectra}.
\end{con}
With these constructions we can now formulate a striking theorem of Thomas Nikolaus!
\begin{thm}[{\cite[Theorem~4.11 and Theorem~5.7]{NikolausStable}}]
	\label{thm:NikolausStabilisationOfOperads}
	The $\infty$-operads $\Oo_*^{\Op_\infty}$, $\CMon^{\Op_\infty}(\Oo)$, $\CGrp^{\Op_\infty}(\Oo)$, and $\Sp^{\Op_\infty}(\Oo)$ from \cref{con:Operads*CMonCGrpSp} are pointed, semi-additive, additive and stable respectively. Moreover, the respective constructions assemble into functors
	\begin{align*}
		(-)_*^{\Op_\infty}\colon \Op_\infty^*&\morphism \Op_\infty^\mathrm{pt}\,,&\CMon^{\Op_\infty}(-)\colon \Op_\infty^\times&\morphism \Op_\infty^\mathrm{semi\mhyph add}\,,\\
		\CGrp^{\Op_\infty}(-)\colon \Op_\infty^\times&\morphism \Op_\infty^\mathrm{add}\,,&\Sp^{\Op_\infty}(-)\colon \Op_\infty^\mathrm{lex}&\morphism \Op_\infty^\mathrm{st}\,,
	\end{align*}
	each of which is right-adjoint to the respective fully faithful inclusion in the other direction \embrace{i.e., they are all right Bousfield localisations}.
\end{thm}
\begin{proof*}[Proof sketch]
	Once we know that our four functors take values in the correct $\infty$-categories, we can proceed as in \cref{thm:CMonCGrpAdjunctions} and \cref{thm:SpLeftAdjoint} to show that they are right Bousfield localisations. So we only need to prove the first part of the theorem.
	
	The most subtle part is to show that the required colimits in the $\infty$-operads $\Oo_*^{\Op_\infty}$, $\CMon^{\Op_\infty}(\Oo)$, $\CGrp^{\Op_\infty}(\Oo)$, and $\Sp^{\Op_\infty}(\Oo)$ are actually operadic. Thomas Nikolaus does this with a beautiful trick: Let's first assume $\Oo\simeq \Cc^\otimes$, where $\Cc$ is a symmetric monoidal $\infty$-category for which \cref{prop:Sp(C)Monoidal} below is applicable. Then this proposition says that in fact all small colimits are operadic and we're happy. In general, one can find a fully faithful map of $\infty$-operads $i\colon\Oo\morphism \Cc^\otimes$ such that \cref{prop:Sp(C)Monoidal} is applicable to $\Cc^\otimes$ and $i_1\colon \Oo_1\morphism\Cc$ preserves the required finite limits, see \cite[Proposition~2.7]{NikolausStable} and \cref{rem*:PropII.51ApplicableForP(D)}. The required finite colimits in $*/\Oo_1$, $\CMon(\Oo_1)$, $\CGrp(\Oo_1)$, and $\Sp(\Oo_1)$ are also finite colimits in $*/\Cc$, $\CMon(\Cc)$, $\CGrp(\Cc)$, and $\Sp(\Cc)$ since each of them (zero objects, finite coproducts in semi-additive $\infty$-categories, finite colimits in stable $\infty$-categories) can be built from finite limits which $i_1\colon \Oo_1\morphism\Cc$ preserves. Hence they are operadic and we win.
\end{proof*}
In general, neither of the constructions from \cref{con:Operads*CMonCGrpSp} gives a symmetric monoidal category, even when $\Oo$ itself is symmetric monoidal. The following proposition gives a criterion for the symmetric monoidal structure to be inherited.
\begin{prop}\label{prop:Sp(C)Monoidal}
	Suppose $\Oo\simeq\Cc^\otimes$ is a symmetric monoidal $\infty$-categories with all small operadic colimits \embrace{this is a short way of saying that $\Cc$ is cocomplete and $-\otimes_\Cc-$ commutes with colimits in either variable, see \cref{rem*:OperadicLimits}}. Suppose furthermore that the inclusions
	\begin{align*}
		*/\Cc&\subseteq \Ar(\Cc)\,,&\CMon(\Cc)&\subseteq \Fun(\IGamma^\op,\Cc)\,,\\
		\CGrp(\Cc)&\subseteq \Fun(\IGamma^\op,\Cc)\,,& \Sp(\Cc)&\subseteq \Fun(\IDigamma^\op,\Cc)
	\end{align*}
	have left adjoints. Then $(\Cc^\otimes)_*^{\Op_\infty}$, $\CMon^{\Op_\infty}(\Cc^\otimes)$, $\CGrp^{\Op_\infty}(\Cc^\otimes)$, and $\Sp^{\Op_\infty}(\Cc^\otimes)$ are symmetric monoidal $\infty$-categories again and have all small operadic colimits. Moreover, there are maps $\infty$-operads
	\begin{align*}
		(-)_+\colon\Cc^\otimes&\morphism(\Cc^\otimes)_*^{\Op_\infty}\,,&\Free^\CMon\colon \Cc^\otimes&\morphism\CMon^{\Op_\infty}(\Cc^\otimes)\,,\\
		\Free^\CGrp(-)\colon \Cc^\otimes&\morphism \CGrp^{\Op_\infty}(\Cc^\otimes)\,,& \Sigma^\infty\colon \Cc^\otimes&\morphism \Sp^{\Op_\infty}(\Cc^\otimes)
	\end{align*}
	which are strongly monoidal and left-adjoint to the respective maps in the other direction constructed at the end of \cref{con:Operads*CMonCGrpSp}. Similar assertions hold for other left adjoints among these. For instance, in the case $\Cc^\otimes=\An^\times$ we get a fully faithful strongly monoidal left adjoint $B^\infty\colon \CGrp^{\Op_\infty}(\An^\times)\morphism \Sp^{\Op_\infty}(\An^\times)$ of $\Omega^\infty$.
\end{prop}
\begin{proof*}
	We only prove the assertions about $\Sp$, since the arguments can be copied verbatim for the other cases. By \cref{prop:DayConvolution}\itememph{c}, $\operatorname{Day}((\IDigamma^\op)^\wedge,\Cc^\otimes)$ is symmetric monoidal and has all small operadic colimits. Since $\Sp(\Cc)\subseteq\Fun(\IDigamma^\op,\Cc)$ has a left adjoint, we may apply \labelcref{par:LaxMonoidalAdjoints}\itememph{b} to see that $\Sp^{\Op_\infty}(\Cc^\otimes)$ is again symmetric monoidal and its inclusion into $\operatorname{Day}((\IDigamma^\op)^\wedge,\Cc^\otimes)$ has a left adjoint $\operatorname{Day}((\IDigamma^\op)^\wedge,\Cc^\otimes)\morphism\Sp^{\Op_\infty}(\Cc^\otimes)$ which is strongly monoidal. Together with our computation of colimits in Bousfield localisations (\cref{cor:CoLimitsInBousfield}) this shows that $\Sp^{\Op_\infty}(\Cc^\otimes)$ has all small operadic colimits.
	
	Moreover, to produce the strongly monoidal left adjoint $\Sigma^\infty\colon \Cc^\otimes\morphism \Sp^{\Op_\infty}(\Cc^\otimes)$, it suffices to construct a stronglyy monoidal left adjoint of the canonical map
	\begin{equation*}
		\operatorname{Day}\big((\IDigamma^\op)^\wedge,\Cc\big)\morphism\operatorname{Day}(\cat{\IC omm},\Cc^\otimes)\simeq \Cc^\otimes
	\end{equation*}
	induced by the strongly monoidal map $\cat{\IC omm}\morphism (\IDigamma^\op)^\wedge$. But this is precisely what \cref{prop:DayConvolution}\itememph{d} does! The additional assertion about $B^\infty$ can be shown in the same way.
\end{proof*}
\begin{rem*}\label{rem*:PropII.51ApplicableForP(D)}
	\cref{prop:Sp(C)Monoidal} is always applicable when $\Cc^\otimes\simeq \operatorname{Day}(\Dd^\otimes,\An^\times)$ for some small symmetric monoidal $\infty$-category $\Dd$. That is, whenever $\Cc\simeq \Pp(\Dd)$, equipped with the Day convolution structure. Since $\Cc^\otimes$ in the proof of \cref{thm:NikolausStabilisationOfOperads} can be always chosen in this way (see the proof of \cite[Proposition~2.7]{NikolausStable}), this is all we need.
	
	So let's prove that the assumptions of \cref{prop:Sp(C)Monoidal} are really satisfied for $\Cc\simeq \Pp(\Dd)$! First of all, $\Pp(\Dd)$ has all small operadic colimits by \cref{prop:DayConvolution}\itememph{c}, so only the required left adjoints have to be constructed. Let's first do the case of $\Sp$. By Proposition~\labelcref{prop:OtherModelSpectrification}, the inclusion $\Sp(\Cc)\subseteq \Fun_*(\IDigamma^\op,*/\Cc)$ into the reduced functors has a left adjoint whenever $*/\Cc$ has sequential colimits and $\Omega\colon {*/\Cc}\morphism*/\Cc$ commutes with them. This is satisfied for $*/\An$, as we checked after \cref{prop:Spectrification}, hence also for $\const */\Pp(\Dd)\simeq \Fun(\Dd^\op,*/\An)$ since everything is pointwise. Now $\Fun_*(\IDigamma^\op,*/\Cc)\simeq \Fun_*(\IDigamma^\op,\Cc)$, since every reduced functor $F\colon \IDigamma^\op\morphism\Cc$ canonically lifts to $*/\Cc$, so it suffices to construct a left adjoint of $\Fun_*(\IDigamma^\op,\Cc)\subseteq\Fun(\IDigamma^\op,\Cc)$. If $\Cc$ has pushouts, we can send an arbitrary $F\colon \IDigamma^\op\morphism\Cc$ to $\cofib(F(*)\morphism F(-))\colon \IDigamma^\op\morphism \Cc$ to obtain the desired left adjoint. Again, $\Pp(\Dd)$ clearly has pushouts, whence we are done.
	
	Next, we treat $\CMon$. Denote $i\colon \IGamma^\op_{\leq 1}\monomorphism \IGamma^\op$. If $\Cc$ has small limits, then restriction and right Kan extension along $i$ gives a functor $i_*i^*\colon \Fun_*(\IGamma^\op,\Cc)\morphism \CMon(\Cc)$. This is a left Bousfield localisation, which can be easily checked using Proposition~\labelcref{prop:LLaddendum} and an assertion similar to claim~\itememph{\boxtimes} from the proof of \cref{prop:CMonOfSemiAdditive}. Composing this with a left Bousfield localisation $\Fun(\IGamma^\op,\Cc)\morphism\Fun_*(\IGamma^\op,\Cc)$ as above gives the required localisation $\Fun(\IGamma^\op,\Cc)\morphism\CMon(\Cc)$ in the $\CMon$ case. This immediately implies the $\CGrp$ case, since $\CGrp(\An)\subseteq\CMon(\An)$ has a left adjoint by \cref{cor:CommutativeHorizontalAdjoints}, hence the same is true for $\CGrp(\Pp(\Dd))\simeq \Fun(\Dd,\CGrp(\An))\subseteq\Fun(\Dd,\CMon(\An))\simeq \CMon(\Pp(\Dd))$.
	
	I'll leave the final case, i.e.\ showing that $*/\Cc\subseteq\Ar(\Cc)$ has a left adjoint in the special case $\Cc\simeq\Pp(\Dd)$, to you.
\end{rem*}
\refstepcounter{smallerdummy}
\numpar*{\thesmallerdummy. Example}\label{exm:TensorProductCalculations}
Let's discuss two special cases of \cref{prop:Sp(C)Monoidal}. The first one wasn't in the lecture, but it will show up later.
\begin{alphanumerate}
	\item We claim that $\CGrp^{\Op_\infty}(\Set^\times)\simeq \Ab^\otimes$ is the symmetric monoidal structure on the $1$-category of abelian groups given by the usual tensor product. Indeed, let $A,B\in\Ab$, thought of as functors $A,B\colon \IGamma^\op\morphism\Set$. Let $A\otimes B$ denote their Day convolution $\operatorname{Day}((\IGamma^\op)^\times,\Set^\times)$. By \cref{prop:DayConvolution}\itememph{c}, we can compute it as the left Kan extension of
	\begin{equation*}
		\begin{tikzcd}
			\IGamma^\op\times\IGamma^\op\dar["\times_{\IGamma^\op}"']\rar["A\times B"] & \Set\times\Set\rar["\times_{\Set}"]& \Set\\
			\IGamma^\op\ar[urr, bend right=15, dashed,"A\otimes B"']
		\end{tikzcd}
	\end{equation*}
	By the universal property of left Kan extension, this means that
	\begin{equation*}
		\Hom_\Ab(A\otimes B,C)\simeq \Nat(A\otimes B,C)\simeq \Nat\left(\times_\Set\circ (A\times B),C\circ \times_{\IGamma^\op}\right)
	\end{equation*}
	for all abelian groups $C$. Unravelling, we find that the right-hand side is given by $\IZ$-bilinear maps $A\times B\morphism C$, hence $A\otimes B$ is indeed the usual tensor product.
	
	The inclusion $\Set\subseteq\An$ induces inclusions $\Ab\simeq \CGrp(\Set)\subseteq\CGrp(\An)$ and $\Ab^\otimes\simeq \CGrp^{\Op_\infty}(\Set^\times)\subseteq \CGrp^{\Op_\infty}(\An)$, which are fully faithful again. To see this, note that the left adjoint $\pi_0\colon \An\morphism\Set$ induces left adjoints for each of the inclusions, and the condition that the counit transformation is an equivalence is inherited, hence so is fully faithfulness by \cref{propdef:BousfieldLocalisation}\itememph{a}.
	\item \cref{prop:Sp(C)Monoidal} shows that $*/\An$ carries a pointed symmetric monoidal structure, called the \emph{smash product}, via $(*/\An)^\wedge\coloneqq (\An^\times)_*^{\Op_\infty}$. We wish to compute  $-\wedge -$ explicitly to show that it coincides with the smash product we know from topology.
	
	So let $X,Y\in */\An\subseteq\Ar(\An)$. Let's first compute the Day convolution $X\times_{\operatorname{Day}}Y$: Using the explicit formula from \cref{prop:DayConvolution}\itememph{c}, we obtain
	\begin{align*}
		(X\times_{\operatorname{Day}}Y)(0)&\simeq\colimit\left(\begin{tikzcd}[ampersand replacement=\&]
			*\times *\dar\rar \& *\times Y\\
			X\times *
		\end{tikzcd}\right)\simeq X\vee Y\,,\\
		(X\times_{\operatorname{Day}}Y)(1)&\simeq\colimit\left(\begin{tikzcd}[ampersand replacement=\&]
			*\times *\dar\rar\drar  \& *\times Y\dar\\
			X\times *\rar \& X\times Y
		\end{tikzcd}\right)\simeq X\times Y\,.
	\end{align*}
	Since the left adjoint of $*/\An\subseteq\Ar(\An)$ sends any $T\colon [1]\morphism\An$ to $\cofib(T(0)\morphism T(1))$, we obtain $X\wedge Y\simeq \cofib(X\wedge Y\morphism X\times Y)$ after unravelling what the proofs* of \cref{prop:Sp(C)Monoidal} and \labelcref{par:LaxMonoidalAdjoints}\itememph{b} actually do. Hence $X\wedge Y$ is what we would expect.
\end{alphanumerate}


\begin{defi}\label{def:SpTensorProduct}
	The \emph{tensor product} (or traditionally \emph{smash product}) of spectra is defined by the $\infty$-operad $\Sp^\otimes\coloneqq \Sp^{\Op_\infty}(\An^\times)$, which is a symmetric monoidal $\infty$-category by \cref{prop:Sp(C)Monoidal}.
\end{defi}
\section{Some Brave New Algebra}
\lecture[Some tensor product calculations. $\IE_\infty$-ring spectra. The $\infty$-operads $\cat{\IA ssoc}$ and $\cat{\IL Mod}$, categories of modules over $\IE_\infty$-ring spectra.]{2021-01-07}The term \enquote{brave new algebra} roughly refers to the idea that $\Sp$ should behave like the category of abelian groups, and more generally to notion of \emph{$\IE_\infty$-ring spectra} and modules over them, which should behave very much like ordinary commutative rings and modules. Making this precise and proving it rigorously takes Lurie about 400 pages in \cite{HA}! Of course there's no way to even remotely cover this in this course. So be warned that proofs will become more and more sparse towards the end of this final subsection of \cref{chap:Monoidal}. Nevertheless, Fabian made it his goal to introduce all relevant definitions and theorems in the lecture, and even more material in the lecture notes \cite[Chapter~II pp.\:104--132]{KTheory}, to give us at least a guide for how to read Lurie. Starting from \cref{chap:GroupCompletion}, we will then just assume that homological algebra works with $\IE_\infty$-ring spectra just as it does for ordinary rings.


\numpar{First Steps with the Tensor Product on $\Sp$}\label{par:SpTensorProduct}
Recall the symmetric monoidal structure on $\Sp$ from \cref{def:SpTensorProduct}.
%\begin{alphanumerate}
	%\item 
	
	The tensor unit $1_\Sp$ is given by the sphere spectrum $\IS$! %To see this, we compute the tensor unit $1_{\operatorname{Day}}$ of $\operatorname{Day}((\IF^\op)^\wedge,\An^\times)$ first. Since $(\IS^0,*)\in \IF^\op$ and $*\in \An$ are the respective tensor units,  $1_{\operatorname{Day}}$ can be computed as the left Kan extension of $*\colon *\morphism \An$ along $(\IS^0,*)\colon *\morphism\IF^\op$, hence
	%\begin{equation*}
	%	1_{\operatorname{Day}}(X,x)\simeq \colimit_{(\IS^0,*)\morphism (X,x)} \const *\simeq \colimit_X \const *\simeq X\,.
	%\end{equation*}
	%This is already reduced, hence $1_\Sp$ can be obtained by applying the left adjoint from Proposition~\labelcref{prop:OtherModelSpectrification} to it. This shows
	%\begin{equation*}
	%	1_\Sp(\IS^i)\simeq \colimit_{n\in\IN}\Omega^n\Sigma^n\IS^i\simeq\colimit_{n\in\IN}\Omega^n\IS^{n+i}\,.
	%\end{equation*}
	%Hence $1_\Sp$ becomes indeed the sphere spectrum under the equivalence from \cref{prop:OtherModelForSpectra}. 
	The quickest way to see this is that $\IS[-]\colon \An^\times\morphism\Sp^\otimes$ is strongly symmetric monoidal because so are both its constituents $(-)_+\colon \An^\times\morphism (*/\An)^\wedge$ and $\Sigma^\infty\colon (*/\An)^\wedge\morphism\Sp^\otimes$ by \cref{prop:Sp(C)Monoidal}, and the tensor unit $*\in \An$ is sent to $\IS$.
	%\item 
	
	As an easy consequence, we obtain
	\begin{equation*}
		X\otimes E\simeq \IS[X]\otimes E\quad\text{and}\quad (X,x)\otimes E\simeq \Sigma^\infty(X,x)\otimes E\,,
	\end{equation*}
	for all $(X,x)\in */\An$ and $E\in \Sp$, where $X\otimes E$ and $(X,x)\otimes E$ are defined as in the proof of \cref{prop:OtherModelForSpectra}. Indeed, for the left-hand equivalence it's enough to show that both $-\otimes E\colon \An\morphism\Sp$ and $\IS[-]\otimes E\colon \An\morphism\Sp$ preserve colimits and that they agree on $*\in \An$, for then they must agree by \cref{thm:ColimitPreservingRepresentable}. That both sides agree on $*$ precisely says that $\IS$ is the tensor unit, which we just checked. That $-\otimes E$ preserves colimits is clear by inspection, whereas for $\IS[-]\otimes E$ it follows from the fact that the tensor product on spectra commutes with colimits in either variable by \cref{prop:Sp(C)Monoidal}. Hence the left-hand equivalence holds. The right-hand one follows, as $(X,x)\otimes E\simeq \cofib(\{x\}\otimes E\morphism X\otimes E)$ and $\Sigma^\infty(X,x)\simeq \cofib(\IS[\{x\}]\morphism\IS[X])$.
	%\item 
	
	This can be used to derive a general formula for the tensor product of spectra. Any spectrum $E$ can be written as
	\begin{equation*}
		E\simeq \colimit_{n\in\IN} \big(\Sigma^\infty(\Omega^{\infty-n}E)\big)[-n]\,.
	\end{equation*}
	Indeed, using the $(\Sigma^\infty,\Omega^\infty)$-adjunction, we can calculate
	\begin{align*}
		\Hom_\Sp\left(\colimit_{n\in\IN} \big(\Sigma^\infty\big(\Omega^{\infty-n}E)\big)[-n],E'\right)&\simeq\limit_{n\in\IN^\op}\Hom_\Sp\big(\Sigma^\infty(\Omega^{\infty-n} E),E'[n]\big)\\
		&\simeq \limit_{n\in\IN^\op}\Hom_{*/\An}(\Omega^{\infty-n} E,\Omega^{\infty-n} E')\,,
	\end{align*}
	and the right-hand side is $\Hom_\Sp(E,E')$ since $\Hom$ anima in limits are limits of $\Hom$ anima. Hence the claimed equivalence follows from Yoneda's lemma.
	
	Together with the fact that the tensor product of spectra commutes with arbitrary colimits, this shows the general formula
	\begin{equation*}
		E\otimes E'\simeq \colimit_{n\in \IN} \big(\Sigma^\infty(\Omega^{\infty-n}E)\big)[-n]\otimes E'\simeq \colimit_{n\in\IN}\big(\Omega^{\infty-n}E\otimes E'\big)[-n]\,,
	\end{equation*}
	as was already announced after \cref{def:SpectraCohomologyTheory}. Note that the tensor products on the right-hand side are no longer those in $\Sp$, but defined as in the proof of \cref{prop:OtherModelForSpectra}, so this formula gives---at least in theory---a way to compute tensor products of spectra.
	
	In real life though, the colimits in question are seldom computable. To give an example of how complicated even simple tensor products can become, write 
	\begin{equation*}
		\Tor_i^\IS(E,E')\coloneqq \pi_*(E\otimes E')\quad\text{and}\quad\Ext_\IS^i(E,E')\simeq \pi_i\hom_\Sp(E,E')\,,
	\end{equation*}
	and consider $\Tor_*^\IS(H\IF_p,H\IF_p)=\Aa_p^\vee$ and $\Ext_\IS^*(H\IF_p,H\IF_p)=\Aa_p$, called the (\emph{dual} and \emph{non-dual}) \emph{Steenrod algebra}. These are infinite-dimensional graded $\IF_p$-algebras, but completely computed (albeit not by the formula above). For example, $\Aa_2^\vee=\IF_2\left[x_i\st i \in\IN\right]$ with $\deg(x_i)=2^i-1$ is a polynomial ring in infinitely many variables.
%\end{alphanumerate}

To show that our formula isn't completely useless, I'd like to use it to prove the tensor--$\Hom$ adjunction. This wasn't discussed in the lecture, but Fabian has also written out a proof in the official lecture notes, \cite[Remarks~II.53(iv)]{KTheory}.
\begin{lem*}\label{lem*:TensorHomAdjunction}
	For all $E,E',E''\in\Sp$, we have an equivalence
	\begin{equation*}
		\hom_\Sp(E\otimes E',E'')\simeq \hom_\Sp\big(E,\hom_\Sp(E',E'')\big)
	\end{equation*}
	It is functorial in all three variables, i.e.\ an equivalence in $\Fun(\Sp^\op\times\Sp^\op\times\Sp,\Sp)$.
\end{lem*}
\begin{proof*}
	Let's first check it in the special case $E'\simeq \IS[X]$. Note that $\hom_\Sp(\IS,-)\colon \Sp\morphism\Sp$ is the identity functor. Indeed, we've already verified $\Hom_\Sp(\IS,-)\simeq \Omega^\infty$ (see before \cref{lem:HomologyOfS}), and $\Omega^\infty\colon \Sp\morphism\An$ clearly lifts to $\id_\Sp$ via \cref{thm:SpLeftAdjoint}. Hence the functors
	\begin{equation*}
		\hom_\Sp(E\otimes\IS[-],E'')^\op,\,\hom_\Sp\big(E,\hom_\Sp(\IS[-],E'')\big)^\op\colon \An\morphism\Sp^\op
	\end{equation*}
	agree on $*\in\An$ and both clearly preserve colimits, hence they must agree by \cref{thm:ColimitPreservingRepresentable}. This settles the special case $E'\simeq \IS[X]$. More generally, this shows
	\begin{equation*}
		\hom_\Sp(-\otimes\IS[-],-)^\op\simeq \hom_\Sp(-,\hom_\Sp(\IS[-],-))^\op\,.
	\end{equation*}
	Indeed, we can interpret both sides as functors $\Sp^\op\times\Sp\morphism\Fun(\An,\Sp^\op)$. Upon inspection, they land in the full sub-$\infty$-category $\Fun^{\colimit}(\An,\Sp^\op)$ of colimit-preserving functors. But $\Fun^{\colimit}(\An,\Sp^\op)\simeq \Fun(*,\Sp^\op)$, and if we interpret the two functors above as functors $\Sp^\op\times\Sp\morphism\Fun(*,\Sp^\op)$ then they clearly agree since both $-\otimes\IS$ and $\hom_\Sp(\IS,-)$ are equivalent to the identity. The case $E'\simeq \Sigma^\infty(X,x)$, and more generally the equivalence of functors
	\begin{equation*}
		\hom_\Sp(-\otimes\Sigma^\infty\,-,-)\simeq \hom_\Sp\big(-,\hom_\Sp(\Sigma^\infty\,-,-)\big)
	\end{equation*}
	in $\Fun(\Sp^\op\times(*/\An)^\op\times\Sp,\Sp)$ immediately follows.
	
	Finally, we can write $E'\simeq \colimit_{n\in\IN} (\Sigma^\infty(\Omega^{\infty-n}E'))[-n]$, and then the formula from \cref{par:SpTensorProduct} reduces the general case to the case $E'\simeq \Sigma^\infty(X,x)$. This also proves the full functoriality statement, since we can similarly write $\id_{\Sp}\simeq\colimit_{n\in\IN}(\Sigma^\infty(\Omega^{\infty-n}\,-))[-n]$.
\end{proof*}

\subsection{\texorpdfstring{$\IE_\infty$}{E-infinity}-Ring Spectra}

\numpar{Definition and Examples}\label{par:EinftyRingSpectra}
For any symmetric monoidal $\infty$-operad, we define $\cat{CAlg}(\Cc^\otimes)\coloneqq \Alg_\IComm(\Cc^\otimes)$. We already introduced the notation $\CMon(\Cc^\otimes)$ for this in \cref{exm:MyFirstAlgebrasOverOperads}\itememph{c}, but nevermind. In the special case $\Cc^\otimes\simeq \Sp^\otimes$, we put
\begin{equation*}
	\cat{CAlg}\coloneqq \CAlg(\Sp^\otimes)=\cat{Alg}_{\cat{\IC omm}}(\Sp^\otimes)
\end{equation*}
and call this the \emph{$\infty$-category of $\IE_\infty$-ring spectra}. Any  $\IE_\infty$-ring spectrum has an underlying spectrum, given by its image under $\ev_{\langle 1\rangle}\colon \CAlg\simeq \Fun^{\Op_\infty}(\IComm,\Sp^\otimes)\morphism \Sp$. Occasionally (and abusingly) we won't distinguish between an $\IE_\infty$-ring spectrum and its underlying spectrum.

In our picture where $\Sp$ corresponds to the $1$-category $\Ab$ of abelian groups, $\cat{CAlg}$ takes the place of the $1$-category $\cat{CRing}$ of commutative rings (which makes a lot of sense, as $\cat{CRing}\simeq\CAlg(\Ab^\otimes)$ by unwinding of definitions). Now come \enquote{my first $\IE_\infty$-ring spectra}:
\begin{alphanumerate}
	\item Recall from \cref{par:SpTensorProduct} that $\IS[-]\colon \An^\times\morphism\Sp^\otimes$ is strongly monoidal, and that $\CAlg(\An^\times)\simeq \CMon(\An)$ by \cref{thm:OMon}. Hence $\IS[-]$ upgrades to a functor
	\begin{equation*}
		\IS[-]\colon \CMon(\An)\morphism \cat{CAlg}\,.
	\end{equation*}
	For $M\in \CMon(\An)$, we think of $\IS[M]\in\cat{CAlg}$ as the \enquote{spherical monoid ring on $M$}.
	\item Every ordinary commutative ring defines an $\IE_\infty$-ring spectrum, and more generally so does every commutative differential graded algebra (CGDA in the following). We will discuss this in detail in \cref{lem:HLaxMonoidal}, \cref{cor:CRingToCAlgFullyFaithful}, and the space in between.
\end{alphanumerate}
Let me also mention another example, that came only up at a later point in the course, but fits here very well.
\begin{alphanumerate}
	\item[\itememph{c^*}] Let $R$ be a commutative ring and $k(R)$ its algebraic $K$-theory space from \cref{def:AlgebraicKTheory}. We know from \cref{exm:CommutativeHorizontalAdjoints}\itememph{a} that it is an $\IE_\infty$-group. But $B^\infty k(R)$ even has a canonical refinement as an $\IE_\infty$-ring spectrum! To see this, put
	\begin{equation*}
		\CMon(\Grpd)^\otimes\coloneqq\CMon^{\Op_\infty}(\Grpd^\times)\quad\text{and}\quad\CMon(\An)^\otimes\coloneqq \CMon^{\Op_\infty}(\An^\times)\,. 
	\end{equation*}
	The tensor product $-\otimes_R-$ turns $\Proj(R)$, which is already an object of $\CMon(\Grpd)$ via $-\oplus -$, into an object of $\CAlg(\CMon(\Grpd)^\otimes)\subseteq \CAlg(\CMon(\An)^\otimes)$. The functors $(-)^\inftygrp\colon \CMon(\An)^\otimes\morphism \CGrp(\An)^\otimes$ and $B^\infty\colon \CGrp(\An)^\otimes\morphism \Sp^\otimes$ are strongly monoidal by \cref{prop:Sp(C)Monoidal} (the former is hidden under \enquote{similar assertions hold for other left adjoints among these}), hence $B^\infty k(R)\simeq B^\infty \Proj(R)^\inftygrp$ is indeed an element of $\CAlg$.
\end{alphanumerate}
Let's discuss now how to turn ordinary commutative rings and CDGAs into $\IE_\infty$-ring spectra, as promised in \itememph{b}:
\begin{lem}\label{lem:HLaxMonoidal}
	The Eilenberg--MacLane functor $H\colon \Dd(\IZ)\morphism \Sp$ refines to a map
	\begin{equation*}
		H\colon \Dd(\IZ)^{\otimes_\IZ^L}\morphism\Sp^\otimes
	\end{equation*}
	of $\infty$-operad, i.e.\ it is a lax symmetric monoidal functor. More generally, for arbitrary commutative rings $R$, the functor $C_\bullet(-,R)=C_\bullet(-)\otimes_\IZ^LR\colon \Sp\morphism \Dd(R)$ has a left adjoint $H\colon \Dd(R)\morphism \Sp$, which refines to a lax symmetric functor
	\begin{equation*}
		H \colon \Dd(R)^{\otimes_R^L}\morphism\Sp^\otimes\,.
	\end{equation*}
\end{lem}
\begin{proof}
	This isn't anything special about $\Dd(R)$, but an instance of the following more general principle:
	\begin{alphanumerate}
		\item[\itememph{*}] \itshape If $\Cc^\otimes$ is a symmetric monoidal $\infty$-category, then $\Hom_\Cc(1_\Cc,-)\colon \Cc\morphism\An$ has a canonical lax symmetric monoidal refinement. If $\Cc^\otimes$ is stable as an $\infty$-operad, then the same is true for the enriched $\Hom$ functor $\hom_\Cc(1_\Cc,-)\colon \Cc\morphism\Sp$.
	\end{alphanumerate}
	The derived base change $-\otimes_\IZ^LR\colon \Dd(\IZ)\morphism\Dd(R)$ has the forgetful functor $\Dd(R)\morphism\Dd(\IZ)$ as a left adjoint, hence $C_\bullet(-,R)$ has indeed a right adjoint, which deserves to be named $H$ as well since it really does nothing but forgetting to $\Dd(\IZ)$ and then applying $H\colon \Dd(\IZ)\morphism \Sp$. Corollary~\labelcref{cor:homDRIsRHom} implies $H\simeq \hom_{\Dd(R)}(R[0],-)$. Since $R[0]\in \Dd(R)$ is the tensor unit, we see that \itememph{*} indeed implies the assertion of the lemma.
	
	To prove \itememph{*}, let's first consider an arbitrary symmetric monoidal $\infty$-category $\Dd$ with all small operadic colimits. Then the Day convolution $\operatorname{Day}(\Cc^\otimes,\Dd^\otimes)$ is symmetric monoidal and its tensor unit is given by
	\begin{equation*}
		\Hom_\Cc(1_\Cc,-)\otimes 1_\Dd\colon \Cc\morphism\Dd\,.
	\end{equation*}
	Note that the tensor product here is neither $\otimes_\Cc$, nor $\otimes_\Dd$, but the one defined in the proof of \cref{prop:OtherModelForSpectra}! That is, the functor under consideration is actually given by $\colimit_{\Hom_\Cc(1_\Cc,-)}\const {1_\Dd}$, and one immediately checks that this coincides with the left Kan extension of $1_\Dd\colon *\morphism\Dd$ along $1_\Cc\colon *\morphism\Cc$, which is how we constructed the tensor unit in the proof of \cref{prop:DayConvolution}\itememph{c}. Now, per construction (just take a sharp look at \cref{con:DayConvolution} again), there is a map
	\begin{equation*}
		\cat{Alg}_{\cat{\IC omm}}\big(\operatorname{Day}(\Cc^\otimes,\Dd^\otimes)\big)\morphism \Fun^{\Op_\infty}(\Cc^\otimes,\Dd^\otimes)
	\end{equation*}
	(it is even an equivalence, but we won't need that). By Lemma/Definition~\labelcref{lemdef:UnitInAlg}, the tensor unit of $\operatorname{Day}(\Cc^\otimes,\Dd^\otimes)$ defines canonically an element of the left-hand side, hence also an element of the right-hand side. This shows that $\Hom_\Cc(1_\Cc,-)\otimes 1_\Dd$ has a canonical lax symmetric monoidal refinement.
	
	Now plug in $\Dd^\otimes\simeq \An^\times$. Then $1_\Dd\simeq *$ and $X\otimes *\simeq \colimit_X \const *\simeq X$ for all $X\in\An$, hence we obtain that $\Hom_\Cc(1_\Cc,-)\colon \Cc\morphism\An$ is the Day convolution tensor unit and thus has a lax symmetric monoidal refinement $\Hom_\Cc(1_\Cc,-)\colon \Cc^\otimes\morphism\An^\times$, as required. Moreover, it preserves limits (which are automatically operadic), hence the refinement lifts to $\hom_\Cc(1_\Cc,-)\colon \Cc^\otimes\morphism\Sp^\otimes$ by \cref{thm:NikolausStabilisationOfOperads} if $\Cc^\otimes$ is a stable $\infty$-operad.
\end{proof}
Now we can resume our discussion on how to include commutative rings and CDGAs into $\CAlg$. Observe that we have two more lax symmetric monoidal functors
\begin{equation*}
	\Ch(\IZ)^{\otimes_\IZ}\morphism\Kk(\IZ)^{\otimes_\IZ}\morphism\Dd(\IZ)^{\otimes_\IZ^L}\,,
\end{equation*}
where $\Ch(\IZ)$ is the $1$-category of chain complexes over $\IZ$. The left functor comes from the fact that $\Kk(\IZ)\simeq \N^c(\Ch(\IZ))$ is the coherent nerve of the Kan-enrichment of $\Ch(\IZ)$. The right functor was constructed in \labelcref{par:DerivedTensorProduct}. Unwinding definitions, we see that $\CAlg(\Ch(\IZ)^{\otimes_\IZ})\simeq \cat{CDGAlg}$ is the $1$-category of commutative differential graded algebras. Moreover, we have an inclusion $\Ab^\otimes\subseteq \Ch(\IZ)^{\otimes_\IZ}$ of symmetric monoidal $1$-categories, given by interpreting abelian groups as chain complexes concentrated in degree $0$, which gives a map $\cat{CRing}\simeq \CAlg(\Ab^\otimes)\morphism \CAlg(\Ch(\IZ)^{\otimes_\IZ})\simeq \cat{CDGAlg}$. Summarizing, if we apply $\CAlg(-)$ everywhere, we obtain functors
\begin{equation*}
	\cat{CRing}\morphism\cat{CDGAlg}\morphism \CAlg\big(\Dd(\IZ)^{\otimes_\IZ^L}\big)\morphism\cat{CAlg}\,.
\end{equation*}
In fact, it turns out that we can completely import ordinary commutative ring theory:
\begin{smallcor}\label{cor:CRingToCAlgFullyFaithful}\itshape
	The above induces a fully faithful functor $\cat{CRing}\morphism \CAlg$ with essential image those $R\in \CAlg$ for which $\pi_i (R)=0$ for all $i\neq 0$.
\end{smallcor}
Here we define the homotopy groups of an $\IE_\infty$-ring spectrum as those of its underlying spectrum. Also note that in the composition above, the middle functor $\cat{CDGAlg}\morphism \CAlg(\Dd(\IZ)^{\otimes_\IZ^L})$ from the $1$-category of CDGAs to the \emph{$\infty$-category of $\IE_\infty$-chain complexes} is not fully faithful, so this is really a thing about the composition.
\begin{proof*}[Proof sketch of \cref{cor:CRingToCAlgFullyFaithful}]
	The proof wasn't discussed in the lecture, but is taken from Fabian's lecture notes, \cite[Remarks~II.53(ii)]{KTheory}. Recall from Example~\labelcref{exm:TensorProductCalculations} that
	\begin{equation*}
		\Ab^\otimes\simeq \CGrp^{\Op_\infty}(\Set^\times)\morphism\CGrp^{\Op_\infty}(\An^\times)\simeq \Sp_{\geq 0}^\otimes
	\end{equation*}
	is fully faithful with left adjoint $\pi_0$. The adjunction, together with the condition that the counit is an equivalence, persists to taking $\CAlg(-)$ everywhere. This almost shows what we want, except that we have to explain why the functor $i^\otimes\colon \cat{CRing}\morphism \CAlg(\Sp_{\geq 0}^\otimes)\subseteq\CAlg$ that we get out is the one induced by $H$, i.e.\ the one from the formulation of the corollary. It's more or less clear from the constructions that the underlying functor of $i^\otimes$ is indeed $H\simeq \hom_\Ab(\IZ,-)\colon\Ab\morphism \Sp$, but we need to check that the lax symmetric monoidal refinement is the right one.
	
	Fabian outlines two ways to do so. One can use the additional result from \labelcref{par:LaxMonoidalAdjoints}, which says that lax monoidal refinements of right adjoints are equivalent to oplax monoidal refinements of left adjoints, so it suffices that both candidates induce the same refinement of $\pi_0$.
	
	Alternatively, one can argue as follows:  By construction and Lemma/Definition~\labelcref{lemdef:UnitInAlg}, the lax symmetric monoidal refinement of $\Hom_\Ab(\IZ,-)\colon \Ab\morphism \An$ from the proof of \cref{lem:HLaxMonoidal} is initial in
	\begin{equation*}
		\CAlg\big(\operatorname{Day}(\Ab^\otimes,\An^\times)\big)\simeq \Fun^{\Op_\infty}(\Ab^\otimes,\An^\times)
	\end{equation*}
	(we didn't need this to be an equivalence in the proof of \cref{lem:HLaxMonoidal}). Hence there is a unique map to $\Omega^\infty i^\otimes$, which must be an equivalence since it is an equivalence on underlying functor $\Ab\morphism\An$ (and then Segal conditions do the rest). Now use \cref{thm:NikolausStabilisationOfOperads} to see that our lax symmetric monoidal refinement of $\hom_\Ab(\IZ,-)$ coincides with $i^\otimes$.
\end{proof*}
Before we go on and construct module categories, I'd like to go on another detour and prove that the homology and cohomology theories associated to a spectrum really do what they're supposed to do.
\begin{cor*}\label{cor*:SpectraCohomologyTheory}
	The homology and cohomology associated to a spectrum $E$ by \cref{def:SpectraCohomologyTheory} can be described as
	\begin{equation*}
		\begin{aligned}[t]
			E_*(-)&= \pi_*\big(\IS[-]\otimes E\big)\\
			&= \Tor_*^\IS\big(\IS[-],E\big)
		\end{aligned}
		\quad\text{and}\quad
		\begin{aligned}[t]
			E^*(-)&= \pi_{*}\hom_\Sp\big(\IS[-],E\big)\\
			&= \Ext_\IS^{*}\big(\IS[-],E\big)
		\end{aligned}\,.
	\end{equation*}
	They satisfy the Eilenberg--Steenrod axioms. Moreover, the homology and cohomology theory associated to an Eilenberg--MacLane spectrum $HA$ for some abelian group $A$ coincide with singular homology and \embrace{up to a sign swap} singular cohomology. That is,
	\begin{equation*}
		HA_*(-)\simeq H_*(-,A)\colon \An\morphism \Ab\quad\text{and}\quad HA^*(-)\simeq H^{-*}(-,A)\colon \An^\op\morphism \Ab\,.
	\end{equation*}
	Finally, the homology and cohomology of a spectrum $E$ itself, as defined in \cref{par:HomologyOfSpectra}, is given by
	\begin{equation*}
		H_*(E,A)\simeq \pi_*(E\otimes HA)\quad\text{and}\quad H^*(E,A)\simeq \pi_{*}\hom_\Sp(E,HA)\,.
	\end{equation*}
\end{cor*}
The sign swap in cohomology is deliberate. Fabian believes it was a mistake to have ever defined singular cohomology in nonnegative degrees rather than in nonpositive ones (or in fact, to have ever defined cochain complexes). I was late to be convinced, but in the end I was, not at least because in \cref{par:AdamsOperations} we'll encounter a situation where our sign convention is definitely preferable.
\begin{proof*}[Proof of \cref{cor*:SpectraCohomologyTheory}]
	The description of $E_*(-)$ follows from our computations in \cref{par:SpTensorProduct}. Along the same lines one can also prove $E^X\simeq\hom_\Sp(\IS[X],E)$, hence the description of $E^*(-)$.
	
	Next we check the Eilenberg--Steenrod axioms. To construct the long exact sequences and excision, first note that fibre and cofibre sequences in the stable $\infty$-category $\Sp$ coincide, and that to each such sequence there  is an associated long exact sequence on homotopy group (because the functors $\Omega^{\infty-i}\colon \Sp\morphism */\An$ preserve fibre sequences, so the long exact sequences from $*/\An$ are inherited). Now both $\IS[-]\otimes E$ and $\hom_\Sp(\IS[-],E)$ transform cofibre sequences in $\An$ into fibre sequences in $\Sp$, providing the long exact sequences for $E_*(-)$ and $E^*(-)$.
	
	It remains to see that they transform disjoint unions into direct sums/products. For $E^*(-)$, this is easy: $\hom_\Sp(\IS[-],E)$ sends disjoint unions to products, and taking homotopy groups commutes with products. For $E_*(-)$, we can argue similarly that $\IS[-]\otimes E$ sends disjoint unions to coproducts, but now we need to show that $\pi_*$ sends coproducts of spectra to products. Arbitrary coproducts can be written as filtered colimits of finite coproducts, and $\pi_*$ commutes with filtered colimits (\cref{rem*:piPreservesFilteredColimits}), hence it suffices to check the case of finite coproducts. But $\Sp$ is stable, hence finite coproducts agree with finite products, and $\pi_*$ preserves the latter.
	
	Finally, let's do our reality check for Eilenberg--MacLane spectra. The cohomology case is easy: From the $(C_\bullet,H)$-adjunction combined with Corollary~\labelcref{cor:homDRIsRHom}, we get
	\begin{equation*}
		\hom_\Sp(E,HA)\simeq \hom_{\Dd(\IZ)}\big(C_\bullet(E),A\big)\simeq H\big(R\!\Hom_\IZ(C_\bullet(E),A)\big)
	\end{equation*}
	By \cref{exm:MyFirstSpectra}\itememph{c}, the homotopy groups of the right-hand side are the homology groups of $R\!\Hom_\IZ(C_\bullet(E),A)$, which are the singular cohomology groups $H^*(E,A)$ (in negative degrees) by definition. Since $H^*(\IS[X],A)= H^{-*}(X,A)$ for all $X\in\An$ by \cref{lem:HomologyOfS}, plugging in the special case $E\simeq \IS[X]$ also shows $HA^*(X)= H^{-*}(X,A)$.
	
	For the homology case we actually need to use fact that $H$ is lax symmetric monoidal by \cref{lem:HLaxMonoidal}. Combining this with the unit $E\morphism HC_\bullet(E)$ of the $(C_\bullet,H)$-adjunction gives functorial maps
	\begin{equation*}
		E\otimes HA\morphism HC_\bullet(E)\otimes HA\morphism H\big(C_\bullet(E)\otimes_\IZ^LA\big)\,.
	\end{equation*}
	If we can show that the composition is an equivalence, taking homotopy groups on both sides will show $\pi_*(E\otimes HA)= H_*(E,A)$, as desired, and then the special case $E\simeq \IS[X]$ together with \cref{lem:HomologyOfS} will also show $HA_*(X)= H_*(X,A)$.
	
	To show that the composition is indeed an equivalence, first note that it is an equivalence for $E\simeq \IS$ by \cref{lem:HomologyOfS}, and that both sides commute with shifts and colimits in $E$. For this we need that $H$ commutes with colimits, which is perhaps a bit surprising at first, given that $H$ is only a right adjoint, but we give an argument in the proof sketch* of \cref{thm:D(R)IsModOverHR} below (and once you know this theorem, $H$ preserving colimits is not that surprising any more). Now all of $\Sp$ can be generated from $\IS$ using shifts and colimits, hence
	\begin{equation*}
		E\otimes HA\isomorphism H\big(C_\bullet(E)\otimes_\IZ^LA\big)
	\end{equation*}
	is an equivalence everywhere.
\end{proof*}


\subsection{\texorpdfstring{$\infty$}{Infinity}-Categories of Modules}
$\IE_\infty$-ring spectra have categories of modules that behave like the derived categories of modules over ordinary rings. In particular, as we'll see  in \cref{thm:D(R)IsModOverHR} below, the category of modules over $HR$ for some $R\in\cat{CRing}$ \emph{is} nothing else but $\Dd(R)$.

As a first step, we'll construct certain operads $\cat{\IA ssoc}$ and $\cat{\IL Mod}$. These have already been teasered in \cref{exm:MyFirstAlgebrasOverOperads}\itememph{d} and \itememph{e}, so we'll hijack that numbering and define them properly.\label{exm:MyFirstAlgebrasOverOperadsII}

\numpar*{\labelcref*{exm:MyFirstAlgebrasOverOperads}. Example}What it should have been:
\begin{alphanumerate}\setcounter{enumi}{3}
	\item We would like to have an $\infty$-operad that describes associative monoids. So we define $\IE_1=\cat{\IA ssoc}\morphism\IGamma^\op$ as follows: The objects of $\cat{\IA ssoc}$ are finite sets and a morphism $I\morphism J$ is a map in $\IGamma^\op$ together with a total ordering on each preimage $\alpha^{-1}(j)$. Composition is given by composition in $\IGamma^\op$ and lexicogaphic ordering. If we equip the $1$-category $\cat{\IA ssoc}$ with the obvious functor $\cat{\IA ssoc}\morphism \IGamma^\op$, it becomes an $\infty$-operad with underlying category $\cat{\IA ssoc}_1=\{*\}$ and
	\begin{equation*}
		\Hom_{\cat{\IA ssoc}}^\mathrm{act}\big(\underbrace{(*,\dotsc,*)}_{n\text{ entries}},*\big)=\left\{\text{total orders on }\langle n\rangle\right\}\,.
	\end{equation*}
	(just as there are $n!$ ways to multiply $n$ factors in an associative, but not necessarily commutative monoid).
	\item We would also like to have an $\infty$-operad that describes an associative monoid $A$ together with an object $M$ it acts upon (we think of this as an algebra together with a module over it). So we define $\cat{\IL Mod}\morphism \IGamma^\op$ as follows: Its objects are pairs $(I,S)$ with $I\in\IGamma^\op$ and $S\subseteq I$ (think of $I$ as the set of factors we'd like to multiply, and of $S$ as the set of those factors that come from $M$; of course, we can't multiply more than one element from $M$). A map $(I,S)\morphism(J,T)$ is given by a map $\alpha\colon I\morphism J$ in $\cat{\IA ssoc}$ such that $\alpha(S)=T$ and for all $t\in T$ the set $\alpha^{-1}(t)\cap S$ consists exactly of the maximal element of $\alpha^{-1}(t)$. Composition is inherited from $\cat{\IA ssoc}$. Again, if we equip the $1$-category $\cat{\IL Mod}$ with the obvious map $\cat{\IL Mod}\morphism \IGamma^\op$, it becomes an $\infty$-operad. The underlying category $\cat{\IL Mod}_1$ has precisely two objects: $a=(\langle 1\rangle,\emptyset)$ and $m=(\langle 1\rangle,\langle 1\rangle)$, which we think of corresponding to $A$ and $M$ respectively. We have
	\begin{equation*}
		\Hom_{\cat{\IL Mod}}^\mathrm{act}\big((b_1,\dotsc,b_n),a\big)=\begin{cases*}
			\emptyset & if $b_i=m$ for some $i$\\
			\{\text{total orders on }\langle n\rangle\}& if $b_i=a$ for all $i$
		\end{cases*}
	\end{equation*}
	This makes sense since \enquote{as soon as one of the factors is from $M$, the result will be no longer an element of $A$, but one of $M$}. Similarly,
	\begin{equation*}
		\Hom_{\cat{\IL Mod}}^\mathrm{act}\big((b_1,\dotsc,b_n),m\big)=\begin{cases*}
			\emptyset & unless exactly one $b_i=m$\\
			\left\{\!\!\!\begin{tabular}{c}
				total orders on $\langle n\rangle$\\
				with max.\ element $i$
			\end{tabular}\!\!\!\right\}& if $b_i=m$
		\end{cases*}
	\end{equation*}
	since there can be only one factor from $M$ in any product, and if there is, it must be the right-most factor.
	
	Sending the unique element $*\in\IAssoc_1$ to $a\in\ILMod_1$ defines a map $\cat{\IA ssoc}\morphism \cat{\IL Mod}$ of $\infty$-operads, which induces a map $a^*\colon \cat{Alg}_{\cat{\IL Mod}}(\Cc^\otimes)\morphism \cat{Alg}_{\cat{\IA ssoc}}(\Cc^\otimes)$ (\enquote{we forget $M$ and only retain $A$}) for any symmetric monoidal category $\Cc^\otimes$. For any $A\in \cat{Alg}_{\cat{\IA ssoc}}(\Cc^\otimes)$, the pullback
	\begin{equation*}
		\begin{tikzcd}
			\cat{LMod}_{A}(\Cc^\otimes)\rar\dar\drar[pullback] & \cat{Alg}_{\cat{\IL Mod}}(\Cc^\otimes)\dar["a^*"]\\
			*\rar["A"]& \cat{Alg}_{\cat{\IA ssoc}}(\Cc^\otimes)
		\end{tikzcd}
	\end{equation*}
	is called the \emph{$\infty$-category of left modules in $\Cc$ over $A$}. We can convince ourselves in examples that this terminology really makes sense: If $\Cc^\otimes=\Set^\times$, then an algebra $A\in \Alg_\IAssoc(\Set^\times)\simeq\Mon(\Set)$ is a monoid and elements of $\cat{LMod}_{A}(\Set^\times)$ are really left $A$-sets. Similarly, if $\Cc^\otimes=\Ab^\otimes$, then $A$ is an associative (not necessarily commutative) ring and $\cat{LMod}_{A}(\Ab^\otimes)$ is really the $1$-category of left $A$-modules.
\end{alphanumerate}

\numpar{General Properties of Module $\infty$-Categories}
We will now discuss (and partially prove) a series of results to show that $\IAssoc$ and $\ILMod$ really do what they are supposed to do, and to see how one works with the left module $\infty$-categories $\LMod_A(\Cc^\otimes)$. After that, we are ready for modules over $\IE_\infty$-ring spectra!

First off, recall the functor $\Cut\colon\IDelta^\op\morphism\IGamma^\op$ from \cref{def:CartesianCommutativeMonoids}, which sends a totally ordered set $I$ to its set of Dedekind cuts. Since the set of Dedekind cuts inherits a total order, $\Cut$ lifts to a functor $\Cut\colon \IDelta^\op\morphism\cat{\IA ssoc}$ over $\IGamma^\op$. Also recall the notation introduced before \cref{thm:OMon}: For any $\infty$-operad $\Oo$ and any $\infty$-category $\Cc$ with finite products, we denote by $\Oo\Mon(\Cc)\subseteq\Fun(\Oo,\Cc)$ the full sub-$\infty$-category of functors satisfying the Segal condition.
\begin{smallthm}[Lurie, {\cite[Proposition~\HAthm{4.1.2.10}]{HA}}]\label{thm:AlgAssocDescription}
	If $\Cc$ has finite products, then the induced functor
	\begin{equation*}
		\Cut^*\colon \cat{\IA ssoc}\Mon(\Cc)\simeq \cat{Alg}_{\cat{\IA ssoc}}(\Cc^\times)\isomorphism \Mon(\Cc)
	\end{equation*}
	is an equivalence. More generally, if $\Cc^\otimes$ is any symmetric monoidal $\infty$-category, then
	\begin{equation*}
		\Cut^*\colon \Alg_\IAssoc(\Cc^\otimes)\morphism \Fun(\IDelta^\op,\Cc^\otimes)\times_{\Fun(\IDelta^\op,\IGamma^\op)}\{\Cut\}
	\end{equation*}
	is fully faithful, with essential image those functors $F\colon \IDelta^\op\morphism\Cc^\otimes$ that send inerts to inerts. That is, whenever $\alpha\colon [n]\morphism{} [m]$ is inert in $\IDelta$ \embrace{without $^\op$}, the induced morphism $F(\alpha^\op)\colon F([m])\morphism F([n])$ is inert in $\Cc^\otimes$.
\end{smallthm}
\begin{rem*}\label{rem*:AlgAssocDescription}
	We won't prove \cref{thm:AlgAssocDescription}, but let me at least give some additional motivation. The condition that $F$ preserves inerts is just a somewhat unusual formulation of the Segal condition. Indeed, if we denote $A=F([1])\in \Cc_1^\otimes\simeq \Cc$, then the fact that the Segal maps $e_i\colon [1]\morphism {[n]}$ induce inert maps in $\Cc^\otimes$ shows
	\begin{equation*}
		F\big([n]\big)\simeq (A,\dotsc,A)\in\Cc_n^\otimes\simeq \Cc^n
	\end{equation*}
	(in particular, $F([0])\simeq 1_\Cc$). For general $\alpha\colon [n]\morphism{[m]}$ in $\IDelta$, we can think of the induced map $F(\alpha^\op)\colon F([m])\morphism F([n])$ as follows: Let $\Cut(\alpha^\op)_*\colon (A,\dotsc,A)\morphism (B_1,\dotsc,B_n)$ be a cocartesian lift of $\Cut(\alpha^\op)\colon \langle m\rangle\morphism\langle n\rangle$. A quick unravelling shows
	\begin{equation*}
		B_i\simeq \bigotimes_{\alpha(i-1)<j\leq \alpha(i)}A
	\end{equation*}
	for $i=1,\dotsc,n$. Then $\Cut(\alpha^\op)_*$ and $F(\alpha^\op)$ induce a map $\Lambda_0^2\morphism \Cc^\otimes$, which has an extension to $\Delta^2$ since $\Cc^\otimes\morphism \IGamma^\op$ is a cocartesian fibration. Restricting the extension to $\Delta^{\{1,2\}}$ gives maps $B_i\morphism A$ in $\Cc$, which we think of as the \enquote{multiplication maps in the algebra $A$}.
\end{rem*}

 
 \lecture[A user's guide to Lurie's treatment of brave new algebra. Localisations of $\IE_\infty$-ring spectra.]{2021-01-12}There is a similar description of $\Alg_{\ILMod}(\Cc^\otimes)$. To this end, we can extend $\Cut$ to a functor $\operatorname{LCut}\colon \IDelta^\op\times[1]\morphism \ILMod$  that fits into a diagram
\begin{equation*}
	\begin{tikzcd}[row sep=small]
		\IDelta^\op\rar["\Cut"]\ar[dd,"-\times \{1\}"'] & \IAssoc\ar[dd,"a"]\drar\\
		& & \IGamma^\op\\
		\IDelta^\op\times[1]\rar["\operatorname{LCut}"] & \ILMod\urar["p"]
	\end{tikzcd}
\end{equation*}
$\operatorname{LCut}$ sends elements of the form $([n],1)$ to the \enquote{algebra-like} element $(a,\dotsc,a)\in\ILMod_n$ (this guy has $n$ repetitions of $a$), and elements of the form $([n],0)$ to the \enquote{module-like} element $(a,\dotsc,a,m)\in \ILMod_{n+1}$ (this guy also has $n$ repetitions of $a$). In particular, and this might be confusing at first, the structure map $p\circ \operatorname{LCut}\colon \IDelta^\op\times[1]\morphism \IGamma^\op$ is \emph{not} given by projecting to the first factor and applying $\Cut\colon \IDelta^\op\morphism \IGamma^\op$. But if you think about this, it makes sense to have it that way: The morphisms $\operatorname{LCut}([n],0)\morphism \operatorname{LCut}([n],1)$ should be given by forgetting the \enquote{module part} of $(a,\dotsc,a,m)$ and only retaining the algebra part $(a,\dotsc,a)$---which only works if the morphism in question runs from $\ILMod_{n+1}$ to $\ILMod_n$.
\begin{smallthm}[Lurie, {\cite[Proposition~\HAthm{4.2.2.12}]{HA}}]\label{thm:AlgLModDescription}
	The functor
	\begin{equation*}
		\operatorname{LCut}^*\colon \Alg_\ILMod(\Cc^\otimes)\morphism \Fun\big(\IDelta^\op\times[1],\Cc^\otimes\big)\times_{\Fun(\IDelta^\op\times[1],\IGamma^\op)}\{p\circ \operatorname{LCut}\}
	\end{equation*}
	is fully faithful, with essential image those functors $F\colon \IDelta^\op\times[1]\morphism\Cc^\otimes$ that fulfill the following conditions:
	\begin{alphanumerate}
		\item The \enquote{algebra-like} part $F_{|\IDelta^\op\times\{1\}}$ of $F$ satisfies the condition from \cref{thm:AlgAssocDescription}. That is, whenever $\alpha\colon [n]\morphism{} [m]$ is inert in $\IDelta$ \embrace{without $^\op$}, the induced morphism $F(\alpha^\op,\id_1)\colon F([m],1)\morphism F([n],1)$ is inert in $\Cc^\otimes$.  
		\item $F([n],0)\morphism F([n],1)$ is always inert in $\Cc^\otimes$.
		\item If $\alpha^\op\colon [n]\morphism{}[m]$ is an inert morphism in $\IDelta$ satisfying $\alpha(n)=m$, then the induced morphism $F(\alpha^\op,\id_0)\colon F([m],0)\morphism F([n],0)$ is inert in $\Cc^\otimes$.
	\end{alphanumerate}
\end{smallthm}
\begin{rem*}\label{rem*:AlgLModDescription}
	Again, we won't prove \cref{thm:AlgLModDescription}, but at least explain how the structure of an algebra and a module over it are encoded in the right-hand side. Let $M=F([0],0)$ and $A=F([1],1)$; both are elements of $\Cc^\otimes_1\simeq \Cc$ by construction. By condition \itememph{a}, the part  $F_{|\IDelta^\op\times\{1\}}$ encodes the algebra structure on $A$, so in particular, $F([n],1)\simeq (A,\dotsc,A)\in\Cc^n$ as in \cref{rem*:AlgAssocDescription}. Combining this with conditions \itememph{b} and \itememph{c} shows
	\begin{equation*}
		F\big([n],0\big)\simeq (A,\dotsc,A,M)\in \Cc^{n+1}\,.
	\end{equation*}
	Given a general map $\alpha\colon [n]\morphism{[m]}$ in $\IDelta$, we can think of $F(\alpha^\op,\id_0)\colon F([m],0)\morphism F([n],0)$ as follows: Choose a cocartesian lift $p(\operatorname{LCut}(\alpha^\op))_*\colon (A,\dotsc,A,M)\morphism (B_1,\dotsc,B_n,N)$ of $p(\operatorname{LCut}(\alpha^\op))\colon \langle m+1\rangle\morphism\langle n+1\rangle$. Then
	\begin{equation*}
		B_i\simeq \bigotimes_{\alpha(i-1)<j\leq \alpha(i)}A\,,\quad\text{and}\quad N\simeq \Bigg(\bigotimes_{\alpha(n)<j\leq m}A\Bigg) \otimes M\,.
	\end{equation*}
	As in \cref{rem*:AlgAssocDescription}, we get maps $B_i\morphism A$ and $N\morphism M$, which encode the multiplication in $A$ and the left action of $A$ on $M$.
\end{rem*}
In particular \cref{thm:AlgAssocDescription} says that any algebra $A\in\Alg_\IAssoc(\Cc^\otimes)$ defines an element $\Cut^*(A)\in\Fun_{\IGamma^\op}(\IDelta^\op,\Cc^\otimes)$ which satisfies the Segal condition. And then by \cref{thm:AlgLModDescription}, we find that $\LMod_A(\Cc^\otimes)$ is a full sub-$\infty$-category of the pullback
\begin{equation*}
	\begin{tikzcd}
		P\rar\dar\drar[pullback] & \Fun_{\IGamma^\op}\big(\IDelta^\op\times[1],\Cc^\otimes\big)\dar["(-\times\{1\})^*"]\\
		*\rar["\Cut^*(A)"] & \Fun_{\IGamma^\op}(\IDelta^\op,\Cc^\otimes)
	\end{tikzcd}
\end{equation*}

Note that besides the operad map $a\colon \IAssoc\morphism \ILMod$ of extracting algebras, there is also a map $m\colon \cat{\IT riv}\morphism \ILMod$ that extracts the module object. Also recall that $\Alg_{\cat{\IT riv}}(\Cc^\otimes)\simeq \Cc$ as shown in \cref{exm:MyFirstAlgebrasOverOperads}\itememph{b}.
\begin{smallcor}\label{cor:LModHasCoLimits}
	If $\Cc^\otimes$ has small/finite operadic colimits, then $\LMod_A(\Cc^\otimes)$ has small/finite colimits, and these are computed \enquote{underlyingly}. That is, the functor
	\begin{equation*}
		m^*\colon \LMod_A(\Cc^\otimes)\morphism \Alg_{\cat{\IT riv}}(\Cc^\otimes)\simeq \Cc
	\end{equation*}
	preserves small/finite colimits. Likewise, if $\Cc$ has small/finite limits \embrace{which are automatically operadic}, then so has $\cat{LMod}_A(\Cc^\otimes)$, and $m^*$ preserves small/finite limits. In particular, if $\Cc^\otimes$ is a stable $\infty$-operad, then $\LMod_A(\Cc^\otimes)$ is a stable $\infty$-category for all $A\in \Alg_\IAssoc(\Cc^\otimes)$.
\end{smallcor}
\begin{proof}[Proof sketch]
	Let $F\colon \Ii\morphism \LMod_A(\Cc^\otimes)$ be a diagram. Then \cref{thm:AlgLModDescription} induces a diagram $F\colon \Ii\morphism \Fun(\IDelta^\op\times[1],\Cc^\otimes)$ satisfying
	\begin{equation*}
		F(i)\big([n],1\big)\simeq(A,\dotsc,A)\quad\text{and}\quad F\big([n],0\big)\simeq \big(A,\dotsc,A,M(i)\big)\,,
	\end{equation*}
	where $M\colon \Ii\morphism \Cc$ is another diagram (the \enquote{module part} of $F$). We have to check that the there are functors
	\begin{align*}
		\ov{F}_{\limit},\ov{F}_{\colimit}\colon \Ii\times\IDelta^\op\times[1]&\morphism \Cc^\otimes\\
		(i,[n],0)& \longmapsto (A,\dotsc,A)\\
		(i,[n],1)&\longmapsto \begin{cases*}
			(A,\dotsc,A,\limit_{i\in\Ii} M(i)) & for $\ov{F}_{\limit}$\\
			(A,\dotsc,A,\colimit_{i\in\Ii} M(i)) & for $\ov{F}_{\colimit}$
		\end{cases*}
	\end{align*}
	Moreover, we need to verify that $\ov{F}_{\limit}$, $\ov{F}_{\colimit}$ constitute the limit and colimit over $F$, taken inside the pullback $P$ above, and that both satisfy the conditions from \cref{thm:AlgLModDescription} so that they lie inside the full subcategory $\LMod_A(\Cc^\otimes)\subseteq P$. We didn't work this out in the lecture, but in these notes I would like to give a rough sketch of what I think happens.
	
	Naively, I would expect that the limit and colimit over $F$, taken in $P$, should be given by some kind of pointwise limit in $\Fun(\IDelta^\op\times [1],\Cc^\otimes)$. There are two problems though: First, we see from \cref{rem*:OperadicLimits} that $(A,\dotsc,A,\colimit_{i\in\Ii} M(i))$ doesn't necessarily coincide with the colimit $\colimit_{i\in\Ii}(A,\dotsc,A,M(i))$ in $\Cc^\otimes$, even though colimits in $\Cc$ are operadic, unless $\Ii$ is weakly contractible. This problem doesn't occur in the limit case, but there is a second one, which affects both cases: We aren't taking limits/colimits in $\Fun(\IDelta^\op\times[1],\Cc^\otimes)$, but in some pullback $P$ of it. And if $\Ii$ isn't weakly contractible, then the limit/colimit over the constant $A$-diagram in $\Fun_{\IGamma^\op}(\IDelta^\op,\Cc^\otimes)$ might not be $A$ any more. Hence we can't expect that limits/colimits in the pullback $P$ can be computed in each factor separately.
	
	Fortunately, there is a trick that solves both problems at once. Since $\Cc$ is cocomplete by assumption, it has an initial object, hence the diagram $M\colon \Ii\morphism \Cc$ extends canonically to a diagram $M^\triangleleft\colon \Ii^\triangleleft\morphism \Cc$ that sends the tip of the cone to the initial object. Moreover,
	\begin{equation*}
		\colimit_{i\in\Ii}M(i)\simeq \colimit_{i\in \Ii^\triangleleft}M^\triangleleft(i)\,,
	\end{equation*}
	and the right-hand side is also a colimit in $\Cc^\otimes$ by \cref{rem*:OperadicLimits}, since $\Ii^\triangleleft$ is weakly contractible. Similarly, taking the colimit over an $\Ii^\triangleleft$-shaped in $\Fun_{\IGamma^\op}(\IDelta^\op,\Cc^\otimes)$ which is constant on $A$ gives $A$ again. After some unravelling, we see (I think) that this gives a way of computing $\Ii$-shaped colimits in $P$ by factorwise $\Ii^\triangleleft$-shaped colimits. A similar trick works for limits. I'll leave it to you to make this argument precise and also to check that the conditions from \cref{thm:AlgLModDescription} are satisfied.
\end{proof}
The structure map $\IAssoc\morphism
\IGamma^\op$ defines an $\infty$-operad map $\IAssoc\morphism\IComm$, which in turn induces a map $ \CAlg(\Cc^\otimes)\simeq\Alg_\IComm(\Cc^\otimes)\morphism\Alg_\IAssoc(\Cc^\otimes)$, as one would expect. In particular, the tensor unit $1_\Cc\in \Cc$ also defines an associative algebra. In fact, one can show similar to Lemma/Definition~\labelcref{lemdef:UnitInAlg} that $1_\Cc$ is also initial in $\Alg_\IAssoc(\Cc^\otimes)$.
\begin{smallcor}\label{cor:LMod_1C=C}
	For any symmetric monoidal $\infty$-category $q\colon\Cc^\otimes\morphism\IGamma^\op$, the forgetful functor 
	\begin{equation*}
		m^*\colon \cat{LMod}_{1_\Cc}(\Cc^\otimes)\isomorphism \Alg_\cat{\IT riv}(\Cc^\otimes)\simeq\Cc
	\end{equation*}
	is an equivalence.
\end{smallcor}
\begin{proof}[Proof sketch]
	Since the map $\emptyset\morphism (1_\Cc)$ in $\Cc^\otimes$ is a $q$-cocartesian lift of $\langle 0\rangle \morphism\langle 1\rangle$, we see that a functor $F\colon \IDelta^\op\times[1]\morphism\Cc^\otimes$ fulfills the conditions from \cref{thm:AlgLModDescription} iff all induced maps in $\Cc^\otimes$ are $q$-cocartesian. But since $([0],0)$ is initial in $\IDelta^\op\times[1]$, a functor $F\colon \IDelta^\op\times [1]\morphism \Cc^\otimes$ with $F([1],1)\simeq 1_\Cc$ is the same thing as a natural transformation
	\begin{equation*}
		\eta_F\colon \const {F([0],0)}\Longrightarrow F|_{\IDelta^\op\times[1]\smallsetminus\{([0],0)\}}
	\end{equation*}
	in $\Fun(\IDelta^\op\times[1]\smallsetminus\{([0],0)\},\Cc^\otimes)$. In other words, $F$ corresponds to a lift in the diagram
	\begin{equation*}
		\begin{tikzcd}[column sep=7em]
			{[0]}\rar["\const {F([0],0)}"]\dar & \Fun\big(\IDelta^\op\times[1]\smallsetminus\{([0],0)\},\Cc^\otimes\big)\dar["q_*"]\\
			{[1]}\rar["\const {\langle 0\rangle}\Rightarrow p\circ \operatorname{LCut}"'] \urar[dashed,"\eta_F"] & \Fun\big(\IDelta^\op\times[1]\smallsetminus\{([0],0)\},\IGamma^\op\big)
		\end{tikzcd}
	\end{equation*}
	(in the bottom arrow, $p\colon \ILMod\morphism\IGamma^\op$ denotes the structure map of $\ILMod$).
	
	As we've seen above, $F$ defines an element of $\LMod_{1_\Cc}(\Cc^\otimes)$ iff it takes all maps in $\IDelta^\op\times[1]$ to $q$-cocartesian ones. Using \cite[Corollary~IX.25]{HigherCatsII}, this is equivalent to $\eta_F$ being a $q_*$-cocartesian edge. Hence  the space of lifts $\eta_F$, and thus the space of functors $F$ with given $F([0],0)$, is the space of $q_*$-cocartesian lifts in the above diagram. Now the functor $m^*\colon \LMod_{1_\Cc}(\Cc^\otimes)\morphism\Cc$ corresponds to the choice of $F([0],0)$, i.e.\ to the choice of starting point. Since $q_*$ is a cocartesian fibration, we see that $m^*$ is essentially surjective, and since a map between starting points defines a contractible space of maps between $q_*$-cocartesian lifts (you can make this precise using \cite[Proposition~IX.24]{HigherCatsII}), we see that $m^*$ is fully faithful.
\end{proof}
Fabian's script mentions a (stronger version of) another corollary, which we'll need to prove \cref{thm:D(R)IsModOverHR} below. I thought about whether there is an easy proof using what we know so far, but didn't succeed, so for now I'll just refer to Lurie.
\begin{cor*}[Lurie, {\cite[Proposition~\HAthm{4.2.4.2}]{HA}}]\label{cor*:LModFreeAdjunction}
	Let  $\Cc^\otimes$ be a symmetric monoidal $\infty$-category and $A\in \Alg_\IAssoc(\Cc^\otimes)$ an algebra in it \embrace{whose underlying object we will also denote $A\in\Cc$}. Then the functor $A\otimes -\colon \Cc\morphism \Cc$ lifts to a functor
	\begin{equation*}
		\begin{tikzcd}
			& \LMod_A(\Cc^\otimes)\dar["m^*"]\\
			\Cc\urar["A\otimes -"]\rar["A\otimes-"] & \Cc
		\end{tikzcd}
	\end{equation*}
	which is right-adjoint to the forgetful functor $m^*\colon \LMod_A(\Cc^\otimes)\morphism \Cc$. In particular, we see that $A\in\LMod_A(\Cc^\otimes)$.
\end{cor*}
With that out of the way, let's discuss a cool application. Any $\IE_\infty$-ring spectrum $R$ defines an element of $\Alg_\IAssoc(\Sp^\otimes)$ via the canonical map $\CAlg\morphism\Alg_\IAssoc(\Sp^\otimes)$. Consequently, we may define the \emph{$\infty$-category of modules} over $R$ as
\begin{equation*}
	\Mod_R\coloneqq \LMod_R(\Sp^\otimes)\,.
\end{equation*}
We know from \cref{cor:LModHasCoLimits} that this is a stable $\infty$-category.

\begin{thm}\label{thm:D(R)IsModOverHR}
	Let $R$ be an ordinary commutative ring. Then the canonical functor
	\begin{equation*}
		\Dd(R)\simeq \cat{LMod}_{R[0]}\big(\Dd(R)^{\otimes_R^L}\big)\morphism[H]\cat{LMod}_{HR}(\Sp^\otimes)\simeq \cat{Mod}_{HR}
	\end{equation*}
	induced by \cref{cor:LMod_1C=C} and \cref{lem:HLaxMonoidal} is an equivalence. 
\end{thm}
We can interpret this theorem as follows: Since $\IS$ is the tensor unit in $\Sp$ (by \cref{par:SpTensorProduct}), it is the initial element in $\CAlg$ (Lemma/Definition~\labelcref{lemdef:UnitInAlg}) and we have $\Sp\simeq \Mod_\IS$ (\cref{cor:LMod_1C=C}). Hence, what \cref{thm:D(R)IsModOverHR} says is that the passage from $\Dd(R)$ to $\Sp$ is nothing else but the forgetful functor $\Mod_{HR}\morphism \Mod_\IS$ with respect to the unique map $\IS\morphism HR$ in $\CAlg$.

As an easy consequence, we obtain that also
\begin{equation*}
	K\colon \Dd_{\geq 0}(R)\isomorphism\cat{LMod}_R(\CGrp(\An)^\otimes)
\end{equation*}
is an equivalence, where $\CGrp(\An)^\otimes\simeq \CGrp^{\Op_\infty}(\An^\times)$ denotes the symmetric monoidal structure on $\CGrp(\An)$ obtained via \cref{prop:Sp(C)Monoidal}.

\begin{proof*}[Proof of \cref{thm:D(R)IsModOverHR}]
	Fabian has sketched out a proof in the lecture notes \cite[Theorem~II.57]{KTheory}. But then about a month after the final lecture, Fabian has come up with a more straightforward argument, which he explained to me and I'll include it here.
	
	Recall that an object $c$ in an $\infty$-category $\Cc$ is called \emph{compact} if $\Hom_\Cc(c,-)$ commutes with filtered colimits (which were introduced in \cref{rem*:piPreservesFilteredColimits}). Note that a finite colimit of compact objects is compact again; this follows from the fact that $\Hom_\Cc(-,-)$ transforms finite colimits in the first variable into finite limits, and finite limits commute with filtered colimits in $\An$. This is proved in \cite[Proposition~\HTTthm{5.3.3.3}]{HTT}, but I think it also follows from \cref{rem*:piPreservesFilteredColimits} after some fiddling. With that out of the way, the proof of \cref{thm:D(R)IsModOverHR} is based on the following three observations:
	\begin{alphanumerate}
		\item[\itememph{1}]\itshape The objects $R[i]\in\Dd(R)$ are compact, and the functors $\pi_0\Hom_{\Dd(R)}(R[i],-)\colon \Dd(R)\morphism \cat{Ab}$ for $i\in\IZ$ are jointly conservative. The same is true for the objects $HR[i]\in \Mod_{HR}$ and the functors $\pi_0\hom_{\Mod_{HR}}(HR[i],-)\colon \Mod_{HR}\morphism \cat{Ab}$.
		\item[\itememph{2}] The functor $H\colon \Dd(R)\morphism \Mod_{HR}$ preserves colimits.
		\item[\itememph{3}] $\Dd(R)$ is generated under colimits by the $R[i]$, and $\Mod_{HR}$ is generated under colimits by the $HR[i]$.
	\end{alphanumerate}
	To show that $\Hom_{\Dd(R)}(R[i],-)\simeq \Omega^\infty\hom_{\Dd(R)}(R[i],-)$ commutes with filtered colimits, it suffices to show the same for $\pi_j\hom_{\Dd(R)}(R[i],-)$, using that $\Omega^\infty$ and $\pi_j$ preserve filtered colimits (\cref{rem*:piPreservesFilteredColimits}) and that equivalences of spectra can be detected on homotopy groups (\cref{lem*:WhiteheadForSpectra}). We've seen in the proof of \cref{lem:HLaxMonoidal} that $\hom_{\Dd(R)}(R[0],-)\simeq H\colon \Dd(R)\morphism \Sp$. Hence
	\begin{equation*}
		\pi_j\hom_{\Dd(R)}\big(R[i],-\big)\simeq \pi_j H\big(-[-i]\big)\simeq H_{i+j}\colon \Dd(R)\morphism \Ab
	\end{equation*}
	is the functor taking homology in degree $i+j$. These functors preserve fitered colimits. The above description also immediately shows that the functors $\pi_0\Hom_{\Dd(R)}(R[i],-)\colon \Dd(R)\morphism \cat{Ab}$ for $i\in\IZ$ are jointly conservative.
	
	The corresponding assertions for $HR$ hold more generally for any $\IE_\infty$-ring spectrum $T$. Indeed, \cref{cor*:LModFreeAdjunction} implies
	\begin{equation*}
		\Hom_{\Mod_T}\big(T[i],-\big)\simeq \Hom_\Sp\big(\IS[i],-\big)\simeq \Omega^{\infty+i}\,.
	\end{equation*}
	$\Omega^{\infty+i}$ preserves filtered colimits by \cref{rem*:piPreservesFilteredColimits}, and the functors $\pi_0\Omega^{\infty+i}\simeq \pi_i\colon \Sp\morphism \cat{Ab}$ for $i\in \IZ$ are jointly conservative by \cref{lem*:WhiteheadForSpectra}. Combine this with the fact that colimits and equivalences in $\Mod_{HR}$ can be detected on underlying spectra by \cref{cor:LModHasCoLimits} and the Segal condition from \cref{thm:AlgLModDescription}, respectively, to  finish the proof of \itememph{1}.
	
	The above arguments also show that $H\colon \Dd(R)\morphism \Mod_{HR}$ commutes with filtered colimits. But $H$ is an exact functor between stable $\infty$-categories, hence it also commutes with finite colimits. But every colimit can be obtained from coproducts (which are filtered colimits of finite coproducts) and pushouts (which are finite), proving \itememph{2}.
	
	It remains to show \itememph{3}. We only show the part about $HR$, since the part about $\Dd(R)$ is similar. Again, we may replace $HR$ by an arbitrary $\IE_\infty$-ring spectrum $T$. Let $\Cc\subseteq \Mod_T$ denote the full sub-$\infty$-category generated by the $T[i]$ under finite colimits. We claim that the canonical map
	\begin{equation*}
		\colimit_{X\in \Cc/M}X\isomorphism M
	\end{equation*}
	is an equivalence for all $M\in\Mod_T$, which will imply that $\Mod_T$ is generated by the $T[i]$ under colimits. By \itememph{1}, the equivalence above may be checked after applying $\pi_0\Hom_{\Mod_T}(T[i],-)$ on both sides. Note that the indexing $\infty$-category $\Cc/M$ is filtered. Indeed, any map $K\morphism \Cc/M$ from a finite simplicial set $K$ can be extended over $K^\triangleright$ by taking the colimit in $\Cc$. Using that $T[i]$ is compact, we thus need to show that
	\begin{equation*}
		\colimit_{X\in \Cc/M}\pi_0\Hom_{\Mod_T}\big(T[i],X\big)\isomorphism \pi_0\Hom_{\Mod_T}\big(T[i],M\big)
	\end{equation*}
	is an isomorphism of abelian groups. It is surjective because each map $f\colon T[i]\morphism M$ on the right-hand side also defines an element of $\Cc/M$, the image of $\id_{T[i]}\in \pi_0\Hom_{\Mod_T}(T[i],T[i])$ in the colimit maps to $f$. For injectivity, assume $g\colon T[i]\morphism X$ is mapped to $0$ on the right-hand side. Then $X\morphism M$ extends over $\cofib(g)$. Note that $\cofib(g)$ is an element of $\Cc$ again, and that $g$ maps to $0$ in $\pi_0\Hom_{\Mod_T}(T[i],\cofib(g))$. Hence the image of $g$ in the colimit is $0$. This proves \itememph{3}.
	
	Now note that $\hom_{\Dd(R)}(R[0],M)\isomorphism \hom_{HR}(HR,HM)$ is an equivalence for all $M\in \Dd(R)$, since both sides coincide with $HM$ by the arguments in \itememph{1}. The full sub-$\infty$-category of all $X\in \Dd(R)$ for which $\hom_{\Dd(R)}(R[0],M)\isomorphism \hom_{HR}(HR,HM)$ is an equivalence is clearly closed under shifts and colimits, hence it must be all of $\Dd(R)$ by \itememph{3}. Thus $H$ is fully faithful. Its essential image is closed under colimits by \itememph{2} and contains all $HR[i]$. Hence $H$ is also essentially surjective by \itememph{3} again.	
	%The proof is based on the following general fact (which we'll later discuss in detail in \cref{par:Ind}):
	%\begin{alphanumerate}
	%	\item[\itememph{\boxtimes}]\itshape Let $\Cc$ be an $\infty$-category with finite colimits and a set $S$ of compact objects such that the functors 
	%	\begin{equation*}
	%		\Hom_\Cc(s,-)\colon \Cc\morphism\An
	%	\end{equation*}
	%	are jointly conservative. Let $\Cc^\omega\subseteq\Cc$ denote the full sub-$\infty$-category of compact objects. Then the restricted Yoneda embedding $\Cc\morphism\Pp(\Cc^\omega)$ is fully faithful with essential image described equivalently as
	%	\begin{alphanumerate}
	%		\item those functors $(\Cc^\omega)^\op\morphism \An$ that preserve finite limits, or
	%		\item the smallest subcategory of $\Pp(\Cc^\omega)$ that contains $\Cc^\omega$ and is closed under filtered colimits \embrace{also known as the \emph{ind-completion} $\operatorname{Ind}(\Cc^\omega)$}.
	%	\end{alphanumerate}
	%\end{alphanumerate}
	%Such $\infty$-categories are called \emph{compactly generated}. Note that $\Dd(R)$ is compactly generated, with $S=\left\{R[i]\st i\in\IZ\right\}$. Indeed, to show that $\Hom_{\Dd(R)}(R[i],-)\simeq \Omega^\infty\hom_{\Dd(R)}(R[i],-)$ are jointly conservative and commute with filtered colimits, it suffices to show the same for $\pi_j\hom_{\Dd(R)}(R[i],-)$, using that $\Omega^\infty$ and $\pi_j$ preserve filtered colimits (\cref{rem*:piPreservesFilteredColimits}) and that equivalences of spectra can be detected on homotopy groups (\cref{lem*:WhiteheadForSpectra}). We've seen in the proof of \cref{lem:HLaxMonoidal} that $\hom_{\Dd(R)}(R[0],-)\simeq H\colon \Dd(R)\morphism \Sp$. Hence
	%\begin{equation*}
	%	\pi_j\hom_{\Dd(R)}\big(R[i],-\big)\simeq \pi_j H\big(-[-i]\big)\simeq H_{i+j}\colon \Dd(R)\morphism \Ab
	%\end{equation*}
	%is the functor taking homology in degree $i+j$. These functors preserve fitered colimits and are clearly jointly conservative, as claimed.
	%
	%We remark for future use that this argument also proves that $H\colon \Dd(R)\morphism \Sp$ is conservative and commutes with filtered colimits. In fact, it even shows that $H$ commutes with arbitrary colimits! Indeed, $H$ is an exact functor between stable $\infty$-categories, hence it also preserves finite colimits. But every colimit can be obtained from coproducts (which are filtered colimits of finite coproducts) and pushouts (which are finite). The upshot is that also the functor $H\colon \Dd(R)\morphism \Mod_{HR}$ we're interested in is conservative and colimit-preserving, since colimits and equivalences in $\Mod_{HR}$ can be detected on underlying spectra by \cref{cor:LModHasCoLimits}, and the Segal condition from \cref{thm:AlgLModDescription}, respectively.
	%
	%Similarly, $\Mod_T$ is a compactly generated $\infty$-category for any $\IE_\infty$-ring spectrum $T$, with $S=\left\{T[i]\st i\in\IZ\right\}$ a set of compact generators. Indeed, \cref{cor*:LModFreeAdjunction} implies
	%\begin{equation*}
	%	\Hom_{\Mod_T}\big(T[i],-\big)\simeq \Hom_\Sp\big(\IS[i],-\big)\simeq \Omega^{\infty+i}\,,
	%\end{equation*}
	%and the right-hand sides are clearly jointly conservative and preserve filtered colimits by \cref{rem*:piPreservesFilteredColimits}.
	%
	%Now we can begin with the actual proof. It's a general fact that a conservative functor is an equivalence iff it has a left adjoint and the counit is an equivalence (use the triangle relations to show that the unit must be an equivalence as well). By Lurie's presentability magic (more precisely, \cite[Corollary~\HTTthm{5.5.2.9}]{HTT}), a functor between compactly generated $\infty$-categories has a left adjoint if it preserves preserves limits and filtered colimits, and a right adjoint if it preserves colimits. This is certainly true for $H\colon \Dd(R)\morphism \Mod_{HR}$: It preserves limits since these can be computed on underlying spectra by \cref{cor:LModHasCoLimits} and the underlying functor $H\colon \Dd(R)\morphism\Sp$ is a right adjoint. Preservation of colimits was checked above. Hence we get a left adjoint (and a right adjoint, which we won't need) for free and all we need to check is that the counit is an equivalence.
	%
	%If $L\colon \Mod_{HR}\morphism \Dd(R)$ denotes the left adjoint, then
	%\begin{align*}
	%	\Hom_{\Dd(R)}\big(L(HR),C\big)&\simeq \Hom_{\Mod_{HR}}(HR,HC)\\
	%	&\simeq \Hom_\Sp(\IS,HC)\\
	%	&\simeq \Hom_{\Dd(R)}\big(R[0],C\big)\,,
	%\end{align*}
	%using the adjunctions from \cref{cor*:LModFreeAdjunction} and \cref{lem:HLaxMonoidal}. Hence the counit $LH\Rightarrow \id_{\Dd(R)}$ is an equivalence on $R[0]$. Since $\Mod_{HR}$ is stable by \cref{cor:LModHasCoLimits}, $L$ is a colimit-preserving functor between stable $\infty$-categories, hence exact by \cref{cor:ExactFunctors}. Thus both $LH$ and $\id_{\Dd(R)}$ are exact and preserve colimits, and therefore the sub-$\infty$-category of $\Dd(R)$ where the counit is an equivalence must be stable under shifts and colimits. But then it must be all of $\Dd(R)$: It contains $R[0]$, hence all $R[i]$, hence everything as $\Dd(R)$ is compactly generated.
\end{proof*}
Fabian also mentioned in the lecture (and elaborates in the script, see \cite[Theorem~II.58]{KTheory}) that one can more or less the same argument as in the proof of \cref{thm:D(R)IsModOverHR} to show a vast generalisation: The \emph{Schwede--Shipley theorem}. 
\begin{thm}[Schwede--Shipley]
	If $\Cc$ is a complete stable $\infty$-category containing an object $x\in \Cc$ such that $\hom_\Cc(x,-)\colon \Cc\morphism\Sp$ is conservative and commutes with colimits, then this functor lifts to an equivalence
	\begin{equation*}
		\hom_\Cc(x,-)\colon \Cc\isomorphism \Mod_{\operatorname{end}(x)^\op}\,,
	\end{equation*}
	where $\operatorname{end}(x)^\op$ denotes the opposite algebra of the canonical refinement of $\operatorname{end}(x)\coloneqq \hom_\Cc(x,x)\in\Sp$ to an object of $\Alg_\IAssoc(\Sp^\otimes)$.
\end{thm}

\subsection{Tensor Products over Arbitrary Bases}
By \cref{thm:D(R)IsModOverHR},  the symmetric monoidal structure on $\Dd(R)$ can be carried over to $\Mod_{HR}$. More generally, we would like to define symmetric monoidal structures $\Mod_T^{\otimes_T}$ on $\Mod_T$ for any $\IE_\infty$-ring spectrum $T$. In the special case $T\simeq HR$ it should be given by
\begin{equation*}
	HM\otimes_{HR}HN\simeq H(M\otimes_R^LN)\,.
\end{equation*}
for any $M,N\in \Dd(R)$. In fact, constructing symmetric monoidal structures on $\infty$-categories of algebras works ridiculous generality: Given a symmetric monoidal $\infty$-category $q\colon \Cc^\otimes\morphism\IGamma^\op$ and an $\infty$-operad $p\colon\Oo\morphism\IGamma^\op$, the $\infty$-category of algebras $\cat{Alg}_\Oo(\Cc^\otimes)$ inherits a symmetric monoidal structure $\cat{Alg}_\Oo(\Cc^\otimes)^\otimes$,

\begin{con}
	With notation as above, define $\ov{\Alg_\Oo(\Cc^\otimes)}^\otimes\morphism \IGamma^\op$ via \labelcref{par:PresheafConstruction} applied to the presheaf $F\colon \IDelta/\IGamma^\op\morphism\An$ which is given by
	\begin{equation*}
		F \big(f\colon [n]\rightarrow\IGamma^\op\big)\simeq \Hom_{\Cat_\infty}\big([n]\times\Oo,\Cc^\otimes\big)\times_{\Hom_{\Cat_\infty}([n]\times\Oo,\IGamma^\op)}\{f\times p\}\,,
	\end{equation*}
	where $f\times p\colon [n]\times \Oo\morphism \IGamma^\op$ denotes the composition $[n]\times\Oo\morphism \IGamma^\op\times\IGamma^\op\morphism[\times]\IGamma^\op$. Let's compute the fibres
	\begin{equation*}
		\ov{\Alg_\Oo(\Cc^\otimes)}^\otimes_n\simeq \ov{\Alg_\Oo(\Cc^\otimes)}^\otimes\times_{\IGamma^\op}\{\langle n\rangle\}\,.
	\end{equation*}
	Using the method from the proof sketch of \cref{prop:CocartesianMonoidalStructure}, we find that the Rezk nerve of the fibre over $\langle n\rangle$ is a simplicial anima $Y$ given by $Y_k\simeq F(\const {\langle n\rangle}\colon [k]\morphism \IGamma^\op)$. Plugging in the definitions, we get
	\begin{align*}
		Y_k&\simeq \Hom_{\Cat_\infty}\big([k]\times \Oo,\Cc^\otimes\big)\times_{\Hom_{\Cat_\infty}([k]\times \Oo,\IGamma^\op)}\{\const {\langle n\rangle}\times p\}\\
		&\simeq \Hom_{\Cat_\infty}\big([k],\Fun(\Oo,\Cc^\otimes)\big)\times_{\Hom_{\Cat_\infty}([k],\Fun(\Oo,\IGamma^\op))}\{\const {\langle n\rangle \times p}\}\\
		&\simeq \Hom_{\Cat_\infty}\big([k],\Fun(\Oo,\Cc^\otimes)\times_{\Fun(\Oo,\IGamma^\op)}\{\langle n\rangle\times p\}\big)\\
		&\simeq \Hom_{\Cat_\infty}\big([k],\Fun_{\IGamma^\op}(\Oo,\underbrace{\Cc^\otimes\times_{\IGamma^\op}\dotsb\times_{\IGamma^\op}\Cc^\otimes}_{n\text{ times}})\big)\,.
	\end{align*}
	For the second and third equivalence, we use currying and the fact that $\Hom_{\Cat_\infty}([k],-)$ commutes with pullbacks. For the fourth equivalence, we use that maps $\Oo\morphism \Cc^\otimes$ over $\IGamma^\op$, where $\Oo$ is equipped with the structure map $\langle n\rangle \times p\colon \Oo\morphism\IGamma^\op$, are the same as maps from $\Oo$ with its usual structure map $p$ to the pullback of $\Cc^\otimes$ along $\langle n\rangle \times -\colon \IGamma^\op\morphism\IGamma^\op$. Via the Segal maps, said pullback is easily identified with the $n$-fold fibre product of $\Cc^\otimes$ over $\IGamma^\op$.
	
	The calculation above shows that $Y\simeq \N^r(\Fun_{\IGamma^\op}(\Oo,\Cc^\otimes)^n)$, hence the fibre we're interested in is given by
	\begin{equation*}
		\ov{\Alg_\Oo(\Cc^\otimes)}^\otimes_n\simeq \Fun_{\IGamma^\op}(\Oo,\Cc^\otimes)^n\,.
	\end{equation*}
	We let $\Alg_\Oo(\Cc^\otimes)^\otimes\subseteq\ov{\Alg_\Oo(\Cc^\otimes)}^\otimes$ be the full sub-$\infty$-category spanned fibrewise by those $(F_1,\dotsc,F_n)$ such that each $F_i$ is a map of $\infty$-operads.
\end{con}
\begin{prop}[Lurie, {\cite[Proposition~\HAthm{3.2.4.3}]{HA}}]\label{thm:AlgOSymmetricMonoidal}
	The map $\Alg_\Oo(\Cc^\otimes)^\otimes\morphism\IGamma^\op$ is always a symmetric monoidal monoidal $\infty$-category, and for every $x\in\Oo_1$, evaluation at $x$ refines to a strongly monoidal functor
	\begin{equation*}
		\ev_x\colon \Alg_\Oo(\Cc^\otimes)^\otimes\morphism \Cc^\otimes\,.
	\end{equation*}
\end{prop}
\begin{obs}\label{obs:AlgOCocartesian}
	In the special case $\Oo\simeq \IComm$, the symmetric monoidal structure on $\Alg_\IComm(\Cc^\otimes)\simeq \CAlg(\Cc^\otimes)$ is the cocartesian symmetric monoidal structure as defined in \cref{prop:CocartesianMonoidalStructure}. In particular, the tensor product of two algebras is also their coproduct in $\CAlg(\Cc^\otimes)$.
\end{obs}
\begin{proof}[Proof sketch of \cref{obs:AlgOCocartesian}]
	Use a generalisation of \cref{lem:SemiAddCriterion}, which we used to show that $\CMon(\Cc)$ is semi-additive. In fact, Lurie's version (\cite[Proposition~\HAthm{2.4.3.19}]{HA}) of that lemma gives a criterion for a symmetric monoidal structure to be cocartesian.
\end{proof}
By \cref{obs:AlgOCocartesian} the second half of \cref{thm:OMon}, we get
\begin{align*}
	\CAlg\big(\CAlg(\Cc^\otimes)^\otimes\big)&\simeq \Fun^{\Op_\infty}\big(\IComm,\CAlg(\Cc^\otimes)^\otimes\big)\\
	&\simeq \Fun\big(\IComm_1,\CAlg(\Cc^\otimes)\big)\\
	&\simeq \CAlg(\Cc^\otimes)\,.
\end{align*}
Hence every commutative algebra $A\in\CAlg(\Cc^\otimes)$ canonically refines to a map of $\infty$-operads $A\colon \IComm\morphism \CAlg(\Cc^\otimes)^\otimes$.  Now consider the following pullback diagram:
\begin{equation*}
	\begin{tikzcd}
		\LMod_A(\Cc^\otimes)\dar\rar\drar[pullback] & \LMod_A(\Cc^\otimes)^{\otimes_A}\dar\ar[rr]\ar[drr,pullback] & & \Alg_{\ILMod}(\Cc^\otimes)^\otimes\dar["a^*"]\\
		*\rar["\langle 1\rangle"] & \IComm\rar["A"] & \Alg_\IComm(\Cc^\otimes)^\otimes \rar & \Alg_\IAssoc(\Cc^\otimes)^\otimes
	\end{tikzcd}
\end{equation*}
\begin{thm}\label{thm:LModSymmetricMonoidal}
	Suppose $\Cc^\otimes$ has all small operadic colimits. Then
	\begin{equation*}
		a^*\colon \Alg_{\ILMod}(\Cc^\otimes)^\otimes\morphism\Alg_\IAssoc(\Cc^\otimes)^\otimes
	\end{equation*}
	is a cocartesian fibration. In particular, $\LMod_A(\Cc^\otimes)^{\otimes_A}\morphism \IComm$ defines a symmetric monoidal structure on $\LMod_A(\Cc^\otimes)$ for all $A\in\CAlg(\Cc^\otimes)$. Moreover, $\LMod_A(\Cc^\otimes)^{\otimes_A}$ has again all operadic colimits.
\end{thm}\refstepcounter{smallerdummy}
\numpar*{\thesmallerdummy. The Bar Construction}\label{par:BarConstruction}
The tensor product in $\LMod_A(\Cc^\otimes)$ can be computed by the \emph{bar construction}: For $M,N\in \LMod_A(\Cc^\otimes)$, the underlying $\Cc$-object of $M\otimes_AN$ (which we'll abusingly also denote that way) is given by
\begin{equation*}
	M\otimes_AN\simeq \colimit_{\IDelta^\op}\operatorname{Bar}(M,A,N)\,.
\end{equation*}
Informally, the functor $\operatorname{Bar}(M,A,N)\colon \IDelta^\op\morphism \Cc$ sends $[n]\in\IDelta^\op$ to $M\otimes A^{\otimes n}\otimes N$, with the tensor product being taken in $\Cc$. 

In the script \cite[Chapter~II pp.\:123--124]{KTheory}, Fabian explains how to define the bar construction formally. Let $\operatorname{rev}\colon \IDelta^\op\morphism\Delta^\op$ denote the functor that reverses the order of a totally ordered set. Using \cref{thm:AlgAssocDescription}, it induces a functor
\begin{equation*}
	\operatorname{rev}^*\colon \Alg_\IAssoc(\Cc^\otimes)\morphism\Alg_\IAssoc(\Cc^\otimes)\,.
\end{equation*}
But one easily checks that $\Cut\colon \IDelta^\op\morphism\IGamma^\op$ and $\Cut\circ \operatorname{rev}\colon \IDelta^\op\morphism\IDelta^\op$ are naturally equivalent, hence $\operatorname{rev}^*$ becomes equivalent to the identity upon composition with the canonical map $\CAlg(\Cc^\otimes)\morphism \Alg_\IAssoc(\Cc^\otimes)$. Of course, this is nothing but the $\infty$-categorical way of saying that a commutative algebra is naturally equivalent to its opposite.

Using \cref{thm:AlgLModDescription}, we may now consider the functors 
\begin{equation*}
	M\circ (\operatorname{rev}\times\id_{[1]}), N\colon \IDelta^\op\times[1]\morphism \Cc^\otimes\,.
\end{equation*}
Let's also write $M\circ (\operatorname{rev}\times\id_{[1]})\eqqcolon M_A$ for simplicity (indicating that we now consider the right-$A$ module structure on $M$, although we never defined anything of that sort). Then $M_{A|\IDelta^\op\times\{1\}}\simeq A^\mathrm{rev}\simeq A\simeq N_{|\IDelta^\op\times\{1\}}$ by what we just showed. In particular, if we regard the functors $M_A$ and $N$ as natural transformations $M_{A|\IDelta^\op\times\{0\}}\Rightarrow M_{A|\IDelta^\op\times\{1\}}$ and $N_{|\IDelta^\op\times\{0\}}\Rightarrow N_{|\IDelta^\op\times\{1\}}$ in $\Fun(\IDelta^\op,\Cc^\otimes)$, then their endpoints coincide, and we can form the pullback $P\in\Fun(\IDelta^\op,\Cc^\otimes)$. Pullbacks in functor categories are formed pointwise (\cref{lem:f^*preservesColimits}), hence $P([n])$ fits into a pullback diagram
\begin{equation*}
	\begin{tikzcd}
		P\big([n]\big)\rar\dar\drar[pullback] & N\big([n],0\big)\dar\\
		M_A\big([n],0\big)\rar & A\big([n]\big)
	\end{tikzcd}
\end{equation*}
in $\Cc^\otimes$. By construction, we have $M_A([n],0)\simeq (M,A,\dotsc,A)$, $N([n],0)\simeq (A,\dotsc,A,N)$, and $A([n])\simeq (A,\dotsc,A)$, where we abusingly identify algebras and modules with their underlying $\Cc$-objects, and the maps between them are the evident projections. Thus, using \cref{def:Operad}\itememph{c} one easily checks
\begin{equation*}
	P\big([n]\big)\simeq (M,A\dotsc,A,N)\in \Cc_{n+2}^\otimes\,.
\end{equation*}
Now observe that $P\colon \IDelta^\op\morphism \Cc^\otimes$ factors over $\Cc_\mathrm{act}^\otimes\coloneqq \Cc^\otimes\times_{\IGamma^\op}\IGamma^\op_\mathrm{act}$. Indeed, for a general $\alpha\colon [n]\morphism {[m]}$ in $\IDelta$, the morphism $\Cut(\alpha^\op)\colon \langle m\rangle \morphism\langle n\rangle$ is only undefined on those $i\in\langle m\rangle$ for which $i\leq \alpha(0)$ or $\alpha(n)<i$, see \cref{def:CartesianCommutativeMonoids}. But those $i$ with $\alpha(n)<i$ are used to encode the left-action of $A$ on $N$ (compare this to \cref{rem*:AlgLModDescription}), and those $i$ with $i\leq \alpha(0)$ are used to encode the right action of $A$ on $M_A$. Hence the image of $P(\alpha^\op)$ in $\IGamma^\op$ must be active, as claimed.

The unique active maps $f_n\colon \langle n\rangle\morphism \langle 1\rangle$ define a natural transformation $\id_{\IGamma^\op_\mathrm{act}}\Rightarrow \const {\langle 1\rangle}$ in $\Fun(\IGamma_\mathrm{act}^\op,\IGamma_\mathrm{act}^\op)$. By unstraightening magic, it defines a map of cocartesian fibrations from $\Cc_\mathrm{act}^\otimes$ to its pullback along $\const {\langle 1\rangle}$, which is $\Cc\times\IGamma^\op_\mathrm{act}$. Composing $P$ with that map gives a functor
\begin{equation*}
	\operatorname{Bar}(M,A,N)\colon \IDelta^\op\morphism \Cc\,,
\end{equation*}
which is the one we're looking for. Since cocartesian lifts of the active maps $f_{n+2}\colon \langle n+2\rangle\morphism \langle 1\rangle$ just tensor stuff together, we see that $\operatorname{Bar}(M,A,N)$ does indeed take $[n]$ to $M\otimes A^{\otimes n}\otimes N$.

We won't explain why the bar construction computes the tensor product in $\LMod_A(\Cc^\otimes)$, and refer instead to \cite{HA} again: In Proposition~\HAthm{4.4.2.8}, Lurie proves that the tensor product in \emph{bimodules} is given by the bar construction, and in Theorem~\HAthm{4.5.2.1} he shows that the tensor product on $\LMod_A(\Cc^\otimes)$ for $A\in \CAlg(\Cc^\otimes)$ factors over the specialisation from bimodules to left modules.
\refstepcounter{smallerdummy}

\numpar*{\thesmallerdummy. The Tensor Unit*}\label{par:TensorUnit}
Unravelling the construction in \cref{thm:LModSymmetricMonoidal}, we see that $A$ is the tensor unit in $\LMod_A(\Cc^\otimes)^{\otimes_A}$, as it should be. As a reality check, let's see that this also comes out of the Bar construction. We must show
\begin{equation*}
	N\simeq \colimit_{\IDelta^\op}\operatorname{Bar}(A,A,N)\,.
\end{equation*}

This is, in fact, a special case of a general principle. Observe that the bar construction $B\coloneqq \operatorname{Bar}(A,A,N)$ is the \emph{décalage} of another simplicial object in $\Cc$. That is, there is another functor $B'\colon \IDelta^\op\morphism\Cc$ such that the following diagram commutes:
\begin{equation*}
	\begin{tikzcd}[column sep=small]
		\IDelta^\op\ar[rr,"\sigma^\op"]\drar["B"'] & & \IDelta^\op\dlar["B'"]\\
		& \Cc
	\end{tikzcd}
\end{equation*}
Here $\sigma\colon \IDelta \morphism\IDelta $ denotes the functor taking objects $[n]$ to $[n+1]$ and morphisms $\alpha\colon [n]\morphism{[m]}$ to $\sigma(\alpha)\colon [n+1]\morphism {[m+1]}$ given by $\sigma(\alpha)(i)=\alpha(i)$ for $i\leq n$ and $\sigma(\alpha)(n+1)=m+1$.

It's not hard to construct $B'$ in our example. Basically, one has to place another $M$ in degree $0$ and add those maps $ A^{\otimes n}\otimes N\morphism A^{\otimes n+1}\otimes N$ that are given by $1_\Cc\morphism A$ on the first factor and the identity on the rest, as well as those maps $ A^{\otimes n+1}\otimes N\morphism A^{\otimes n}\otimes N $ that multiply the first factor with $N$ (so we are forming some kind of cyclic bar construction). I'll leave the formal construction to you.

In general, if $B,B'\colon \IDelta^\op\morphism \Cc$ are simplicial objects in some $\infty$-category $\Cc$, and if $B\simeq \operatorname{d\acute{e}c}(B')$ is the décalage of $B'$, then
\begin{equation*}
	\colimit_{\IDelta^\op}B\simeq \colimit_{\IDelta^\op}\const {B_0'}\simeq B_0'\,.
\end{equation*}
The right equivalence follows from the fact that $\IDelta^\op$ is weakly contractible (with $[0]$ an initial object). For the left-hand side, observe that there are canonical transformations $B \Rightarrow\const {B_0'}$ and $\const {B_0'} \Rightarrow B$ in $\Fun(\IDelta^\op,\Cc)$, induced by suitable natural transformations $\eta\colon \sigma^\op\Rightarrow \const {[0]}$ and $\tau\colon \const {[0]}\Rightarrow \sigma^\op$ in $\Fun(\IDelta^\op,\IDelta^\op)$. Using higher order transformations between $\eta$ and $\sigma$, one shows that
\begin{equation*}
	\tau^*\colon\Nat(B,\const x)\doublelrmorphism[\sim][\sim] \Nat(\const {B_0'},\const x)\noloc\eta^*
\end{equation*}
are mutually inverse homotopy equivalences, which is what we need.

\refstepcounter{smallerdummy}

\numpar*{\thesmallerdummy. Relative Tensor Products of Spectra*}\label{par:TensorHomInCAlg}
We can now define symmetric monoidal structures on the stable $\infty$-categories $\Mod_R^{\otimes_R}$ for all $\IE_\infty$-ring spectra $R$. We'd hope the following properties to hold:
\begin{alphanumerate}\itshape
	\item The functor $\hom_{\Mod_R}\colon \Mod_R^\op\times \Mod_R\morphism\Sp$ has a natural refinement to a functor $\hom_R\colon \Mod_R^\op\times\Mod_R\morphism\Mod_R$ satisfying the tensor-$\Hom$ adjunction
	\begin{equation*}
		\hom_R(M\otimes_RN,L)\simeq \hom_R\big(M,\hom_R(N,L)\big)\,.
	\end{equation*}
	\item If $R\morphism S$ is a map in $\CAlg$, we get a symmetric monoidal functor
	\begin{equation*}
		S\otimes_R-\colon \Mod_R^{\otimes_R}\morphism \Mod_S^{\otimes_S}\,,
	\end{equation*}
	which does what it says on underlying spectra. If moreover $S\morphism T$ is another map in $\CAlg$, then $T\otimes_S(S\otimes_R -)\simeq  T\otimes_R-$.
	\item The functor $S\otimes_R-\colon \Mod_R\morphism\Mod_S$ has a right adjoint, the forgetful functor
	\begin{equation*}
		F_{S/R}\colon \Mod_S\morphism\Mod_R\,,
	\end{equation*}
	which is the identity on underlying spectra. Moreover, $F_{S/R}$ has a right adjoint itself, which is a refinement of $\hom_R(S,-)\colon \Mod_R\morphism\Mod_R$.
\end{alphanumerate}
\begin{proof*}
Part~\itememph{b} works in fact in much greater generality, see \cite[Theorem~\HAthm{4.5.3.1}]{HA}. We will see that the rest are formal consequences.

The existence of a functor $\hom_R\colon \Mod_R^\op\times \Mod_R\morphism\Mod_R$ in part~\itememph{a} follows from adjoint functor theorem magic
(initially, we only get a tensor-$\Hom$ adjunction with the outer $\hom_R$ replaced by the actual $\Hom$ functor $\Hom_{\Mod_R}\colon \Mod_R^\op\times \Mod_R\morphism \An$, but then it formally follows for $\hom_R$ as well). To see that $\hom_R$ equals $\hom_{\Mod_R}$ on underlying spectra, we have to prove \itememph{c} first.

The existence of $F_{S/R}$ follows again from Lurie's adjoint functor theorem: We know that $S\otimes_R-$, as an endofunctor of $\Mod_R$, preserves colimits by \cref{thm:LModSymmetricMonoidal}, hence the same is true for $S\otimes_R-\colon \Mod_R\morphism\Mod_S$, since both equivalences and colimits in $\Mod_S$ can be detected on underlying spectra by the Segal condition from \cref{thm:AlgLModDescription}, and \cref{cor:LModHasCoLimits}, respectively.  Hence $F_{S/R}$ exists. To see that it is the identity on underlying spectra, note that $S\otimes_R(R\otimes_\IS -)\simeq S\otimes_\IS -$ implies $F_{R/\IS}\circ F_{S/R}\simeq F_{S/\IS}$, and both $F_{R/\IS}$ and $F_{S/\IS}$ are just extraction of underlying spectra by \cref{cor*:LModFreeAdjunction}.  This also implies that $F_{S/R}$ preserves colimits, since these can be detected on underlying spectra. Hence $F_{S/R}$ has itself a right adjoint  $h_{S/R}\colon \Mod_R\morphism\Mod_S$. Since right adjoints compose, $F_{S/R}\circ h_{S/R}$ is a right adjoint of $S\otimes_R-\colon \Mod_R\morphism \Mod_R$, hence $h_{S/R}$ is indeed a refinement of $\hom_R(S,-)\colon \Mod_R\morphism\Mod_R$. This proves \itememph{c}.

Finally, to show that the underlying spectrum of $\hom_R(M,N)$ is $\hom_{\Mod_R}(M,N)$, it suffices to consider the case $M=R$, since then the general case follows by writing $M$ as a colimit of shifts of $R$ (see the proof of \cref{thm:D(R)IsModOverHR} for why that's possible). But $\hom_R(R,N)\simeq N$ by the tensor-$\Hom$ adjunction, and $\hom_{\Mod_R}(R,N)\simeq \hom_\Sp(\IS,N)\simeq N$ by the adjunction from \cref{cor*:LModFreeAdjunction}.
\end{proof*}
It remains to see why in the case of an ordinary ring $R$, the derived tensor product on $\Dd(R)$ and the tensor product on $\Mod_{HR}$ coincide. To this end, observe that for arbitrary symmetric monoidal $\infty$-categories, there's a strongly monoidal functor
\begin{equation*}
	\Cc^\otimes\morphism \Alg_\ILMod(\Cc^\otimes)^\otimes\,,
\end{equation*}
which informally sends objects $(c_1,\dotsc,c_n)$ to $((1_\Cc,c_1),\dotsc,(1_\Cc,c_n))$, considering each $c_i$ as a module over the trivial algebra $1_\Cc$ via \cref{cor:LMod_1C=C}.
\begin{prop}\label{prop:HStronglyMonoidal}
	The composite
	\begin{equation*}
		\Dd(R)^{\otimes_R^L}\morphism\Alg_\ILMod\big(\Dd(R)^{\otimes_R^L}\big)^{\otimes_R^L}\morphism[H]\Alg_\ILMod(\Sp^\otimes)^\otimes
	\end{equation*}
	induces an equivalence of symmetric monoidal $\infty$-categories
	\begin{equation*}
		H\colon \Dd(R)^{\otimes_R^L}\isomorphism\Mod_{HR}^{\otimes_{HR}}\,.
	\end{equation*}
\end{prop}
\begin{proof*}[Proof sketch]
	We already know that $H$ is an equivalence of underlying $\infty$-categories. Hence it suffices to show
	\begin{equation*}
		H\big(R[0]\big)\simeq HR\quad\text{and}\quad H(M\otimes_R^LN)\simeq HM\otimes_{HR}HN
	\end{equation*}
	(since then it formally follows that $H$ preserves cocartesian lifts of active morphisms, hence all cocartesian lifts). There's nothing to show for the first equivalence. For the second, we'll use the bar construction from \labelcref{par:BarConstruction}. Since $H$, being an equivalence, preserves colimits and shifts, it suffices to consider the case $M=R[0]$, which follows immediately from the fact that $R$ is the tensor unit in $\Dd(R)^{\otimes_R^L}$ and $HR$ is the tensor unit in $\Mod_{HR}^{\otimes_{HR}}$.
\end{proof*}

\subsection{Miscellanea}
The following stuff was only implicitly mentioned in the lecture (if at all), but Fabian gave a detailed account in his notes, \cite[Remarks~II.53]{KTheory}.
\numpar{Graded Structures on Homotopy Groups*}\label{par:pi*GradedRing}
The functor $\pi_0\colon \Sp\morphism \Ab$ is lax symmetric monoidal, since it is the underlying functor of the composition
\begin{equation*}
	\Sp^\otimes\morphism[\Omega^\infty] \CGrp(\An)^\otimes\morphism[\pi_0]\Ab^\otimes\,,
\end{equation*}
where we denote $\CGrp(\An)^\otimes\simeq \CGrp^{\Op_\infty}(\An^\times)$ (which makes sense, since the right-hand side is symmetric monoidal by \cref{prop:Sp(C)Monoidal}), and use $\CGrp^{\Op_\infty}(\Set^\times)\simeq \Ab^\otimes$ by Example~\labelcref{exm:TensorProductCalculations}\itememph{a}.

This implies that $\pi_0$ induces functors
\begin{align*}
	\pi_0\colon \CAlg\simeq \CAlg(\Sp^\otimes)&\morphism \CAlg(\Ab^\otimes)\simeq \cat{CRing}\,,\\
	\pi_0\colon \Mod_R\simeq \LMod_R(\Sp^\otimes)&\morphism \LMod_{\pi_0(R)}(\Ab^\otimes)\simeq \Mod_{\pi_0(R)}
\end{align*}
for every $\IE_\infty$-ring spectrum $R$. In particular, $\pi_0(R)$ is an ordinary ring, and if $M$ is an $R$-module, then $\pi_0(M)$ is an ordinary $\pi_0(R)$-module. But we can do better! For all $i,j\in\IZ$, the multiplication $\mu\colon R\otimes M\morphism M$ induces morphisms
\begin{equation*}
	\pi_0\big(\IS[-i]\otimes R\big)\otimes \pi_0\big(\IS[-j]\otimes M\big)\morphism \pi_0\big(\IS[-(i+j)]\otimes R\otimes M\big)\morphism[\mu] \pi_0\big(\IS[-(i+j)]\otimes M\big)\,,
\end{equation*}
which provide a multiplication $\pi_i(R)\otimes \pi_j(M)\morphism \pi_{i+j}(M)$. In the special case $R=M$, it turns $\pi_*(R)$ into a graded ring, and for general $M$ we get a graded $\pi_*(R)$-module structure on $\pi_*(M)$!

In fact, $\pi_*(R)$ is graded commutative. To see this, we need to investigate the effect of the canonical equivalence $\IS[i]\otimes \IS[j]\simeq \IS[j]\otimes \IS[i]$ (we drop the minus signs for convenience) on homotopy groups. This equivalence corresponds to an element of
\begin{equation*}
	\pi_0\Hom_\Sp\big(\IS[i+j],\IS[i+j]\big)\simeq \pi_{i+j}\big(\IS[i+j]\big)\simeq \IZ\,.
\end{equation*}
Now $\IS[i]\simeq \Sigma^\infty\IS^i$ and $\IS[j]\simeq \Sigma^\infty\IS^j$. Since $\Sigma^\infty$ is strongly monoidal (\cref{prop:Sp(C)Monoidal}), we see that the equivalence $\IS[i]\otimes \IS[j]\simeq \IS[j]\otimes\IS[i]$ corresponds to
\begin{equation*}
	\IS^i\wedge \IS^j\simeq (\IS^1)^{\wedge i}\wedge (\IS^1)^{\wedge j}\simeq (\IS^1)^{\wedge j}\wedge (\IS^1)^{\wedge i}\simeq \IS^j\wedge \IS^i\,,
\end{equation*}
given by permutation of factors. The permutation in question can be written as $ij$ transpositions, hence its sign is $(-1)^{ij}$. Hence its action on $\pi_{i+j}(\IS^{i+j})\simeq \IZ$ is given by $(-1)^{ij}$ as well. This proves graded commutativity.

Combining these considerations with \labelcref{par:EinftyRingSpectra}\itememph{c^*} for example, we see that the algebraic $K$-theory $K_*(R)=\pi_*(B^\infty k(R))$ of any commutative ring $R$ carries a graded commutative ring structure, called the \emph{cup product} on $K$-theory. In the next paragraph we'll see that also the cup product on singular cohomology arises in this way.

\numpar{The Cup Product and the Pontryagin Product*}\label{par:CupProduct}
If $M\in \CGrp(\An)$, then $\IS[M]$ is an $\IE_\infty$-ring spectrum, as noted way back in \cref{par:EinftyRingSpectra}. Using \cref{thm:AlgOSymmetricMonoidal}, this implies that $\IS[M]\otimes HR$ is an $\IE_\infty$-ring spectrum too for any ordinary ring $R$. Hence, by \cref{par:pi*GradedRing} and \cref{cor*:SpectraCohomologyTheory} there is a graded commutative ring structure on
\begin{equation*}
	\pi_*\big(\IS[M]\otimes HR\big)\simeq H_*(M,R)\,.
\end{equation*}
This is (our definition of) the \emph{Pontryagin product}. A natural question is whether the cup product in cohomology arises in a similar way. It does! The main ingredient is a construction due to Glasman, which upgrades the Yoneda embedding $\Yo^\Cc\colon \Cc\morphism \Fun(\Cc^\op,\An)$ to a lax symmetric monoidal functor
\begin{equation*}
	\Cc^\otimes\morphism\operatorname{Day}(\Cc_\otimes^\op,\An^\times)\,,
\end{equation*}
where $\Cc_\otimes^\op$ denotes the induced symmetric monoidal structure on $\Cc^\op$ (i.e., compose the functor $\St^\cocart(\Cc^\otimes)\colon \IGamma^\op\morphism\Cat_\infty$ with $(-)^\op$ and observe that the Segal condition survives). In particular, if $A\in \CAlg(\Cc^\otimes)$ is a commutative algebra, which we abusingly identify with its underlying $\Cc$-object, then
\begin{equation*}
	\Hom_\Cc(-,A)\colon \Cc^\op\morphism\An
\end{equation*}
is an element of $\CAlg(\operatorname{Day}(\Cc_\otimes^\op,\An^\times))\simeq \Fun^{\Op_\infty}(\Cc_\otimes^\op,\An^\times)$ (as in the proof of \cref{lem:HLaxMonoidal}, we don't even need that this is an equivalence), hence a lax symmetric monoidal functor. Thus it defines a functor
\begin{equation*}
	\Hom_\Cc(-,A)\colon \CAlg(\Cc_\otimes^\op)\morphism\CAlg(\An^\times)\,.
\end{equation*}
The right-hand side equals $\CMon(\An)$ by \cref{thm:OMon}. In particular, if $B\in \CAlg(\Cc_\otimes^\op)$ is a coalgebra in $\Cc$, then $\Hom_\Cc(B,A)$ has a canonical refinement in $\CMon(\An)$.

Moreover one can check that $\Sp^{\Op_\infty}(\operatorname{Day}(\Cc_\otimes,\Oo))\simeq \operatorname{Day}(\Cc_\otimes^\op,\Sp^{\Op_\infty}(\Oo))$ for all $\infty$-operads $\Oo\in\Op_\infty^\mathrm{lex}$. Hence, if $\Cc^\otimes$ is a stable $\infty$-operad, we can apply \cref{thm:NikolausStabilisationOfOperads} to the functors $\Hom_\Cc(-,A)$ above and obtain upgraded lax symmetric monoidal functors
\begin{align*}
	\hom_\Cc(-,A)\colon \Cc_\otimes^\op&\morphism \Sp^\otimes\,,\\
	\hom_\Cc(-,A)\colon \CAlg(\Cc_\otimes^\op)&\morphism \CAlg(\Sp^\otimes)\simeq \CAlg\,.
\end{align*}
If $\Cc^\otimes\simeq\Cc^\times$ is the cartesian monoidal structure from \cref{prop:CartesianMonoidalStructure}, then $\Cc_\times^\op\simeq (\Cc^\op)^\sqcup$ is the cocartesian monoidal structure, and $\CAlg(\Cc_\times^\op)\simeq\CAlg((\Cc^\op)^\sqcup)\simeq \Cc^\op$ by the second half of \cref{thm:OMon}. In particular, every anima $X\in \An$ is a coalgebra in the cartesian monoidal structure. Hence $\IS[X]\in \CAlg(\Sp_\otimes^\op)$, and then we see that $\hom_\Sp(\IS[X],E)$ is canonically an $\IE_\infty$-ring spectrum for every $\IE_\infty$-ring spectrum $E$. Now \cref{cor*:SpectraCohomologyTheory} implies
\begin{equation*}
	\pi_{*}\hom_\Sp\big(\IS[X],HR\big)\simeq H^{-*}(X,R)\,,
\end{equation*}
whence \cref{par:pi*GradedRing} endows the cohomology of $X$ with a graded commutative ring structure. This is (our definition of) the \emph{cup product}.

\numpar{$\IE_1$-Ring Spectra*}\label{par:E1RingSpectra}
Elements of $\Alg_\IAssoc(\Sp^\otimes)$ are called \emph{$\IE_1$-ring spectra}. Many of our results about $\IE_\infty$-ring spectra can be generalised to $\IE_1$-ring spectra.  But let me only mention those that we'll need in \cref{chap:GroupCompletion}.
\begin{alphanumerate}
	\item Since $\IS[-]\colon \An^\times\morphism\Sp^\otimes$ is symmetric monoidal by \cref{prop:Sp(C)Monoidal}, it induces a functor
	\begin{equation*}
		\IS[-]\colon \Mon(\An)\simeq \Alg_\IAssoc(\An^\times)\morphism \Alg_\IAssoc(\Sp^\otimes)\,.
	\end{equation*}
	In particular, if $M\in\Mon(\An)$ is an $\IE_1$-monoid, then its spherical group ring $\IS[M]$ is an $\IE_1$-ring spectrum.
	\item Tensor products of $\IE_1$-ring spectra inherit a $\IE_1$-ring structure by \cref{thm:AlgOSymmetricMonoidal}.
	\item If $A$ is an $\IE_1$-ring spectrum and $M$ a left module over it, then $\pi_*(A)$ is a graded (but of course not necessarily commutative) ring and $\pi_*(M)$ a graded left $\pi_*(A)$-module.
\end{alphanumerate}
