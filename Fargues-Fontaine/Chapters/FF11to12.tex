\chapter{Yet to be named}
\section{Ramified Witt Vectors}
\lecture[The abstract of this lecture is left as an exercise.]{2019-10-23}
Let $p$ be a prime, $E/\IQ_p$ a finite extension with ring of integers $\Oo_E$. We fix a choice of uniformizer $\pi$ and let $\IF_q=\Oo_E/\pi\Oo_E$ be the residue field of $\Oo_E$, where $q=p^f$. The goal for today is to prove
\begin{prop}\label{prop:FqAlgebrasEquivalence}
	There is an equivalence of categories
	\begin{align*}
		\left\{\begin{tabular}{c}
			$\pi$-torsionfree $\pi$-adically complete $\Oo_E$-alge-\\
			bras $A$ with perfect residue ring $A/\pi A$
		\end{tabular}
		\right\}&\isomorphism\left\{\text{perfect $\IF_q$-algebras}\right\}\\
		A&\longmapsto R=A/\pi A\,.
	\end{align*}
\end{prop}
For the proof, we will construct an inverse functor $R\mapsto W_{\Oo_E}(R)$ that somehow \enquote{reconstructs} $A$ from $A/\pi A$.
\begin{rem}
	The most important case is the unramified one, i.e., $E=\IQ_p$, in which case we obtain an equivalence
	\begin{align*}
	\left\{\begin{tabular}{c}
	$p$-torsionfree $p$-adically complete rings\\
	$A$ with perfect residue ring $A/pA$
	\end{tabular}
	\right\}&\isomorphism\left\{\text{perfect $\IF_p$-algebras}\right\}\\
	A&\longmapsto R=A/p A\,.
	\end{align*}
	We will see (in \cref{cor:unramifiedWitt}) that the general case can be reduced to this one. Also we put $W\coloneqq W_{\IZ_p}$ for brevity.
\end{rem}
\numpar*{Example}
We will see $W(\IF_p)=\IZ_p$ and $W(\IF_q)=\Oo_{E_0}$ where $E_0$ is the maximal unramified subextension of $E/\IQ_p$ (i.e., the unique unramified extension with residue field $\IF_q$). Moreover, we will see
	\begin{equation*}
		W\big(\IF_p\big\llbracket T^{1/p^\infty}\big\rrbracket\big)=\IZ_p\big\llbracket T^{1/p^\infty}\big\rrbracket\,.
	\end{equation*}
	
\subsection{The construction of \texorpdfstring{$W_{\Oo_E}$}{W}}
\begin{lem}\label{lem:LTE}
	Let $A$ be any $\Oo_E$-algebra and $x,y\in A$ such that $x\equiv y\mod \pi$. Then
	\begin{equation*}
		x^{q^k}\equiv y^{q^k}\mod \pi^{k+1}\quad\text{for all }k\geq 0\,.
	\end{equation*}
\end{lem}
\begin{proof}
	By induction on $k$, this boils down to the following question: if $x\equiv y\mod \pi^k$, show $x^q\equiv y^q\mod \pi^{k+1}$. To see this, write $x=y+\pi^ka$ for some $a\in A$. As all binomial coefficients $\binom{q}{i}$ except for $i=0,q$ are divisible by $p$, we obtain 
	\begin{equation*}
		x^q=(y+\pi^ka)^q=y^q+p\pi^k(\ldots)+\pi^{kq}a^q\,.
	\end{equation*}
	As $\pi\mid p$, the assertions follows.
\end{proof}
\begin{deflem}\label{deflem:Teichmüller}
	Let $A$ be a $p$-adically complete $\Oo_E$-algebra with $R=A/\pi A$ perfect. Let $a\in R$. Choose any sequence of lifts $\alpha_n\in A$ of $a^{1/q^n}\in R$. Then the sequence $(\alpha_n^{q^n})_{n\in \IN}$ converges in $A$ to a lift of $a$, which is independent of the choices of $\alpha_n$. The map
	\begin{align*}
		[-]\colon R&\morphism A\\
		a&\longmapsto [a]\coloneqq\lim_{n\to\infty}\alpha_n^{q^n}
	\end{align*}
	is well-defined and called the \defemph{Teichmüller representative}. It defines a natural multiplicative section of $A\epimorphism R$.
\end{deflem}
\begin{proof}
	We have $\alpha_{n+1}^q\equiv \alpha_n\mod \pi$, hence
	\begin{equation*}
		\alpha_{n+1}^{q^{n+1}}\equiv \alpha_n^{q^n}\mod \pi^{n+1}
	\end{equation*}
	by \cref{lem:LTE}. This shows convergence of the sequence in question. To show that it doesn't depend on the choice of lifts can be seen by a similar argument. Now if $a,b\in R$ are given together with a choice of lifts $\alpha_n$ and $\beta_n$, we can choose $\alpha_n\beta_n$ as lifts of $(ab)^1/q^n$, since the choice of lifts doesn't matter. From this argument, multiplicativity is clear. Naturality is similar.
\end{proof}
\begin{lem}\label{lem:TeichmüllerRep}
	In our usual situation, every $x\in A$ admits a unique representation
	\begin{equation*}
		x=\sum_{n=0}^{\infty}[x_n]\pi^n\quad\text{for some }x_n\in R\,.
	\end{equation*}
\end{lem}
\begin{proof}
	Let $x_0\in R$ be the reduction of $x$. Then $x\equiv[x_0]\mod \pi$, so $x-[x_0]=\pi y_1$ for some $y_1\in A$, which is unique as $A$ is $\pi$-torsionfree. Now let $x_1\in R$ be the reduction of $y_1$. Similar as above, write $y_1=[x_1]+\pi y_2$. Now repeat this process to get a representation of the desired type. Uniqueness can be shown along the lines of the construction.
\end{proof}
\begin{urem}
	We can think of $A\morphism R$ in a similar way as we think about $R\llbracket T\rrbracket\morphism R$ with its canonical section $R\morphism R\llbracket T\rrbracket$ given by $a\mapsto a$. Since in our situation $A$ has characteristic $0$ but $R$ has characteristic $p$, there is no way $[-]\colon R\morphism A$ can be additive. So it being multiplicative is really the best we could hope for.
\end{urem}
At this point, \cref{lem:TeichmüllerRep} allows us to recover $A$ as a \emph{set} from $R=A/\pi A$. But what about the ring structure? Let's try! Say we have sequences $(x_n),(y_n)\in R^{\IN}$ and we want to find the unique sequence $(s_n)\in R^{\IN}$ such that
\begin{equation*}
	\sum_{n=0}^{\infty}[x_n]\pi^n+\sum_{n=0}^{\infty}[y_n]\pi^n=\sum_{n=0}^\infty [s_n]\pi^n\,.
\end{equation*}
One could naively assume that $s_n$ is just $x_n+y_n$. Spoiler: \emph{it's not}. For $n=0$, we calculate modulo $\pi$. We should have $[x_0]+[y_0]=[s_0]$, hence $s_0=x_0+y_0$. That was easy! Now for $n=1$. We calculate modulo $\pi^2$:
\begin{equation*}
	[x_0]+[x_1]\pi+[y_0]+[y_1]\pi\equiv [s_0]+[s_1]\pi\equiv [x_0+y_0]+[s_1]\pi\mod \pi^2\,.
\end{equation*}
Hence we want to put
\begin{equation*}
	``s_1=x_1+y_1+\frac{[x_0]+[y_0]-[x_0+y_0]}{\pi}\text{''}\,,
\end{equation*}
except it's not clear at all how to define this formally. Here we use a trick: since $R$ is perfect and $[-]$ is multiplicative, we have
\begin{align*}
	[x_0]+[y_0]-[x_0+y_0]=\big[x_0^{1/q}\big]^q+\big[y_0^{1/q}\big]^q-\big[x_0^{1/q}+y_0^{1/q}\big]^q\,.
\end{align*}
Since $\big[x_0^{1/q}\big]+\big[y_0^{1/q}\big]\equiv\big[x_0^{1/q}+y_0^{1/q}\big]\mod \pi$, \cref{lem:LTE} shows
\begin{equation*}
	\big[x_0^{1/q}\big]^q+\big[y_0^{1/q}\big]^q\equiv\big[x_0^{1/q}+y_0^{1/q}\big]^q\mod \pi^2\,.
\end{equation*}
Hence we can choose
\begin{equation*}
	s_1=x_1+y_1-\sum_{i=1}^{q-1}\frac{1}{\pi}\binom{q}{i}\big[x_0^{1/q}\big]^i\big[y_0^{1/q}\big]^{q-i}\,,
\end{equation*}
where the $\pi^{-1}\binom{q}{i}$ are considered as elements of $\Oo_E$. In the very unpleasant Germany of 1936, the mathematician and SA member Ernst Witt understood this pattern and extended it to higher $n$ as follows.
\begin{defi}
	For $n\geq 0$, define the \emph{$n\ordinalth$ ghost component} as
	\begin{equation*}
		W_n(X_0,\dotsc,X_n)=\sum_{i=0}^{n}X_i^{q^{n-i}}\pi^i\in\Oo_E[X_0,\dotsc,X_n]\,.
	\end{equation*}
\end{defi}
\begin{urem}
The idea behind the $W_n$ is that 
\begin{equation*}
	\sum_{i=0}^n[a_i]\pi^i=W_n\Big(\big[a_0^{1/q^n}\big],\dotsc,\big[a_n^{1/q^0}\big]\Big)\,.
\end{equation*}
\end{urem}
\begin{prop}\label{prop:WittPolynomials}
	There are unique sequences of polynomials $(S_n)_{n\in \IN}$, $(P_n)_{n\in\IN}$ in the polynomial ring $\Oo_E[X_0,\ldots,X_n,Y_0,\ldots,Y_n]$, such that
	\begin{align*}
		W_n(X_0,\dotsc,X_n)+W_n(Y_0,\dotsc,Y_n)&=W_n(S_0,\dotsc,S_n)\\
		W_n(X_0,\dotsc,X_n)\cdot W_n(Y_0,\dotsc,Y_n)&=W_n(P_0,\dotsc,P_n)\,.
	\end{align*}
\end{prop}
\begin{proof}%nitl
	We show more generally that for any polynomial $\Phi\in\Oo_E[X,Y]$ there is a unique sequence $(\Phi)_{n\in\IN}$ of polynomials $\Phi_n\in\Oo_E[X_0,\dotsc,X_n,Y_0,\dotsc,Y_n]$ such that
	\begin{equation*}
		\Phi\big(W_n(X_0,\dotsc,X_n),W_n(Y_0,\dotsc,Y_n)\big)=W_n(\Phi_0,\dotsc,\Phi_n)\,.
	\end{equation*}
	We show this via induction on $n$. For $n=0$ we have to take $\Phi_0(X_0,Y_0)=\Phi(X_0,Y_0)$. Now suppose $\Phi_0,\dotsc,\Phi_n$ are already constructed. We need to check that
	\begin{equation}\label{eq:Witt1}
		\Phi\big(W_{n+1}(X_0,\dotsc,X_{n+1}),W_{n+1}(Y_0,\dotsc,Y_{n+1})\big)-W_{n+1}(\Phi_0,\dotsc,\Phi_n,0)
	\end{equation}
	is a polynomial divisible by $\pi^{n+1}$; for then $\pi^{-(n+1)}\cdot(\text{this polynomial})$ is the unique choice for $\Phi_{n+1}$. Note that 
	\begin{equation}\label{eq:Witt2}
		W_{n+1}(X_0,\dotsc,X_{n+1})\equiv W_n\left(X_0^q,\dotsc,X_n^q\right)\mod\pi^{n+1}\,.
	\end{equation}
	Using \cref{eq:Witt1} together with the induction hypothesis, we obtain
	\begin{align*}
		\Phi\big(W_{n+1}(X_0,\dotsc,X_{n+1}),W_{n+1}(Y_0,\dotsc,Y_{n+1})\big)&\equiv \Phi\big(W_n(X_0^q,\dotsc,X_n^q),W_n(Y_0^q,\dotsc,Y_n^q)\big)\\
		&\equiv W_n\big(\Phi_0^{(q)},\dotsc,\Phi_n^{(q)}\big)\mod \pi^{n+1}\,,
	\end{align*}
	where $\Phi_i^{(q)}$ is the polynomial obtained from $\Phi_i$ by replacing every variable by its $q\ordinalth$ power. Note that $\Phi_i^{(q)}\equiv \Phi_i^q\mod \pi$. Thus, using \cref{lem:LTE} we get
	\begin{equation*}
		\pi^i\big(\Phi_i^{(q)}\big)^{q^{n-i}}\equiv \pi^i\Phi_i^{q^{n+1-i}}\mod \pi^{n+1}\,.
	\end{equation*}
	But this shows $W_n\big(\Phi_0^{(q)},\dotsc,\Phi_n^{(q)}\big)\equiv W_{n+1}(\Phi_0,\dotsc,\Phi_n,0)\mod \pi^{n+1}$. Now putting everything together shows that the polynomial in \cref{eq:Witt1} is indeed divisible by $\pi^{n+1}$, as required.
\end{proof}
\begin{cor}\label{cor:snpn}
	Let $(x_n)_{n\in\IN}$ and $(y_n)_{n\in\IN}$ be sequences in $R^\IN$, where $R=A/\pi A$. For all $n\geq 0$ put
	\begin{align*}
		s_n&=S_n\left(x_0^{1/q^n},\dotsc,x_n^{1/q^0},y_0^{1/q^n},\dotsc,y_n^{1/q^0}\right)\\
		p_n&=P_n\left(x_0^{1/q^n},\dotsc,x_n^{1/q^0},y_0^{1/q^n},\dotsc,y_n^{1/q^0}\right)\,. 
	\end{align*}
	Then these sequences $(s_n)_{n\in\IN}$ and $(p_n)_{n\in\IN}$ satisfy
	\begin{align*}
		\sum_{n=0}^\infty[x_n]\pi^n+\sum_{n=0}^\infty[y_n]\pi^n&=\sum_{n=0}^\infty[s_n]\pi^n\\
		\Bigg(\sum_{n=0}^\infty[x_n]\pi^n\Bigg)\cdot\Bigg(\sum_{n=0}^\infty[y_n]\pi^n\Bigg)&=\sum_{n=0}^\infty[p_n]\pi^n\,.
	\end{align*}
\end{cor}
\begin{proof*}
	Again, we show the assertion more generally for an arbitrary $\Phi\in\Oo_E[X,Y]$ and its associated Witt polynomials $(\Phi_n)_{n\in\IN}$ constructed in the proof of \cref{prop:WittPolynomials}. The key observation is the following:
	\begin{alphanumerate}
		\item[$(*)$] If $a_0,\dotsc,a_n$ and $a_0',\dotsc,a'_n$ are elements of $A$ such that $a_i\equiv a_i'\mod \pi$, then
		\begin{align*}
			W_n(a_0,\dotsc,a_n)\equiv W_n(a'_0,\dotsc,a'_n)\mod \pi^{n+1}\,.
		\end{align*}
	\end{alphanumerate}
	Indeed, if you think about it, this immediately follows from \cref{lem:LTE} and the definition of the $W_n$. Now fix some $N$ and put
	\begin{align*}
		\varphi_n&=\Phi_n\left(x_0^{1/q^n},\dotsc,x_n^{1/q^0},y_0^{1/q^n},\dotsc,y_n^{1/q^0}\right)\\
		\varphi_n'&=\Phi_n\left(\big[x_0^{1/q^N}\big],\dotsc,\big[x_n^{1/q^{N-n}}\big],\big[y_0^{1/q^N}\big],\dotsc,\big[y_n^{1/q^{N-n}}\big]\right)\,.
	\end{align*}
	By construction of the Witt polynomials $(\Phi_n)_{n\in\IN}$ (see the proof of \cref{prop:WittPolynomials}) we immediately have 
	\begin{equation*}
		\Phi\left(W_N\left(\big[x_0^{1/q^N}\big],\dotsc,\big[x_N^{1/q^0}\big]\right),W_N\left(\big[y_0^{1/q^N}\big],\dotsc,\big[y_N^{1/q^0}\big]\right)\right)= W_N(\varphi_0',\dotsc,\varphi_N')\,.
	\end{equation*}
	But also $\varphi_n'\equiv\big[\varphi_n^{1/q^{N-n}}\big]\mod \pi$. Hence, by \itememph{*}, we obtain
	\begin{align*}
		W_N(\varphi_0',\dotsc,\varphi_N')\equiv W_N\left(\big[\varphi_0^{1/q^N}\big],\dotsc,\big[\varphi_n^{1/q^{0}}\big]\right)\mod \pi^{N+1}\,.
	\end{align*}
	Taking $N\rightarrow\infty$, this shows
	\begin{equation*}
		\Phi\Bigg(\sum_{n=0}^\infty[x_n]\pi^n,\sum_{n=0}^\infty[y_n]\pi^n\Bigg)=\sum_{n=0}^\infty[\varphi_n]\pi^n\,.
	\end{equation*}
	For $\Phi=X+Y$ resp.\ $\Phi=XY$ we retain the assertion of this corollary.
\end{proof*}
The upshot is that we can now reconstruct $A$ as a ring from $R=A/\pi A$. The next goal is to start with an arbitrary $R$ and construct an $A$ in a functorial way. In particular, we will allow $R$ to be an $\Oo_E$-algebra instead of an $\IF_q$-algebra (recall that $\IF_q=\Oo_E/\pi\Oo_E$). In the end, we will only be interested in the latter case, but allowing for rings of characteristic $0$ too gives us some nice uniqueness properties.
\begin{defi}\label{def:W_OE}
	For any $\Oo_E$-algebra $R$ write $W_{\Oo_E}(R)=R^\IN$. Its elements (which are sequences) are denoted $x=[x_0,x_1,\dotsc]$.
\end{defi}
\begin{prop}\label{prop:W_OE}
	The functor from \cref{def:W_OE} admits a unique factorization
	\begin{equation*}
		\begin{tikzcd}
		\cat{Alg}_{\Oo_E}\drar[dashed, "W_{\Oo_E}(-)"{swap}]\ar[rr, "(-)^\IN"] & & \cat{Set}\\
		& \cat{Alg}_{\Oo_E} \urar["\mathrm{forget}"{swap}]&
		\end{tikzcd}
	\end{equation*}
	such that the natural transformation $\Ww$ given by
	\begin{align*}
		\Ww_R\colon W_{\Oo_E}(R)&\morphism R^\IN\\
		[x_n]_{n\in\IN}&\longmapsto \big(W_n(x_0,\dotsc,x_n)\big)_{n\in\IN}
	\end{align*}
	is a morphism of $\Oo_E$-algebras. Here $R^\IN$ is equipped with its natural component-wise $\Oo_E$-algebra structure.
\end{prop}
\begin{proof}
	We first construct a natural $\Oo_E$-algebra structure on $W_{\Oo_E}(R)$. If two sequences $x=[x_n]_{n\in\IN}$ and $[y_n]_{n\in\IN}$ are given, we define $x+y=[s_n]_{n\in\IN}$ and $xy=[p_n]_{n\in\IN}$, where---you might have guessed it---we put
	\begin{equation*}
		s_n=S_n(x_0,\dotsc,x_n,y_0,\dotsc,y_n)\quad\text{and}\quad p_n=P_n(x_0,\dotsc,x_n,y_0,\dotsc,y_n)\,.
	\end{equation*}
	To see that this is determines a ring structure, the crucial thing to notice is that the proof of \cref{prop:WittPolynomials} works just the same if $\Phi\in\Oo_E[X_1,\ldots,X_N]$ is a polynomial in arbitrary many variables instead of just $N=2$. So by choosing suitable $\Phi$, we can verify all ring axioms. For example, $\Phi=-X_1$ constructs additive inverses, $\Phi=(X_1+X_2)+X_3=X_1+(X_2+X_3)$ shows the associativity law of addition, $\Phi=X_1(X_2+X_3)=X_1X_2+X_1X_3$ shows distributivity, and so on. Also, if $\alpha\in\Oo_E$, then $\Phi=\alpha X_1$ defines multiplication by $\alpha$ on $W_{\Oo_E}(R)$, turning it into an $\Oo_E$-algebra.
	
	This provides a factorization through $\cat{Alg}_{\Oo_E}$. It is clear from the construction that $\Ww_R$ is an $\Oo_E$-algebra morphism. So it remains to show that this factorization is unique. If $R$ is $\pi$-torsionfree, then $\Ww_R\colon W_{\Oo_E}(R)\morphism R^\IN$ is easily seen to be injective, hence the $\Oo_E$-algebra structure on $W_{\Oo_E}(R)$ is uniquely determined by the one on $R^\IN$. In general, every $R$ admits a surjection $R'\epimorphism R$ from a $\pi$-torsionfree $\Oo_E$-algebra; e.g., $R'=\Oo_E\left[T_a\st a\in R\right]$ does it. Then $W_{\Oo_E}(R')\epimorphism W_{\Oo_E}(R)$ uniquely determines the $\Oo_E$-algebra structure on $W_{\Oo_E}(R)$. This shows uniqueness.
\end{proof}
\begin{urem}
	\begin{numerate}
		\item For the uniqueness part it was crucial to have \enquote{enough} $\pi$-torsionfree $\Oo_E$-algebras. If we had worked with $\IF_q$-algebras, where $\pi=0$, this wouldn't have been possible. In this case, $W_n(x_0,\dotsc,x_n)$ is just $x_0^{q^n}$. Hence the name \enquote{ghost components}.
		
		\item Also, \cref{prop:W_OE} gives the functor $W_{\Oo_E}(-)$ the structure of a ring scheme.
	\end{numerate}
\end{urem}	
\begin{lem}\label{lem:W_OETeichmüller}
	The natural map (which we will also call \enquote{Teichmüller lift})
	\begin{align*}
		[-]\colon R&\morphism W_{\Oo_E}(R)\\
		x&\longmapsto [x,0,0,\dotsc]
	\end{align*}
	is multiplicative.
\end{lem}
\begin{proof*}
	It's easy to see $P_0(X_0,Y_0)=X_0Y_0$. So to prove the assertion it suffices to check that $P_n(X_0,0,\dotsc,0,Y_0,0,\dotsc,0)=0$ for all $n>0$. But
	\begin{align*}
		W_n(X_0,0,\dotsc,0)\cdot W_n(Y_0,0,\dotsc,0)=X_0^{q^n}Y_0^{q^n}=W_n(X_0Y_0,0,\dotsc,0)\,,
	\end{align*}
	so this is easy to check by induction on $n$ (and using that polynomial rings over $\Oo_E$ are $\pi$-torsionfree).
\end{proof*}
\subsection{Frobenius and Verschiebung}
If $R$ happens to be an $\IF_q$-algebra, then we have the Frobenius $(-)^q$ on $R$. By functoriality, it extends to an endomorphism $F\colon W_{\Oo_E}(R)\morphism W_{\Oo_E}(R)$. The next lemma shows that $F$ actually exists for arbitrary $R$ and can be explicitly described.
\begin{lem}\label{lem:WittFrob}
	\begin{numerate}
		\item There is a unique natural transformation $F\colon W_{\Oo_E}(-)\morphism W_{\Oo_E}(-)$ of $\Oo_E$-algebras making the following diagram commute:
		\begin{equation*}
			\begin{tikzcd}
				W_{\Oo_E}(R)\rar["\Ww"]\dar["F"{swap}] & R^\IN\dar &[-2.4em] (x_n)_{n\in\IN}\dar[|->]\\
				W_{\Oo_E}(R)\rar["\Ww"] & R^\IN &[-2.4em] (x_{n+1})_{n\in \IN}
			\end{tikzcd}
		\end{equation*}
		\item If $R$ is an $\IF_q$-algebra, then $F$ is given by $F([x_0,x_1,\dotsc])=[x_0^q,x_1^q,\dotsc]$ and it is induced by the Frobenius on $R$.
	\end{numerate}
\end{lem}
\begin{proof*}
	We first construct a sequence $(F_n)_{n\in\IN}$ of polynomials $F_n\in \Oo_E[X_0,\dotsc,X_{n+1}]$ satisfying $W_{n+1}(X_0,\dotsc,X_{n+1})=W_n(F_0,\dotsc,F_n)$ and that $F_n\equiv X_n^q\mod \pi$. This is done by induction on $n$, the case $n=0$ being trivial. Suppose $F_0,\dotsc,F_{n-1}$ have already been constructed and have the required property. If we could prove that
	\begin{equation}\label{eq:Fn}
		W_{n+1}(X_0,\dotsc,X_{n+1})-W_n(F_0,\dotsc,F_{n-1},0)\equiv \pi^n X_0^q\mod \pi^{n+1}\,,
	\end{equation}
	this would show existence of $F_n$ and $F_n\equiv X_n^q\mod \pi$ at once. To prove \cref{eq:Fn}, we may equivalently show
	\begin{align}\label{eq:Fn2}
		\begin{split}
			0&\equiv W_{n+1}(X_0,\dotsc,X_{n-1},0,0)-W_n(F_0,\dotsc,F_{n-1},0)\\
			&\equiv W_{n-1}\big(X_0^{q^2},\dotsc,X_{n-1}^{q^2}\big)-W_{n-1}\left(F_0^q,\dotsc,F_{n-1}^q\right)\mod \pi^{n+1}\,.
		\end{split}
	\end{align}
	But $F_i\equiv X_i^q\mod \pi$ shows $F_i^q\equiv X_i^{q^2}\mod \pi^2$ by \cref{lem:LTE}, hence the bottom line of \cref{eq:Fn2} is indeed $0$ modulo $\pi^{n+1}$ by another application of \cref{lem:LTE}.
	
	Thus we can construct a sequence $F=(F_n)_{n\in \IN}$ with the required properties. By construction, $F$ makes the diagram in \itememph{1} commute and satisfies \itememph{2}. So it remains to show that $F$ is unique with this property and a morphism of $\Oo_E$-algebras. This can be done by the same argument as in the proof of \cref{prop:W_OE}. If $R$ is $\pi$-torsionfree, $W_{\Oo_E}(R)$ injects into $R^\IN$, hence it is uniquely determined and an $\Oo_E$-algebra morphism. In general, we take a surjection $R'\epimorphism R$ from a $\pi$-torsionfree $\Oo_E$-algebra.
\end{proof*}
\begin{lem}
	There is a natural transformation $V\colon W_{\Oo_E}(-)\morphism W_{\Oo_E}(-)$ of $\Oo_E$-modules that makes the following diagram commute:
	\begin{equation*}
		\begin{tikzcd}
			{[x_0,x_1,\dotsc]}\dar[|->] &[-2.4em] W_{\Oo_E}(R)\dar["V"{swap}]\rar["\Ww"] & R^\IN\dar &[-2.4em] (x_n)_{n\in\IN}\dar[|->]\\
			{[0,x_0,x_1,\dotsc]} &[-2.4em] W_{\Oo_E}(R) \rar["\Ww"] & R^\IN &[-2.4em] (\pi x_{n-1})_{n\in\IN}
		\end{tikzcd}\,,
	\end{equation*}
	where we put $x_{-1}=0$. Moreover, $V$ is unique with this property.
\end{lem}
\begin{proof*}
	It's immediately clear that $V$ as constructed makes the diagram commute. To show that $V$ is unique, we use the usual trick: for $\pi$-torsionfree $\Oo_E$-algebras $R$, this is clear; in general, consider a surjection $R'\epimorphism R$ where $R'$ is $\pi$-torsionfree.
\end{proof*}
\numpar*{Remark}
The letter $V$ stands for the German word \enquote{Verschiebung}. In contrast to $F$, $V$ is no ring endomorphism and it does depend on the choice of $\pi$.\footnote{Well, $W_n$ and thus $\Ww$ depend on $\pi$ too, so we cannot really say that $F$ is \enquote{independent} of $\pi$. But at least its image in $R^\IN$ is, in contrast to the image of $V$ in $R^\IN$.}
\begin{lem}\label{lem:FVidentities}
	The following identities hold for $F$ and the Verschiebung $V$.
	\begin{numerate}
		\item $FV=\pi$.
		\item $V(F(x)y)=xV(y)$ for all $x,y\in W_{\Oo_E}(R)$.
		\item $\pi F(x)y=F(xV(y))$ for all $x,y\in W_{\Oo_E}(R)$. 
	\end{numerate}
\end{lem}
\begin{proof}
	If $R$ is $\pi$-torsionfree, these can be checked in $R^\IN$. In general, take a surjection $R'\epimorphism R$ where $R'$ is $\pi$-torsionfree to reduce everything to the $\pi$-torsionfree case.
\end{proof}
\begin{lem}\label{lem:imVn}
	\begin{numerate}
		\item For all $n\in\IN$, the image of $V^n$ is an ideal in $W_{\Oo_E}(R)$.
		\item We have $W_{\Oo_E}(R)\cong \lim_{n\in\IN}W_{\Oo_E}(R)/\im V^n$.
		\item Every $x\in W_{\Oo_E}(R)$ admits a unique representation
		\begin{equation*}
			x=\sum_{n=0}^\infty V^n[x_n]
		\end{equation*}
		for some $x_n\in R$, where $[-]\colon R\morphism W_{\Oo_E}(R)$ is the Teichmüller lift from \cref{lem:W_OETeichmüller}. In fact, the $x_n$ are determined by $x=[x_n]_{n\in\IN}$.
	\end{numerate}
\end{lem}
\begin{proof*}
	Since $V$ is $\Oo_E$-linear, $\im V^n$ is a subgroup of $W_{\Oo_E}(R)$. Moreover, \cref{lem:FVidentities}\itememph{1} shows $xV^n(y)=V^n(F^n(x)y)$ for all $x,y\in W_{\Oo_E}(R)$, hence $\im V^n$ is closed under scalar multiplication. This shows \itememph{1}.
	
	Now part~\itememph{2}. We claim that the canonical map of sets $W_{\Oo_E}(R)\morphism R^N$ given by $[x_n]_{n\in\IN}\mapsto (x_0,\dotsc,x_{N-1})$ descends to a bijection
	\begin{equation*}
		W_{\Oo_E}(R)/\im V^N\isomorphism R^N\,.
	\end{equation*}
	Let's first check that it is well-defined. Let $y=[y_n]_{n\in \IN}$ be in the image of $V^n$, i.e., $y_n=0$ for all $n< N$. Let $x+y=[s_n]_{n\in \IN}$. Then what we need to show is that $s_n=x_n$ for all $n<N$. Thus, it suffices to check the polynomial identity
	\begin{equation*}
		S_n(X_0,\dotsc,X_n,0,\dotsc,0)=X_n\,.
	\end{equation*}
	However, this is easily seen from induction and the trivial identity
	\begin{equation*}
		W_n(X_0,\dotsc,X_n)+W_n(0,\dotsc,0)=W_n(X_0,\dotsc,X_n)\,.
	\end{equation*}
	Since $W_{\Oo_E}(R)/\im V^N\morphism R^N$ is automatically surjective, it remains to show injectivity. So let $x,y\in W_{\Oo_E}(R)$ be such that $x_n=y_n$ for all $n<N$. Let $x-y=[\delta_n]_{n\in\IN}$. To show that $\delta$ is in the image of $V^n$, we need to check $\delta_n=0$ for $n<N$. Thus, it suffices to check the polynomial identity
	\begin{equation*}
		\Delta_n(X_0,\dotsc,X_n,X_0,\dotsc,X_n)=0\,,
	\end{equation*}
	where $\Delta=X-Y\in\Oo_E[X,Y]$ and $(\Delta_n)_{n\in\IN}$ are the associated Witt polynomials constructed in the proof of \cref{prop:WittPolynomials}. This can be done in the same way as above.
	
	Now since $R^\IN\cong \lim_{n\in\IN}R^n$, the bijection $W_{\Oo_E}(R)/\im V^n\cong R^n$ for all $n\in \IN$ shows that $W_{\Oo_E}(R)\cong \lim_{n\in\IN}W_{\Oo_E}(R)/\im V^n$ is true as a limit of sets. However, the limit in the category of $\Oo_E$-algebras can be taken on the level of sets. This shows \itememph{2}.
	
	Finally, we show \itememph{3}. First we prove that for all $N\in \IN$ we have
	\begin{equation}\label{eq:sumVn}
		\sum_{n=0}^NV^n[x_n]=[x_0,\dotsc,x_N,0,0,\dotsc]\,.
	\end{equation}
	We use induction on $N$. The case $N=0$ is trivial. Now suppose the assertion is true for $N-1$. To prove it for $N$, it suffices to check the following polynomial identity: if $(X_n)_{n\in\IN}$ and $(Y_n)_{n\in\IN}$ are sequences of variables such that $X_N=0$ and $Y_n=0$ for all $n\neq N$, then
	\begin{equation*}
		S_n(X_0,\dotsc,X_n,Y_0,\dotsc,Y_n)=\begin{cases}
		Y_N&\text{if }n= N\\
		X_n&\text{else}
		\end{cases}\,.
	\end{equation*}
	For $n<N$, we obtain an identity that was already seen in the proof of \itememph{2}. For $n\geq N$, this easily follows by induction on $n$, using the identity
	\begin{equation*}
		W_n(X_0,\dotsc,X_n)+W_n(Y_0,\dotsc,Y_n)=W_n(X_0,\dotsc,X_{N-1},Y_N,X_{N+1},\dotsc,X_n)\,.
	\end{equation*}
	This shows \cref{eq:sumVn}. Now let $x-[x_0,\dotsc,x_N,0,0,\dotsc]=\delta=[\delta_n]_{n\in\IN}$. As in the proof of \itememph{2} we see that $\delta_n=0$ for $n\leq N$. Hence $\delta\in\im V^n$. This shows \itememph{3} except for the uniqueness part. But uniqueness is also clear from \cref{eq:sumVn}.
\end{proof*}
\begin{urem*}
	\cref{lem:imVn} holds for arbitrary $R$, despite what was claimed in the lecture. We leave it as an exercise to relate this error to the lecture's overall rushed style.
\end{urem*}
Now that the general theory of $W_{\Oo_E}(-)$ is set up, we restrict ourselves to the case where $R$ has characteristic $p$, i.e., $\pi=0$ on $R$ and $R$ is an $\IF_q$-algebra.
\begin{lem}\label{lem:Vincharp}
	Suppose $\pi=0$ on $R$. Then the following hold:
	\begin{numerate}
		\item For $x=[x_n]_{n\in\IN}\in R$ we have $x=\sum_{n=0}^\infty V^n[x_n]$.
		\item $VF=\pi$. Hence $V$ and $F$ commute.
		\item $F\big(\sum_{n=0}^\infty V^n[x_n]\big)=\sum_{n=0}^\infty V^n[x_n^q]$.
	\end{numerate}
\end{lem}
\begin{proof*}
	Part~\itememph{1} was already seen in \cref{lem:imVn}\itememph{3}. Now \itememph{3} is an immediate consequence of \itememph{1} and \cref{lem:WittFrob}\itememph{2}. For \itememph{2}, note that $VF$ sends $[x_0,x_1,\dotsc]$ to $[0,x_0^q,x_1^q,\dotsc]$. Thus, it suffices to show that the Witt polynomials $(\Pi_n)_{n\in\IN}$ associated to $\Pi=\pi X\in\Oo_E[X]$ satisfy
	\begin{equation*}
		\Pi_n(X_0,\dotsc,X_n)\equiv X_{n-1}^q\mod \pi\quad\text{for }n\geq 1
	\end{equation*}
	and $\Pi_0\equiv 0\mod \pi$. We show this by induction on $n$, the case $n=0$ being trivial. Now suppose the assertions holds up to $n$. Then $\Pi_i\equiv X_{i-1}^q\mod \pi$ for all $i\leq n$ shows, by \cref{lem:LTE}, that
	\begin{equation*}
		W_{n+1}(\Pi_1,\dotsc,\Pi_{n+1})\equiv \pi X_0^{q^{n+1}}+\dotsb+\pi^{n}X_{n-1}^{q^2}+\pi^{n+1}\Pi_{n+1}\mod \pi^{n+2}\,.
	\end{equation*}
	However, the left-hand side can, by definition, be computed as
	\begin{align*}
		W_{n+1}(\Pi_1,\dotsc,\Pi_{n+1})&\equiv\pi W_{n+1}(X_0,\dotsc,X_{n+1})\\
		&\equiv \pi X_0^{q^{n+1}}+\dotsb+\pi^nX_{n-1}^{q^2}+\pi^{n+1}X_n^q\mod \pi^{n+2}\,.
	\end{align*}
	This shows indeed $\Pi_{n+1}\equiv X_n^q\mod \pi$, as claimed.
\end{proof*}
\begin{lem}\label{lem:W_OEpi}
	If $R$ is a perfect $\IF_q$-algebra, then $W_{\Oo_E}(R)$ is $\pi$-adically complete, and if $x=[x_n]_{n\in\IN}$, then
	\begin{equation*}
		x=\sum_{n=0}^\infty\big[x_n^{1/q^n}\big]\pi^n\,.
	\end{equation*}
\end{lem}
\begin{proof*}
	Since $R$ is perfect, the Frobenius is an automorphism, hence the same is true for $F$ on $W_{\Oo_E}(R)$. Thus \cref{lem:Vincharp}\itememph{2} shows that the image of $V^n$ is the image of $\pi^n$. Thus \cref{lem:imVn}\itememph{2} proves that $W_{\Oo_E}(R)$ is $\pi$-adically complete.
	
	To see the second assertion, note that by \cref{lem:Vincharp}\itememph{2} we have
	\begin{equation*}
		[x_n]\pi^n=V^nF^n\big[x_n^{1/q^n}\big]=V^n[x_n]\,,
	\end{equation*}
	and use \cref{lem:imVn}\itememph{3}.
\end{proof*}
Finally we have everything together to prove \cref{prop:FqAlgebrasEquivalence}.
\begin{proof}[Proof of \cref{prop:FqAlgebrasEquivalence}]
	We claim that $W_{\Oo_E}(-)$ defines an inverse functor. If $R$ is a perfect $\IF_q$-algebra, \cref{lem:W_OEpi} shows $W_{\Oo_E}(R)/\pi W_{\Oo_E}(R)= W_{\Oo_E}(R)/\im V$. The right-hand side is isomorphic $R$ as an $\Oo_E$-algebra. On the level of sets this was seen in the proof of \cref{lem:imVn}\itememph{2}. Rs $\Oo_E$-algebra this follows from $S_0=X_0+Y_0$, $P_0=X_0Y_0$, and $(aX)_0=aX_0$ for all $a\in\Oo_E$.
	
	Thus, the image of $W_{\Oo_E}(-)$ is as desired. It remains to provide a natural isomorphism between $A$ and $W_{\Oo_E}(R)$ if $R=A/\pi A$. We define it via
	\begin{align*}
		W_{\Oo_E}(R)&\morphism A\\
		\sum_{n=0}^\infty[x_n]\pi^n&\longmapsto\sum_{n=0}^\infty[x_n]\pi^n\,.
	\end{align*}
	By \cref{lem:TeichmüllerRep} and \cref{lem:W_OEpi}, it is a natural bijection. By \cref{cor:snpn} it is $\Oo_E$-linear. We are done.
\end{proof}
\begin{cor}\label{cor:unramifiedWitt}
	Let $E_0$ be the maximal unramified subextension of $E/\IQ_p$ (or in other words, the unique unramified extension of $\IQ_p$ with residue field $\IF_q$). Then there is a natural isomorphism
	\begin{equation*}
		W(R)\otimes_{\Oo_{E_0}}\Oo_E\isomorphism W_{\Oo_E}(R)\,.
	\end{equation*}
\end{cor}
\begin{proof*}
	Since $p$ is a uniformizer of $\Oo_{E_0}$, the Witt vectors $W(R)$ taken over $\IZ_p$ are the same as if they were taken over $\Oo_{E_0}$. Now the diagram
	\begin{equation*}
		\begin{tikzcd}[row sep=normal]
			\left\{\begin{tabular}{c}
			$p$-torsionfree $p$-adically complete \\
			$\Oo_E$-algebras $A$ s.th.\ $A/p A$ is perfect
			\end{tabular}
			\right\}\ar[dd,"-\otimes_{\Oo_{E_0}}\Oo_E"{swap}]
			\drar[start anchor=south east, end anchor=north west, "-/p-"{swap}] & \\
			 & \left\{\text{perfect $\IF_q$-algebras}\right\}\ular[start anchor=168, end anchor=355, bend right, dotted, "W(-)"{swap}]\dlar[start anchor=192, end anchor=5, bend left, dotted, "W_{\Oo_E}(-)"]\\
			\left\{\begin{tabular}{c}
			$\pi$-torsionfree $\pi$-adically complete \\
			$\Oo_E$-algebras $A$ s.th.\ $A/\pi A$ is perfect
			\end{tabular}
			\right\}\urar[start anchor=north east, end anchor=south west,"-/\pi-"] & 
		\end{tikzcd}
	\end{equation*}
	of functors between categories commutes. Hence the diagram formed by the vertical arrow and the two dotted quasi-inverses commutes up to natural isomorphism, which is precisely what we want to show.
\end{proof*}
\begin{exm*}
	Now we can easily verify the examples given at the beginning of the section. To prove
	\begin{equation*}
		W(\IF_p)=\IZ_p\,,\quad W(\IF_q)=\Oo_{E_0}\,,\quad\text{and}\quad W\big(\IF_p\big\llbracket T^{1/p^\infty}\big\rrbracket\big)=\IZ_p\big\llbracket T^{1/p^\infty}\big\rrbracket\,,
	\end{equation*}
	it suffices to see that the respective right-hand sides are $p$-complete, $p$-torsionfree and that modding out $p$ gives $\IF_p$, $\IF_q$, and $\IF_p\big\llbracket T^{1/p^\infty}\big\rrbracket$ respectively. This is easy to check.
\end{exm*}

\section{The Ring \texorpdfstring{$\IA_\inf$}{Ainf}}
\lecture[Definition of $\IA_\inf$. Tilting as an adjoint to $W_{\Oo_E}(-)$. Perfectoid $\Oo_E$-algebras: tilting equivalence, examples. A picture of $\Spec \IA_\inf$.]{2019-10-30}
Apparently, $\IA_\inf$ is so awesome that Pierre Colmez titled it \enquote{The One Ring to rule them all} (\href{https://www.facebook.com/cyclotomicmemes/photos/a.189056291880728/347547606031595/?type=3&theater}{somewhat related}). For example, it already determines $B_\cris$ and $B_\dR$.

Throughout this section, let $p$ be a prime, $E/\IQ_p$ a finite extension, $\pi\in \Oo_E$ a uniformizer and $\IF_q=\Oo_E/\pi\Oo_E$ for $q=p^f$. Moreover, let $F/\IF_q$ be a non-archimedean algebraically closed extension. For us, \defemph{non-archimedean} always means that $F$ is complete with respect to a non-archimedean non-trivial valuation $|\blank|\colon F\morphism\IR_{\geq 0}$. As usual, the \defemph{ring of integers} $\Oo_F$ is defined as
\begin{equation*}
	\Oo_F=\left\{x\in F\st |x|\leq 1\right\}\,.
\end{equation*}
Note that $\Oo_F$ is local with maximal ideal $\mm_F=\left\{x\in F\st|x|<1\right\}$.
\begin{defi}
	In the above setting, we define
	\begin{equation*}
		\IA_\inf=\IA_{\inf,E,F}\coloneqq W_{\Oo_E}(\Oo_F)\,.
	\end{equation*}
\end{defi}
\begin{rem}
	\begin{numerate}
		\item $\IA_\inf$ should be thought of a \enquote{power series ring over $\Oo_F$ in the indeterminate $\pi$}. So its equal characteristic analogue should be $\Oo_F\llbracket z\rrbracket$.
		\item $\IA_\inf$ has a natural Frobenius action $\phi$, given by the Witt vector Frobenius, which is, in turn, given by the Frobenius on $\Oo_F$.
	\end{numerate}
\end{rem}
In the proof of \cref{prop:FqAlgebrasEquivalence} we have seen that $W_{\Oo_E}(-)$ is a quasi-inverse to $-/\pi -$ on some suitable category. In general, $W_{\Oo_E}(-)$ still possesses an adjoint, the \defemph{tilt functor}.
\begin{defi}
	Let $A$ be a $\pi$-complete $\Oo_E$-algebra. Then the \defemph{tilt of $A$} is
	\begin{equation*}
		A^\flat\coloneqq \lim_{x\mapsto x^q}A/\pi A=\left\{(a_0,a_1,\dotsc)\in\prod_{n\in\IN}A/\pi A\st a_i^q=a_{i-1}\text{ for all }i>0\right\}\,.
	\end{equation*}
\end{defi}
Note that $A^\flat$ is always a perfect $\IF_q$-algebra (in fact, that's a purely category-theoretical statement): the Frobenius on $A^\flat$ is given by $\Frob_{q,A^\flat}(a_0,a_1,\dotsc)=(a_0^q,a_0,a_1,\dotsc)$ and it has an inverse defined by $\Frob_{q,A^\flat}^{-1}(a_0,a_1,\dotsc)=(a_1,a_2,\dotsc)$.
\begin{prop}\label{prop:tiltWittAdjunction}
	There is an adjunction
	\begin{equation*}
		W_{\Oo_E}(-)\colon \left\{\text{$\pi$-complete $\Oo_E$-algebras}\right\}\doublelrmorphism \left\{\text{perfect $\IF_q$-algebras}\right\}\noloc (-)^\flat\,.
	\end{equation*}
\end{prop}
\begin{rem}
	Before we sketch a proof of \cref{prop:tiltWittAdjunction}, let us leave two remarks.
	\begin{numerate}
		\item If $R$ is a perfect $\IF_q$-algebra, then the unit $R\morphism W_{\Oo_E}(R)^\flat$ of the adjunction is given by $r\mapsto (r,r^{1/q},r^{1/q^2},\dotsc)$. Thus it is an isomorphism. In particular, this shows that $W_{\Oo_E}(-)$ is fully faithful by abstract nonsense. However, we have already seen that in the proof of \cref{prop:FqAlgebrasEquivalence}, where moreover the essential image of $W_{\Oo_E}(-)$ was identified as the class of $\pi$-complete $\pi$-torsionfree $\Oo_E$-algebras $A$ such that $A/\pi A$ is perfect.
		\item The counit $\theta\colon W_{\Oo_E}(A^\flat)\morphism A$ is usually called \defemph{Fontaine's map}.
	\end{numerate}
\end{rem}
\begin{proof}[Sketch of a proof of \cref{prop:tiltWittAdjunction}]
	First we state the following slightly more general form of the key \cref{lem:LTE} (actually, this proof only uses the previous formulation, but for future use the more general version will be handy). It can be proved in the exact same way as \cref{lem:LTE}.
	\begin{lem}[\enquote{$q$-power map is $\pi$-adically contracting}]\label{lem:keyLemma}
		Let $B$ be any $\Oo_E$-algebra and $I\subseteq B$ an ideal such that $\pi\in I$. If $x,y\in B$ such that $x\equiv y\mod I$, then
		\begin{equation*}
			x^{q^n}\equiv y^{q^n}\mod I^{n+1}\quad\text{for all }n\geq 0\,.
		\end{equation*}
	\end{lem}
	We construct the counit $\theta$ as follows. Fix $n>0$. By $W_{\Oo_E,n}(A)$ we denote the truncated Witt vectors of length $n+1$. These are obtained by cutting off everything after the first $n+1$ components. In other words, $W_{\Oo_E,n}(A)=W_{\Oo_E}(A)/\im V^{n+1}$. Consider the map
	\begin{align*}
		\Ww_n\colon W_{\Oo_E,n}(A)&\morphism A/\pi^{n+1}A\\
		[a_0,\dotsc,a_n]&\longmapsto W_n(a_0,\dotsc,a_n)\mod \pi^{n+1}\,.
	\end{align*}
	If $a_i\equiv 0\mod \pi$ for all $i=0,\dotsc,n$, then \cref{lem:keyLemma} shows $W_n(a_0,\dotsc,a_n)\equiv 0\mod \pi^ {n+1}$. Thus, we get an induced map $\theta_n\colon W_{\Oo_E,n}(A/\pi A)\morphism A/\pi^{n+1}A$. We check that the diagram
	\begin{equation*}
		\begin{tikzcd}
			W_{\Oo_E,n+1}(A/\pi A)\dar["F"{swap}]\rar["\theta_{n+1}"] & A/\pi^{n+2}A\dar\\
			W_{\Oo_E,n}(A/\pi A)\rar["\theta_n"] & A/\pi^{n+1}A
		\end{tikzcd}
	\end{equation*}
	commutes. Indeed, given $[\ov{a}_0,\dotsc,\ov{a}_{n+1}]\in W_{\Oo_E,n+1}(A/\pi A)$ with lifts $[a_0,\dotsc,a_{n+1}]$, we have
	\begin{equation*}
		W_{n+1}(a_0,\dotsc,a_{n+1})\equiv W_n(a_0^q,\dotsc,a_n^q)\mod \pi^{n+1}\,,
	\end{equation*}
	which is precisely what we want. Passing to the limit, we obtain a map
	\begin{equation*}
		\theta\colon W_{\Oo_E}(A^\flat)\cong \lim_FW_{\Oo_E,n}(A/\pi A)\morphism \lim_{n \in\IN}A/\pi^{n+1}A\cong A\,.
	\end{equation*}
	The isomorphism on the left is easy to check, and the isomorphism on the right follows from $A$ being $\pi$-complete. In the lecture, that was the end of the proof sketch. In these notes we will finish the proof, but only after we understand the map $\theta$ a little better.
\end{proof}
Another application of \cref{lem:keyLemma} is the following.
\begin{prop}\label{prop:(A/I)b}
	Let $A$ be a $\pi$-complete $\Oo_E$-algebra. Let $I\subseteq A$ be an ideal containing $I$, such that $A$ is also $I$-complete. Then the canonical map
	\begin{equation*}
		\lim_{x\mapsto x^q}A\isomorphism (A/I)^\flat
	\end{equation*}
	is an isomorphism. In particular, the left-hand side (which is a priori only a multiplicative monoid) inherits a natural ring structure.
\end{prop}
\begin{proof}
	Let $x=(\ov{x}_0,\ov{x}_1,\dotsc)\in (A/I)^\flat$. For every $n\geq 0$ choose a lift $x_n\in A$ of $\ov{x}_n$. By \cref{lem:keyLemma}, $(x_n^{q^n})_{n\in\IN}$ is a Cauchy sequence in the $I$-adic topology. Put
	\begin{equation*}
		x^\sharp=\lim_{n\to\infty}x_n^{q^n}\,.
	\end{equation*}
	As in the proof of \cref{deflem:Teichmüller}, $x^\sharp$ is independent of the choice of lifts and $(-)^\sharp$ is multiplicative. Now it's easy to see that the map
	\begin{align*}
		(A/I)^\flat&\morphism \lim_{x\mapsto x^q}A\\
		x&\longmapsto \big(x^\sharp,(x^{1/q})^\sharp,\dotsc\big)
	\end{align*}
	is a multiplicative inverse of the map in question. This proves the assertion.
\end{proof}
\begin{lem}\label{lem:WAb->A}
	The counit $\theta\colon W_{\Oo_E}(A^\flat)\morphism A$ can be explicitly described as 
	\begin{equation*}
		\sum_{n=0}^\infty[a_n]\pi^n\longmapsto \sum_{n=0}^\infty a_n^\sharp \pi^n\,.
	\end{equation*}
\end{lem}
\begin{proof*}
	Let us first describe the isomorphism $W_{\Oo_E}(A^\flat)\cong \lim_FW_{\Oo_E,n}(A/\pi A)$ that is part of the definition of $\theta$. The underlying set of $W_{\Oo_E}(A^\flat)$ consists of sequences $[a_0,a_1,\dotsc]$, where each $a_n\in A^\flat$ is itself a sequence $a_n=(\ov{a}_{n,i})_{i\in\IN}$ in $A/\pi A$ such that $\ov{a}_{n,i}^q=\ov{a}_{n,i-1}$. The underlying set of $\lim_FW_{\Oo_E,n}(A/\pi A)$ consists of sequences $([\ov{a}_{0,0}],[\ov{a}_{0,1},\ov{a}_{1,1}],[\ov{a}_{0,2},\ov{a}_{1,2},\ov{a}_{2,2}],\dotsc)$ that are compatible under $F$. The isomorphism in question is given by
	\begin{align*}
		W_{\Oo_E}(A^\flat)&\isomorphism\lim_FW_{\Oo_E,n}(A/\pi A)\\
		[\ov{a}_0,\ov{a}_1,\ov{a}_2,\dotsc]&\longmapsto \big([\ov{a}_{0,0}],[\ov{a}_{0,1},\ov{a}_{1,1}],[\ov{a}_{0,2},\ov{a}_{1,2},\ov{a}_{2,2}],\dotsc\big)\,.
	\end{align*}
	Indeed, it is clear that this defines a bijection on set level and one may check that it is also compatible with the ring structures on either side.
	
	Now let $a=[a_0,a_1,\dotsc]\in W_{\Oo_E}(A^\flat)$ be as above. We unwind what $\theta(a)$ actually is. By definition of the map $\theta_N\colon W_{\Oo_E,N}(A/\pi A)\morphism A/\pi^{N+1}A$, we have
	\begin{equation*}
		\theta_N[\ov{a}_{0,N},\dotsc,\ov{a}_{N,N}]\equiv\sum_{n=0}^Na_{n,N}^{q^{N-n}}\pi^n\mod \pi^{N+1}\,,
	\end{equation*}
	where the $a_{n,N}$ are arbitrary lifts of $\ov{a}_{n,N}$. 
	Thus, the coefficient of $\pi^n$ in $\theta(a)$ is given by
	\begin{equation*}
		\lim_{N\to\infty}a_{n,N}^{q^{N-n}}=\big(a_n^{1/q^n}\big)^\sharp\,.
	\end{equation*}
	The exponent $1/q^n$ seems off at first glance, but according to \cref{lem:W_OEpi} this is exactly what we want.
\end{proof*}
\begin{proof*}[End of proof of \cref{prop:tiltWittAdjunction}]
	Let $A$ be a $\pi$-complete $\Oo_E$-algebra and $R$ a perfect $\IF_q$-algebra. By \cref{prop:FqAlgebrasEquivalence} we have a bijection
	\begin{equation*}
		\Hom(R,A^\flat)\cong \Hom\big(W_{\Oo_E}(R),W_{\Oo_E}(A^\flat)\big)\,,
	\end{equation*}
	so it suffices to see that every $\Oo_E$-algebra morphism $\alpha\colon W_{\Oo_E}(R)\morphism A$ factors uniquely over $\theta$. Let such an $\alpha$ be given. Modulo $\pi$ we get an induced morphism $\ov{\alpha}\colon R\morphism A/\pi A$. Since $R$ is perfect, $R^\flat\cong R$. Also $A^\flat\cong (A/\pi A)^\flat$. Hence we get an induced morphism $\ov{\alpha}^\flat\colon R\morphism A^\flat$. We claim that
	\begin{equation*}
		\begin{tikzcd}
			W_{\Oo_E}(R)\dar["W_{\Oo_E}(\ov{\alpha}^\flat)"{swap}]\rar["\alpha"]& A\\
			W_{\Oo_E}(A^\flat)\urar["\theta"{swap}]&
		\end{tikzcd}
	\end{equation*}
	commutes. In view of \cref{lem:WAb->A} we only need to check that $\alpha[x]=\ov{\alpha}^\flat(x)^\sharp$ for all $x\in R$. By construction, $\ov{\alpha}^\flat(x)$ is the sequence $(\ov{\alpha}(x),\ov{\alpha}(x^{1/q}),\dotsc)\in A^\flat$. Moreover, $\alpha[x^{1/q^n}]$ is a lift of $\ov{\alpha}(x^{1/q^n})$ for all $n\in\IN$. Raising $\ov{\alpha}(x^{1/q^n})$ to the $(q^n)\ordinalth$ power gives $\alpha[x]$ back, since both $\alpha$ and the Teichmüller lift $[-]$ are multiplicative. This shows indeed $\alpha[x]=\ov{\alpha}^\flat(x)^\sharp$.
	
	To finish the proof, it's left to see why $\ov{\alpha}^\flat$ is the only choice. Suppose $\beta\colon R\morphism A^\flat$ leads to a commutative diagram as above. Reducing modulo $\pi$ we see that the composition of $\beta$ with $A^\flat\morphism A/\pi$ must coincide with $\ov{\alpha}$. In other words, the $0\ordinalth$ component of $\beta\colon R\morphism A^\flat$ must be given by $\ov{\alpha}$. By naturality of the Witt vector Frobenius, the diagram
	\begin{equation*}
		\begin{tikzcd}
		W_{\Oo_E}(R)\rar["F^{-1}"]\dar["W_{\Oo_E}(\beta)"{swap}"]& W_{\Oo_E}(R)\dar["W_{\Oo_E}(\beta)"{swap}]\rar["\alpha"]& A\\
		W_{\Oo_E}(A^\flat)\rar["F^{-1}"] & W_{\Oo_E}(A^\flat)\urar["\theta"{swap}]&
		\end{tikzcd}
	\end{equation*}
	commutes as well. Reducing modulo $\pi$ and walking around the perimeter, we see that the $1\ordinalst$ component of $R\morphism A^\flat$ must be given by $\ov{\alpha}((-)^{1/q})$. Repeating this argument, we see that $\beta=\ov{\alpha}^\flat$, as desired.
\end{proof*}
\subsection{Perfectoid \texorpdfstring{$\Oo_E$}{O}-Algebras}
\begin{defi}
	\begin{numerate}
		\item A \defemph{perfect prism} over $\Oo_E$ is a pair $(W_{\Oo_E}(R),I)$, where $R$ is a perfect $\IF_q$-algebra, $I\subseteq W_{\Oo_E}(R)$ is a principal ideal generated by an element $d$ such that
		\begin{equation*}
			\frac{F(d)-d^q}{\pi}\in W_{\Oo_E}(R)^\times
		\end{equation*}
		(such $d$ is called \defemph{distinguished}), and such that $W_{\Oo_E}(R)$ is $(\pi,I)$-adically complete.
		\item An $\Oo_E$-algebra $A$ is a \defemph{perfectoid $\Oo_E$-algebra} if it can be written as $A\cong W_{\Oo_E}(R)/I$ for some perfect prism $(W_{\Oo_E}(R),I)$ over $\Oo_E$.
	\end{numerate}
\end{defi}
\begin{rem}\label{rem:perfectoid}
	\begin{numerate}
		\item To see that $F(d)-d^q$ is always divisible by $\pi$, note that $F$ is the lift of the Frobenius on $R$. In particular, $F$ and $(-)^q$ become equal after reducing modulo $\pi$.
		\item An element $d=\sum_{n=0}^\infty[r_n]\pi^n\in W_{\Oo_E}(R)$ is distinguished iff $r_1\in R^\times$. Indeed, by \cref{lem:WittFrob}\itememph{2} and \cref{lem:W_OEpi} we have $F(d)\equiv [r_0^q]+[r_1^q]\pi\mod \pi^2$ and from the key \cref{lem:keyLemma} we get $d^q\equiv [r_0^q]\mod \pi^2$. Hence
		\begin{equation*}
			\frac{F(d)-d^q}{\pi}\equiv [r_1^q]\mod \pi\,.
		\end{equation*}
		By $\pi$-completeness, an element $x\in W_{\Oo_E}(R)$ is invertible iff its modulo-$\pi$ reduction is invertible. And $r_1^q\in R$ is invertible iff so is $r_1$. Moreover, $W_{\Oo_E}(R)$ is $(\pi,d)$-adically complete iff $R$ is $r_0$-complete. Since this seems rather non-trivial to me, we give it a proper proof in \cref{lem*:nonTrivial} below.
		\item Perfect rings are perfectoid. Indeed, if $R$ is perfect, we have $R\cong W_{\Oo_E}(R)/\pi W_{\Oo_E}(R)$, and $(W_{\Oo_E}(R),\pi)$ is clearly a perfect prism (by \itememph{2} for example). Conversely, if an algebra $A$ over $\IF_q=\Oo_E/\pi\Oo_E$ is perfectoid, then it is also perfect. This too was not trivial for me, so we prove it in \cref{lem*:perfectoid=perfect} below.
		\item If $A$ is perfectoid, say, $A\cong W_{\Oo_E}(R)/I$, then
		\begin{equation*}
			A^\flat\cong (W_{\Oo_E}(R)/I)^\flat\cong \big(W_{\Oo_E}(R)/(\pi,I)\big)^\flat\cong (R/IR)^\flat\cong R^\flat\cong R\,.
		\end{equation*}
		The only non-obvious step is $(R/IR)^\flat\cong R^\flat$. To see this, first note that $IR$ is an ideal containing the image of $\pi$ in $R$ since this image is $0$. Moreover, $R$ is $IR$-adically complete by \cref{lem*:nonTrivial}. Hence the isomorphism follows from \cref{prop:(A/I)b}.
	\end{numerate}
\end{rem}
\begin{lem*}\label{lem*:nonTrivial}
	Let $R$ be a perfect $\IF_q$-algebra and $d=\sum_{n=0}^\infty[r_n]\pi^n$ be an element of $W=W_{\Oo_E}(R)$. Then $W$ is $(\pi,d)$-adically complete iff $R$ is $r_0$-complete.
\end{lem*}
\begin{proof*}
	Let's first assume $W$ is $(\pi,d)$-complete. Then $R$ being $r_0$-complete is equivalent to $R$ being $(\pi,d)$-complete too. By \cite[\stackstag{031A}]{stacks-project}, we need to check that
	\begin{equation*}
		\pi W=\bigcap_{n\geq 1}\big(\pi W+(\pi,d)^n\big)\,.
	\end{equation*}
	Suppose some $w\in W$ is contained in $\pi W+(\pi,d)^n$ for all $n\in \IN$. Then its image $\ov{w}\in R$ is divisible by $r_0^n$ for all $n\geq 0$, hence also $[\ov{w}]$ is divisible by $[r_0]^n$ for all $n\geq 0$. By a well-known argument, $W$ being $(\pi,d)$-complete is equivalent to $W$ being complete with respect to the ideals $\{(\pi^n,d^n)\}_{n\geq 1}$. By abstract nonsense, we may replace this family of ideals by $\{(\pi^{n+1},d^{q^n})\}_{n\geq 1}$. But $d^{q^n}\equiv [r_0]^{q^n}\mod \pi^{n+1}$ by the key \cref{lem:keyLemma}, hence $W$ is also complete with respect to the ideals $\{(\pi^{n+1},[r_0]^{q^n})\}_{n\geq 1}$. Since $[\ov{w}]$ lies in all of them by assumption, we get $\ov{w}=0$, hence $w\in\pi W$, as required.
	
	Now assume $R$ is $r_0$-complete. It suffices to show that $W$ is complete with respect to the ideals $\{(\pi^n,d^n)\}_{n\geq 1}$. By an abstract nonsense argument, this is equivalent to $W$ being complete with respect to $\{(\pi^n,d^m)\}_{n,m\geq 1}$. Since $W$ is $\pi$-complete, it thus suffices to show that $W/\pi^nW$ is $d$-complete for all $n\geq 1$. The key \cref{lem:keyLemma} shows $d^{q^m}\equiv [r_0]^{q^m}\mod \pi^n$ for all $m\geq n-1$. Thus we may equivalently show that $W/\pi^nW$ is $[r_0]$-complete.
	
	We argue by induction over $n$. The case $n=1$ is just the assumption. Now assume the assertion holds up to $n$. Consider the short exact sequence
	\begin{equation*}
		0\morphism W/\pi^nW\morphism[\pi]W/\pi^{n+1}W\morphism R\morphism 0\,.
	\end{equation*}
	Suppose $x\in W/\pi^nW$ has the property that $\pi x\in W/\pi^{n+1}W$ is divisible by $[r_0]^m$, say, $\pi x=[r_0^m]y$. Write $x=[x_0]+[x_1]\pi+\dotsb+[x_{n-1}]\pi^{n-1}$ and $y=[y_0]+[y_1]\pi+\dotsb+[y_n]\pi^n$. Then
	\begin{equation*}
		[x_0]\pi+[x_1]\pi^2+\dotsb+[x_{n-1}]\pi^n=[r_0^my_0]+[r_0^my_1]\pi+\dotsb+[r_0^my_n]\pi^n\,.
	\end{equation*}
	By uniqueness of these representations, we get $0=r_0^my_0$, $x_0=r_0^my_1$ and so on up to $x_{n-1}=r_0^my_n$. In particular, $x=[r_0]^m([y_1]+\dotsb+[y_n]\pi^{n-1})$ is divisible by $[r_0]^m$! We conclude that the sequence
	\begin{equation*}
		0\morphism W/(\pi^n,[r_0]^m)\morphism[\pi] W/(\pi^{n+1},[r_0]^m)\morphism R/r_0^mR\morphism 0
	\end{equation*}
	is exact again. Taking limits over $m$ we obtain a diagram
	\begin{equation*}
		\begin{tikzcd}
			0 \rar & W/\pi^nW\dar[iso]\rar["\pi"] & W/\pi^{n+1}W \dar\rar & R\rar\dar[iso] & 0\\
			0 \rar &\lim\limits_{m\geq 1}W/(\pi^n,[r_0]^m)\rar["\pi"] & \lim\limits_{m\geq 1}W/(\pi^{n+1},[r_0]^m)\rar & \lim\limits_{m\geq 1}R/r_0^mR\rar & 0
		\end{tikzcd}
	\end{equation*}
	in which the outer vertical arrows are isomorphisms by the induction hypothesis. Thus the middle vertical arrow is an isomorphism as well by the five lemma (note that the bottom sequence is exact by the Mittag-Leffler condition, but this isn't even needed for the argument).
\end{proof*}
\begin{lem*}\label{lem*:perfectoid=perfect}
	If an algebra $A$ over $\IF_q=\Oo_E/\pi\Oo_E$ is perfectoid, then $A$ is already a perfect $\IF_q$-algebra.
\end{lem*}
\begin{proof*}
	Write $A\cong W_{\Oo_E}(R)/I$. Since $\pi$ vanishes on $A$, we have $\pi\in A$. By \cref{rem:perfectoid}\itememph{4}, $A^\flat\cong R\cong W_{\Oo_E}(R)/\pi W_{\Oo_E}(R)$. Hence it suffices to prove that $I$ is generated by $\pi$, since then $A\cong A^\flat$ is perfect.
	
	The argument that follows is stolen from \cite[Lemma~3.10]{BMS}. Write $\pi=dw$, where $d\in I$ is a distinguished generator and $w=\sum_{n=0}^\infty [w_n]\pi^n$ is some element of $W_{\Oo_E}(A^\flat)$. The Witt polynomial $P_1$ is given by $P_1(X,Y)=X_0^qY_1+X_1Y_0^q+\pi X_1Y_1$. Thus $\pi=dw$ yields
	\begin{equation*}
		1=r_0^qw_1+r_1w_0^q
	\end{equation*}
	(note that $\pi r_1w_1$ vanishes in $A^\flat$). We claim that $r_1w_0^q=1-r_0^qw_1$ is a unit in $A^\flat$. It suffices to check that it is mapped to a unit under the projection $A^\flat\morphism A/\pi A=A$ to the $0\ordinalth$ component. But $A\cong A^\flat/r_0A^\flat$, hence $1-r_0^qw_1$ is mapped to $1\in A$, which is indeed a unit. Thus also $r_1$ and $w_0$ are units in $A^\flat$. But $w_0$ being a unit implies that $w$ itself is a unit in $W_{\Oo_E}(A^\flat)$, hence $\pi$ is indeed a generator of $I$.
\end{proof*}
The following fact wasn't mentioned in the lecture, making it hard for me to read some of the literature that uses the \enquote{old} definition of perfectoid rings. So we prove it here.
\begin{lem*}\label{lem*:perfectoidComplete}
	Let $(W_{\Oo_E}(R),I)$ be a perfect prism over $\Oo_E$ and $A=W_{\Oo_E}(R)/I$.
	\begin{alphanumerate}
		\item If $\xi$ is a distinguished generator of $I$, then $\xi$ is a non-zero divisor in $W_{\Oo_E}(R)$.
		\item $A$ is $\pi$-complete.
	\end{alphanumerate}
\end{lem*}
\begin{proof*}
	Put $W=W_{\Oo_E}(R)$ for convenience. Both \itememph{a} and \itememph{b} are based on the following observation.
	\begin{alphanumerate}
		\item[\itememph{*}] Let $(x_n)_{n\in \IN}$ be a sequence such that $\xi x_n\equiv 0\mod \pi^n$. Then the $x_n$ converge to $0$ in the $(\pi,\xi)$-adic topology.
	\end{alphanumerate}
	Claim \itememph{*} immediately implies \itememph{a}. Also  \itememph{b} is not far: by \cite[\stackstag{031A}]{stacks-project}, we need to check that
	\begin{equation*}
		\xi W=\bigcap_{n\geq 1}(\xi W+\pi^nW)\,.
	\end{equation*}
	So suppose $y$ lies in the intersection and choose $(x_n)_{n\in \IN}$ such that $y\equiv \xi x_n\mod \pi^n$. Then $\xi(x_{n+1}-x_n)\equiv 0\mod \pi^n$. Thus the $(x_{n+1}-x_n)$ converge to $0$ in the $(\pi,\xi)$-adic topology. Hence $(x_n)_{n\in \IN}$ converges to some $x\in W$ satisfying $y=\xi x$. This shows \itememph{a}.
	
	It remains to show \itememph{*}. Write $\xi=[r_0]+\pi u$, where $u\in W$ is a unit. If $\xi x_n\equiv 0\mod \pi^n$, then also $([r_0]^s+\pi^su^s)x_n\equiv 0\mod \pi^n$ for all odd $s$, since $[r_0]+\pi u$ divides $[r_0]^s+\pi^su^s$ for odd $s$. Now $\pi^sx_n\equiv -[r_0]^su^{-s}x_n\mod \pi^n$ shows that the first $n$ coefficients in $\pi$-adic expansion of $\pi^sx_n$ must be divisible by $r_0^s$. In other words, we can write
	\begin{equation*}
		x_n=[r_0^sy_0]+[r_0^sy_1]\pi+\dotsb+[r_0^sy_{n-s-1}]\pi^{n-s-1}+\pi^{n-s}z\,.
	\end{equation*}
	Thus, $x_n\in(\pi^{n-s},[r_0]^s)$ for all odd $s$. Choosing $s$ roughly equal to $n/2$, we see that $(x_n)_{n\in \IN}$ converges with respect to the ideals $\{(\pi^m,[r_0]^m)\}_{m\geq 1}$. But these ideals generate $(\pi,\xi)$-adic topology, as seen in the proof of \cref{lem*:nonTrivial}. 
\end{proof*}


\cref{rem:perfectoid}\itememph{4} suggests the following definition.
\begin{defi}
	Let $R$ be a perfect $\IF_q$-algebra. An \defemph{untilt} of $R$ is a pair $(A,\iota)$, where $A$ is a perfectoid $\Oo_E$-algebra and $\iota$ an isomorphism $\iota\colon R\morphism A^\flat$.
\end{defi}
Again by \cref{rem:perfectoid}\itememph{4} we get a bijection
\begin{equation*}
	\left\{\begin{tabular}{c}
		isomorphism classes of\\
		untilts $(A,\iota)$ of $R$
	\end{tabular}\right\}\lrisomorphism \left\{\begin{tabular}{c}
	ideals $I\subseteq W_{\Oo_E}(R)$ such that\\ $(W_{\Oo_E}(R),I)$
	is a perfect prism over $\Oo_E$
	\end{tabular}\right\}
\end{equation*}
\begin{exc}[Tilting equivalence]\label{exc:tilting}
	If $A$ is a perfectoid $\Oo_E$-algebra, then there is an equivalence of categories
	\begin{align*}
		\left\{\text{perfectoid $A$-algebras}\right\}&\lrisomorphism \left\{\text{perfect(oid) $A^\flat$-algebras}\right\}\\
		B&\longmapsto B^\flat\\
		W_{\Oo_E}(S)\otimes_{W_{\Oo_E}(A^\flat)}A&\longmapsfrom S
	\end{align*}
	(on the left-hand side, $A$ gets a $W_{\Oo_E}(A^\flat)$-algebra structure via $\theta$).
\end{exc}
\begin{proof*}[Disproof]
	%Let $T=W_{\Oo_E}(S)\otimes_{W_{\Oo_E}(A^\flat)}A$. We first check $T^\flat\cong S$. We have 
	%\begin{equation*}
	%	T/\pi T\cong S\otimes_{A^\flat}A/\pi A\,.
	%\end{equation*}
	The assertion as stated is wrong. Take $A^\flat=\IF_p\llbracket T^{1/p^\infty}\rrbracket$ and $A$ comes from the perfect prism $(W(A^\flat),T-p)$. This works by \cref{lem*:nonTrivial} since $T-p$ is clearly distinguished and $A^\flat$ is $T$-complete. We claim that there is a perfect $A^\flat$-algebra $S$ such that $W(S)\otimes_{W(A^\flat)}A$ is not perfectoid. Indeed, for it to be perfectoid, $(W(S),(T-p)W(S))$ would need to be a perfect prism, which again needs $S$ to be $T$-complete by \cref{lem*:nonTrivial} again. However, there are perfect $A^\flat$ algebras $S$ which are not $T$-complete; for example, the Laurent series ring $S=\IF_p(\!(T^{1/p^\infty})\!)$.
\end{proof*}
\numpar{Corrected exercise* \textmd{(The actual tilting equivalence)}}
By a \defemph{perfectoid $A^\flat$-algebra} $S$ we don't just understand an $A^\flat$-algebra that is perfectoid. The topology on $S$ must also be induced by the topology on $A^\flat$, i.e., $S$ must be $(\pi,I)$-complete, where $I$ is the kernel of $\theta\colon W_{\Oo_E}(A^\flat)\morphism A$ (so that $(W_{\Oo_E}(A^\flat),I)$ is a perfect prism that gives $A$). Then there is an equivalence of categories
\begin{equation*}
	\left\{\text{perfectoid $A$-algebras}\right\}\lrisomorphism \left\{\text{perfect $A^\flat$-algebras}\right\}
\end{equation*}
as in \cref{exc:tilting}.
\begin{proof*}
	Put $W_A=W_{\Oo_E}(A^\flat)$ and $W_S=W_{\Oo_E}(S)$ for convenience. Let $\xi$ be a distinguished generator of $I$. First note that $W_S\otimes_{W_A}A\cong W_S/\xi W_S$ is again perfectoid. Indeed, we need to check that $(W_S,\xi W_S)$ is a perfect prism. Clearly $\xi W_S$ is a distinguishedly generated ideal. Also $S$ is $(\pi,I)$-complete and hence $\xi$-complete, so $W_S$ is $(\pi,\xi W_S)$-complete by \cref{lem*:nonTrivial}. This shows that $(W_S,\xi W_S)$ is a perfect prism, as required. Now the calculation from \cref{rem:perfectoid}\itememph{4} shows $(W_S/\xi W_S)^\flat\cong S$.
	
	Conversely, we have to show that for a perfectoid $A$-algebra $B$ we get $B\cong W_B\otimes_{W_A} A$, where $W_B=W_{\Oo_E}(B^\flat)$ for brevity, and that $B^\flat$ is $(\pi,I)$-complete. Write $B\cong W_B/J$. Then  $(W_A,I)\morphism (W_B,J)$ is a morphism of perfect prisms in the sense that it is a $\Oo_E$-algebra morphism that maps $I$ into $J$. An argument analogous to the stolen one from the proof of \cref{lem*:perfectoid=perfect} (hint: replace $1$ be the coefficient of $\pi$ in $\xi$, which is still a unit) shows that actually $J=IW_B$. But this immediately shows $B\cong W_B\otimes_{W_A}A$ and we are done.
\end{proof*}

\numpar{Example \smash{\Attention}}\label{exm:OCperfectoid}
If $C/E$ is a non-archimedean (recall that this requires $C$ to be complete) algebraically closed field extension, then the ring of integers $\Oo_C$ is a perfectoid $\Oo_E$-algebra.
\begin{proof}
	We first formulate two claims which together will imply the assertion.
	\begin{numerate}
		\item Let $\{\pi^{1/q^n}\}_{n\geq 0}$ be a compatible system of $(q^n)\ordinalth$ roots of $\pi$ in $\Oo_C$. They define an element $\pi^\flat=(\pi,\pi^{1/q},\dotsc)\in\Oo_C^\flat$. Then 
		\begin{equation*}
			\Oo_C^\flat/\pi^\flat\Oo_C^\flat\cong\Oo_C/\pi\Oo_C\,.
		\end{equation*}
		\item The kernel of $\theta\colon W_{\Oo_E}(\Oo_C^\flat)\morphism\Oo_C$ is generated by $\pi-[\pi^\flat]$.
	\end{numerate}
	We start with \itememph{1}. Note that by \cref{prop:(A/I)b} we may write $\Oo_C^\flat\cong \lim_{x\mapsto x^q}\Oo_C$. Now let $y=(y_0,y_1,\dotsc)\in\Oo_C^\flat$. Then $\pi^\flat\mid y$ iff $\pi^{1/q^n}\mid y_n$ for all $n\geq 0$. Since $\Oo_C$ is a valuation ring, this is equivalent to $|\pi|^{1/q^n}\geq |y_n|=|y_0|^{1/q^n}$. Thus, $\pi^\flat\mid y$ is equivalent to the single condition $y_0\equiv 0\mod \pi$. Therefore, the kernel of $(-)^\sharp\colon \Oo_C^\flat\morphism \Oo_C/\pi\Oo_C$ is generated by $\pi^\flat$. However, $\Oo_C^\flat\epimorphism \Oo_C/\pi\Oo_C$ is clearly surjective (since $C$ is algebraically closed), hence indeed
	\begin{equation*}
		\Oo_C^\flat/\pi^\flat\Oo_C^\flat\cong \Oo_C/\pi\Oo_C\,.
	\end{equation*}
	For \itememph{2}, \cref{lem:WAb->A} shows $\theta(\pi-[\pi^\flat])=\pi-(\pi^\flat)^\sharp=\pi-\pi=0$. So $\pi-[\pi^\flat]\in\ker\theta$. Conversely, let $x=\sum_{n=0}^\infty[x_n]\pi^n$ be an element of $\ker\theta$. Hence
	\begin{equation*}
		0\equiv \theta(x)\equiv \sum_{n=0}^\infty x_n^\sharp\pi^n\equiv x_0^\sharp\mod \pi\,.
	\end{equation*}
	From \itememph{1} we get $\pi^\flat\mid x_0$, say, $x_0=\pi^\flat y$. Write $z^{(0)}=\sum_{n=1}^\infty [x_n]\pi^{n-1}$ and $x^{(1)}=[y]+z^{(0)}$. Then $x=[\pi^\flat]x^{(1)}+(\pi-[\pi^\flat])z^{(0)}$. We obtain
	\begin{equation*}
		0=\theta(x)=\theta\big([\pi^\flat]x^{(1)}\big)=\pi\theta\big(x^{(1)}\big)\,,
	\end{equation*}
	hence also $\theta(x^{(1)})=0$ since $\Oo_C$ is $\pi$-torsionfree. Repeating this process with $x^{(1)}$ and iterating, we get an expression
	\begin{equation*}
		x=\xi\big(z^{(0)}+[\pi^\flat]z^{(1)}+\dotsb\big)\,,
	\end{equation*}
	where $\xi=\pi-[\pi^\flat]$. This shows that $x$ lies in the ideal generated by $\xi$, proving \itememph{2}.
	
	It remains to see that $\theta\colon W_{\Oo_E}(\Oo_C^\flat)\morphism\Oo_C$ is surjective and that $(W_{\Oo_E}(\Oo_C^\flat),\xi)$ is a perfect prism. The first assertion is because $(-)^\sharp\colon \Oo_C^\flat\morphism\Oo_C$ is surjective since $C$ is algebraically closed. For the second assertion, $\xi=\pi-[\pi^\flat]$ is clearly distinguished by \cref{rem:perfectoid}\itememph{2}, so it remains to show that $\Oo_C^\flat$ is $\pi^\flat$-complete. Observe that for all $c\geq 0$ the $c\ordinalth$ component of $(\pi^\flat)^{q^n}$ is $0$ for all $n\geq c$. From this observation, $\pi^\flat$-completeness of $\Oo_C^\flat$ easily follows.
\end{proof}
Next time we proof the first half of the following \cref{lem:perfectoidOC} (see \cref{lem:OcflatisOF}). The other half will have to wait until the $4\ordinalth$ lecture.
\begin{lem}\label{lem:perfectoidOC}
	Let $A$ be a perfectoid $\Oo_E$-algebra. Then $A$ is isomorphic to $\Oo_C$ for some non-archimedean algebraically closed extension $C/E$ if and only if $A^\flat$ is isomorphic to $\Oo_F$ for some non-archimedean algebraically closed extension $F/\IF_q$.
\end{lem}
\begin{rem}\label{rem:AinfProperties}
	Recall that for $F$ as in \cref{lem:perfectoidOC} we put $\IA_\inf=W_{\Oo_E}(\Oo_F)$.
	\begin{numerate}
		\item $\IA_\inf$ is a local integral domain. This is in fact true for any $W_{\Oo_E}(R)$ if $R$ itself is a local integral domain over $\IF_q$ (this follows from \cref{lem:W_OEpi} for example).
		\item $\IA_\inf$ is $(\pi,[\varpi])$-complete for any $\varpi\in\mm_F\setminus \{0\}$. Indeed, this follows from \cref{rem:perfectoid}\itememph{2} as $\Oo_F$ is easily seen to be $\varpi$-complete. Such $\varpi$ is called a \defemph{pseudo-uniformizer}.
		\item By a theorem of Ludwig--Lang, $\IA_\inf$ has infinite Krull dimension (and is, in particular, non-noetherian). We can actually see by hand that $\IA_\inf$ is at least three-dimensional: there is a chain
		\begin{equation*}
			0\subsetneq\bigcup_{x\in\mm_F}[x]\IA_\inf\subsetneq W_{\Oo_E}(\mm_F)\subsetneq (\pi,W_{\Oo_E}(\mm_F))
		\end{equation*}
		of prime ideals. Also note that $(\pi,W_{\Oo_E}(\mm_F))$ is the unique maximal ideal of $\IA_\inf$ since an element of $\IA_\inf$ is invertible iff its image in $\IA_\inf/\pi\IA_\inf\cong \Oo_F$ is invertible.
	\end{numerate}
\end{rem}
Despite \cref{rem:AinfProperties}\itememph{3}, we should think of $\IA_\inf$ as a two-dimensional ring, except for some \enquote{bad} primes. Here's a \enquote{picture} of $\Spec\IA_\inf$. The left picture shows a select choice of prime ideals of $\IA_\inf$. In the right picture the corresponding residue fields are shown and the Frobenius action $\phi$ is indicated.
\begin{center}
	\tabcolsep=0pt
	\begin{tabularx}{\textwidth}{X c X c X}
		& \begin{tikzpicture}[line width=rule_thickness, line cap=round, line join =round, x=1cm,y=1cm]
		\draw[-to] (0,0) -- (5,0) node[below] {$[\varpi]$};
		\draw[-to] (0,0) -- (0,5) node[left] {$\pi$};
		\fill (0,0) circle (0.5ex) node[below=4] (null) {$(\pi,W_{\Oo_E}(\mm_F))$};
		\fill (4,0) circle (0.5ex) node[below=4] (pi) {$(\pi)$};
		%\path (null) -- (pi) node[pos=0.5] {\scriptsize$\Spec W_{\Oo_E}(k)$};
		\fill (0,4) circle (0.5ex) node[right=4] {$W_{\Oo_E}(\mm_F)$};
		\draw[rounded corners, thick] (-1ex,-1ex) rectangle (4cm+1ex,1ex) node[pos=0.5, below=4] {\scriptsize$\Spec \Oo_F$};
		\draw[rotate=30,rounded corners, thick] (-1ex,-1ex) rectangle (3.5cm+1ex,1ex) node[pos=0.5,rotate=30] {\scriptsize$\Spec \Oo_C$};
		\draw[pattern=north west lines, rounded corners, thick] (1ex,-1ex) rectangle (-1ex,4cm+1ex) node[pos=0.5, rotate=90, above=4] {\scriptsize$\Spec W_{\Oo_E}(k)$} node[pos=0.5, rotate=90, below=4] {\scriptsize \enquote{bad} primes};
		\fill (30:3.5cm) circle (0.5ex) node[right=4,align=left] {$(\pi-[\pi^\flat])$,\\[0.5ex] $\Oo_C$ untilt of $\Oo_F$};
		\draw[rotate=60,rounded corners, thick] (-1ex,-1ex) rectangle (3.5cm+1ex,1ex) node[pos=0.5, rotate=60] {\scriptsize$\Spec \Oo_{\smash{C'}}$};
		\fill (60:3.5cm) circle (0.5ex) node[right=4, align=left] {$(\pi-[a])$,\\[0.5ex] $a\in\mm_F\setminus\{0\}$};
		\end{tikzpicture} & & \begin{tikzpicture}[line width=rule_thickness, line cap=round, line join =round, x=1cm,y=1cm]
		\draw[-to] (0,0) -- (5,0) node[below] {$[\varpi]$};
		\draw[-to] (0,0) -- (0,5) node[left] {$\pi$};
		\fill (0,0) circle (0.5ex) node[below=4] (null) {$k$};
		\fill (4,0) circle (0.5ex) node[below=4] (pi) {$F$};
		%\path (null) -- (pi) node[pos=0.5] {\scriptsize$\Spec W_{\Oo_E}(k)$};
		\fill (0,4) circle (0.5ex) node[right=4] {$W_{\Oo_E}(k)\left[\frac1\pi\right]$};
		\draw[rounded corners, thick] (-1ex,-1ex) rectangle (4cm+1ex,1ex);
		\draw[rotate=30,rounded corners, thick] (-1ex,-1ex) rectangle (3.5cm+1ex,1ex);
		\draw[pattern=north west lines, rounded corners, thick] (1ex,-1ex) rectangle (-1ex,4cm+1ex);
		\fill (30:3.5cm) circle (0.5ex) node[right=4] {$C$};
		\draw[->, shift={(0,0)}] (30:4.5cm) arc (30:50:4.5cm) node[pos=0.5,above right] {$\phi$};
		\end{tikzpicture} &
	\end{tabularx}
\end{center}
We put $k=\Oo_F/\mm_F$ for convenience. We will see next time that $\Spec\IA_\inf$ is indeed \enquote{two-dimensional away from $[\varpi]=0$}. More precisely, we will show the following: let $(\Oo_C,\iota)$ be an untilt of $\Oo_F$ and $\xi$ a generator of $\ker(\theta\colon \IA_\inf\morphism \Oo_C)$. Put
\begin{equation*}
	B_\dR^+=\IA_\inf\left[\textstyle \frac 1\pi\right]_\xi^\complete\,.
\end{equation*}
Then $B_\dR^+$ is always a DVR and the same is true for $\IA_{\inf,(\pi-[\pi^\flat])}$ (see \cref{lem:BdR+DVR} below). Moreover, in the lecture after the next one we will show that all $(\pi-[a])$ for $a\in\mm_F\setminus\{0\}$ are prime ideals, and in fact $\IA_\inf/(\pi-[a])$ is isomorphic to another untilt $\Oo_{C'}$ of $\Oo_F$ (as indicated in the left picture), with $C'/E$ an algebraically closed non-archimedean extension.
\numpar{Side remark}
\lecture[$B_\dR^+$ is a DVR. A universal property for $\IA_\inf$. $p$-adic PD-thickenings and $\IA_\cris$.]{2019-11-06}
Why this setup? Let $K/\IQ_p$ be a discretely valued non-archime-dean field extension with perfect residue field and let $X/K$ be a smooth proper scheme.  The objects of interest in $p$-adic hodge theory are the $p$-adic cohomology groups $H_\et^*(X_{\ov{K}},\IQ_p)$. We will replace $\IQ_p$ by $E$ and $\ov{K}$ by $C=\roof{\ov{K}}$, with $F=C^\flat=\Frac(\Oo_C^\flat)$.
\begin{defi}
	An element $x=\sum_{n=0}^\infty [x_n]\pi^n$ of $\IA_\inf$ is called \defemph{primitive} if $x_0\neq 0$ and there exists a $d\geq 0$ such that $x_d\in \Oo_F^\times$. If $x$ is primitive, the smallest such $d$ is called the \defemph{degree} of $x$. The set primitive elements of degree $d$ is denoted $\Prim_d$.
\end{defi}
\begin{exm}
	We have $\Prim_0=\IA_\inf^\times$. Moreover, any element $x\in \Prim_1$ is distinguished. The converse is true iff $[x_0]\neq 0$.
\end{exm}
Next time we will see that if $a\in \Prim_1$, then $a\IA_\inf$ is  a prime ideal and $\IA_\inf/a\IA_\inf\cong \Oo_C$ for some non-archimedean algebraically closed extension $C/E$ (which generalizes the claim about the $(\pi-[a])$ above). For now let $C/E$ be such an extension and $|\blank|\colon C\morphism \IR_{\geq 0}$ its norm. Recall that \cref{prop:(A/I)b} provides an isomorphism
\begin{equation*}
	\Oo_C^\flat\cong \lim_{x\mapsto x^q}\Oo_C\,,
\end{equation*}
sending an element $x\in \Oo_C^\flat$ of the left-hand side to $(x^\sharp, (x^{1/q})^\sharp,\dotsc)$ contained in the right-hand side.
\begin{lem}\label{lem:OcflatisOF}
	Assume we are in the above situation.
	\begin{numerate}
		\item The map $|\blank|^\flat\colon \Oo_C^\flat\morphism\IR_{\geq 0}$ given by $x\mapsto |x^\sharp|$ is a norm on $\Oo_C^\flat$. Moreover, $\Oo_C^\flat$ is complete with respect to the topology induced by $|\blank|^\flat$.
		\item $C^\flat=\Frac(\Oo_C^\flat)$ is a non-archimedean algebraically closed extension of $\IF_q$.
	\end{numerate}
\end{lem}
\begin{proof}
	It is clear that $|\blank|^\flat$ is multiplicative, that $|1|^\flat=1$, and that $|x|^\flat=0$ iff $x=0$. So only the triangle inequality remains. We calculate
	\begin{align*}
		|x+y|^\flat=|(x+y)^\sharp|&=\lim_{n\to\infty}\bigg|\Big(\big(x^{1/q^n}\big)^\sharp+\big(y^{1/q^n}\big)\Big)^{q^n}\bigg|\\
		&=\lim_{n\to\infty}\max\left\{\big|\big(x^{1/q^n}\big)^\sharp\big|^{q^n},\big|\big(y^{1/q^n}\big)^\sharp\big|^{q^n}\right\}\\
		&=\lim_{n\to\infty}\max\left\{|x^\sharp|,|y^\sharp|\right\}\\
		&=\max\big\{|x|^\flat,|y|^\flat\big\}
	\end{align*}
	This shows that $|\blank|^\flat$ is a norm in $\Oo_C^\flat$. To show that $\Oo_C^\flat$ is complete, we claim that the topology generated by $|\blank|^\flat$ is the inverse limit topology on $\Oo_C^\flat\cong \lim_{x\mapsto x^q}\Oo_C$. A neighbourhood basis of $0$ in the topology generated by $|\blank|^\flat$ is given by the sets
	\begin{equation*}
		\big\{x\ \big|\ |x|^\flat<\epsilon\big\}\quad\text{for all }\epsilon>0\,.
	\end{equation*}
	In the inverse limit topology, a neighbourhood basis of $0$ is given by the sets
	\begin{equation*}
		\left\{x\in \Oo_C^\flat\st \big|\big(x^{1/q^n}\big)^\sharp\big|<\delta\right\}\quad\text{for all }\delta>0\text{, }n\geq 0\,.
	\end{equation*}
	But $|(x^{1/q^n})^\sharp|=(|x|^\flat)^{1/q^n}$, so its easy to see that these topology bases not only generate the same topology, but even coincide on the nose.
	
	
	For \itememph{2}, it remains to show that $C^\flat$ is algebraically closed, and for this it suffices to show that $\Oo_C^\flat$ is integrally closed. So let $f\in \Oo_C^\flat[T]$ be a monic polynomial. Write $f(T)=T^d+a_{d-1}T^{d-1}+\dotsb+a_0$. For all $n\geq 0$ put
	\begin{equation*}
		f_n(T)=T^d+\big(a_{d-1}^{1/q^n}\big)^\sharp T^{d-1}+\dotsb+\big(a_0^{1/q^n}\big)^\sharp\in \Oo_C[T]\,.
	\end{equation*}
	Then $f_{n+1}(T)^q\equiv f_n(T^q)\mod \pi$. Now fix $n\geq 0$ and let $x\in \Oo_C$ be a zero of $f_n$, which exists as $\Oo_C$ is integrally closed. Choose $y\in \Oo_C$ such that $y^q=x$. Although $y$ need not be a root of $f_{n+1}$, we certainly have $|f_{n+1}(y)|\leq |\pi|^{1/q}$. Let $z_1,\dotsc,z_n\in \Oo_C$ be the actual roots of $f_{n+1}$. Then
	\begin{equation*}
		|f_{n+1}(y)|=\prod_{i=1}^d|y-z_i|\leq |\pi|^{1/q}\,.
	\end{equation*}
	hence there exists an index $i$ such that $|y-z_i|\leq |\pi|^{1/dq}$, or equivalently $|y-z_i|^q\leq |\pi|^{1/d}$. Then also $|x-z_i^q|\leq |\pi|^{1/d}$ as all other terms in the expansion of $(y-z_i)^q$ are divisible by $\pi$. By induction, we obtain a sequence $(x_n)_{n\in \IN}$ such that $x_n\in\Oo_C$, $f_n(x_n)=0$, and the $x_n$ are \enquote{close} to being $q$-power compatible in the sense that $|x_{n+1}-x_n^q|\leq |\pi|^{1/d}$. But this is actually sufficient! Indeed, put $\aa=\left\{y\in \Oo_C\st |y|\leq |\pi|^{1/d}\right\}$. Then $x=(x_n)_{n\in\IN}$ is an element of
	\begin{equation*}
		\lim_{x\mapsto x^q}\Oo_C/\aa\cong \lim_{x\mapsto x^q}\Oo_C/\pi\Oo_C=\Oo_C^\flat\,,
	\end{equation*}
	where we use \cref{prop:(A/I)b} to obtain the isomorphism on the left. Hence $x$ corresponds to an element $x\in \Oo_C^\flat$, which clearly satisfies $f(x)=0$.
\end{proof}
\begin{lem}\label{lem:BdR+DVR}
	Let $\Oo_C$ be an untilt of $\Oo_F$ and let $\xi$ be a distinguished generator of the kernel of $\theta\colon \IA_\inf\morphism\Oo_C$. As above, we put $B_\dR^+=\IA_\inf\localize{\pi}_\xi^\complete$. Then the following holds.
	\begin{numerate}
		\item The canonical map $\IA_\inf\monomorphism B_\dR^+$ is an injection.
		\item $B_\dR^+$ and $\IA_{\inf,(\xi)}$ are discrete valuation rings.
	\end{numerate}
\end{lem}
\begin{proof}
	Since we are not in a noetherian setting, we need to be careful with completion. As $(\xi)$ is obviously a finitely generated ideal, \cite[\stackstag{05GG}]{stacks-project} shows that $B_\dR^+$ is $\xi$-complete. Moreover,
	\begin{equation*}
		B_\dR^+/(\xi^n)\cong \IA_\inf\big[\textstyle\frac1\pi\big]/(\xi^n)\cong \IA_\inf/(\xi^n)\big[\textstyle\frac1\pi\big]\,,
	\end{equation*}
	by exactness of localization. We claim that $\IA_\inf/(\xi^n)\monomorphism \IA_\inf/(\xi^n)\localize{\pi}$ is injective for all $n$. To show this, we need to check that $\IA_\inf/(\xi^n)$ is $\pi$-torsionfree. We use induction on $n$. For $n=1$ we get $\IA_\inf/(\xi)\cong \Oo_C$, which is $\pi$-torsionfree. Now suppose $\pi x=\xi^ny$ for some $x,y\in\IA_\inf$. By the $n=1$ case we see that $x$ must be divisible by $\xi$, say, $x=\xi x'$. Since $\IA_\inf$ is a domain this implies $\pi x'=\xi^{n-1}y$. But then the induction hypothesis shows that $x'$ itself must be divisible by $\xi^{n-1}$, proving the claim.
	Now since limits are left exact, we see that
	\begin{align*}
		\IA_\inf\cong \lim_{n\in \IN}\IA_\inf/(\xi^n)\monomorphism \IA_\inf/(\xi^n)\localize{\pi}\cong B_\dR^+
	\end{align*}
	is injective, as required. The isomorphism on the left-hand side uses that $\IA_\inf$ is $\xi$-complete by \cite[\stackstag{09OT}]{stacks-project} and the fact that $\IA_\inf$ is $(\pi,\xi)$-complete. This shows \itememph{1}.
	
	For \itememph{2}, first note that $B_\dR^+/(\xi)\cong \Oo_C\localize{\pi}\cong C$. Hence \cite[\stackstag{05GH}]{stacks-project} implies that $B_\dR^+$ is noetherian. Moreover, we know that $B_\dR^+$ is local with maximal ideal $(\xi)$, because it is $\xi$-adically complete and its quotient by $\xi$ is $C$, which is a field. This implies $\dim B_\dR^+\leq 1$. Moreover, we are done once we show $\dim B_\dR^+\geq 1$, since then $B_\dR^+$ is a one-dimensional noetherian local ring whose maximal ideal is principal, hence regular, hence a DVR.
	
	For $\dim B_\dR^+\geq 1$ it suffices to see that $B_\dR^+$ is a domain, since then $0\subsetneq (\xi)$ is a chain of prime ideals. From \itememph{1} and the fact that $\IA_\inf$ is a domain, it's easy to see that $B_\dR^+$ is $\xi$-torsionfree. Now if $xy=0$ for $x,y\in B_\dR^+$, then $x$ or $y$ must be divisible by $\xi$ as $B_\dR^+/(\xi)\cong C$. Say $x=\xi x'$. Then $B_\dR^+$ being $\xi$-torsionfree shows $x'y=0$. Iterating the argument shows $x=0$ or $y=0$ as $B_\dR^+$ is $\xi$-complete. This finishes the proof that $B_\dR^+$ is indeed a DVR.
	
	Now for $\IA_{\inf,(\xi)}$. Take any prime ideal $\pp\subseteq \IA_{\inf,(\xi)}$ such that $\xi\notin \pp$. Still $\pp\subseteq (\xi)$ as $(\xi)$ is the maximal ideal of $\IA_{\inf,(\xi)}$. Hence, if $a\in \pp$, then $a=b\xi$. But since $\pp$ is prime and $\xi\notin\pp$, this implies $b\in\pp$. Thus $\xi\pp=\pp$. Now let $\qq=\pp B_\dR^+$. Then $\xi\qq=\qq$ shows $\qq=0$ as $B_\dR^+$ is a DVR. But $\IA_{\inf,(\xi)}\monomorphism B_\dR^+$ is injective by \itememph{1} as localizations of injections stay injective. This shows $\pp=0$.
	
	What we have shown is that $\Spec \IA_{\inf,(\xi)}$ has exactly two points, namely $\{0,(\xi)\}$. But then all prime ideals of $\IA_\inf$ are finitely generated, which implies that $\IA_\inf$ is noetherian by the rather obscure fact \cite[\stackstag{05KG}]{stacks-project}. Now it's clear that $\IA_{\inf,(\xi)}$ is one-dimensional and regular, hence a DVR.
\end{proof}

Have you ever wondered what the \enquote{$\inf$} in $\IA_\inf$ actually means? It stands for \emph{infinitesimal}. In fact, this leads to a description of $\IA_\inf$ as a universal thickening of $\Oo_C$!
\begin{defi}
	Let $R$ be a $\pi$-complete $\Oo_E$-algebras. A \defemph{$\pi$-adic pro-infinitesimal thickening of $R$} is a  surjection $D\epimorphism R$ of $\Oo_E$-algebras with kernel $I$ such that $D$ is $(\pi,I)$-adically complete.
\end{defi}
\begin{exm}
	For $R\in\left\{\Oo_C,\Oo_C/\pi\Oo_C\right\}$, the natural map $\IA_\inf\epimorphism R$ is a $\pi$-adic pro-infinitesimal thickening. Indeed, its kernel is given by $(\xi)$ and $(\pi,\xi)$ respectively. Actually, $\IA_\inf$ is the universal $\pi$-adic pro-infinitesimal thickening of $R$, as shown in the following lemma!
\end{exm}
\begin{lem}\label{lem:AinfUniversal}
	Let $R\in\{\Oo_C,\Oo_C/\pi\Oo_C\}$ and let $D\epimorphism R$ be a $\pi$-adic pro-infinitesimal thickening. Then it factors uniquely as
	\begin{equation*}
		\begin{tikzcd}
			\IA_\inf\rar[epi]\dar[dashed,"\exists!"{swap}] & R\\
			D\urar[epi] &
		\end{tikzcd}
	\end{equation*}
\end{lem}
\begin{proof}[Sketch of a proof]
	By \cref{prop:(A/I)b} we have $\lim_{x\mapsto x^q}D\cong (D/(\pi,I))^\flat\cong R^\flat$. By the same argument $\lim_{x\mapsto x^q} D\cong D^\flat$. Hence $D^\flat\cong R^\flat$. Thus, the Witt-tilting adjunction (\cref{prop:tiltWittAdjunction}) provides a unique map 
	\begin{align*}
		\IA_\inf\cong W_{\Oo_E}(R^\flat)\morphism D\,.
	\end{align*}
	It's easily verified that this map has the required properties.
\end{proof}
\subsection{\texorpdfstring{$p$}{p}-adic PD-thickenings and \texorpdfstring{$\IA_\cris$}{Acris}}
From now on, we restrict our attention to the case $E=\IQ_p$ and $\pi=p$. As above, let $R\in\{\Oo_C,\Oo_C/p\Oo_C\}$.
\begin{defi}
	A \defemph{$p$-adic PD-thickening} of $R$ is a triple $(D,D\epimorphism R,(\gamma_n)_{n\in\IN})$, where $D$ is $p$-complete and $(\gamma_n)_{n\in\IN}$ a PD-structure on $J=\ker(D\epimorphism R)$ which is compatible with the canonical PD-structure on $pR$.
\end{defi}
\begin{rem}
	\begin{numerate}
		\item If $D$ is $p$-torsionfree, then necessarily $\gamma_n(x)=x^n/n!$.
		\item Normalize $|\blank|\colon C\morphism \IR_{\geq 0}$ such that $|p|=p^{-1}$. Then a well-known calculation shows $|n!|\geq p^{(n-1)/(p-1)}$ for all $n\in\IN$. In fact, the $1$ in $n-1$ can be replaced by the digit sum of the $p$-adic expansion of $n$. Thus, it's easy to check that
		$|x^n/n!|\leq 1$ for all $n\in\IN$ iff $|x|<p^{-1/(p-1)}$. Moreover, one may check that
		\begin{equation*}
			\left\{x\in\Oo_C\st|x|<p^{-1/(p-1)}\right\}
		\end{equation*}
		is the largest ideal in $\Oo_C$ admitting divided powers.
	\end{numerate}
\end{rem}
\begin{defi}
	The ring $\IA_\cris$ denotes the universal $p$-adic PD-thickening of $\Oo_C$, or equivalently, of $\Oo_C/p\Oo_C$. In fancy words,
	\begin{equation*}
		\IA_\cris=H_\cris^0(\Oo_C/\IZ_p)\cong H_\cris^0\big((\Oo_C/p\Oo_C)/\IZ_p\big)\,.
	\end{equation*}
\end{defi}
Concretely, $\IA_\cris$ is the $p$-adic divided power envelope of $\ker\theta=(\xi)\subseteq \IA_\inf$. This follows more or less from \cref{lem:AinfUniversal}, but this requires an additional argument, since a $p$-adic PD-thickening $D$ of $R$ need not be $(p,J)$-complete, so $D\epimorphism R$ need not be a $p$-adic pro-infinitesimal thickening. But the conclusion of that lemma is still true: we get a unique map $\IA_\inf\morphism D$ over $R$, and then its formal to see that $\IA_\cris$ can be described as above.

So where does the map $\IA_\inf\morphism D$ come from? A closer inspection of the proof of \cref{lem:AinfUniversal} shows that we only need to show that $D^\flat\morphism R^\flat$ is an isomorphism. We can't use \cref{prop:(A/I)b} to prove this. However, we can still construct an inverse $R^\flat\morphism D^\flat$ in the same way as in the proof of that proposition. This is based on the following observation, that serves as a replacement for \cref{lem:keyLemma}.
\begin{lem*}
	If $x,y\in D$ such that $x\equiv y\mod (p,J)$, then $(x^{p^n}-y^{p^n})_{n\in\IN}$ converges to $0$ in the $p$-adic topology.
\end{lem*}
\begin{proof*}
	Observe that for $d\in J$ we have $d^t=t!\gamma_t(d)$, so $d^t$ is divisible by $p^{v_p(t!)}$. Now put $x=y+pz+d$, where $z\in R$ and $d\in J$. Then a typical term in the multinomial expansion of $x^{p^n}-y^{p^n}$ looks like
	\begin{equation*}
		\binom{p^n}{r,s,t}y^r(pz)^sd^t\,,
	\end{equation*}
	where $r+s+t=p^n$. Fix some $N>0$. If $t>p^N$, then the above consideration shows that $d^t$ is at least divisible by $p^N$ (we are very permissive here). If $t\leq p^N$, then the multinomial coefficient is at least divisible by $p^{n-N}$. Hence if $n\geq 2N$, every term will at least be divisible by $p^N$, and we're done.
\end{proof*}
Now that we know $\IA_\cris$ is the $p$-adic divided power envelope of $(\xi)$, we can write it down explicitly as
\begin{equation*}
	\IA_\cris\cong \IA_\inf\big[\textstyle\frac{\xi^n}{n!}\ \big|\  n\in\IN\big]_p^\complete\cong \IA_\inf\cotimes_{\IZ[x]}D_{\IZ[x]}(x)\,,
\end{equation*}
using that $\xi$ is a non-zero divisor in $\IA_\inf$. Also $-\cotimes_{\IZ[x]}-$ refers to the $p$-adic completed tensor product, with $\IZ[x]\morphism\IA_\inf$ sending $x\mapsto \xi$. Finally, the tensor factor on the right is defined as
\begin{equation*}
	D_{\IZ[x]}(x)=\IZ\langle x\rangle =\bigoplus_{n\in\IN}\IZ\big\{\textstyle\frac{x^n}{n!}\big\}\,.
\end{equation*}
Then
\begin{equation*}
	D_{\IZ[x]}(x)_p^\complete\cong\big(\IZ[y_0,y_1,\dotsc]/(y_0-x,y_n^p-py_{n+1} \text{ for }n\in\IN)\big)_p^\complete\,.
\end{equation*}
In particular, we can calculate
\begin{equation*}
	\IA_\cris\cong \Oo_C/p\Oo_C\otimes_{\IF_p}\IF_p[y_1,y_2,\dotsc]/(y_1^p,y_2^p,\dotsc)\,.
\end{equation*}
Some intuition: the image of $\Spec \IA_\cris$ in $\Spec \IA_\inf$ is roughly described by the following picture.
\begin{center}
	\begin{tikzpicture}[line width=rule_thickness, line cap=round, line join =round, x=1cm,y=1cm]
	\draw[-to] (0,0) -- (5,0) node[below] {$[\varpi]$};
	\draw[-to] (0,0) -- (0,5) node[left] {$\pi$};
	\fill (0,0) circle (0.5ex);
	%\fill (4,0) circle (0.5ex);
	\fill (2,2) circle (0.5ex) node[right=4] {$(\xi)$};
	\fill (1.6,2.4) circle (0.4ex) node[above right=4] (1) {$\phi^{-1}(\xi)$};
	\fill (1.28,2.72) circle (0.32ex);
	\fill (1.024,2.976) circle (0.256ex);
	\fill (0.819,3.181) circle (0.205ex);
	\fill (0.655,3.345) circle (0.164ex);
	\fill (0.524,3.476) circle (0.13ex) node[above right=4] (n) {$\phi^{-n}(\xi)$};
	\path (1.south west) -- (n.south west) node[pos=0.5,sloped, above=4] {$\dotsc$};
	\draw[thick, rounded corners=2.5] (-1ex,-1ex) -- (0,-1.414ex) -- (1ex,-1ex) --  (2cm+1ex,2cm-1ex) -- (2cm+1.414ex,2) -- (2cm+1ex,2cm+1ex) -- (1ex,4cm+1ex) -- (0,4cm+1.414ex) -- (-1ex,4cm+1ex) -- cycle;
	\draw[->, shift={(0,0)}] (30:4.5cm) arc (30:50:4.5cm) node[pos=0.5,above right] {$\phi$};
	\path (0,0) -- (2,2) node[pos=0.5,sloped] {$\Spec \IA_\cris$};
	\fill (3,1) circle (0.5ex) node[right=4, align=left] {$(p-[a])$\\$a\in\mm_F\setminus\{0\}$};
	\end{tikzpicture}
\end{center}
Note that $\phi^{-1}(\xi)=(p-[p^\flat]^{1/p})$. Concretely, if $a\in\mm_F\setminus\{0\}$ such that $|a|\leq |p^\flat|^p=|p|^p$, then $(p-[a])\IA_\cris=(p)$. We may think of this as \enquote{$1-[a]/p\in\IA_\cris$}. And if $a=\phi^{-n}(p^\flat)$ for some $n\in \IN$, then $\IA_\inf\epimorphism\IA_\inf/(p-[a])$ factors over $\IA_\cris$.

Recall that $\IA_\inf$ should be thought of as a mixed characteristic analogue of $\Oo_F\llbracket z\rrbracket$. In fact, we see a similar picture for $\Oo_F\llbracket z\rrbracket$.
\begin{center}
	\begin{tikzpicture}[line width=rule_thickness, line cap=round, line join =round, x=1cm,y=1cm]
	\draw[-to] (0,0) -- (5,0) node[below] {$\varpi$};
	\draw[-to] (0,0) -- (0,5) node[left] {$z$};
	\fill (4,0) circle (0.5ex) node[below=4] {$(z)$};
	\fill (2.5,1.5) circle (0.5ex) node[above right=4] {$(z-a)$, $a\in\mm_F$};
	%\fill (20:3.75cm) circle (0.5ex) node[right=4, align=left] {$(p-[a])$\\\scriptsize for $a\in\mm_F\setminus\{0\}$};
	\draw[->, shift={(0,0)}] (35:4.5cm) arc (35:55:4.5cm) node[pos=0.5,above right] {$\phi$};
	\draw[rotate around={135:(4,0)},thick,rounded corners, shift={(4,0)}] (-1ex,-1ex) rectangle (5.25cm+1ex,1ex);
	\end{tikzpicture}
\end{center}
The surrounded area may be described as $\Prim_1/\Oo_F\llbracket z\rrbracket^\times\cong\mm_F=\left\{x\in F\st |x|<1\right\}$. This is also the \enquote{open rigid-analytic disc} $\ID_F$. It contains the \enquote{punctured disc} $\ID_F^*=\mm_F\setminus\{0\}$. Then the equal characteristic analogue of the Fargues--Fontaine curve is the quotient $\ID_F^*/\phi^\IZ$.

However, for $\IA_\inf$ the canonical map $\mm_F\epimorphism\Prim_1/\IA_\inf^\times$ sending $a\in\mm_F$ to $(\pi-[a])$ is not bijective! For example, $(\pi-[\pi^\flat])$ depends on choices of $(q^n)\ordinalth$ roots of $\pi$ to get $\pi^\flat=(\pi,\pi^{1/q},\dotsc)$.