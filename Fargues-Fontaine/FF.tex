\documentclass[a4paper, 10pt, oneside, DIV=9, chapterprefix=true, numbers=enddot,bibliography=totoc]{scrbook}
\usepackage{StyleFF}
\usepackage{ShortcutsFF}

\usetikzlibrary{shapes.geometric,patterns}
\DeclareRobustCommand{\Attention}{\tikz[baseline, anchor=base]\node[draw, regular polygon, regular polygon sides=3, rounded corners=2, thick, inner sep=-0.25pt] at (0,0) {\textbf{!}};}


\subject{Lecture Notes for}
\title{The Fargues--Fontaine Curve}
\subtitle{Or: \enquote{The Fundamental Curve of $p$-adic Hodge Theory}}
\author{{\normalsize Lecturer}\\
	Johannes Anschütz}
\date{{\normalsize Notes typed by}\\
	Ferdinand Wagner}
\publishers{Winter Term 2019/20\\
University of Bonn}

%\includeonly{nothingtoseehere}
\begin{document}
\frontmatter
\KOMAoption{chapterprefix}{false}
\maketitle
\noindent This text consists of notes on the lecture Selected Topics in Algebra (The Fargues--Fontaine Curve), taught at the University of
Bonn by Dr.\ Johannes Anschütz in the winter term (Wintersemester) 2019/20.

Some changes and some additions have been made by the author. To distinguish them from the lecture's actual contents, they are labelled with an asterisk. So any \emph{Lemma}* or \emph{Remark}* or \emph{Proof}* that the reader might encounter are wholly the author's responsibility.\\[\thmsep]Please report errors, typos etc.\ through the \emph{Issues} feature of GitHub.


\tableofcontents
\listoftoc{lol}
\setcounter{llecture}{-1}
\chapter{Introduction and Motivation}
\renewcommand{\thedummy}{\thechapter.\thesection.\arabic{dummy}}
\lecture[What is this \enquote{$p$-adic Hodge theory} and what does it have to do with this lecture?]{2019-10-16}
The $0\ordinalth$ lecture had a lot of hard theorems and deep facts thrown at us---for purely motivational purposes! That is, none of the following is a prerequisite for this lecture; rather it shows where we're going, and parts of it will be discussed in detail.

Fix a prime $p$ and a finite extension $K/\IQ_p$. Let $C$ be the completion of an algebraic closure $\ov{K}$ of $K$. We put $G_K=\Gal(\ov{K}/K)$. Note that the $G_K$-action on $\ov{K}$ can be continuously extended to $C$.
\begin{thm}[Faltings, Tsuji$,\dotsc$]\label{thm:HTdecomp}
	Let $X/K$ be a proper smooth scheme. For $n\geq 0$ there exists a natural $G_K$-equivariant \enquote{Hodge--Tate decomposition}
	\begin{equation*}
		H_\et^n\big(X_{\ov{K}},\IQ_p\big)\otimes_{\IQ_p}C\cong \bigoplus_{i+j=n}H^i\big(X,\Omega_{X/K}^j\big)\otimes_KC(-j)
	\end{equation*}
\end{thm}
\begin{rem}\label{rem:HTdecomp}
	There are a \emph{lot} of things in \cref{thm:HTdecomp} that demand clarification.
	\begin{numerate}
		\item $H_\et^n(X_{\ov{K}},\IQ_p)$ is the $p$-adic étale cohomology, defined as
		\begin{equation*}
			H_\et^n\big(X_{\ov{K}},\IQ_p\big)\coloneqq \Big(\lim_{k\geq 0} H_{\et}^n\big(X_{\ov{K}},\IZ/p^k\IZ\big)\Big)\otimes_{\IZ_p}\IQ_p\,.
		\end{equation*}
		\item $G_K$ acts diagonally on the left-hand side and via $C(-j)$ on the right-hand side. Here, $M(-j)$ is a \defemph{Tate twist}. In general this is defined as $M(j)\coloneqq M\otimes_{\IZ_p}\IZ_p(1)^{\otimes j}$, where
		\begin{equation*}
			\IZ_p(1)=\lim_{k\geq 0}\mu_{p^k}(C)\,,
		\end{equation*}
		equipped with its natural $G_K$-action.
		\item \cref{thm:HTdecomp} got its name from the analogous assertion in complex Hodge theory: If $Y$ is a compact Kähler manifold, then
		\begin{equation*}
			H^n(Y,\IZ)\otimes_{\IZ}\IC\cong\bigoplus_{i+j=n}H^i\big(Y,\Omega_{Y/\IC}^j\big)\,.
		\end{equation*}
		\item The Tate twists are necessary to get $G_K$-invariance of the decomposition. To see this, take for example $X=\IP_K^1$, $n=2$. As $\IG_{m,\ov{K}}$ is $\IP_{\ov{K}}^1\setminus\{\text{two points}\}$, the left-hand side can be calculated as
		\begin{align*}
			H_\et^2\big(\IP_{\ov{K}}^1,\IQ_p\big)\cong H_\et^1\big(\IG_{m,\ov{K}},\IQ_p\big)&\cong \Hom\big(\pi_1^\et(\IG_{m,\ov{K}}),\IQ_p\big)\\
			&\cong \Hom\big(\IZ_p(1),\IQ_p\big)\\
			&\cong \IQ_p(-1)
		\end{align*}
		On the right-hand side, the only non-vanishing summand is $H^1(X,\Omega_{X/K}^1)\cong K$. So far, everything is ok as both sides in \cref{thm:HTdecomp} are one-dimensional $C$-vector spaces. However, there can't be an $G_K$-equivariant isomorphism $C(-1)\cong C$, as can be seen from the following theorem. 
	\end{numerate}
\end{rem}
\begin{thm}[Tate]
	Let $H_\cts^*(G_K,-)$ denote continous group cohomology/Galois cohomology. With notation as above, we have
	\begin{numerate}
		\item $H_\cts^*(G_K,C(j))=0$ for all $j\neq 0$.
		\item $K\cong H_\cts^0(G_K,C)\cong H_\cts^1(G_K,C)$. In particular, $K\cong C^{G_K}$ (and not even this is trivial to prove).
	\end{numerate}
\end{thm}
\begin{cor}\label{cor:etKnowsHodge}
	For all $n\geq 0$ and $j\geq 0$ we have
	\begin{equation*}
		H^{n-j}\big(X,\Omega_{X/K}^j\big)\cong \Big(H_\et^n\big(X_{\ov{K}},\IQ_p\big)\otimes_{\IQ_p}C(j)\Big)^{\smash{G_K}}\,.
	\end{equation*}
\end{cor}
\begin{cntx}
	As a slogan, \cref{cor:etKnowsHodge} shows that \enquote{$p$-adic étale cohomology knows Hodge cohomology}. The converse, however, is not true, and in fact, it fails almost always. Here are two counterexamples.
	\begin{numerate}
		\item If $X$ is an elliptic curve over $K$, then the Hodge--Tate decomposition shows
		\begin{equation*}
			H_\et^1\big(X_{\ov{K}},\IQ_p\big)\cong C\oplus C(-1)\,,
		\end{equation*}
		independent of $X$. However, the $G_K$-action on $H_\et^1\big(X_{\ov{K}},\IQ_p\big)$ knows if $X$ has good or semistable reduction. So this is not seen by Hodge cohomology.
		\item If $X=\Spec L$, where $L/K$ is finite, then
		\begin{equation*}
			H_\et^0\big(X_{\ov{K}},\IQ_p\big)\cong \prod_{L\monomorphism\ov{K}}\IQ_p\,,
		\end{equation*}
		on which $G_K$ acts by permuting the factors. This action determines $X$. However, $H_\et^0\big(X_{\ov{K}},\IQ_p\big)\otimes_{\IQ_p}\cong C^{[L:K]}$ only knows $[L:K]$ and not $L$.
	\end{numerate}
\end{cntx}
A nice application of \cref{thm:HTdecomp} and \cref{cor:etKnowsHodge} is the following theorem.
\begin{thm}[Ito, Veys, Kontsevich$,\dotsc$]\label{thm:MinimalModels}
	Let $Y$, $Y'$ be smooth minimal models (i.e., smooth projective schemes over $\IC$ with nef canonical bundle). If $Y$, $Y'$ are birational, then
	\begin{equation*}
		\dim_\IC H^i\big(Y,\Omega_{Y/\IC}^j\big)=\dim_\IC H^i\big(Y',\Omega_{Y'/\IC}^j\big)\quad\text{for all }i,j\geq 0\,.
	\end{equation*}
\end{thm}
\begin{proof}[Idea of the proof]
	It's well-known that if $Y$, $Y'$ are birational and smooth minimal models, then they are \defemph{$K$-equivalent}. That is, there exists a diagram
	\begin{equation}\label{diag:K-eq1}
		\begin{tikzcd}
			 & Z\dlar["f",swap]\drar["g"] & \\
			 Y & & Y'
		\end{tikzcd}
	\end{equation}
	such that $Z$ is proper and smooth over $\IC$, the morphisms $f$ and $g$ are proper and birational, and $f^*K_Y\cong g^*K_{Y'}$ holds for the respective canonical bundles (or rather canonical divisors in this notation).
	
	Now we \defemph{spread out} over some finitely generated $\IZ$-algebra $A\subseteq \IC$. This means the following: all data---the schemes $Y$, $Y'$, $Z$ together with the morphisms $f$ and $g$---can be described by finitely many polynomials. Taking $A=\IZ[\{\text{all their finitely many coefficients}\}]$ we see that all these polynomials are already defined over $A$. Hence also the corresponding schemes are already defined over $A$. To make this precise: there is a diagram
	\begin{equation}\label{diag:K-eq2}
		\begin{tikzcd}
			& \Zz\dlar["\snake{f}",swap]\drar["\snake{g}"] & \\
			\Yy & & \Yy'
		\end{tikzcd}
	\end{equation}
	of schemes over $A$, such that \cref{diag:K-eq1} is the base-change of \cref{diag:K-eq2} along $\Spec \IC\morphism\Spec A$. Since Hodge numbers are constant for proper smooth morphisms in characteristic $0$, we can replace $A$ by some suitable localization. Hence we may assume $A=\Oo_F[N^{-1}]$ for some number field $F/\IQ$. By a $p$-adic integration black box we have $\Yy(\IF_{\ell^k})=\Yy'(\IF_{\ell^k})$ for all primes $\ell$ such that $(\ell,N)=1$ and all $k\geq 1$. Fix a prime $p$. If $(p,N)=1$, then
	\begin{equation*}
		H_\et^*\big(\Yy_{\ov{\Ff}_\ell},\IQ_p\big)^{\mathrm{ss}}\cong
		H_\et^*\big(\Yy'_{\ov{\Ff}_\ell},\IQ_p\big)^{\mathrm{ss}}
	\end{equation*}
	are isomorphic as Galois representations for all primes $\ell$ such that $(\ell,pN)=1$. This is somehow implied by the Weil conjectures. Also $(-)^{\mathrm{ss}}$ denotes semisimplification. By Chebotarev's density theorem we thus obtain
	\begin{equation*}
		H_\et^*\big(\Yy_{\ov{F}},\IQ_p\big)^{\mathrm{ss}}\cong
		H_\et^*\big(\Yy'_{\ov{F}},\IQ_p\big)^{\mathrm{ss}}\,.
	\end{equation*}
	Now pick a prime ideal $\pp\mid p$ in $\Oo_F$ and put $K=F_\pp$. Then also
	\begin{equation*}
		H_\et^*\big(\Yy_{\ov{K}},\IQ_p\big)^{\mathrm{ss}}\cong
		H_\et^*\big(\Yy'_{\ov{K}},\IQ_p\big)^{\mathrm{ss}}\,.
	\end{equation*}
	Finally, the Hodge decomposition from \cref{thm:HTdecomp} together with \cref{cor:etKnowsHodge} and a \enquote{small argument $\epsilon$} (to get rid of the semisimplifications) implies
	\begin{align*}
		\dim_KH^i\big(\Yy_K,\Omega_{\Yy_K/K}^j\big)\cong \dim_KH^i\big(\Yy'_K,\Omega_{\Yy'_K/K}^j\big)\quad\text{for all }i,j\geq 0\,.
	\end{align*}
	Base-changing (in a zig-zag) back to $\IC$ finally proves the assertion.
\end{proof}
Another nice application is the degeneration of the \emph{Hodge--de Rham spectral sequence}. Let $Y/k$ be a proper smooth scheme over a field $k$. The \defemph{de Rham cohomology} of $Y$ is defined as the (hyper-)cohomology of the de Rham complex $\Omega_{Y/k}^\bullet$,
\begin{equation*}
	H_\dR^n(Y/k)=H^n\left(0\morphism \Oo_Y\morphism[\d]\Omega_{Y/k}^1\morphism[\d]\Omega_{Y/k}^2\morphism[\d]\dotso\right)\,.
\end{equation*}
Then, more or less by definition, there is a spectral sequence
\begin{equation*}
	E_1^{i,j}=H^j\big(Y,\Omega_{Y/k}^i\big)\converge H_\dR^{i+j}(Y/k)\,,
\end{equation*}
called \defemph{Hodge--de Rham spectral sequence}. This sequence is degenerate, which can be proved by similar methods as \cref{thm:MinimalModels}.
\begin{qst}
	Again, one can ask whether in our original situation $H_\et^n\big(X_{\ov{K}},\IQ_p\big)$ \enquote{knows} $H_\dR^n(X/K)$ including its Hodge filtration? This question is in part answered by the following theorem.
\end{qst}
\begin{thm}[Faltings, Tsuji$,\dotsc$]\label{thm:deRhamComp}
	For $n\geq 0$ there exists a natural $G_K$-equivariant filtered \enquote{de Rham comparison} isomorphism
	\begin{equation*}
		H_\et^n\big(X_{\ov{K}},\IQ_p\big)\otimes_{\IQ_p}B_\dR\cong H_\dR^n(X/K)\otimes_KB_\dR\,.
	\end{equation*}
\end{thm}
\begin{rem}\label{rem:deRhamComp}
	Again, a lot of clarifications need to be done.
	\begin{numerate}
		\item $B_\dR$ is Fontaine's field of \emph{$p$-adic periods} and comes with a $G_K$-action. It is the fraction field of some complete DVR $B_\dR^+$ with residue field $C$ (thus, abstractly, $B_\dR^+\cong C\llbracket t\rrbracket$, but this isomorphism is \emph{not} $G_K$-equivariant). We have a natural filtration $\Fil^jB_\dR=\xi^jB_\dR^+$, where $\xi\in B_\dR^+$ is a uniformizer. The associated graded object is
		\begin{equation*}
			B_\HT\coloneqq \gr B_\dR=\bigoplus_{j\in\IZ}C(j)\,.
		\end{equation*}
		Thus, the de Rham comparison (\cref{thm:deRhamComp}) implies the Hodge--Tate decomposition (\cref{thm:HTdecomp}).
		\item The $G_K$-action is diagonally on the left-hand side and via $B_\dR$ on the right-hand side. Conversely, the filtration on the right-hand side is diagonally, whereas on the left-hand side it comes from $B_\dR$.
		
		\item If $X=\IP_K^1$ and $n=2$, we obtain $\IQ_p(-1)\otimes_{\IQ_p}B_\dR\cong B_\dR$ (we use the calculations from \cref{rem:HTdecomp}\itememph{1}). Hence there exists a canonical $G_K$-stable line $\IQ_pt\subseteq B_\dR$ such that $G_K$ acts via a cyclotomic character $\chi_\cycl\colon G_K\morphism\IZ_p^\times$ (i.e.\ $\IQ_pt\cong \IQ_p(1)$). 
		
		For some $\epsilon\in \IZ_p(1)\setminus\{0\}$ we thus get $t=\log{}[\epsilon]\in B_\dR$. Such an element is also called \enquote{Fontaine's $2\pi\mathrm{i}$}.
	\end{numerate}
\end{rem}
From now on, we will talk about stuff that will be the actual contents of the lecture. Assume that, additionally to the usual assumptions, $X$ has \defemph{good reduction}. That is, $X=\XX_K$ for some smooth proper $\XX\morphism\Spec \Oo_K$. Let $\XX_0$ be the special fibre. Then we get refinement of the de Rham comparison theorem (\cref{thm:deRhamComp}):
\begin{thm}[Faltings, Niziol, Tsuji]\label{thm:crystallineStuff}
	For $n\geq 0$ there exists a natural $G_K$-equivariant filtered $\phi$-equivariant isomorphism
	\begin{equation*}
		H_\et^n\big(X_{\ov{K}},\IQ_p\big)\otimes_{\IQ_p}B_\cris\cong H_\cris^n\big(\XX_0/\Oo_{K_0}\big)\otimes_{\Oo_{K_0}}B_\cris\,.
	\end{equation*}
\end{thm}
\begin{rem}
	As usual, we should explain a lot of notation.
	\begin{numerate}
		\item Here, $K_0\subseteq K$ is the maximal subextension that is unramified over $\IQ_p$ (so $p$ is a uniformizer of $\Oo_{K_0}$). There exists a (unique) Frobenius lift $\phi$, which acts on $\Oo_{K_0}$.
		\item $H_\cris^n(\XX_0/\Oo_{K_0})$ is the \emph{crystalline cohomology} of $\XX_0$ over $\Oo_{K_0}$. Roughly, this is the \enquote{de Rham cohomology of a smooth lift}. It has the Frobenius $\phi$ acting on it. Moreover,
		\begin{equation*}
			\Big(H_\cris^n\big(\XX_0/\Oo_{K_0}\big)\big[\textstyle\frac 1p\big], \phi, \Fil^\bullet\Big)
		\end{equation*}
		is a \defemph{filtered $\phi$-module} (or \defemph{Frobenius isocrystal}), that is, a finite-dimensional $K_0$-vector space $D$, with an automorphism $\phi_D\colon D\morphism D$ that satisfies $\phi_D(\lambda d)=\phi(\lambda)\phi_D(d)$ for all $\lambda\in K_0$, $d\in D$ (this is called \defemph{$\phi$-semilinear}), and a filtration $\Fil^\bullet(D_K)$ (coming from the Hodge filtration) on $D_K\coloneqq D\otimes_{K_0}K$.
		\item $B_\cris$ is Fontaine's ring of \defemph{crystalline $p$-adic periods}. It is constructed as follows. Let
		\begin{equation*}
			\IA_\cris\coloneqq H_\cris^0\big((\Oo_C/p\Oo_C)/\IZ_p\big)\,,
		\end{equation*}
		with a Frobenius action $\phi$ on it. Put $B_\cris^+\coloneqq \IA_\cris\big[\frac1p\big]$. Then $B_\cris^+$ is actually a $G_K$-stable subring of $B_\dR^+$, and it contains $t=\log{}[\epsilon]$ from \cref{rem:deRhamComp}\itememph{3}. Then we can finally define $B_\cris=B_\cris^+\big[\frac1t\big]$. Also note that $\phi(t)=pt$.
		
		One cool feature of the Fargues--Fontaine curve is that all these strange period rings appear as rings of functions on it!
		\item \cref{thm:crystallineStuff} is analogous to the following statement in $\ell$-adic cohomology (where $\ell\neq p$ is a prime). Let $\XX\morphism\Spec \Oo_K$ be smooth proper, and $s,\eta\in\Spec \Oo_K$ the special resp.\ the generic point. Then there exists a $G_K$-equivariant isomorphism
		\begin{equation*}
			H_\et^*(\XX_{\ov{\eta}},\IQ_\ell)\cong H_\et^*(\XX_{\ov{s}},\IQ_\ell)\,.
		\end{equation*}
		In particular, $H_\et^*(\XX_{\ov{\eta}},\IQ_\ell)$ is unramified.
		\item By Grothendieck's philosophy of \enquote{motives} we should expect that $H_\et^n(X_{\ov{K}},\IQ_p)$ and $H_\cris^n(\XX_0/\Oo_{K_0})\big[\frac1p\big]$ contain the \enquote{same information}. Which raises the question, how to pass from $G_K$ representations on finite-dimensional $\IQ_p$-vector spaces (that's what étale cohomology is) to $K_0$-vector spaces with Frobenius and a filtration over $K$ (that's the crystalline side of things)? This question became famously known as \enquote{Grothendieck's question about the \emph{mysterious functor}}, and was eventually resolved by Fontaine as follows: there are functors
		\begin{equation*}
			D_\cris\colon \Rep_{\IQ_p}G_K \doublelrmorphism \left\{\text{filtered }\phi\text{-modules}\right\}\noloc V_\cris
		\end{equation*} 
		given by $D_\cris(V)=(V\otimes_{\IQ_p}B_\cris)^{G_K}$ and $V_\cris(D)=\Fil^0(D\otimes_{K_0}B_\cris)^{\phi=1}$. They satisfy the following theorem, which will be the main goal of the lecture.
	\end{numerate}
\end{rem}
\begin{thm}[Colmez/Fontaine]\label{thm:ColmerzFontaine}
	\enquote{Weakly admissible implies admissible}. That is, $D_\cris$ and $V_\cris$ restrict to equivalences
	\begin{equation*}
		D_\cris\colon\left\{
		\begin{tabular}{c}
			crystalline $G_K$-\\
			representations
		\end{tabular}\right\}\lrisomorphism \left\{
		\begin{tabular}{c}
			weakly admissible\\
			filtered $\phi$-modules
		\end{tabular}
		\right\}\noloc V_\cris\,.
	\end{equation*}
\end{thm}
\begin{rem}
	\begin{numerate}
		\item $V\in \Rep_{\IQ_p}G_K$ is called \defemph{crystalline} if $\dim_{K_0}D_\cris(V)=\dim_{\IQ_p}V$.
		\item Being \defemph{weakly admissible} has something to do with \enquote{the Newton polygon lying above the Hodge polygon}.
		\item The essential ingredient in the proof of \cref{thm:ColmerzFontaine} will be the \defemph{Fargues--Fontaine curve} (duh!), together with the relation between its $G_K$-invariant vector bundles and $\Rep_{\IQ_p}G_K$ resp.\ $\left\{\text{filtered }\phi\text{-modules}\right\}$. We can already define it as
		\begin{equation*}
			X_\FFC\coloneqq \Proj\Bigg(\bigoplus_{d\geq 0}(B_\cris^+)^{\phi=p^d}\Bigg)\,.
		\end{equation*}
		We will see that this is a Dedekind scheme over $\IQ_p$, and the completions of the local rings at its closed points are $B_\dR^+$.
	\end{numerate}
\end{rem}



\mainmatter\KOMAoption{chapterprefix}{true}
\renewcommand{\thedummy}{\thesection.\arabic{dummy}}

\chapter{Construction of the Fargues--Fontaine Curve}
\section{Ramified Witt Vectors}
\lecture[The abstract of this lecture is left as an exercise.]{2019-10-23}
Let $p$ be a prime, $E/\IQ_p$ a finite extension with ring of integers $\Oo_E$. We fix a choice of uniformizer $\pi$ and let $\IF_q=\Oo_E/\pi\Oo_E$ be the residue field of $\Oo_E$, where $q=p^f$. The goal for today is to prove
\begin{prop}\label{prop:FqAlgebrasEquivalence}
	There is an equivalence of categories
	\begin{align*}
		\left\{\begin{tabular}{c}
			$\pi$-torsionfree $\pi$-adically complete $\Oo_E$-alge-\\
			bras $A$ with perfect residue ring $A/\pi A$
		\end{tabular}
		\right\}&\isomorphism\left\{\text{perfect $\IF_q$-algebras}\right\}\\
		A&\longmapsto R=A/\pi A\,.
	\end{align*}
\end{prop}
For the proof, we will construct an inverse functor $R\mapsto W_{\Oo_E}(R)$ that somehow \enquote{reconstructs} $A$ from $A/\pi A$.
\begin{rem}
	The most important case is the unramified one, i.e., $E=\IQ_p$, in which case we obtain an equivalence
	\begin{align*}
	\left\{\begin{tabular}{c}
	$p$-torsionfree $p$-adically complete rings\\
	$A$ with perfect residue ring $A/pA$
	\end{tabular}
	\right\}&\isomorphism\left\{\text{perfect $\IF_p$-algebras}\right\}\\
	A&\longmapsto R=A/p A\,.
	\end{align*}
	We will see (in \cref{cor:unramifiedWitt}) that the general case can be reduced to this one. Also we put $W\coloneqq W_{\IZ_p}$ for brevity.
\end{rem}
\numpar*{Example}
We will see $W(\IF_p)=\IZ_p$ and $W(\IF_q)=\Oo_{E_0}$ where $E_0$ is the maximal unramified subextension of $E/\IQ_p$ (i.e., the unique unramified extension with residue field $\IF_q$). Moreover, we will see
	\begin{equation*}
		W\big(\IF_p\big\llbracket T^{1/p^\infty}\big\rrbracket\big)=\IZ_p\big\llbracket T^{1/p^\infty}\big\rrbracket\,.
	\end{equation*}
	
\subsection{The construction of \texorpdfstring{$W_{\Oo_E}$}{W}}
\begin{lem}\label{lem:LTE}
	Let $A$ be any $\Oo_E$-algebra and $x,y\in A$ such that $x\equiv y\mod \pi$. Then
	\begin{equation*}
		x^{q^k}\equiv y^{q^k}\mod \pi^{k+1}\quad\text{for all }k\geq 0\,.
	\end{equation*}
\end{lem}
\begin{proof}
	By induction on $k$, this boils down to the following question: if $x\equiv y\mod \pi^k$, show $x^q\equiv y^q\mod \pi^{k+1}$. To see this, write $x=y+\pi^ka$ for some $a\in A$. As all binomial coefficients $\binom{q}{i}$ except for $i=0,q$ are divisible by $p$, we obtain 
	\begin{equation*}
		x^q=(y+\pi^ka)^q=y^q+p\pi^k(\ldots)+\pi^{kq}a^q\,.
	\end{equation*}
	As $\pi\mid p$, the assertions follows.
\end{proof}
\begin{deflem}\label{deflem:Teichmuller}
	Let $A$ be a $p$-adically complete $\Oo_E$-algebra with $R=A/\pi A$ perfect. Let $a\in R$. Choose any sequence of lifts $\alpha_n\in A$ of $a^{1/q^n}\in R$. Then the sequence $(\alpha_n^{q^n})_{n\in \IN}$ converges in $A$ to a lift of $a$, which is independent of the choices of $\alpha_n$. The map
	\begin{align*}
		[-]\colon R&\morphism A\\
		a&\longmapsto [a]\coloneqq\lim_{n\to\infty}\alpha_n^{q^n}
	\end{align*}
	is well-defined and called the \defemph{Teichmüller representative}. It defines a natural multiplicative section of $A\epimorphism R$.
\end{deflem}
\begin{proof}
	We have $\alpha_{n+1}^q\equiv \alpha_n\mod \pi$, hence
	\begin{equation*}
		\alpha_{n+1}^{q^{n+1}}\equiv \alpha_n^{q^n}\mod \pi^{n+1}
	\end{equation*}
	by \cref{lem:LTE}. This shows convergence of the sequence in question. To show that it doesn't depend on the choice of lifts can be seen by a similar argument. Now if $a,b\in R$ are given together with a choice of lifts $\alpha_n$ and $\beta_n$, we can choose $\alpha_n\beta_n$ as lifts of $(ab)^1/q^n$, since the choice of lifts doesn't matter. From this argument, multiplicativity is clear. Naturality is similar.
\end{proof}
\begin{lem}\label{lem:TeichmullerRep}
	In our usual situation, every $x\in A$ admits a unique representation
	\begin{equation*}
		x=\sum_{n=0}^{\infty}[x_n]\pi^n\quad\text{for some }x_n\in R\,.
	\end{equation*}
\end{lem}
\begin{proof}
	Let $x_0\in R$ be the reduction of $x$. Then $x\equiv[x_0]\mod \pi$, so $x-[x_0]=\pi y_1$ for some $y_1\in A$, which is unique as $A$ is $\pi$-torsionfree. Now let $x_1\in R$ be the reduction of $y_1$. Similar as above, write $y_1=[x_1]+\pi y_2$. Now repeat this process to get a representation of the desired type. Uniqueness can be shown along the lines of the construction.
\end{proof}
\begin{urem}
	We can think of $A\morphism R$ in a similar way as we think about $R\llbracket T\rrbracket\morphism R$ with its canonical section $R\morphism R\llbracket T\rrbracket$ given by $a\mapsto a$. Since in our situation $A$ has characteristic $0$ but $R$ has characteristic $p$, there is no way $[-]\colon R\morphism A$ can be additive. So it being multiplicative is really the best we could hope for.
\end{urem}
At this point, \cref{lem:TeichmullerRep} allows us to recover $A$ as a \emph{set} from $R=A/\pi A$. But what about the ring structure? Let's try! Say we have sequences $(x_n),(y_n)\in R^{\IN}$ and we want to find the unique sequence $(s_n)\in R^{\IN}$ such that
\begin{equation*}
	\sum_{n=0}^{\infty}[x_n]\pi^n+\sum_{n=0}^{\infty}[y_n]\pi^n=\sum_{n=0}^\infty [s_n]\pi^n\,.
\end{equation*}
One could naively assume that $s_n$ is just $x_n+y_n$. Spoiler: \emph{it's not}. For $n=0$, we calculate modulo $\pi$. We should have $[x_0]+[y_0]=[s_0]$, hence $s_0=x_0+y_0$. That was easy! Now for $n=1$. We calculate modulo $\pi^2$:
\begin{equation*}
	[x_0]+[x_1]\pi+[y_0]+[y_1]\pi\equiv [s_0]+[s_1]\pi\equiv [x_0+y_0]+[s_1]\pi\mod \pi^2\,.
\end{equation*}
Hence we want to put
\begin{equation*}
	``s_1=x_1+y_1+\frac{[x_0]+[y_0]-[x_0+y_0]}{\pi}\text{''}\,,
\end{equation*}
except it's not clear at all how to define this formally. Here we use a trick: since $R$ is perfect and $[-]$ is multiplicative, we have
\begin{align*}
	[x_0]+[y_0]-[x_0+y_0]=\big[x_0^{1/q}\big]^q+\big[y_0^{1/q}\big]^q-\big[x_0^{1/q}+y_0^{1/q}\big]^q\,.
\end{align*}
Since $\big[x_0^{1/q}\big]+\big[y_0^{1/q}\big]\equiv\big[x_0^{1/q}+y_0^{1/q}\big]\mod \pi$, \cref{lem:LTE} shows
\begin{equation*}
	\big[x_0^{1/q}\big]^q+\big[y_0^{1/q}\big]^q\equiv\big[x_0^{1/q}+y_0^{1/q}\big]^q\mod \pi^2\,.
\end{equation*}
Hence we can choose
\begin{equation*}
	s_1=x_1+y_1-\sum_{i=1}^{q-1}\frac{1}{\pi}\binom{q}{i}\big[x_0^{1/q}\big]^i\big[y_0^{1/q}\big]^{q-i}\,,
\end{equation*}
where the $\pi^{-1}\binom{q}{i}$ are considered as elements of $\Oo_E$. In the very unpleasant Germany of 1936, the mathematician and SA member Ernst Witt understood this pattern and extended it to higher $n$ as follows.
\begin{defi}
	For $n\geq 0$, define the \emph{$n\ordinalth$ ghost component} as
	\begin{equation*}
		W_n(X_0,\dotsc,X_n)=\sum_{i=0}^{n}X_i^{q^{n-i}}\pi^i\in\Oo_E[X_0,\dotsc,X_n]\,.
	\end{equation*}
\end{defi}
\begin{urem}
The idea behind the $W_n$ is that 
\begin{equation*}
	\sum_{i=0}^n[a_i]\pi^i=W_n\Big(\big[a_0^{1/q^n}\big],\dotsc,\big[a_n^{1/q^0}\big]\Big)\,.
\end{equation*}
\end{urem}
\begin{prop}\label{prop:WittPolynomials}
	There are unique sequences of polynomials $(S_n)_{n\in \IN}$, $(P_n)_{n\in\IN}$ in the polynomial ring $\Oo_E[X_0,\dotsc,X_n,Y_0,\dotsc,Y_n]$, such that
	\begin{align*}
		W_n(X_0,\dotsc,X_n)+W_n(Y_0,\dotsc,Y_n)&=W_n(S_0,\dotsc,S_n)\\
		W_n(X_0,\dotsc,X_n)\cdot W_n(Y_0,\dotsc,Y_n)&=W_n(P_0,\dotsc,P_n)\,.
	\end{align*}
\end{prop}
\begin{proof}%nitl
	We show more generally that for any polynomial $\Phi\in\Oo_E[X,Y]$ there is a unique sequence $(\Phi)_{n\in\IN}$ of polynomials $\Phi_n\in\Oo_E[X_0,\dotsc,X_n,Y_0,\dotsc,Y_n]$ such that
	\begin{equation*}
		\Phi\big(W_n(X_0,\dotsc,X_n),W_n(Y_0,\dotsc,Y_n)\big)=W_n(\Phi_0,\dotsc,\Phi_n)\,.
	\end{equation*}
	We show this via induction on $n$. For $n=0$ we have to take $\Phi_0(X_0,Y_0)=\Phi(X_0,Y_0)$. Now suppose $\Phi_0,\dotsc,\Phi_n$ are already constructed. We need to check that
	\begin{equation}\label{eq:Witt1}
		\Phi\big(W_{n+1}(X_0,\dotsc,X_{n+1}),W_{n+1}(Y_0,\dotsc,Y_{n+1})\big)-W_{n+1}(\Phi_0,\dotsc,\Phi_n,0)
	\end{equation}
	is a polynomial divisible by $\pi^{n+1}$; for then $\pi^{-(n+1)}\cdot(\text{this polynomial})$ is the unique choice for $\Phi_{n+1}$. Note that 
	\begin{equation}\label{eq:Witt2}
		W_{n+1}(X_0,\dotsc,X_{n+1})\equiv W_n\left(X_0^q,\dotsc,X_n^q\right)\mod\pi^{n+1}\,.
	\end{equation}
	Using \cref{eq:Witt1} together with the induction hypothesis, we obtain
	\begin{align*}
		\Phi\big(W_{n+1}(X_0,\dotsc,X_{n+1}),W_{n+1}(Y_0,\dotsc,Y_{n+1})\big)&\equiv \Phi\big(W_n(X_0^q,\dotsc,X_n^q),W_n(Y_0^q,\dotsc,Y_n^q)\big)\\
		&\equiv W_n\big(\Phi_0^{(q)},\dotsc,\Phi_n^{(q)}\big)\mod \pi^{n+1}\,,
	\end{align*}
	where $\Phi_i^{(q)}$ is the polynomial obtained from $\Phi_i$ by replacing every variable by its $q\ordinalth$ power. Note that $\Phi_i^{(q)}\equiv \Phi_i^q\mod \pi$. Thus, using \cref{lem:LTE} we get
	\begin{equation*}
		\pi^i\big(\Phi_i^{(q)}\big)^{q^{n-i}}\equiv \pi^i\Phi_i^{q^{n+1-i}}\mod \pi^{n+1}\,.
	\end{equation*}
	But this shows $W_n\big(\Phi_0^{(q)},\dotsc,\Phi_n^{(q)}\big)\equiv W_{n+1}(\Phi_0,\dotsc,\Phi_n,0)\mod \pi^{n+1}$. Now putting everything together shows that the polynomial in \cref{eq:Witt1} is indeed divisible by $\pi^{n+1}$, as required.
\end{proof}
\begin{cor}\label{cor:snpn}
	Let $(x_n)_{n\in\IN}$ and $(y_n)_{n\in\IN}$ be sequences in $R^\IN$, where $R=A/\pi A$. For all $n\geq 0$ put
	\begin{align*}
		s_n&=S_n\left(x_0^{1/q^n},\dotsc,x_n^{1/q^0},y_0^{1/q^n},\dotsc,y_n^{1/q^0}\right)\\
		p_n&=P_n\left(x_0^{1/q^n},\dotsc,x_n^{1/q^0},y_0^{1/q^n},\dotsc,y_n^{1/q^0}\right)\,. 
	\end{align*}
	Then these sequences $(s_n)_{n\in\IN}$ and $(p_n)_{n\in\IN}$ satisfy
	\begin{align*}
		\sum_{n=0}^\infty[x_n]\pi^n+\sum_{n=0}^\infty[y_n]\pi^n&=\sum_{n=0}^\infty[s_n]\pi^n\\
		\Bigg(\sum_{n=0}^\infty[x_n]\pi^n\Bigg)\cdot\Bigg(\sum_{n=0}^\infty[y_n]\pi^n\Bigg)&=\sum_{n=0}^\infty[p_n]\pi^n\,.
	\end{align*}
\end{cor}
\begin{proof*}
	Again, we show the assertion more generally for an arbitrary $\Phi\in\Oo_E[X,Y]$ and its associated Witt polynomials $(\Phi_n)_{n\in\IN}$ constructed in the proof of \cref{prop:WittPolynomials}. The key observation is the following:
	\begin{alphanumerate}
		\item[$(*)$] If $a_0,\dotsc,a_n$ and $a_0',\dotsc,a'_n$ are elements of $A$ such that $a_i\equiv a_i'\mod \pi$, then
		\begin{align*}
			W_n(a_0,\dotsc,a_n)\equiv W_n(a'_0,\dotsc,a'_n)\mod \pi^{n+1}\,.
		\end{align*}
	\end{alphanumerate}
	Indeed, if you think about it, this immediately follows from \cref{lem:LTE} and the definition of the $W_n$. Now fix some $N$ and put
	\begin{align*}
		\varphi_n&=\Phi_n\left(x_0^{1/q^n},\dotsc,x_n^{1/q^0},y_0^{1/q^n},\dotsc,y_n^{1/q^0}\right)\\
		\varphi_n'&=\Phi_n\left(\big[x_0^{1/q^N}\big],\dotsc,\big[x_n^{1/q^{N-n}}\big],\big[y_0^{1/q^N}\big],\dotsc,\big[y_n^{1/q^{N-n}}\big]\right)\,.
	\end{align*}
	By construction of the Witt polynomials $(\Phi_n)_{n\in\IN}$ (see the proof of \cref{prop:WittPolynomials}) we immediately have 
	\begin{equation*}
		\Phi\left(W_N\left(\big[x_0^{1/q^N}\big],\dotsc,\big[x_N^{1/q^0}\big]\right),W_N\left(\big[y_0^{1/q^N}\big],\dotsc,\big[y_N^{1/q^0}\big]\right)\right)= W_N(\varphi_0',\dotsc,\varphi_N')\,.
	\end{equation*}
	But also $\varphi_n'\equiv\big[\varphi_n^{1/q^{N-n}}\big]\mod \pi$. Hence, by \itememph{*}, we obtain
	\begin{align*}
		W_N(\varphi_0',\dotsc,\varphi_N')\equiv W_N\left(\big[\varphi_0^{1/q^N}\big],\dotsc,\big[\varphi_n^{1/q^{0}}\big]\right)\mod \pi^{N+1}\,.
	\end{align*}
	Taking $N\rightarrow\infty$, this shows
	\begin{equation*}
		\Phi\Bigg(\sum_{n=0}^\infty[x_n]\pi^n,\sum_{n=0}^\infty[y_n]\pi^n\Bigg)=\sum_{n=0}^\infty[\varphi_n]\pi^n\,.
	\end{equation*}
	For $\Phi=X+Y$ resp.\ $\Phi=XY$ we retain the assertion of this corollary.
\end{proof*}
The upshot is that we can now reconstruct $A$ as a ring from $R=A/\pi A$. The next goal is to start with an arbitrary $R$ and construct an $A$ in a functorial way. In particular, we will allow $R$ to be an $\Oo_E$-algebra instead of an $\IF_q$-algebra (recall that $\IF_q=\Oo_E/\pi\Oo_E$). In the end, we will only be interested in the latter case, but allowing for rings of characteristic $0$ too gives us some nice uniqueness properties.
\begin{defi}\label{def:W_OE}
	For any $\Oo_E$-algebra $R$ write $W_{\Oo_E}(R)=R^\IN$. Its elements (which are sequences) are denoted $x=[x_0,x_1,\dotsc]$.
\end{defi}
\begin{prop}\label{prop:W_OE}
	The functor from \cref{def:W_OE} admits a unique factorization
	\begin{equation*}
		\begin{tikzcd}
		\cat{Alg}_{\Oo_E}\drar[dashed, "W_{\Oo_E}(-)"{swap}]\ar[rr, "(-)^\IN"] & & \cat{Set}\\
		& \cat{Alg}_{\Oo_E} \urar["\mathrm{forget}"{swap}]&
		\end{tikzcd}
	\end{equation*}
	such that the natural transformation $\Ww$ given by
	\begin{align*}
		\Ww_R\colon W_{\Oo_E}(R)&\morphism R^\IN\\
		[x_n]_{n\in\IN}&\longmapsto \big(W_n(x_0,\dotsc,x_n)\big)_{n\in\IN}
	\end{align*}
	is a morphism of $\Oo_E$-algebras. Here $R^\IN$ is equipped with its natural component-wise $\Oo_E$-algebra structure.
\end{prop}
\begin{proof}
	We first construct a natural $\Oo_E$-algebra structure on $W_{\Oo_E}(R)$. If two sequences $x=[x_n]_{n\in\IN}$ and $[y_n]_{n\in\IN}$ are given, we define $x+y=[s_n]_{n\in\IN}$ and $xy=[p_n]_{n\in\IN}$, where---you might have guessed it---we put
	\begin{equation*}
		s_n=S_n(x_0,\dotsc,x_n,y_0,\dotsc,y_n)\quad\text{and}\quad p_n=P_n(x_0,\dotsc,x_n,y_0,\dotsc,y_n)\,.
	\end{equation*}
	To see that this is determines a ring structure, the crucial thing to notice is that the proof of \cref{prop:WittPolynomials} works just the same if $\Phi\in\Oo_E[X_1,\dotsc,X_N]$ is a polynomial in arbitrary many variables instead of just $N=2$. So by choosing suitable $\Phi$, we can verify all ring axioms. For example, $\Phi=-X_1$ constructs additive inverses, $\Phi=(X_1+X_2)+X_3=X_1+(X_2+X_3)$ shows the associativity law of addition, $\Phi=X_1(X_2+X_3)=X_1X_2+X_1X_3$ shows distributivity, and so on. Also, if $\alpha\in\Oo_E$, then $\Phi=\alpha X_1$ defines multiplication by $\alpha$ on $W_{\Oo_E}(R)$, turning it into an $\Oo_E$-algebra.
	
	This provides a factorization through $\cat{Alg}_{\Oo_E}$. It is clear from the construction that $\Ww_R$ is an $\Oo_E$-algebra morphism. So it remains to show that this factorization is unique. If $R$ is $\pi$-torsionfree, then $\Ww_R\colon W_{\Oo_E}(R)\morphism R^\IN$ is easily seen to be injective, hence the $\Oo_E$-algebra structure on $W_{\Oo_E}(R)$ is uniquely determined by the one on $R^\IN$. In general, every $R$ admits a surjection $R'\epimorphism R$ from a $\pi$-torsionfree $\Oo_E$-algebra; e.g., $R'=\Oo_E\left[T_a\st a\in R\right]$ does it. Then $W_{\Oo_E}(R')\epimorphism W_{\Oo_E}(R)$ uniquely determines the $\Oo_E$-algebra structure on $W_{\Oo_E}(R)$. This shows uniqueness.
\end{proof}
\begin{urem}
	\begin{numerate}
		\item For the uniqueness part it was crucial to have \enquote{enough} $\pi$-torsionfree $\Oo_E$-algebras. If we had worked with $\IF_q$-algebras, where $\pi=0$, this wouldn't have been possible. In this case, $W_n(x_0,\dotsc,x_n)$ is just $x_0^{q^n}$. Hence the name \enquote{ghost components}.
		
		\item Also, \cref{prop:W_OE} gives the functor $W_{\Oo_E}(-)$ the structure of a ring scheme.
	\end{numerate}
\end{urem}	
\begin{lem}\label{lem:W_OETeichmuller}
	The natural map (which we will also call \enquote{Teichmüller lift})
	\begin{align*}
		[-]\colon R&\morphism W_{\Oo_E}(R)\\
		x&\longmapsto [x,0,0,\dotsc]
	\end{align*}
	is multiplicative.
\end{lem}
\begin{proof*}
	It's easy to see $P_0(X_0,Y_0)=X_0Y_0$. So to prove the assertion it suffices to check that $P_n(X_0,0,\dotsc,0,Y_0,0,\dotsc,0)=0$ for all $n>0$. But
	\begin{align*}
		W_n(X_0,0,\dotsc,0)\cdot W_n(Y_0,0,\dotsc,0)=X_0^{q^n}Y_0^{q^n}=W_n(X_0Y_0,0,\dotsc,0)\,,
	\end{align*}
	so this is easy to check by induction on $n$ (and using that polynomial rings over $\Oo_E$ are $\pi$-torsionfree).
\end{proof*}
\subsection{Frobenius and Verschiebung}
If $R$ happens to be an $\IF_q$-algebra, then we have the Frobenius $(-)^q$ on $R$. By functoriality, it extends to an endomorphism $F\colon W_{\Oo_E}(R)\morphism W_{\Oo_E}(R)$. The next lemma shows that $F$ actually exists for arbitrary $R$ and can be explicitly described.
\begin{lem}\label{lem:WittFrob}
	\begin{numerate}
		\item There is a unique natural transformation $F\colon W_{\Oo_E}(-)\morphism W_{\Oo_E}(-)$ of $\Oo_E$-algebras making the following diagram commute:
		\begin{equation*}
			\begin{tikzcd}
				W_{\Oo_E}(R)\rar["\Ww"]\dar["F"{swap}] & R^\IN\dar &[-2.4em] (x_n)_{n\in\IN}\dar[|->]\\
				W_{\Oo_E}(R)\rar["\Ww"] & R^\IN &[-2.4em] (x_{n+1})_{n\in \IN}
			\end{tikzcd}
		\end{equation*}
		\item If $R$ is an $\IF_q$-algebra, then $F$ is given by $F([x_0,x_1,\dotsc])=[x_0^q,x_1^q,\dotsc]$ and it is induced by the Frobenius on $R$.
	\end{numerate}
\end{lem}
\begin{proof*}
	We first construct a sequence $(F_n)_{n\in\IN}$ of polynomials $F_n\in \Oo_E[X_0,\dotsc,X_{n+1}]$ satisfying $W_{n+1}(X_0,\dotsc,X_{n+1})=W_n(F_0,\dotsc,F_n)$ and that $F_n\equiv X_n^q\mod \pi$. This is done by induction on $n$, the case $n=0$ being trivial. Suppose $F_0,\dotsc,F_{n-1}$ have already been constructed and have the required property. If we could prove that
	\begin{equation}\label{eq:Fn}
		W_{n+1}(X_0,\dotsc,X_{n+1})-W_n(F_0,\dotsc,F_{n-1},0)\equiv \pi^n X_0^q\mod \pi^{n+1}\,,
	\end{equation}
	this would show existence of $F_n$ and $F_n\equiv X_n^q\mod \pi$ at once. To prove \cref{eq:Fn}, we may equivalently show
	\begin{align}\label{eq:Fn2}
		\begin{split}
			0&\equiv W_{n+1}(X_0,\dotsc,X_{n-1},0,0)-W_n(F_0,\dotsc,F_{n-1},0)\\
			&\equiv W_{n-1}\big(X_0^{q^2},\dotsc,X_{n-1}^{q^2}\big)-W_{n-1}\left(F_0^q,\dotsc,F_{n-1}^q\right)\mod \pi^{n+1}\,.
		\end{split}
	\end{align}
	But $F_i\equiv X_i^q\mod \pi$ shows $F_i^q\equiv X_i^{q^2}\mod \pi^2$ by \cref{lem:LTE}, hence the bottom line of \cref{eq:Fn2} is indeed $0$ modulo $\pi^{n+1}$ by another application of \cref{lem:LTE}.
	
	Thus we can construct a sequence $F=(F_n)_{n\in \IN}$ with the required properties. By construction, $F$ makes the diagram in \itememph{1} commute and satisfies \itememph{2}. So it remains to show that $F$ is unique with this property and a morphism of $\Oo_E$-algebras. This can be done by the same argument as in the proof of \cref{prop:W_OE}. If $R$ is $\pi$-torsionfree, $W_{\Oo_E}(R)$ injects into $R^\IN$, hence it is uniquely determined and an $\Oo_E$-algebra morphism. In general, we take a surjection $R'\epimorphism R$ from a $\pi$-torsionfree $\Oo_E$-algebra.
\end{proof*}
\begin{lem}
	There is a natural transformation $V\colon W_{\Oo_E}(-)\morphism W_{\Oo_E}(-)$ of $\Oo_E$-modules that makes the following diagram commute:
	\begin{equation*}
		\begin{tikzcd}
			{[x_0,x_1,\dotsc]}\dar[|->] &[-2.4em] W_{\Oo_E}(R)\dar["V"{swap}]\rar["\Ww"] & R^\IN\dar &[-2.4em] (x_n)_{n\in\IN}\dar[|->]\\
			{[0,x_0,x_1,\dotsc]} &[-2.4em] W_{\Oo_E}(R) \rar["\Ww"] & R^\IN &[-2.4em] (\pi x_{n-1})_{n\in\IN}
		\end{tikzcd}\,,
	\end{equation*}
	where we put $x_{-1}=0$. Moreover, $V$ is unique with this property.
\end{lem}
\begin{proof*}
	It's immediately clear that $V$ as constructed makes the diagram commute. To show that $V$ is unique, we use the usual trick: for $\pi$-torsionfree $\Oo_E$-algebras $R$, this is clear; in general, consider a surjection $R'\epimorphism R$ where $R'$ is $\pi$-torsionfree.
\end{proof*}
\numpar*{Remark}
The letter $V$ stands for the German word \enquote{Verschiebung}. In contrast to $F$, $V$ is no ring endomorphism and it does depend on the choice of $\pi$.\footnote{Well, $W_n$ and thus $\Ww$ depend on $\pi$ too, so we cannot really say that $F$ is \enquote{independent} of $\pi$. But at least its image in $R^\IN$ is, in contrast to the image of $V$ in $R^\IN$.}
\begin{lem}\label{lem:FVidentities}
	The following identities hold for $F$ and the Verschiebung $V$.
	\begin{numerate}
		\item $FV=\pi$.
		\item $V(F(x)y)=xV(y)$ for all $x,y\in W_{\Oo_E}(R)$.
		\item $\pi F(x)y=F(xV(y))$ for all $x,y\in W_{\Oo_E}(R)$. 
	\end{numerate}
\end{lem}
\begin{proof}
	If $R$ is $\pi$-torsionfree, these can be checked in $R^\IN$. In general, take a surjection $R'\epimorphism R$ where $R'$ is $\pi$-torsionfree to reduce everything to the $\pi$-torsionfree case.
\end{proof}
\begin{lem}\label{lem:imVn}
	\begin{numerate}
		\item For all $n\in\IN$, the image of $V^n$ is an ideal in $W_{\Oo_E}(R)$.
		\item We have $W_{\Oo_E}(R)\cong \limit_{n\in\IN}W_{\Oo_E}(R)/\im V^n$.
		\item Every $x\in W_{\Oo_E}(R)$ admits a unique representation
		\begin{equation*}
			x=\sum_{n=0}^\infty V^n[x_n]
		\end{equation*}
		for some $x_n\in R$, where $[-]\colon R\morphism W_{\Oo_E}(R)$ is the Teichmüller lift from \cref{lem:W_OETeichmuller}. In fact, the $x_n$ are determined by $x=[x_n]_{n\in\IN}$.
	\end{numerate}
\end{lem}
\begin{proof*}
	Since $V$ is $\Oo_E$-linear, $\im V^n$ is a subgroup of $W_{\Oo_E}(R)$. Moreover, \cref{lem:FVidentities}\itememph{1} shows $xV^n(y)=V^n(F^n(x)y)$ for all $x,y\in W_{\Oo_E}(R)$, hence $\im V^n$ is closed under scalar multiplication. This shows \itememph{1}.
	
	Now part~\itememph{2}. We claim that the canonical map of sets $W_{\Oo_E}(R)\morphism R^N$ given by $[x_n]_{n\in\IN}\mapsto (x_0,\dotsc,x_{N-1})$ descends to a bijection
	\begin{equation*}
		W_{\Oo_E}(R)/\im V^N\isomorphism R^N\,.
	\end{equation*}
	Let's first check that it is well-defined. Let $y=[y_n]_{n\in \IN}$ be in the image of $V^n$, i.e., $y_n=0$ for all $n< N$. Let $x+y=[s_n]_{n\in \IN}$. Then what we need to show is that $s_n=x_n$ for all $n<N$. Thus, it suffices to check the polynomial identity
	\begin{equation*}
		S_n(X_0,\dotsc,X_n,0,\dotsc,0)=X_n\,.
	\end{equation*}
	However, this is easily seen from induction and the trivial identity
	\begin{equation*}
		W_n(X_0,\dotsc,X_n)+W_n(0,\dotsc,0)=W_n(X_0,\dotsc,X_n)\,.
	\end{equation*}
	Since $W_{\Oo_E}(R)/\im V^N\morphism R^N$ is automatically surjective, it remains to show injectivity. So let $x,y\in W_{\Oo_E}(R)$ be such that $x_n=y_n$ for all $n<N$. Let $x-y=[\delta_n]_{n\in\IN}$. To show that $\delta$ is in the image of $V^n$, we need to check $\delta_n=0$ for $n<N$. Thus, it suffices to check the polynomial identity
	\begin{equation*}
		\Delta_n(X_0,\dotsc,X_n,X_0,\dotsc,X_n)=0\,,
	\end{equation*}
	where $\Delta=X-Y\in\Oo_E[X,Y]$ and $(\Delta_n)_{n\in\IN}$ are the associated Witt polynomials constructed in the proof of \cref{prop:WittPolynomials}. This can be done in the same way as above.
	
	Now since $R^\IN\cong \limit_{n\in\IN}R^n$, the bijection $W_{\Oo_E}(R)/\im V^n\cong R^n$ for all $n\in \IN$ shows that $W_{\Oo_E}(R)\cong \limit_{n\in\IN}W_{\Oo_E}(R)/\im V^n$ is true as a limit of sets. However, the limit in the category of $\Oo_E$-algebras can be taken on the level of sets. This shows \itememph{2}.
	
	Finally, we show \itememph{3}. First we prove that for all $N\in \IN$ we have
	\begin{equation}\label{eq:sumVn}
		\sum_{n=0}^NV^n[x_n]=[x_0,\dotsc,x_N,0,0,\dotsc]\,.
	\end{equation}
	We use induction on $N$. The case $N=0$ is trivial. Now suppose the assertion is true for $N-1$. To prove it for $N$, it suffices to check the following polynomial identity: if $(X_n)_{n\in\IN}$ and $(Y_n)_{n\in\IN}$ are sequences of variables such that $X_N=0$ and $Y_n=0$ for all $n\neq N$, then
	\begin{equation*}
		S_n(X_0,\dotsc,X_n,Y_0,\dotsc,Y_n)=\begin{cases}
		Y_N&\text{if }n= N\\
		X_n&\text{else}
		\end{cases}\,.
	\end{equation*}
	For $n<N$, we obtain an identity that was already seen in the proof of \itememph{2}. For $n\geq N$, this easily follows by induction on $n$, using the identity
	\begin{equation*}
		W_n(X_0,\dotsc,X_n)+W_n(Y_0,\dotsc,Y_n)=W_n(X_0,\dotsc,X_{N-1},Y_N,X_{N+1},\dotsc,X_n)\,.
	\end{equation*}
	This shows \cref{eq:sumVn}. Now let $x-[x_0,\dotsc,x_N,0,0,\dotsc]=\delta=[\delta_n]_{n\in\IN}$. As in the proof of \itememph{2} we see that $\delta_n=0$ for $n\leq N$. Hence $\delta\in\im V^n$. This shows \itememph{3} except for the uniqueness part. But uniqueness is also clear from \cref{eq:sumVn}.
\end{proof*}
\begin{urem*}
	\cref{lem:imVn} holds for arbitrary $R$, despite what was claimed in the lecture. We leave it as an exercise to relate this error to the lecture's overall rushed style.
\end{urem*}
Now that the general theory of $W_{\Oo_E}(-)$ is set up, we restrict ourselves to the case where $R$ has characteristic $p$, i.e., $\pi=0$ on $R$ and $R$ is an $\IF_q$-algebra.
\begin{lem}\label{lem:Vincharp}
	Suppose $\pi=0$ on $R$. Then the following hold:
	\begin{numerate}
		\item For $x=[x_n]_{n\in\IN}\in R$ we have $x=\sum_{n=0}^\infty V^n[x_n]$.
		\item $VF=\pi$. Hence $V$ and $F$ commute.
		\item $F\big(\sum_{n=0}^\infty V^n[x_n]\big)=\sum_{n=0}^\infty V^n[x_n^q]$.
	\end{numerate}
\end{lem}
\begin{proof*}
	Part~\itememph{1} was already seen in \cref{lem:imVn}\itememph{3}. Now \itememph{3} is an immediate consequence of \itememph{1} and \cref{lem:WittFrob}\itememph{2}. For \itememph{2}, note that $VF$ sends $[x_0,x_1,\dotsc]$ to $[0,x_0^q,x_1^q,\dotsc]$. Thus, it suffices to show that the Witt polynomials $(\Pi_n)_{n\in\IN}$ associated to $\Pi=\pi X\in\Oo_E[X]$ satisfy
	\begin{equation*}
		\Pi_n(X_0,\dotsc,X_n)\equiv X_{n-1}^q\mod \pi\quad\text{for }n\geq 1
	\end{equation*}
	and $\Pi_0\equiv 0\mod \pi$. We show this by induction on $n$, the case $n=0$ being trivial. Now suppose the assertions holds up to $n$. Then $\Pi_i\equiv X_{i-1}^q\mod \pi$ for all $i\leq n$ shows, by \cref{lem:LTE}, that
	\begin{equation*}
		W_{n+1}(\Pi_1,\dotsc,\Pi_{n+1})\equiv \pi X_0^{q^{n+1}}+\dotsb+\pi^{n}X_{n-1}^{q^2}+\pi^{n+1}\Pi_{n+1}\mod \pi^{n+2}\,.
	\end{equation*}
	However, the left-hand side can, by definition, be computed as
	\begin{align*}
		W_{n+1}(\Pi_1,\dotsc,\Pi_{n+1})&\equiv\pi W_{n+1}(X_0,\dotsc,X_{n+1})\\
		&\equiv \pi X_0^{q^{n+1}}+\dotsb+\pi^nX_{n-1}^{q^2}+\pi^{n+1}X_n^q\mod \pi^{n+2}\,.
	\end{align*}
	This shows indeed $\Pi_{n+1}\equiv X_n^q\mod \pi$, as claimed.
\end{proof*}
\begin{lem}\label{lem:W_OEpi}
	If $R$ is a perfect $\IF_q$-algebra, then $W_{\Oo_E}(R)$ is $\pi$-adically complete, and if $x=[x_n]_{n\in\IN}$, then
	\begin{equation*}
		x=\sum_{n=0}^\infty\big[x_n^{1/q^n}\big]\pi^n\,.
	\end{equation*}
\end{lem}
\begin{proof*}
	Since $R$ is perfect, the Frobenius is an automorphism, hence the same is true for $F$ on $W_{\Oo_E}(R)$. Thus \cref{lem:Vincharp}\itememph{2} shows that the image of $V^n$ is the image of $\pi^n$. Thus \cref{lem:imVn}\itememph{2} proves that $W_{\Oo_E}(R)$ is $\pi$-adically complete.
	
	To see the second assertion, note that by \cref{lem:Vincharp}\itememph{2} we have
	\begin{equation*}
		[x_n]\pi^n=V^nF^n\big[x_n^{1/q^n}\big]=V^n[x_n]\,,
	\end{equation*}
	and use \cref{lem:imVn}\itememph{3}.
\end{proof*}
Finally we have everything together to prove \cref{prop:FqAlgebrasEquivalence}.
\begin{proof}[Proof of \cref{prop:FqAlgebrasEquivalence}]
	We claim that $W_{\Oo_E}(-)$ defines an inverse functor. If $R$ is a perfect $\IF_q$-algebra, \cref{lem:W_OEpi} shows $W_{\Oo_E}(R)/\pi W_{\Oo_E}(R)= W_{\Oo_E}(R)/\im V$. The right-hand side is isomorphic $R$ as an $\Oo_E$-algebra. On the level of sets this was seen in the proof of \cref{lem:imVn}\itememph{2}. Rs $\Oo_E$-algebra this follows from $S_0=X_0+Y_0$, $P_0=X_0Y_0$, and $(aX)_0=aX_0$ for all $a\in\Oo_E$.
	
	Thus, the image of $W_{\Oo_E}(-)$ is as desired. It remains to provide a natural isomorphism between $A$ and $W_{\Oo_E}(R)$ if $R=A/\pi A$. We define it via
	\begin{align*}
		W_{\Oo_E}(R)&\morphism A\\
		\sum_{n=0}^\infty[x_n]\pi^n&\longmapsto\sum_{n=0}^\infty[x_n]\pi^n\,.
	\end{align*}
	By \cref{lem:TeichmullerRep} and \cref{lem:W_OEpi}, it is a natural bijection. By \cref{cor:snpn} it is $\Oo_E$-linear. We are done.
\end{proof}
\begin{cor}\label{cor:unramifiedWitt}
	Let $E_0$ be the maximal unramified subextension of $E/\IQ_p$ (or in other words, the unique unramified extension of $\IQ_p$ with residue field $\IF_q$). Then there is a natural isomorphism
	\begin{equation*}
		W(R)\otimes_{\Oo_{E_0}}\Oo_E\isomorphism W_{\Oo_E}(R)\,.
	\end{equation*}
\end{cor}
\begin{proof*}
	Since $p$ is a uniformizer of $\Oo_{E_0}$, the Witt vectors $W(R)$ taken over $\IZ_p$ are the same as if they were taken over $\Oo_{E_0}$. Now the diagram
	\begin{equation*}
		\begin{tikzcd}[row sep=normal]
			\left\{\begin{tabular}{c}
			$p$-torsionfree $p$-adically complete \\
			$\Oo_E$-algebras $A$ s.th.\ $A/p A$ is perfect
			\end{tabular}
			\right\}\ar[dd,"-\otimes_{\Oo_{E_0}}\Oo_E"{swap}]
			\drar[start anchor=south east, end anchor=north west, "-/p-"{swap}] & \\
			 & \left\{\text{perfect $\IF_q$-algebras}\right\}\ular[start anchor=168, end anchor=355, bend right, dotted, "W(-)"{swap}]\dlar[start anchor=192, end anchor=5, bend left, dotted, "W_{\Oo_E}(-)"]\\
			\left\{\begin{tabular}{c}
			$\pi$-torsionfree $\pi$-adically complete \\
			$\Oo_E$-algebras $A$ s.th.\ $A/\pi A$ is perfect
			\end{tabular}
			\right\}\urar[start anchor=north east, end anchor=south west,"-/\pi-"] & 
		\end{tikzcd}
	\end{equation*}
	of functors between categories commutes. Hence the diagram formed by the vertical arrow and the two dotted quasi-inverses commutes up to natural isomorphism, which is precisely what we want to show.
\end{proof*}
\begin{exm*}
	Now we can easily verify the examples given at the beginning of the section. To prove
	\begin{equation*}
		W(\IF_p)=\IZ_p\,,\quad W(\IF_q)=\Oo_{E_0}\,,\quad\text{and}\quad W\big(\IF_p\big\llbracket T^{1/p^\infty}\big\rrbracket\big)=\IZ_p\big\llbracket T^{1/p^\infty}\big\rrbracket\,,
	\end{equation*}
	it suffices to see that the respective right-hand sides are $p$-complete, $p$-torsionfree and that modding out $p$ gives $\IF_p$, $\IF_q$, and $\IF_p\big\llbracket T^{1/p^\infty}\big\rrbracket$ respectively. This is easy to check.
\end{exm*}

\section{The Ring \texorpdfstring{$\IA_\inf$}{Ainf}}
\lecture[Definition of $\IA_\inf$. Tilting as an adjoint to $W_{\Oo_E}(-)$. Perfectoid $\Oo_E$-algebras: tilting equivalence, examples. A picture of $\Spec \IA_\inf$.]{2019-10-30}
Apparently, $\IA_\inf$ is so awesome that Pierre Colmez titled it \enquote{The One Ring to rule them all} (\href{https://www.facebook.com/cyclotomicmemes/photos/a.189056291880728/347547606031595/?type=3&theater}{somewhat related}). For example, it already determines $B_\cris$ and $B_\dR$.

Throughout this section, let $p$ be a prime, $E/\IQ_p$ a finite extension, $\pi\in \Oo_E$ a uniformizer and $\IF_q=\Oo_E/\pi\Oo_E$ for $q=p^f$. Moreover, let $F/\IF_q$ be a non-archimedean algebraically closed extension. For us, \defemph{non-archimedean} always means that $F$ is complete with respect to a non-archimedean non-trivial valuation $|\blank|\colon F\morphism\IR_{\geq 0}$. As usual, the \defemph{ring of integers} $\Oo_F$ is defined as
\begin{equation*}
	\Oo_F=\left\{x\in F\st |x|\leq 1\right\}\,.
\end{equation*}
Note that $\Oo_F$ is local with maximal ideal $\mm_F=\left\{x\in F\st|x|<1\right\}$.
\begin{defi}
	In the above setting, we define
	\begin{equation*}
		\IA_\inf=\IA_{\inf,E,F}\coloneqq W_{\Oo_E}(\Oo_F)\,.
	\end{equation*}
\end{defi}
\begin{rem}
	\begin{numerate}
		\item $\IA_\inf$ should be thought of a \enquote{power series ring over $\Oo_F$ in the indeterminate $\pi$}. So its equal characteristic analogue should be $\Oo_F\llbracket z\rrbracket$.
		\item $\IA_\inf$ has a natural Frobenius action $\phi$, given by the Witt vector Frobenius, which is, in turn, given by the Frobenius on $\Oo_F$.
	\end{numerate}
\end{rem}
In the proof of \cref{prop:FqAlgebrasEquivalence} we have seen that $W_{\Oo_E}(-)$ is a quasi-inverse to $-/\pi -$ on some suitable category. In general, $W_{\Oo_E}(-)$ still possesses an adjoint, the \defemph{tilt functor}.
\begin{defi}
	Let $A$ be a $\pi$-complete $\Oo_E$-algebra. Then the \defemph{tilt of $A$} is
	\begin{equation*}
		A^\flat\coloneqq \limit_{x\mapsto x^q}A/\pi A=\left\{(a_0,a_1,\dotsc)\in\prod_{n\in\IN}A/\pi A\st a_i^q=a_{i-1}\text{ for all }i>0\right\}\,.
	\end{equation*}
\end{defi}
Note that $A^\flat$ is always a perfect $\IF_q$-algebra (in fact, that's a purely category-theoretical statement): the Frobenius on $A^\flat$ is given by $\Frob_{q,A^\flat}(a_0,a_1,\dotsc)=(a_0^q,a_0,a_1,\dotsc)$ and it has an inverse defined by $\Frob_{q,A^\flat}^{-1}(a_0,a_1,\dotsc)=(a_1,a_2,\dotsc)$.
\begin{prop}\label{prop:tiltWittAdjunction}
	There is an adjunction
	\begin{equation*}
		W_{\Oo_E}(-)\colon \left\{\text{$\pi$-complete $\Oo_E$-algebras}\right\}\doublelrmorphism \left\{\text{perfect $\IF_q$-algebras}\right\}\noloc (-)^\flat\,.
	\end{equation*}
\end{prop}
\begin{rem}
	Before we sketch a proof of \cref{prop:tiltWittAdjunction}, let us leave two remarks.
	\begin{numerate}
		\item If $R$ is a perfect $\IF_q$-algebra, then the unit $R\morphism W_{\Oo_E}(R)^\flat$ of the adjunction is given by $r\mapsto (r,r^{1/q},r^{1/q^2},\dotsc)$. Thus it is an isomorphism. In particular, this shows that $W_{\Oo_E}(-)$ is fully faithful by abstract nonsense. However, we have already seen that in the proof of \cref{prop:FqAlgebrasEquivalence}, where moreover the essential image of $W_{\Oo_E}(-)$ was identified as the class of $\pi$-complete $\pi$-torsionfree $\Oo_E$-algebras $A$ such that $A/\pi A$ is perfect.
		\item The counit $\theta\colon W_{\Oo_E}(A^\flat)\morphism A$ is usually called \defemph{Fontaine's map}.
	\end{numerate}
\end{rem}
\begin{proof}[Sketch of a proof of \cref{prop:tiltWittAdjunction}]
	First we state the following slightly more general form of the key \cref{lem:LTE} (actually, this proof only uses the previous formulation, but for future use the more general version will be handy). It can be proved in the exact same way as \cref{lem:LTE}.
	\begin{lem}[\enquote{$q$-power map is $\pi$-adically contracting}]\label{lem:keyLemma}
		Let $B$ be any $\Oo_E$-algebra and $I\subseteq B$ an ideal such that $\pi\in I$. If $x,y\in B$ such that $x\equiv y\mod I$, then
		\begin{equation*}
			x^{q^n}\equiv y^{q^n}\mod I^{n+1}\quad\text{for all }n\geq 0\,.
		\end{equation*}
	\end{lem}
	We construct the counit $\theta$ as follows. Fix $n>0$. By $W_{\Oo_E,n}(A)$ we denote the truncated Witt vectors of length $n+1$. These are obtained by cutting off everything after the first $n+1$ components. In other words, $W_{\Oo_E,n}(A)=W_{\Oo_E}(A)/\im V^{n+1}$. Consider the map
	\begin{align*}
		\Ww_n\colon W_{\Oo_E,n}(A)&\morphism A/\pi^{n+1}A\\
		[a_0,\dotsc,a_n]&\longmapsto W_n(a_0,\dotsc,a_n)\mod \pi^{n+1}\,.
	\end{align*}
	If $a_i\equiv 0\mod \pi$ for all $i=0,\dotsc,n$, then \cref{lem:keyLemma} shows $W_n(a_0,\dotsc,a_n)\equiv 0\mod \pi^ {n+1}$. Thus, we get an induced map $\theta_n\colon W_{\Oo_E,n}(A/\pi A)\morphism A/\pi^{n+1}A$. We check that the diagram
	\begin{equation*}
		\begin{tikzcd}
			W_{\Oo_E,n+1}(A/\pi A)\dar["F"{swap}]\rar["\theta_{n+1}"] & A/\pi^{n+2}A\dar\\
			W_{\Oo_E,n}(A/\pi A)\rar["\theta_n"] & A/\pi^{n+1}A
		\end{tikzcd}
	\end{equation*}
	commutes. Indeed, given $[\ov{a}_0,\dotsc,\ov{a}_{n+1}]\in W_{\Oo_E,n+1}(A/\pi A)$ with lifts $[a_0,\dotsc,a_{n+1}]$, we have
	\begin{equation*}
		W_{n+1}(a_0,\dotsc,a_{n+1})\equiv W_n(a_0^q,\dotsc,a_n^q)\mod \pi^{n+1}\,,
	\end{equation*}
	which is precisely what we want. Passing to the limit, we obtain a map
	\begin{equation*}
		\theta\colon W_{\Oo_E}(A^\flat)\cong \limit_FW_{\Oo_E,n}(A/\pi A)\morphism \limit_{n \in\IN}A/\pi^{n+1}A\cong A\,.
	\end{equation*}
	The isomorphism on the left is easy to check, and the isomorphism on the right follows from $A$ being $\pi$-complete. In the lecture, that was the end of the proof sketch. In these notes we will finish the proof, but only after we understand the map $\theta$ a little better.
\end{proof}
Another application of \cref{lem:keyLemma} is the following.
\begin{prop}\label{prop:(A/I)b}
	Let $A$ be a $\pi$-complete $\Oo_E$-algebra. Let $I\subseteq A$ be an ideal containing $I$, such that $A$ is also $I$-complete. Then the canonical map
	\begin{equation*}
		\limit_{x\mapsto x^q}A\isomorphism (A/I)^\flat
	\end{equation*}
	is an isomorphism. In particular, the left-hand side (which is a priori only a multiplicative monoid) inherits a natural ring structure.
\end{prop}
\begin{proof}
	Let $x=(\ov{x}_0,\ov{x}_1,\dotsc)\in (A/I)^\flat$. For every $n\geq 0$ choose a lift $x_n\in A$ of $\ov{x}_n$. By \cref{lem:keyLemma}, $(x_n^{q^n})_{n\in\IN}$ is a Cauchy sequence in the $I$-adic topology. Put
	\begin{equation*}
		x^\sharp=\lim_{n\to\infty}x_n^{q^n}\,.
	\end{equation*}
	As in the proof of \cref{deflem:Teichmuller}, $x^\sharp$ is independent of the choice of lifts and $(-)^\sharp$ is multiplicative. Now it's easy to see that the map
	\begin{align*}
		(A/I)^\flat&\morphism \limit_{x\mapsto x^q}A\\
		x&\longmapsto \big(x^\sharp,(x^{1/q})^\sharp,\dotsc\big)
	\end{align*}
	is a multiplicative inverse of the map in question. This proves the assertion.
\end{proof}
\begin{lem}\label{lem:WAb->A}
	The counit $\theta\colon W_{\Oo_E}(A^\flat)\morphism A$ can be explicitly described as 
	\begin{equation*}
		\sum_{n=0}^\infty[a_n]\pi^n\longmapsto \sum_{n=0}^\infty a_n^\sharp \pi^n\,.
	\end{equation*}
\end{lem}
\begin{proof*}
	Let us first describe the isomorphism $W_{\Oo_E}(A^\flat)\cong \limit_FW_{\Oo_E,n}(A/\pi A)$ that is part of the definition of $\theta$. The underlying set of $W_{\Oo_E}(A^\flat)$ consists of sequences $[a_0,a_1,\dotsc]$, where each $a_n\in A^\flat$ is itself a sequence $a_n=(\ov{a}_{n,i})_{i\in\IN}$ in $A/\pi A$ such that $\ov{a}_{n,i}^q=\ov{a}_{n,i-1}$. The underlying set of $\limit_FW_{\Oo_E,n}(A/\pi A)$ consists of sequences $([\ov{a}_{0,0}],[\ov{a}_{0,1},\ov{a}_{1,1}],[\ov{a}_{0,2},\ov{a}_{1,2},\ov{a}_{2,2}],\dotsc)$ that are compatible under $F$. The isomorphism in question is given by
	\begin{align*}
		W_{\Oo_E}(A^\flat)&\isomorphism\limit_FW_{\Oo_E,n}(A/\pi A)\\
		[\ov{a}_0,\ov{a}_1,\ov{a}_2,\dotsc]&\longmapsto \big([\ov{a}_{0,0}],[\ov{a}_{0,1},\ov{a}_{1,1}],[\ov{a}_{0,2},\ov{a}_{1,2},\ov{a}_{2,2}],\dotsc\big)\,.
	\end{align*}
	Indeed, it is clear that this defines a bijection on set level and one may check that it is also compatible with the ring structures on either side.
	
	Now let $a=[a_0,a_1,\dotsc]\in W_{\Oo_E}(A^\flat)$ be as above. We unwind what $\theta(a)$ actually is. By definition of the map $\theta_N\colon W_{\Oo_E,N}(A/\pi A)\morphism A/\pi^{N+1}A$, we have
	\begin{equation*}
		\theta_N[\ov{a}_{0,N},\dotsc,\ov{a}_{N,N}]\equiv\sum_{n=0}^Na_{n,N}^{q^{N-n}}\pi^n\mod \pi^{N+1}\,,
	\end{equation*}
	where the $a_{n,N}$ are arbitrary lifts of $\ov{a}_{n,N}$. 
	Thus, the coefficient of $\pi^n$ in $\theta(a)$ is given by
	\begin{equation*}
		\lim_{N\to\infty}a_{n,N}^{q^{N-n}}=\big(a_n^{1/q^n}\big)^\sharp\,.
	\end{equation*}
	The exponent $1/q^n$ seems off at first glance, but according to \cref{lem:W_OEpi} this is exactly what we want.
\end{proof*}
\begin{proof*}[End of proof of \cref{prop:tiltWittAdjunction}]
	Let $A$ be a $\pi$-complete $\Oo_E$-algebra and $R$ a perfect $\IF_q$-algebra. By \cref{prop:FqAlgebrasEquivalence} we have a bijection
	\begin{equation*}
		\Hom(R,A^\flat)\cong \Hom\big(W_{\Oo_E}(R),W_{\Oo_E}(A^\flat)\big)\,,
	\end{equation*}
	so it suffices to see that every $\Oo_E$-algebra morphism $\alpha\colon W_{\Oo_E}(R)\morphism A$ factors uniquely over $\theta$. Let such an $\alpha$ be given. Modulo $\pi$ we get an induced morphism $\ov{\alpha}\colon R\morphism A/\pi A$. Since $R$ is perfect, $R^\flat\cong R$. Also $A^\flat\cong (A/\pi A)^\flat$. Hence we get an induced morphism $\ov{\alpha}^\flat\colon R\morphism A^\flat$. We claim that
	\begin{equation*}
		\begin{tikzcd}
			W_{\Oo_E}(R)\dar["W_{\Oo_E}(\ov{\alpha}^\flat)"{swap}]\rar["\alpha"]& A\\
			W_{\Oo_E}(A^\flat)\urar["\theta"{swap}]&
		\end{tikzcd}
	\end{equation*}
	commutes. In view of \cref{lem:WAb->A} we only need to check that $\alpha[x]=\ov{\alpha}^\flat(x)^\sharp$ for all $x\in R$. By construction, $\ov{\alpha}^\flat(x)$ is the sequence $(\ov{\alpha}(x),\ov{\alpha}(x^{1/q}),\dotsc)\in A^\flat$. Moreover, $\alpha[x^{1/q^n}]$ is a lift of $\ov{\alpha}(x^{1/q^n})$ for all $n\in\IN$. Raising $\ov{\alpha}(x^{1/q^n})$ to the $(q^n)\ordinalth$ power gives $\alpha[x]$ back, since both $\alpha$ and the Teichmüller lift $[-]$ are multiplicative. This shows indeed $\alpha[x]=\ov{\alpha}^\flat(x)^\sharp$.
	
	To finish the proof, it's left to see why $\ov{\alpha}^\flat$ is the only choice. Suppose $\beta\colon R\morphism A^\flat$ leads to a commutative diagram as above. Reducing modulo $\pi$ we see that the composition of $\beta$ with $A^\flat\morphism A/\pi$ must coincide with $\ov{\alpha}$. In other words, the $0\ordinalth$ component of $\beta\colon R\morphism A^\flat$ must be given by $\ov{\alpha}$. By naturality of the Witt vector Frobenius, the diagram
	\begin{equation*}
		\begin{tikzcd}
		W_{\Oo_E}(R)\rar["F^{-1}"]\dar["W_{\Oo_E}(\beta)"{swap}"]& W_{\Oo_E}(R)\dar["W_{\Oo_E}(\beta)"{swap}]\rar["\alpha"]& A\\
		W_{\Oo_E}(A^\flat)\rar["F^{-1}"] & W_{\Oo_E}(A^\flat)\urar["\theta"{swap}]&
		\end{tikzcd}
	\end{equation*}
	commutes as well. Reducing modulo $\pi$ and walking around the perimeter, we see that the $1\ordinalst$ component of $R\morphism A^\flat$ must be given by $\ov{\alpha}((-)^{1/q})$. Repeating this argument, we see that $\beta=\ov{\alpha}^\flat$, as desired.
\end{proof*}
\subsection{Perfectoid \texorpdfstring{$\Oo_E$}{O}-Algebras}
\begin{defi}
	\begin{numerate}
		\item A \defemph{perfect prism} over $\Oo_E$ is a pair $(W_{\Oo_E}(R),I)$, where $R$ is a perfect $\IF_q$-algebra, $I\subseteq W_{\Oo_E}(R)$ is a principal ideal generated by an element $d$ such that
		\begin{equation*}
			\frac{F(d)-d^q}{\pi}\in W_{\Oo_E}(R)^\times
		\end{equation*}
		(such $d$ is called \defemph{distinguished}), and such that $W_{\Oo_E}(R)$ is $(\pi,I)$-adically complete.
		\item An $\Oo_E$-algebra $A$ is a \defemph{perfectoid $\Oo_E$-algebra} if it can be written as $A\cong W_{\Oo_E}(R)/I$ for some perfect prism $(W_{\Oo_E}(R),I)$ over $\Oo_E$.
	\end{numerate}
\end{defi}
\begin{rem}\label{rem:perfectoid}
	\begin{numerate}
		\item To see that $F(d)-d^q$ is always divisible by $\pi$, note that $F$ is the lift of the Frobenius on $R$. In particular, $F$ and $(-)^q$ become equal after reducing modulo $\pi$.
		\item An element $d=\sum_{n=0}^\infty[r_n]\pi^n\in W_{\Oo_E}(R)$ is distinguished iff $r_1\in R^\times$. Indeed, by \cref{lem:WittFrob}\itememph{2} and \cref{lem:W_OEpi} we have $F(d)\equiv [r_0^q]+[r_1^q]\pi\mod \pi^2$ and from the key \cref{lem:keyLemma} we get $d^q\equiv [r_0^q]\mod \pi^2$. Hence
		\begin{equation*}
			\frac{F(d)-d^q}{\pi}\equiv [r_1^q]\mod \pi\,.
		\end{equation*}
		By $\pi$-completeness, an element $x\in W_{\Oo_E}(R)$ is invertible iff its modulo-$\pi$ reduction is invertible. And $r_1^q\in R$ is invertible iff so is $r_1$. Moreover, $W_{\Oo_E}(R)$ is $(\pi,d)$-adically complete iff $R$ is $r_0$-complete. Since this seems rather non-trivial to me, we give it a proper proof in \cref{lem*:nonTrivial} below.
		\item Perfect rings are perfectoid. Indeed, if $R$ is perfect, we have $R\cong W_{\Oo_E}(R)/\pi W_{\Oo_E}(R)$, and $(W_{\Oo_E}(R),\pi)$ is clearly a perfect prism (by \itememph{2} for example). Conversely, if an algebra $A$ over $\IF_q=\Oo_E/\pi\Oo_E$ is perfectoid, then it is also perfect. This too was not trivial for me, so we prove it in \cref{lem*:perfectoid=perfect} below.
		\item If $A$ is perfectoid, say, $A\cong W_{\Oo_E}(R)/I$, then
		\begin{equation*}
			A^\flat\cong (W_{\Oo_E}(R)/I)^\flat\cong \big(W_{\Oo_E}(R)/(\pi,I)\big)^\flat\cong (R/IR)^\flat\cong R^\flat\cong R\,.
		\end{equation*}
		The only non-obvious step is $(R/IR)^\flat\cong R^\flat$. To see this, first note that $IR$ is an ideal containing the image of $\pi$ in $R$ since this image is $0$. Moreover, $R$ is $IR$-adically complete by \cref{lem*:nonTrivial}. Hence the isomorphism follows from \cref{prop:(A/I)b}.
	\end{numerate}
\end{rem}
\begin{lem*}\label{lem*:nonTrivial}
	Let $R$ be a perfect $\IF_q$-algebra and $d=\sum_{n=0}^\infty[r_n]\pi^n$ be an element of $W=W_{\Oo_E}(R)$. Then $W$ is $(\pi,d)$-adically complete iff $R$ is $r_0$-complete.
\end{lem*}
\begin{proof*}
	Let's first assume $W$ is $(\pi,d)$-complete. Then $R$ being $r_0$-complete is equivalent to $R$ being $(\pi,d)$-complete too. By \cite[\stackstag{031A}]{stacks-project}, we need to check that
	\begin{equation*}
		\pi W=\bigcap_{n\geq 1}\big(\pi W+(\pi,d)^n\big)\,.
	\end{equation*}
	Suppose some $w\in W$ is contained in $\pi W+(\pi,d)^n$ for all $n\in \IN$. Then its image $\ov{w}\in R$ is divisible by $r_0^n$ for all $n\geq 0$, hence also $[\ov{w}]$ is divisible by $[r_0]^n$ for all $n\geq 0$. By a well-known argument, $W$ being $(\pi,d)$-complete is equivalent to $W$ being complete with respect to the ideals $\{(\pi^n,d^n)\}_{n\geq 1}$. By abstract nonsense, we may replace this family of ideals by $\{(\pi^{n+1},d^{q^n})\}_{n\geq 1}$. But $d^{q^n}\equiv [r_0]^{q^n}\mod \pi^{n+1}$ by the key \cref{lem:keyLemma}, hence $W$ is also complete with respect to the ideals $\{(\pi^{n+1},[r_0]^{q^n})\}_{n\geq 1}$. Since $[\ov{w}]$ lies in all of them by assumption, we get $\ov{w}=0$, hence $w\in\pi W$, as required.
	
	Now assume $R$ is $r_0$-complete. It suffices to show that $W$ is complete with respect to the ideals $\{(\pi^n,d^n)\}_{n\geq 1}$. By an abstract nonsense argument, this is equivalent to $W$ being complete with respect to $\{(\pi^n,d^m)\}_{n,m\geq 1}$. Since $W$ is $\pi$-complete, it thus suffices to show that $W/\pi^nW$ is $d$-complete for all $n\geq 1$. The key \cref{lem:keyLemma} shows $d^{q^m}\equiv [r_0]^{q^m}\mod \pi^n$ for all $m\geq n-1$. Thus we may equivalently show that $W/\pi^nW$ is $[r_0]$-complete.
	
	We argue by induction over $n$. The case $n=1$ is just the assumption. Now assume the assertion holds up to $n$. Consider the short exact sequence
	\begin{equation*}
		0\morphism W/\pi^nW\morphism[\pi]W/\pi^{n+1}W\morphism R\morphism 0\,.
	\end{equation*}
	Suppose $x\in W/\pi^nW$ has the property that $\pi x\in W/\pi^{n+1}W$ is divisible by $[r_0]^m$, say, $\pi x=[r_0^m]y$. Write $x=[x_0]+[x_1]\pi+\dotsb+[x_{n-1}]\pi^{n-1}$ and $y=[y_0]+[y_1]\pi+\dotsb+[y_n]\pi^n$. Then
	\begin{equation*}
		[x_0]\pi+[x_1]\pi^2+\dotsb+[x_{n-1}]\pi^n=[r_0^my_0]+[r_0^my_1]\pi+\dotsb+[r_0^my_n]\pi^n\,.
	\end{equation*}
	By uniqueness of these representations, we get $0=r_0^my_0$, $x_0=r_0^my_1$ and so on up to $x_{n-1}=r_0^my_n$. In particular, $x=[r_0]^m([y_1]+\dotsb+[y_n]\pi^{n-1})$ is divisible by $[r_0]^m$! We conclude that the sequence
	\begin{equation*}
		0\morphism W/(\pi^n,[r_0]^m)\morphism[\pi] W/(\pi^{n+1},[r_0]^m)\morphism R/r_0^mR\morphism 0
	\end{equation*}
	is exact again. Taking limits over $m$ we obtain a diagram
	\begin{equation*}
		\begin{tikzcd}
			0 \rar & W/\pi^nW\dar[iso]\rar["\pi"] & W/\pi^{n+1}W \dar\rar & R\rar\dar[iso] & 0\\
			0 \rar &\lim\limits_{m\geq 1}W/(\pi^n,[r_0]^m)\rar["\pi"] & \lim\limits_{m\geq 1}W/(\pi^{n+1},[r_0]^m)\rar & \lim\limits_{m\geq 1}R/r_0^mR\rar & 0
		\end{tikzcd}
	\end{equation*}
	in which the outer vertical arrows are isomorphisms by the induction hypothesis. Thus the middle vertical arrow is an isomorphism as well by the five lemma (note that the bottom sequence is exact by the Mittag-Leffler condition, but this isn't even needed for the argument).
\end{proof*}
\begin{lem*}\label{lem*:perfectoid=perfect}
	If an algebra $A$ over $\IF_q=\Oo_E/\pi\Oo_E$ is perfectoid, then $A$ is already a perfect $\IF_q$-algebra.
\end{lem*}
\begin{proof*}
	Write $A\cong W_{\Oo_E}(R)/I$. Since $\pi$ vanishes on $A$, we have $\pi\in A$. By \cref{rem:perfectoid}\itememph{4}, $A^\flat\cong R\cong W_{\Oo_E}(R)/\pi W_{\Oo_E}(R)$. Hence it suffices to prove that $I$ is generated by $\pi$, since then $A\cong A^\flat$ is perfect.
	
	The argument that follows is stolen from \cite[Lemma~3.10]{BMS}. Write $\pi=dw$, where $d\in I$ is a distinguished generator and $w=\sum_{n=0}^\infty [w_n]\pi^n$ is some element of $W_{\Oo_E}(A^\flat)$. The Witt polynomial $P_1$ is given by $P_1(X,Y)=X_0^qY_1+X_1Y_0^q+\pi X_1Y_1$. Thus $\pi=dw$ yields
	\begin{equation*}
		1=r_0^qw_1+r_1w_0^q
	\end{equation*}
	(note that $\pi r_1w_1$ vanishes in $A^\flat$). We claim that $r_1w_0^q=1-r_0^qw_1$ is a unit in $A^\flat$. It suffices to check that it is mapped to a unit under the projection $A^\flat\morphism A/\pi A=A$ to the $0\ordinalth$ component. But $A\cong A^\flat/r_0A^\flat$, hence $1-r_0^qw_1$ is mapped to $1\in A$, which is indeed a unit. Thus also $r_1$ and $w_0$ are units in $A^\flat$. But $w_0$ being a unit implies that $w$ itself is a unit in $W_{\Oo_E}(A^\flat)$, hence $\pi$ is indeed a generator of $I$.
\end{proof*}
The following fact wasn't mentioned in the lecture, making it hard for me to read some of the literature that uses the \enquote{old} definition of perfectoid rings. So we prove it here.
\begin{lem*}\label{lem*:perfectoidComplete}
	Let $(W_{\Oo_E}(R),I)$ be a perfect prism over $\Oo_E$ and $A=W_{\Oo_E}(R)/I$.
	\begin{alphanumerate}
		\item If $\xi$ is a distinguished generator of $I$, then $\xi$ is a non-zero divisor in $W_{\Oo_E}(R)$.
		\item $A$ is $\pi$-complete.
	\end{alphanumerate}
\end{lem*}
\begin{proof*}
	Put $W=W_{\Oo_E}(R)$ for convenience. Both \itememph{a} and \itememph{b} are based on the following observation.
	\begin{alphanumerate}
		\item[\itememph{*}] Let $(x_n)_{n\in \IN}$ be a sequence such that $\xi x_n\equiv 0\mod \pi^n$. Then the $x_n$ converge to $0$ in the $(\pi,\xi)$-adic topology.
	\end{alphanumerate}
	Claim \itememph{*} immediately implies \itememph{a}. Also  \itememph{b} is not far: by \cite[\stackstag{031A}]{stacks-project}, we need to check that
	\begin{equation*}
		\xi W=\bigcap_{n\geq 1}(\xi W+\pi^nW)\,.
	\end{equation*}
	So suppose $y$ lies in the intersection and choose $(x_n)_{n\in \IN}$ such that $y\equiv \xi x_n\mod \pi^n$. Then $\xi(x_{n+1}-x_n)\equiv 0\mod \pi^n$. Thus the $(x_{n+1}-x_n)$ converge to $0$ in the $(\pi,\xi)$-adic topology. Hence $(x_n)_{n\in \IN}$ converges to some $x\in W$ satisfying $y=\xi x$. This shows \itememph{a}.
	
	It remains to show \itememph{*}. Write $\xi=[r_0]+\pi u$, where $u\in W$ is a unit. If $\xi x_n\equiv 0\mod \pi^n$, then also $([r_0]^s+\pi^su^s)x_n\equiv 0\mod \pi^n$ for all odd $s$, since $[r_0]+\pi u$ divides $[r_0]^s+\pi^su^s$ for odd $s$. Now $\pi^sx_n\equiv -[r_0]^su^{-s}x_n\mod \pi^n$ shows that the first $n$ coefficients in $\pi$-adic expansion of $\pi^sx_n$ must be divisible by $r_0^s$. In other words, we can write
	\begin{equation*}
		x_n=[r_0^sy_0]+[r_0^sy_1]\pi+\dotsb+[r_0^sy_{n-s-1}]\pi^{n-s-1}+\pi^{n-s}z\,.
	\end{equation*}
	Thus, $x_n\in(\pi^{n-s},[r_0]^s)$ for all odd $s$. Choosing $s$ roughly equal to $n/2$, we see that $(x_n)_{n\in \IN}$ converges with respect to the ideals $\{(\pi^m,[r_0]^m)\}_{m\geq 1}$. But these ideals generate $(\pi,\xi)$-adic topology, as seen in the proof of \cref{lem*:nonTrivial}. 
\end{proof*}


\cref{rem:perfectoid}\itememph{4} suggests the following definition.
\begin{defi}
	Let $R$ be a perfect $\IF_q$-algebra. An \defemph{untilt} of $R$ is a pair $(A,\iota)$, where $A$ is a perfectoid $\Oo_E$-algebra and $\iota$ an isomorphism $\iota\colon R\isomorphism A^\flat$.
\end{defi}
Again by \cref{rem:perfectoid}\itememph{4} we get a bijection
\begin{equation*}
	\left\{\begin{tabular}{c}
		isomorphism classes of\\
		untilts $(A,\iota)$ of $R$
	\end{tabular}\right\}\lrisomorphism \left\{\begin{tabular}{c}
	ideals $I\subseteq W_{\Oo_E}(R)$ such that\\ $(W_{\Oo_E}(R),I)$
	is a perfect prism over $\Oo_E$
	\end{tabular}\right\}
\end{equation*}
\begin{exc}[Tilting equivalence]\label{exc:tilting}
	If $A$ is a perfectoid $\Oo_E$-algebra, then there is an equivalence of categories
	\begin{align*}
		\left\{\text{perfectoid $A$-algebras}\right\}&\lrisomorphism \left\{\text{perfect(oid) $A^\flat$-algebras}\right\}\\
		B&\longmapsto B^\flat\\
		W_{\Oo_E}(S)\otimes_{W_{\Oo_E}(A^\flat)}A&\longmapsfrom S
	\end{align*}
	(on the left-hand side, $A$ gets a $W_{\Oo_E}(A^\flat)$-algebra structure via $\theta$).
\end{exc}
\begin{proof*}[Disproof]
	%Let $T=W_{\Oo_E}(S)\otimes_{W_{\Oo_E}(A^\flat)}A$. We first check $T^\flat\cong S$. We have 
	%\begin{equation*}
	%	T/\pi T\cong S\otimes_{A^\flat}A/\pi A\,.
	%\end{equation*}
	The assertion as stated is wrong. Take $A^\flat=\IF_p\llbracket T^{1/p^\infty}\rrbracket$ and $A$ comes from the perfect prism $(W(A^\flat),T-p)$. This works by \cref{lem*:nonTrivial} since $T-p$ is clearly distinguished and $A^\flat$ is $T$-complete. We claim that there is a perfect $A^\flat$-algebra $S$ such that $W(S)\otimes_{W(A^\flat)}A$ is not perfectoid. Indeed, for it to be perfectoid, $(W(S),(T-p)W(S))$ would need to be a perfect prism, which again needs $S$ to be $T$-complete by \cref{lem*:nonTrivial} again. However, there are perfect $A^\flat$ algebras $S$ which are not $T$-complete; for example, the Laurent series ring $S=\IF_p(\!(T^{1/p^\infty})\!)$.
\end{proof*}
\numpar{Corrected exercise* \textmd{(The actual tilting equivalence)}}
By a \defemph{perfectoid $A^\flat$-algebra} $S$ we don't just understand an $A^\flat$-algebra that is perfectoid. The topology on $S$ must also be induced by the topology on $A^\flat$, i.e., $S$ must be $(\pi,I)$-complete, where $I$ is the kernel of $\theta\colon W_{\Oo_E}(A^\flat)\morphism A$ (so that $(W_{\Oo_E}(A^\flat),I)$ is a perfect prism that gives $A$). Then there is an equivalence of categories
\begin{equation*}
	\left\{\text{perfectoid $A$-algebras}\right\}\lrisomorphism \left\{\text{perfect $A^\flat$-algebras}\right\}
\end{equation*}
as in \cref{exc:tilting}.
\begin{proof*}
	Put $W_A=W_{\Oo_E}(A^\flat)$ and $W_S=W_{\Oo_E}(S)$ for convenience. Let $\xi$ be a distinguished generator of $I$. First note that $W_S\otimes_{W_A}A\cong W_S/\xi W_S$ is again perfectoid. Indeed, we need to check that $(W_S,\xi W_S)$ is a perfect prism. Clearly $\xi W_S$ is a distinguishedly generated ideal. Also $S$ is $(\pi,I)$-complete and hence $\xi$-complete, so $W_S$ is $(\pi,\xi W_S)$-complete by \cref{lem*:nonTrivial}. This shows that $(W_S,\xi W_S)$ is a perfect prism, as required. Now the calculation from \cref{rem:perfectoid}\itememph{4} shows $(W_S/\xi W_S)^\flat\cong S$.
	
	Conversely, we have to show that for a perfectoid $A$-algebra $B$ we get $B\cong W_B\otimes_{W_A} A$, where $W_B=W_{\Oo_E}(B^\flat)$ for brevity, and that $B^\flat$ is $(\pi,I)$-complete. Write $B\cong W_B/J$. Then  $(W_A,I)\morphism (W_B,J)$ is a morphism of perfect prisms in the sense that it is a $\Oo_E$-algebra morphism that maps $I$ into $J$. An argument analogous to the stolen one from the proof of \cref{lem*:perfectoid=perfect} (hint: replace $1$ be the coefficient of $\pi$ in $\xi$, which is still a unit) shows that actually $J=IW_B$. But this immediately shows $B\cong W_B\otimes_{W_A}A$ and we are done.
\end{proof*}

\numpar{Example \smash{\Attention}}\label{exm:OCperfectoid}
If $C/E$ is a non-archimedean (recall that this requires $C$ to be complete) algebraically closed field extension, then the ring of integers $\Oo_C$ is a perfectoid $\Oo_E$-algebra.
\begin{proof}
	We first formulate two claims which together will imply the assertion.
	\begin{numerate}
		\item Let $\{\pi^{1/q^n}\}_{n\geq 0}$ be a compatible system of $(q^n)\ordinalth$ roots of $\pi$ in $\Oo_C$. They define an element $\pi^\flat=(\pi,\pi^{1/q},\dotsc)\in\Oo_C^\flat$. Then 
		\begin{equation*}
			\Oo_C^\flat/\pi^\flat\Oo_C^\flat\cong\Oo_C/\pi\Oo_C\,.
		\end{equation*}
		\item The kernel of $\theta\colon W_{\Oo_E}(\Oo_C^\flat)\morphism\Oo_C$ is generated by $\pi-[\pi^\flat]$.
	\end{numerate}
	We start with \itememph{1}. Note that by \cref{prop:(A/I)b} we may write $\Oo_C^\flat\cong \limit_{x\mapsto x^q}\Oo_C$. Now let $y=(y_0,y_1,\dotsc)\in\Oo_C^\flat$. Then $\pi^\flat\mid y$ iff $\pi^{1/q^n}\mid y_n$ for all $n\geq 0$. Since $\Oo_C$ is a valuation ring, this is equivalent to $|\pi|^{1/q^n}\geq |y_n|=|y_0|^{1/q^n}$. Thus, $\pi^\flat\mid y$ is equivalent to the single condition $y_0\equiv 0\mod \pi$. Therefore, the kernel of $(-)^\sharp\colon \Oo_C^\flat\morphism \Oo_C/\pi\Oo_C$ is generated by $\pi^\flat$. However, $\Oo_C^\flat\epimorphism \Oo_C/\pi\Oo_C$ is clearly surjective (since $C$ is algebraically closed), hence indeed
	\begin{equation*}
		\Oo_C^\flat/\pi^\flat\Oo_C^\flat\cong \Oo_C/\pi\Oo_C\,.
	\end{equation*}
	For \itememph{2}, \cref{lem:WAb->A} shows $\theta(\pi-[\pi^\flat])=\pi-(\pi^\flat)^\sharp=\pi-\pi=0$. So $\pi-[\pi^\flat]\in\ker\theta$. Conversely, let $x=\sum_{n=0}^\infty[x_n]\pi^n$ be an element of $\ker\theta$. Hence
	\begin{equation*}
		0\equiv \theta(x)\equiv \sum_{n=0}^\infty x_n^\sharp\pi^n\equiv x_0^\sharp\mod \pi\,.
	\end{equation*}
	From \itememph{1} we get $\pi^\flat\mid x_0$, say, $x_0=\pi^\flat y$. Write $z^{(0)}=\sum_{n=1}^\infty [x_n]\pi^{n-1}$ and $x^{(1)}=[y]+z^{(0)}$. Then $x=[\pi^\flat]x^{(1)}+(\pi-[\pi^\flat])z^{(0)}$. We obtain
	\begin{equation*}
		0=\theta(x)=\theta\big([\pi^\flat]x^{(1)}\big)=\pi\theta\big(x^{(1)}\big)\,,
	\end{equation*}
	hence also $\theta(x^{(1)})=0$ since $\Oo_C$ is $\pi$-torsionfree. Repeating this process with $x^{(1)}$ and iterating, we get an expression
	\begin{equation*}
		x=\xi\big(z^{(0)}+[\pi^\flat]z^{(1)}+\dotsb\big)\,,
	\end{equation*}
	where $\xi=\pi-[\pi^\flat]$. This shows that $x$ lies in the ideal generated by $\xi$, proving \itememph{2}.
	
	It remains to see that $\theta\colon W_{\Oo_E}(\Oo_C^\flat)\morphism\Oo_C$ is surjective and that $(W_{\Oo_E}(\Oo_C^\flat),\xi)$ is a perfect prism. The first assertion is because $(-)^\sharp\colon \Oo_C^\flat\morphism\Oo_C$ is surjective since $C$ is algebraically closed. For the second assertion, $\xi=\pi-[\pi^\flat]$ is clearly distinguished by \cref{rem:perfectoid}\itememph{2}, so it remains to show that $\Oo_C^\flat$ is $\pi^\flat$-complete. Observe that for all $c\geq 0$ the $c\ordinalth$ component of $(\pi^\flat)^{q^n}$ is $0$ for all $n\geq c$. From this observation, $\pi^\flat$-completeness of $\Oo_C^\flat$ easily follows.
\end{proof}
Next time we proof the first half of the following \cref{lem:perfectoidOC} (see \cref{lem:OcflatisOF}). The other half will have to wait until the $4\ordinalth$ lecture.
\begin{lem}\label{lem:perfectoidOC}
	Let $A$ be a perfectoid $\Oo_E$-algebra. Then $A$ is isomorphic to $\Oo_C$ for some non-archimedean algebraically closed extension $C/E$ if and only if $A^\flat$ is isomorphic to $\Oo_F$ for some non-archimedean algebraically closed extension $F/\IF_q$.
\end{lem}
\begin{rem}\label{rem:AinfProperties}
	Recall that for $F$ as in \cref{lem:perfectoidOC} we put $\IA_\inf=W_{\Oo_E}(\Oo_F)$.
	\begin{numerate}
		\item $\IA_\inf$ is a local integral domain. This is in fact true for any $W_{\Oo_E}(R)$ if $R$ itself is a local integral domain over $\IF_q$ (this follows from \cref{lem:W_OEpi} for example).
		\item $\IA_\inf$ is $(\pi,[\varpi])$-complete for any $\varpi\in\mm_F\setminus \{0\}$. Indeed, this follows from \cref{rem:perfectoid}\itememph{2} as $\Oo_F$ is easily seen to be $\varpi$-complete. Such $\varpi$ is called a \defemph{pseudo-uniformizer}.
		\item By a theorem of Ludwig--Lang, $\IA_\inf$ has infinite Krull dimension (and is, in particular, non-noetherian). We can actually see by hand that $\IA_\inf$ is at least three-dimensional: there is a chain
		\begin{equation*}
			0\subsetneq\bigcup_{x\in\mm_F}[x]\IA_\inf\subsetneq W_{\Oo_E}(\mm_F)\subsetneq (\pi,W_{\Oo_E}(\mm_F))
		\end{equation*}
		of prime ideals. Also note that $(\pi,W_{\Oo_E}(\mm_F))$ is the unique maximal ideal of $\IA_\inf$ since an element of $\IA_\inf$ is invertible iff its image in $\IA_\inf/\pi\IA_\inf\cong \Oo_F$ is invertible.
	\end{numerate}
\end{rem}
Despite \cref{rem:AinfProperties}\itememph{3}, we should think of $\IA_\inf$ as a two-dimensional ring, except for some \enquote{bad} primes. Here's a \enquote{picture} of $\Spec\IA_\inf$. The left picture shows a select choice of prime ideals of $\IA_\inf$. In the right picture the corresponding residue fields are shown and the Frobenius action $\phi$ is indicated.
\begin{center}
	\tabcolsep=0pt
	\begin{tabularx}{\textwidth}{X c X c X}
		& \begin{tikzpicture}[line width=rule_thickness, line cap=round, line join =round, x=1cm,y=1cm]
		\draw[-to] (0,0) -- (5,0) node[below] {$[\varpi]$};
		\draw[-to] (0,0) -- (0,5) node[left] {$\pi$};
		\fill (0,0) circle (0.5ex) node[below=4] (null) {$(\pi,W_{\Oo_E}(\mm_F))$};
		\fill (4,0) circle (0.5ex) node[below=4] (pi) {$(\pi)$};
		%\path (null) -- (pi) node[pos=0.5] {\scriptsize$\Spec W_{\Oo_E}(k)$};
		\fill (0,4) circle (0.5ex) node[right=4] {$W_{\Oo_E}(\mm_F)$};
		\draw[rounded corners, thick] (-1ex,-1ex) rectangle (4cm+1ex,1ex) node[pos=0.5, below=4] {\scriptsize$\Spec \Oo_F$};
		\draw[rotate=30,rounded corners, thick] (-1ex,-1ex) rectangle (3.5cm+1ex,1ex) node[pos=0.5,rotate=30] {\scriptsize$\Spec \Oo_C$};
		\draw[pattern=north west lines, rounded corners, thick] (1ex,-1ex) rectangle (-1ex,4cm+1ex) node[pos=0.5, rotate=90, above=4] {\scriptsize$\Spec W_{\Oo_E}(k)$} node[pos=0.5, rotate=90, below=4] {\scriptsize \enquote{bad} primes};
		\fill (30:3.5cm) circle (0.5ex) node[right=4,align=left] {$(\pi-[\pi^\flat])$,\\[0.5ex] $\Oo_C$ untilt of $\Oo_F$};
		\draw[rotate=60,rounded corners, thick] (-1ex,-1ex) rectangle (3.5cm+1ex,1ex) node[pos=0.5, rotate=60] {\scriptsize$\Spec \Oo_{\smash{C'}}$};
		\fill (60:3.5cm) circle (0.5ex) node[right=4, align=left] {$(\pi-[a])$,\\[0.5ex] $a\in\mm_F\setminus\{0\}$};
		\end{tikzpicture} & & \begin{tikzpicture}[line width=rule_thickness, line cap=round, line join =round, x=1cm,y=1cm]
		\draw[-to] (0,0) -- (5,0) node[below] {$[\varpi]$};
		\draw[-to] (0,0) -- (0,5) node[left] {$\pi$};
		\fill (0,0) circle (0.5ex) node[below=4] (null) {$k$};
		\fill (4,0) circle (0.5ex) node[below=4] (pi) {$F$};
		%\path (null) -- (pi) node[pos=0.5] {\scriptsize$\Spec W_{\Oo_E}(k)$};
		\fill (0,4) circle (0.5ex) node[right=4] {$W_{\Oo_E}(k)\left[\frac1\pi\right]$};
		\draw[rounded corners, thick] (-1ex,-1ex) rectangle (4cm+1ex,1ex);
		\draw[rotate=30,rounded corners, thick] (-1ex,-1ex) rectangle (3.5cm+1ex,1ex);
		\draw[pattern=north west lines, rounded corners, thick] (1ex,-1ex) rectangle (-1ex,4cm+1ex);
		\fill (30:3.5cm) circle (0.5ex) node[right=4] {$C$};
		\draw[->, shift={(0,0)}] (30:4.5cm) arc (30:50:4.5cm) node[pos=0.5,above right] {$\phi$};
		\end{tikzpicture} &
	\end{tabularx}
\end{center}
We put $k=\Oo_F/\mm_F$ for convenience. We will see next time that $\Spec\IA_\inf$ is indeed \enquote{two-dimensional away from $[\varpi]=0$}. More precisely, we will show the following: let $(\Oo_C,\iota)$ be an untilt of $\Oo_F$ and $\xi$ a generator of $\ker(\theta\colon \IA_\inf\morphism \Oo_C)$. Put
\begin{equation*}
	B_\dR^+=\IA_\inf\left[\textstyle \frac 1\pi\right]_\xi^\complete\,.
\end{equation*}
Then $B_\dR^+$ is always a DVR and the same is true for $\IA_{\inf,(\pi-[\pi^\flat])}$ (see \cref{lem:BdR+DVR} below). Moreover, in the lecture after the next one we will show that all $(\pi-[a])$ for $a\in\mm_F\setminus\{0\}$ are prime ideals, and in fact $\IA_\inf/(\pi-[a])$ is isomorphic to another untilt $\Oo_{C'}$ of $\Oo_F$ (as indicated in the left picture), with $C'/E$ an algebraically closed non-archimedean extension.

\numpar{Side remark}
Why this setup? Let $K/\IQ_p$ be a discretely valued non-archime-\lecture[$B_\dR^+$ is a DVR. A universal property for $\IA_\inf$. $p$-adic PD-thickenings and $\IA_\cris$.]{2019-11-06}
dean field extension with perfect residue field and let $X/K$ be a smooth proper scheme.  The objects of interest in $p$-adic hodge theory are the $p$-adic cohomology groups $H_\et^*(X_{\ov{K}},\IQ_p)$. We will replace $\IQ_p$ by $E$ and $\ov{K}$ by $C=\roof{\ov{K}}$, with $F=C^\flat=\Frac(\Oo_C^\flat)$.
\begin{defi}
	An element $x=\sum_{n=0}^\infty [x_n]\pi^n$ of $\IA_\inf$ is called \defemph{primitive} if $x_0\neq 0$ and there exists a $d\geq 0$ such that $x_d\in \Oo_F^\times$. If $x$ is primitive, the smallest such $d$ is called the \defemph{degree} of $x$. The set primitive elements of degree $d$ is denoted $\Prim_d$.
\end{defi}
\begin{exm}
	We have $\Prim_0=\IA_\inf^\times$. Moreover, any element $x\in \Prim_1$ is distinguished. The converse is true iff $[x_0]\neq 0$.
\end{exm}
Next time we will see that if $a\in \Prim_1$, then $a\IA_\inf$ is  a prime ideal and $\IA_\inf/a\IA_\inf\cong \Oo_C$ for some non-archimedean algebraically closed extension $C/E$ (which generalizes the claim about the $(\pi-[a])$ above). For now let $C/E$ be such an extension and $|\blank|\colon C\morphism \IR_{\geq 0}$ its norm. Recall that \cref{prop:(A/I)b} provides an isomorphism
\begin{equation*}
	\Oo_C^\flat\cong \limit_{x\mapsto x^q}\Oo_C\,,
\end{equation*}
sending an element $x\in \Oo_C^\flat$ of the left-hand side to $(x^\sharp, (x^{1/q})^\sharp,\dotsc)$ contained in the right-hand side.
\begin{lem}\label{lem:OcflatisOF}
	Assume we are in the above situation.
	\begin{numerate}
		\item The map $|\blank|^\flat\colon \Oo_C^\flat\morphism\IR_{\geq 0}$ given by $x\mapsto |x^\sharp|$ is a norm on $\Oo_C^\flat$. Moreover, $\Oo_C^\flat$ is complete with respect to the topology induced by $|\blank|^\flat$.
		\item $C^\flat=\Frac(\Oo_C^\flat)$ is a non-archimedean algebraically closed extension of $\IF_q$.
	\end{numerate}
\end{lem}
\begin{proof}
	It is clear that $|\blank|^\flat$ is multiplicative, that $|1|^\flat=1$, and that $|x|^\flat=0$ iff $x=0$. So only the triangle inequality remains. We calculate
	\begin{align*}
		|x+y|^\flat=|(x+y)^\sharp|&=\lim_{n\to\infty}\bigg|\Big(\big(x^{1/q^n}\big)^\sharp+\big(y^{1/q^n}\big)\Big)^{q^n}\bigg|\\
		&=\lim_{n\to\infty}\max\left\{\big|\big(x^{1/q^n}\big)^\sharp\big|^{q^n},\big|\big(y^{1/q^n}\big)^\sharp\big|^{q^n}\right\}\\
		&=\lim_{n\to\infty}\max\left\{|x^\sharp|,|y^\sharp|\right\}\\
		&=\max\big\{|x|^\flat,|y|^\flat\big\}
	\end{align*}
	This shows that $|\blank|^\flat$ is a norm in $\Oo_C^\flat$. To show that $\Oo_C^\flat$ is complete, we claim that the topology generated by $|\blank|^\flat$ is the inverse limit topology on $\Oo_C^\flat\cong \limit_{x\mapsto x^q}\Oo_C$. A neighbourhood basis of $0$ in the topology generated by $|\blank|^\flat$ is given by the sets
	\begin{equation*}
		\big\{x\ \big|\ |x|^\flat<\epsilon\big\}\quad\text{for all }\epsilon>0\,.
	\end{equation*}
	In the inverse limit topology, a neighbourhood basis of $0$ is given by the sets
	\begin{equation*}
		\left\{x\in \Oo_C^\flat\st \big|\big(x^{1/q^n}\big)^\sharp\big|<\delta\right\}\quad\text{for all }\delta>0\text{, }n\geq 0\,.
	\end{equation*}
	But $|(x^{1/q^n})^\sharp|=(|x|^\flat)^{1/q^n}$, so its easy to see that these topology bases not only generate the same topology, but even coincide on the nose.
	
	
	For \itememph{2}, it remains to show that $C^\flat$ is algebraically closed, and for this it suffices to show that $\Oo_C^\flat$ is integrally closed. So let $f\in \Oo_C^\flat[T]$ be a monic polynomial. Write $f(T)=T^d+a_{d-1}T^{d-1}+\dotsb+a_0$. For all $n\geq 0$ put
	\begin{equation*}
		f_n(T)=T^d+\big(a_{d-1}^{1/q^n}\big)^\sharp T^{d-1}+\dotsb+\big(a_0^{1/q^n}\big)^\sharp\in \Oo_C[T]\,.
	\end{equation*}
	Then $f_{n+1}(T)^q\equiv f_n(T^q)\mod \pi$. Now fix $n\geq 0$ and let $x\in \Oo_C$ be a zero of $f_n$, which exists as $\Oo_C$ is integrally closed. Choose $y\in \Oo_C$ such that $y^q=x$. Although $y$ need not be a root of $f_{n+1}$, we certainly have $|f_{n+1}(y)|\leq |\pi|^{1/q}$. Let $z_1,\dotsc,z_n\in \Oo_C$ be the actual roots of $f_{n+1}$. Then
	\begin{equation*}
		|f_{n+1}(y)|=\prod_{i=1}^d|y-z_i|\leq |\pi|^{1/q}\,.
	\end{equation*}
	hence there exists an index $i$ such that $|y-z_i|\leq |\pi|^{1/dq}$, or equivalently $|y-z_i|^q\leq |\pi|^{1/d}$. Then also $|x-z_i^q|\leq |\pi|^{1/d}$ as all other terms in the expansion of $(y-z_i)^q$ are divisible by $\pi$. By induction, we obtain a sequence $(x_n)_{n\in \IN}$ such that $x_n\in\Oo_C$, $f_n(x_n)=0$, and the $x_n$ are \enquote{close} to being $q$-power compatible in the sense that $|x_{n+1}-x_n^q|\leq |\pi|^{1/d}$. But this is actually sufficient! Indeed, put $\aa=\left\{y\in \Oo_C\st |y|\leq |\pi|^{1/d}\right\}$. Then $x=(x_n)_{n\in\IN}$ is an element of
	\begin{equation*}
		\limit_{x\mapsto x^q}\Oo_C/\aa\cong \limit_{x\mapsto x^q}\Oo_C/\pi\Oo_C=\Oo_C^\flat\,,
	\end{equation*}
	where we use \cref{prop:(A/I)b} to obtain the isomorphism on the left. Hence $x$ corresponds to an element $x\in \Oo_C^\flat$, which clearly satisfies $f(x)=0$.
\end{proof}
\begin{lem}\label{lem:BdR+DVR}
	Let $\Oo_C$ be an untilt of $\Oo_F$ and let $\xi$ be a distinguished generator of the kernel of $\theta\colon \IA_\inf\morphism\Oo_C$. As above, we put $B_\dR^+=\IA_\inf\localize{\pi}_\xi^\complete$. Then the following holds.
	\begin{numerate}
		\item The canonical map $\IA_\inf\monomorphism B_\dR^+$ is an injection.
		\item $B_\dR^+$ and $\IA_{\inf,(\xi)}$ are discrete valuation rings.
	\end{numerate}
\end{lem}
\begin{proof}
	Since we are not in a noetherian setting, we need to be careful with completion. As $(\xi)$ is obviously a finitely generated ideal, \cite[\stackstag{05GG}]{stacks-project} shows that $B_\dR^+$ is $\xi$-complete. Moreover,
	\begin{equation*}
		B_\dR^+/(\xi^n)\cong \IA_\inf\big[\textstyle\frac1\pi\big]/(\xi^n)\cong \IA_\inf/(\xi^n)\big[\textstyle\frac1\pi\big]\,,
	\end{equation*}
	by exactness of localization. We claim that $\IA_\inf/(\xi^n)\monomorphism \IA_\inf/(\xi^n)\localize{\pi}$ is injective for all $n$. To show this, we need to check that $\IA_\inf/(\xi^n)$ is $\pi$-torsionfree. We use induction on $n$. For $n=1$ we get $\IA_\inf/(\xi)\cong \Oo_C$, which is $\pi$-torsionfree. Now suppose $\pi x=\xi^ny$ for some $x,y\in\IA_\inf$. By the $n=1$ case we see that $x$ must be divisible by $\xi$, say, $x=\xi x'$. Since $\IA_\inf$ is a domain this implies $\pi x'=\xi^{n-1}y$. But then the induction hypothesis shows that $x'$ itself must be divisible by $\xi^{n-1}$, proving the claim.
	Now since limits are left exact, we see that
	\begin{align*}
		\IA_\inf\cong \limit_{n\in \IN}\IA_\inf/(\xi^n)\monomorphism \limit_{n\in\IN}\Big(\IA_\inf/(\xi^n)\localize{\pi}\Big)\cong B_\dR^+
	\end{align*}
	is injective, as required. The isomorphism on the left-hand side uses that $\IA_\inf$ is $\xi$-complete by \cite[\stackstag{09OT}]{stacks-project} and the fact that $\IA_\inf$ is $(\pi,\xi)$-complete. This shows \itememph{1}.
	
	For \itememph{2}, first note that $B_\dR^+/(\xi)\cong \Oo_C\localize{\pi}\cong C$. Hence \cite[\stackstag{05GH}]{stacks-project} implies that $B_\dR^+$ is noetherian. Moreover, we know that $B_\dR^+$ is local with maximal ideal $(\xi)$, because it is $\xi$-adically complete and its quotient by $\xi$ is $C$, which is a field. This implies $\dim B_\dR^+\leq 1$. Moreover, we are done once we show $\dim B_\dR^+\geq 1$, since then $B_\dR^+$ is a one-dimensional noetherian local ring whose maximal ideal is principal, hence regular, hence a DVR.
	
	For $\dim B_\dR^+\geq 1$ it suffices to see that $B_\dR^+$ is a domain, since then $0\subsetneq (\xi)$ is a chain of prime ideals. From \itememph{1} and the fact that $\IA_\inf$ is a domain, it's easy to see that $B_\dR^+$ is $\xi$-torsionfree. Now if $xy=0$ for $x,y\in B_\dR^+$, then $x$ or $y$ must be divisible by $\xi$ as $B_\dR^+/(\xi)\cong C$. Say $x=\xi x'$. Then $B_\dR^+$ being $\xi$-torsionfree shows $x'y=0$. Iterating the argument shows $x=0$ or $y=0$ as $B_\dR^+$ is $\xi$-complete. This finishes the proof that $B_\dR^+$ is indeed a DVR.
	
	Now for $\IA_{\inf,(\xi)}$. Take any prime ideal $\pp\subseteq \IA_{\inf,(\xi)}$ such that $\xi\notin \pp$. Still $\pp\subseteq (\xi)$ as $(\xi)$ is the maximal ideal of $\IA_{\inf,(\xi)}$. Hence, if $a\in \pp$, then $a=b\xi$. But since $\pp$ is prime and $\xi\notin\pp$, this implies $b\in\pp$. Thus $\xi\pp=\pp$. Now let $\qq=\pp B_\dR^+$. Then $\xi\qq=\qq$ shows $\qq=0$ as $B_\dR^+$ is a DVR. But $\IA_{\inf,(\xi)}\monomorphism B_\dR^+$ is injective by \itememph{1} as localizations of injections stay injective. This shows $\pp=0$.
	
	What we have shown is that $\Spec \IA_{\inf,(\xi)}$ has exactly two points, namely $\{0,(\xi)\}$. But then all prime ideals of $\IA_\inf$ are finitely generated, which implies that $\IA_\inf$ is noetherian by the rather obscure fact \cite[\stackstag{05KG}]{stacks-project}. Now it's clear that $\IA_{\inf,(\xi)}$ is one-dimensional and regular, hence a DVR.
\end{proof}

Have you ever wondered what the \enquote{$\inf$} in $\IA_\inf$ actually means? It stands for \emph{infinitesimal}. In fact, this leads to a description of $\IA_\inf$ as a universal thickening of $\Oo_C$!
\begin{defi}
	Let $R$ be a $\pi$-complete $\Oo_E$-algebras. A \defemph{$\pi$-adic pro-infinitesimal thickening of $R$} is a  surjection $D\epimorphism R$ of $\Oo_E$-algebras with kernel $I$ such that $D$ is $(\pi,I)$-adically complete.
\end{defi}
\begin{exm}
	For $R\in\left\{\Oo_C,\Oo_C/\pi\Oo_C\right\}$, the natural map $\IA_\inf\epimorphism R$ is a $\pi$-adic pro-infinitesimal thickening. Indeed, its kernel is given by $(\xi)$ and $(\pi,\xi)$ respectively. Actually, $\IA_\inf$ is the universal $\pi$-adic pro-infinitesimal thickening of $R$, as shown in the following lemma!
\end{exm}
\begin{lem}\label{lem:AinfUniversal}
	Let $R\in\{\Oo_C,\Oo_C/\pi\Oo_C\}$ and let $D\epimorphism R$ be a $\pi$-adic pro-infinitesimal thickening. Then it factors uniquely as
	\begin{equation*}
		\begin{tikzcd}
			\IA_\inf\rar[epi]\dar[dashed,"\exists!"{swap}] & R\\
			D\urar[epi] &
		\end{tikzcd}
	\end{equation*}
\end{lem}
\begin{proof}[Sketch of a proof]
	By \cref{prop:(A/I)b} we have $\limit_{x\mapsto x^q}D\cong (D/(\pi,I))^\flat\cong R^\flat$. By the same argument $\limit_{x\mapsto x^q} D\cong D^\flat$. Hence $D^\flat\cong R^\flat$. Thus, the Witt-tilting adjunction (\cref{prop:tiltWittAdjunction}) provides a unique map 
	\begin{align*}
		\IA_\inf\cong W_{\Oo_E}(R^\flat)\morphism D\,.
	\end{align*}
	It's easily verified that this map has the required properties.
\end{proof}
\subsection{\texorpdfstring{$p$}{p}-adic PD-thickenings and \texorpdfstring{$\IA_\cris$}{Acris}}\label{subsec:Acris}
From now on, we restrict our attention to the case $E=\IQ_p$ and $\pi=p$. As above, let $R\in\{\Oo_C,\Oo_C/p\Oo_C\}$.
\begin{defi}
	A \defemph{$p$-adic PD-thickening} of $R$ is a triple $(D,D\epimorphism R,(\gamma_n)_{n\in\IN})$, where $D$ is $p$-complete and $(\gamma_n)_{n\in\IN}$ a PD-structure on $J=\ker(D\epimorphism R)$ which is compatible with the canonical PD-structure on $pR$.
\end{defi}
\begin{rem}
	\begin{numerate}
		\item If $D$ is $p$-torsionfree, then necessarily $\gamma_n(x)=x^n/n!$.
		\item Normalize $|\blank|\colon C\morphism \IR_{\geq 0}$ such that $|p|=p^{-1}$. Then a well-known calculation shows $|n!|\geq p^{(n-1)/(p-1)}$ for all $n\in\IN$. In fact, the $1$ in $n-1$ can be replaced by the digit sum of the $p$-adic expansion of $n$. Thus, it's easy to check that
		$|x^n/n!|\leq 1$ for all $n\in\IN$ iff $|x|<p^{-1/(p-1)}$. Moreover, one may check that
		\begin{equation*}
			\left\{x\in\Oo_C\st|x|<p^{-1/(p-1)}\right\}
		\end{equation*}
		is the largest ideal in $\Oo_C$ admitting divided powers.
	\end{numerate}
\end{rem}
\begin{defi}
	The ring $\IA_\cris$ denotes the universal $p$-adic PD-thickening of $\Oo_C$, or equivalently, of $\Oo_C/p\Oo_C$. In fancy words,
	\begin{equation*}
		\IA_\cris=H_\cris^0(\Oo_C/\IZ_p)\cong H_\cris^0\big((\Oo_C/p\Oo_C)/\IZ_p\big)\,.
	\end{equation*}
\end{defi}
Concretely, $\IA_\cris$ is the $p$-adic divided power envelope of $\ker\theta=(\xi)\subseteq \IA_\inf$. This follows more or less from \cref{lem:AinfUniversal}, but this requires an additional argument, since a $p$-adic PD-thickening $D$ of $R$ need not be $(p,J)$-complete, so $D\epimorphism R$ need not be a $p$-adic pro-infinitesimal thickening. But the conclusion of that lemma is still true: we get a unique map $\IA_\inf\morphism D$ over $R$, and then its formal to see that $\IA_\cris$ can be described as above.

So where does the map $\IA_\inf\morphism D$ come from? A closer inspection of the proof of \cref{lem:AinfUniversal} shows that we only need to show that $D^\flat\morphism R^\flat$ is an isomorphism. We can't use \cref{prop:(A/I)b} to prove this. However, we can still construct an inverse $R^\flat\morphism D^\flat$ in the same way as in the proof of that proposition. This is based on the following observation, that serves as a replacement for \cref{lem:keyLemma}.
\begin{lem*}
	If $x,y\in D$ such that $x\equiv y\mod (p,J)$, then $(x^{p^n}-y^{p^n})_{n\in\IN}$ converges to $0$ in the $p$-adic topology.
\end{lem*}
\begin{proof*}
	Observe that for $d\in J$ we have $d^t=t!\gamma_t(d)$, so $d^t$ is divisible by $p^{v_p(t!)}$. Now put $x=y+pz+d$, where $z\in R$ and $d\in J$. Then a typical term in the multinomial expansion of $x^{p^n}-y^{p^n}$ looks like
	\begin{equation*}
		\binom{p^n}{r,s,t}y^r(pz)^sd^t\,,
	\end{equation*}
	where $r+s+t=p^n$. Fix some $N>0$. If $t>p^N$, then the above consideration shows that $d^t$ is at least divisible by $p^N$ (we are very permissive here). If $t\leq p^N$, then the multinomial coefficient is at least divisible by $p^{n-N}$. Hence if $n\geq 2N$, every term will at least be divisible by $p^N$, and we're done.
\end{proof*}
Now that we know $\IA_\cris$ is the $p$-adic divided power envelope of $(\xi)$, we can write it down explicitly as
\begin{equation*}
	\IA_\cris\cong \IA_\inf\left[\frac{\xi^n}{n!}\st n\in\IN\right]_p^\complete\cong \IA_\inf\cotimes_{\IZ[x]}D_{\IZ[x]}(x)\,,
\end{equation*}
using that $\xi$ is a non-zero divisor in $\IA_\inf$. Also $-\cotimes_{\IZ[x]}-$ refers to the $p$-adic completed tensor product, with $\IZ[x]\morphism\IA_\inf$ sending $x\mapsto \xi$. Finally, the tensor factor on the right is defined as
\begin{equation*}
	D_{\IZ[x]}(x)=\IZ\langle x\rangle =\bigoplus_{n\in\IN}\IZ\left\{\frac{x^n}{n!}\right\}\,.
\end{equation*}
Then
\begin{equation*}
	D_{\IZ[x]}(x)_p^\complete\cong\big(\IZ[y_0,y_1,\dotsc]/(y_0-x,y_n^p-py_{n+1} \text{ for }n\in\IN)\big)_p^\complete\,.
\end{equation*}
In particular, we can calculate
\begin{equation*}
	\IA_\cris\cong \Oo_C/p\Oo_C\otimes_{\IF_p}\IF_p[y_1,y_2,\dotsc]/(y_1^p,y_2^p,\dotsc)\,.
\end{equation*}
Some intuition: the image of $\Spec \IA_\cris$ in $\Spec \IA_\inf$ is roughly described by the following picture.
\begin{center}
	\begin{tikzpicture}[line width=rule_thickness, line cap=round, line join =round, x=1cm,y=1cm]
	\draw[-to] (0,0) -- (5,0) node[below] {$[\varpi]$};
	\draw[-to] (0,0) -- (0,5) node[left] {$\pi$};
	\fill (0,0) circle (0.5ex);
	%\fill (4,0) circle (0.5ex);
	\fill (2,2) circle (0.5ex) node[right=4] {$(\xi)$};
	\fill (1.6,2.4) circle (0.4ex) node[above right=4] (1) {$\phi^{-1}(\xi)$};
	\fill (1.28,2.72) circle (0.32ex);
	\fill (1.024,2.976) circle (0.256ex);
	\fill (0.819,3.181) circle (0.205ex);
	\fill (0.655,3.345) circle (0.164ex);
	\fill (0.524,3.476) circle (0.13ex) node[above right=4] (n) {$\phi^{-n}(\xi)$};
	\path (1.south west) -- (n.south west) node[pos=0.5,sloped, above=4] {$\dotsc$};
	\draw[thick, rounded corners=2.5] (-1ex,-1ex) -- (0,-1.414ex) -- (1ex,-1ex) --  (2cm+1ex,2cm-1ex) -- (2cm+1.414ex,2) -- (2cm+1ex,2cm+1ex) -- (1ex,4cm+1ex) -- (0,4cm+1.414ex) -- (-1ex,4cm+1ex) -- cycle;
	\draw[->, shift={(0,0)}] (30:4.5cm) arc (30:50:4.5cm) node[pos=0.5,above right] {$\phi$};
	\path (0,0) -- (2,2) node[pos=0.5,sloped] {$\Spec \IA_\cris$};
	\fill (3,1) circle (0.5ex) node[right=4, align=left] {$(p-[a])$\\$a\in\mm_F\setminus\{0\}$};
	\end{tikzpicture}
\end{center}
Note that $\phi^{-1}(\xi)=(p-[p^\flat]^{1/p})$. Concretely, if $a\in\mm_F\setminus\{0\}$ such that $|a|\leq |p^\flat|^p=|p|^p$, then $(p-[a])\IA_\cris=(p)$. We may think of this as \enquote{$1-[a]/p\in\IA_\cris$}. And if $a=\phi^{-n}(p^\flat)$ for some $n\in \IN$, then $\IA_\inf\epimorphism\IA_\inf/(p-[a])$ factors over $\IA_\cris$.

Recall that $\IA_\inf$ should be thought of as a mixed characteristic analogue of $\Oo_F\llbracket z\rrbracket$. In fact, we see a similar picture for $\Oo_F\llbracket z\rrbracket$.
\begin{center}
	\begin{tikzpicture}[line width=rule_thickness, line cap=round, line join =round, x=1cm,y=1cm]
	\draw[-to] (0,0) -- (5,0) node[below] {$\varpi$};
	\draw[-to] (0,0) -- (0,5) node[left] {$z$};
	\fill (4,0) circle (0.5ex) node[below=4] {$(z)$};
	\fill (2.5,1.5) circle (0.5ex) node[above right=4] {$(z-a)$, $a\in\mm_F$};
	%\fill (20:3.75cm) circle (0.5ex) node[right=4, align=left] {$(p-[a])$\\\scriptsize for $a\in\mm_F\setminus\{0\}$};
	\draw[->, shift={(0,0)}] (35:4.5cm) arc (35:55:4.5cm) node[pos=0.5,above right] {$\phi$};
	\draw[rotate around={135:(4,0)},thick,rounded corners, shift={(4,0)}] (-1ex,-1ex) rectangle (5.25cm+1ex,1ex);
	\end{tikzpicture}
\end{center}
The surrounded area may be described as $\Prim_1/\Oo_F\llbracket z\rrbracket^\times\cong\mm_F=\left\{x\in F\st |x|<1\right\}$. This is also the \enquote{open rigid-analytic disc} $\ID_F$. It contains the \enquote{punctured disc} $\ID_F^*=\mm_F\setminus\{0\}$. Then the equal characteristic analogue of the Fargues--Fontaine curve is the quotient $\ID_F^*/\phi^\IZ$.

However, for $\IA_\inf$ the canonical map $\mm_F\epimorphism\Prim_1/\IA_\inf^\times$ sending $a\in\mm_F$ to $(\pi-[a])$ is not bijective! For example, $(\pi-[\pi^\flat])$ depends on choices of $(q^n)\ordinalth$ roots of $\pi$ to get $\pi^\flat=(\pi,\pi^{1/q},\dotsc)$.
\include{./Chapters/FF13to15}



\chapter{Classification of Vector Bundles on the Fargues--Fontaine Curve}
\section{The Vector Bundles \texorpdfstring{$\Oo_{X_\FFC}(\lambda)$}{OX(lambda)}}
\lecture[HN-filtrations and HN-polygons. $\phi$-modules over $\breve{E}$. Construction of the vector bundles $\Oo_{X_\FFC}(\lambda)$.]{2020-01-08}As usual, let $E/\IQ_p$ be a finite extension with uniformizer $\pi$ and residue field $\Oo_E/\pi \Oo_E=\IF_q$, and let $F/\IF_q$ a non-archimedean algebraically closed extension. 

\subsection{The Harder--Narasimhan Formalism}\label{subsec:HNFormalism}
Henceforth, the names Harder--Narasimhan will be abbreviated as HN. Let $\Cc$ be an exact category; roughly speaking, this is an additive category together with a notion of short exact sequences (for example, the categories $\cat{Bun}_{\IP_k^1}$ and $\cat{Bun}_{X_\FFC}$ of vector bundles on $\IP_k^1$ and on the Fargues--Fontaine curve $X_\FFC$ respectively). Moreover, we assume there are:
\begin{alphanumerate}
	\item a function $\rk\colon \Ob(\Cc)\morphism \IN_{\geq 0}$ (the \defemph{rank function}). In the case where $\Cc$ is a category of vector bundles on $\IP_k^1$ or $X_\FFC$, this is just what one would expect.
	\item a function $\deg \colon \Ob(\Cc)\morphism \IZ$ (the \defemph{degree function}). For vector bundles, we would take $\deg \Ee=\deg (\bigwedge^{\rk \Ee}\Ee)$. We have seen last time in \cref{prop:PicX=Z} that for the Fargues--Fontaine curve $X_\FFC$, $\deg \colon \Pic(X_\FFC)\isomorphism\IZ$ is an isomorphism. 
\end{alphanumerate}
Both $\rk$ and $\deg $have to be additive on short exact sequences in $\Cc$. Moreover, we require that there is an exact and faithful functor $F\colon \Cc\morphism \Aa$ (the \defemph{generic fibre functor} in the case where $\Cc$ equals $\cat{Bun}_{\IP_k^1}$ or $\cat{Bun}_{X_\FFC}$) into an abelian category $\Aa$, such that for all $\Ee\in \Cc$, the functor $F$ induces a bijection
\begin{equation*}
	F\colon \left\{\text{strict subobjects of }\Ee\right\}\isomorphism \left\{\text{subobjects of }F(\Ee)\right\}\,.
\end{equation*}
Here a \defemph{strict subobjects} means a monomorphism $\Ee'\monomorphism \Ee$ that is part of a short exact sequence in $\Cc$. Finally, we assume
\begin{numerate}
	\item $\rk\colon \Ob(\Cc)\morphism \IN_{\geq 0}$ is the restriction of another function $\rk\colon \Ob(\Aa)\morphism \IN_{\geq 0}$ along $F$, which again has to be additive on short exact sequences and satisfies $\rk V=0$ iff $V=0$ for all $V\in \Aa$.
	\item If $u\colon \Ee\morphism\Ee'$ is a morphism in $\Cc$ such that $F(u)$ is an isomorphism, then $\deg \Ee\leq \deg \Ee'$ with equality iff $u$ is an isomorphism.
\end{numerate}
%Again, both requirements are easily seen for $\Cc=\cat{Bun}_{\IP_k^1}$ or $\Cc=\cat{Bun}_X$ and $F$ the generic fibre functor.
\begin{defi}
	Let $\Ee\in \Cc$ be an arbitrary object.
	\begin{numerate}
		\item The \defemph{slope of $\Ee$} is the (possibly infinite) number $\mu(\Ee)\coloneqq \deg(\Ee)/\rk(\Ee)\in \IQ\cup\{\infty\}$.
		\item $\Ee$ is called \defemph{semistable} if for all non-zero strict subobjects $\Ff\subseteq \Ee$ we have $\mu(\Ff)\leq \mu(\Ee)$.
	\end{numerate}
\end{defi}
\begin{exm}
	If $\Cc=\cat{Bun}_{X_\FFC}$, then the twisting sheaves $\Oo_{X_\FFC}(n)$ for $n\in \IZ$ are semistable line bundles. Moreover, $\Ee=\Oo_{X_\FFC}(m)\oplus \Oo_{X_\FFC}(n)$ is semistable iff $m=n$. Indeed, its slope is $\mu(\Ee)=(m+n)/2$. Thus for $n\neq m$, either $\Oo_{X_\FFC}(m)$ or $\Oo_{X_\FFC}(n)$ is a subbundle of higher slope. Conversely, for $m=n$, every non-trivial strict subobject of $\Ee$ is of the form $\Oo_{X_\FFC}(m')\monomorphism \Oo_{X_\FFC}(m)$ for $m'\leq m$, hence has slope at most $m$.
\end{exm}
\begin{lem}\label{lem:HN}
	Let $\Ee,\Ee'\in \Cc$ be semistable objects of slopes $\lambda$, $\lambda'$ respectively. If $\lambda>\lambda'$, then we have
	\begin{equation*}
		\Hom_\Cc(\Ee,\Ee')=0\,.
	\end{equation*}
\end{lem}
\begin{proof*}
	Let $\Ee\morphism \Ee'$ be a morphism in $\Cc$. Since $\Aa$ is abelian, we can splice $F(\Ee)\morphism F(\Ee')$ into short exact sequences $0\morphism K\morphism F(\Ee)\morphism K'\morphism 0$ and $0\morphism K'\morphism F(\Ee')\morphism K''\morphism 0$. By our assumption on $F$, the objects $K$ and $K'$ correspond to strict subobjects $\Kk\subseteq \Ee$ and $\Kk'\subseteq \Ee'$. Since $F$ is exact, these guys satisfy $F(\Ee/\Kk)\cong K'\cong F(\Kk')$ and $F(\Ee'/\Kk')\cong K''$. Moreover, we claim that $\Ee\morphism \Ee'$ factors over $\Ee/\Kk$. Indeed, what we need to prove is that $\Kk\morphism\Ee'$ is the zero morphism. Since $F$ is faithful, this may be checked after applying $F$, and for $K\cong F(\Kk)\morphism F(\Ee')$ this is clearly true. In the same way we show that $\Ee\morphism \Ee'$ factors over $\Kk'$.
	
	Hence we get a canonical morphism $\Ee/\Kk\morphism \Kk'$. By construction, this becomes an isomorphism after applying $F$, so $\deg(\Ee/\Kk)\leq \deg(\Kk')$ and $\rk (\Ee/\Kk)=\rk (\Kk')$ by the above properties. In particular, we have $\mu(\Ee/\Kk)\leq \mu(\Kk')$. But $\Ee'$ and $\Ee$ are semistable, hence $\mu(\Kk')\leq \lambda'$ and $\mu(\Kk)\leq \lambda$, except for $\Kk'=0$ (in which case we are done) or $\Kk=0$ (which leads to $\lambda=\mu(\Ee)=\mu(\Ee/\Kk)\leq \lambda'$, a contradiction).
	
	So if no of these two special cases occurs, we get $\mu(\Kk)\leq \lambda$ and $\mu(\Ee/\Kk)<\lambda$. This however contradicts \cref{lem*:MuOnSES} below.
\end{proof*}
\begin{lem*}\label{lem*:MuOnSES}
	For any short exact sequence $0\morphism \Ee'\morphism \Ee\morphism \Ee''\morphism 0$ in $\Cc$ we have 
	\begin{equation*}
		\min\{\mu(\Ee'),\mu(\Ee'')\}\leq \mu(\Ee)\leq \max\{\mu(\Ee'),\mu(\Ee'')\}\,.
	\end{equation*}
	For the left half, equality holds iff $\mu(\Ee')=\mu(\Ee'')$ or one of $\Ee'$, $\Ee''$ is zero. Equality on the right holds iff $\mu(\Ee')=\mu(\Ee'')$.
\end{lem*}
\begin{proof*}
	Put $d'=\deg(\Ee')$, $d''=\deg(\Ee'')$ and $r'=\rk(\Ee')$, $r''=\rk(\Ee'')$. By additivity of $\deg$ and $\rk$ on short exact sequences, we obtain
	\begin{equation*}
		\mu(\Ee)=\frac{d'+d''}{r'+r''}=\frac{r'}{r'+r''}\cdot\mu(\Ee')+\frac{r''}{r'+r''}\cdot\mu(\Ee'')\,.
	\end{equation*}
	Thus, $\mu(\Ee)$ is a convex combination of $\mu(\Ee')$ and $\mu(\Ee'')$ and the inequality as well as the discussion of equality cases follow rather easily.
\end{proof*}

\begin{thm}\label{thm:HN}
	Each $\Ee\in \Cc$ has a unique functorial filtration, called \enquote{HN-filtration}, of the form
	\begin{equation*}
		0=\Ee_0\subsetneq \Ee_1\subsetneq\dotsb\subsetneq \Ee_r=\Ee
	\end{equation*}
	such that $\Ee_i/\Ee_{i-1}$ is semistable for $i=1,\dotsc,r$ and the sequene of slopes $\mu(\Ee_i/\Ee_{i-1})$ is strictly decreasing.
\end{thm}
\begin{proof}[Sketch of a proof]
	Before we start with the proof, we remark that \cref{lem*:MuOnSES} shows $\mu(\Ee_1)> \mu(\Ee_2)> \dotsc \geq \mu(\Ee)$ for any HN-filtration of $\Ee$ as above.
	
	If $F(\Ee)$ is simple in $\Aa$, i.e., has no non-zero subobjects, then $\Ee$ has no non-zero strict subobjects by our assumption on $F$, hence $\Ee$ is semistable for trivial reasons. Then $\Ee$ is its own HN-filtration. Thus we may assume that $F(\Ee)$ is non-simple, so there exists a short exact sequence $0\morphism \Ee'\morphism \Ee\morphism \Ee''\morphism 0$ with $\rk \Ee',\rk\Ee'' <\rk \Ee$. Using induction, we may assume that $\Ee'$ and $\Ee''$ have HN-filtrations. We claim:
	\begin{alphanumerate}
		\item[\itememph{*}] The slopes of strict subobjects of $\Ee$ are bounded.
	\end{alphanumerate}
	To prove \itememph{*}, we first use the above observation to see that it suffices to bound the slopes of stric semistable subobject, because if $\Ff\subseteq \Ee$ is a strict subobject, then it has an HN-filtration by the induction hypothesis, hence $\Ff_1$ is a strict semistable subobject of $\Ee$ satisfying $\mu(\Ff_1)\geq \mu(\Ff)$.\footnote{By the way, here we use that compositions of strict subobjects are strict subobjects again. We didn't mention this in our \enquote{definition} of exact categories, but it's actually one of the axioms.} So w.l.o.g.\ $\Ff=\Ff_1$. Let $0=\Ee''_0\subsetneq\dotsb\subsetneq \Ee_r''=\Ee''$ be the HN-filtration of $\Ee''$. If $\Ff\morphism \Ee\morphism \Ee''_r/\Ee''_{r-1}$ is non-zero, then $\mu(\Ff)\leq \mu(\Ee''_r/\Ee''_{r-1})$ by \cref{lem:HN}. Otherwise, $\Ff\morphism \Ee''$ factors over $\Ee''_{r-1}$. Repeating this argument shows $\mu(\Ff)\leq \mu(\Ee''_{r-1}/\Ee''_{r-2})$ or $\Ff\morphism \Ee''$ factors over $\Ee''_{r-2}$, and so on. So all in all $\mu(\Ff)$ is bounded by the slopes of the HN-filtration of $\Ee''$ or $\Ff\morphism \Ee''$ is zero. But in that case $\Ff\morphism \Ee$ factors over $\Ee'$. Then the argument can be repeated with $\Ee'$, showing that $\mu(\Ff)$ is bounded by the slopes of the HN-filtration of $\Ee'$, or $\Ff$ itself is zero, which is of course excluded. This proves \itememph{*}.
	
	Take $\Ee_1\subseteq \Ee$ a strict subobject of maximal slope, whose rank is also maximal among all strict subobjects of maximal slope. Such an $\Ee_1$ exists since the strict subobjects of $\Ee$ can have rank at most $\rk \Ee$ by additivity of $\rk$, so the denominators are bounded above. As noted above, $\Ee_1$ is necessarily semistable. Moreover, $\Ee/\Ee_1$ has a HN-filtration $0=\Ff_0\subsetneq \dotsb \subsetneq \Ff_r=\Ee/\Ee_1$ by the induction hypothesis. For all $i\geq 2$ let $\Ee_i$ be the kernel of $\Ee\morphism \Ff_r/\Ff_{i-1}$.\footnote{This kernel exists because compositions of \emph{strict quotients}, i.e., epimorphisms $\Ee\morphism\Ee''$ that are part of a short exact sequence, are strict quotients again. This is another axiom we weren't told.} Then $\Ee_1$ and $\Ee_{i+1}/\Ee_i\cong \Ff_i/\Ff_{i-1}$ for $i\geq 1$ are semistable and the slopes of the latter are strictly decreasing, so all that's left to prove is $\mu(\Ee_1)>\mu(\Ee_2/\Ee_1)$. By \cref{lem*:MuOnSES} it suffices to show $\mu(\Ee_1)>\mu(\Ee_2)$. But $\Ee_2$ has larger rank than $\Ee_1\subsetneq \Ee_2$, hence its slope must be strictly smaller by construction of $\Ee_1$.
	
	It remains to show uniqueness and functoriality. Suppose $0=\Ee_0\subsetneq \dotsb\subsetneq \Ee_r=\Ee$ and $0=\Ee'_0\subsetneq \dotsb\subsetneq \Ee'_s=\Ee$ are different HN-filtrations. Without restriction $\mu(\Ee_1')\geq \mu(\Ee_1)$. By \cref{lem:HN}, the morphism $\Ee_1'\monomorphism \Ee\morphism \Ee_r/\Ee_{r-1}$ must be zero, hence $\Ee_1'\monomorphism \Ee$ factors over $\Ee_{r-1}$. Iterating this argument we obtain that it even factors over $\Ee_1$. Thus $\mu(\Ee_1')\leq \mu(\Ee_1)$ by \cref{lem*:MuOnSES} again. So equality must hold we can apply the same argument to $\Ee_1$, ultimately obtaining that $\Ee_1\monomorphism \Ee$ and $\Ee_1'\monomorphism \Ee$ factor over each other. Then $\Ee_1=\Ee_1'$. Now we can repeat the argument for $\Ee/\Ee_1$. I'm not so sure what \enquote{functoriality} means, but it certainly also follows from \cref{lem:HN}.
\end{proof}

\begin{defi}\label{def:HNPolygon}
	The \defemph{HN-polygon} $\HN(\Ee)$ of an object $\Ee\in \Cc$ is the unique polygon in $\IR^2$ with origin $(0,0)$ and slopes $\mu(\Ee_i/\Ee_{i-1})$ with multiplicity $\rk(\Ee_i/\Ee_{i-1})$. In particular, $\Ee$ is semistable iff the HN-polygon is a straight line.
\end{defi}
The following theorem wasn't mentioned in the lecture. I thought it would fix a later argument that went quite wrong. Turns out it doesn't, but by that time I had already typed the proof.
\begin{thm*}\label{thm*:HNPolygon}
	If $\Ff\subseteq \Ee$ is a strict subobject, then the HN-polygon $\HN(\Ff)$ lies below $\HN(\Ee)$. In particular, $\HN(\Ee)$ is the upper concave hull of the points $(\rk(\Ff),\deg(\Ff))\in \IR^2$, where $\Ff$ ranges over all strict subobjects of $\Ff$.
\end{thm*}
\begin{proof*}
	By additivity of $\deg$ and $\rk$ we see that the break points of $\HN(\Ee)$ are precisely the points $(\rk(\Ee_i),\deg(\Ee_i))$ for $i=0,\dotsc,r$. We prove the theorem by induction on the length $s$ of the HN-filtration $0=\Ff_0\subsetneq \dotsb\subsetneq \Ff_s=\Ff$. The case $s=0$ is trivial. Now assume $s\geq 1$ and let $i$ be the minimal index such that $\Ff_{s-1}\morphism \Ee$ factors over $\Ee_i$. Let $j> i$ be minimal such that $\Ff_s\morphism \Ee$ factors over $\Ee_j$. Then $\Ff_s\morphism \Ee_j/\Ee_{j-1}$ is non-zero, and since the image of $\Ff_{s-1}$ is contained in $\Ee_i\subseteq \Ee_{j-1}$, we get a non-zero morphism $\Ff_s/\Ff_{s-1}\morphism \Ee_j/\Ee_{j-1}$. Hence $\mu(\Ff_s/\Ff_{s-1})\leq \mu(\Ee_k/\Ee_{k-1})$ for all $k\leq j$ by \cref{lem:HN} and the fact that the sequence of $\mu(\Ee_k/\Ee_{k-1})$ is strictly decreasing.
	
	Now $\HN(\Ff)$ is obtained by attaching a line of slope $\mu(\Ff_s/\Ff_{s-1})$ to $\HN(\Ff_{s-1})$. Moreover, its endpoint $(\rk(\Ff),\deg(\Ff))$ has $x$-coordinate $\rk(\Ff)\leq \rk(\Ee_j)$. So $\HN(\Ff_{s-1})$ lies below $\HN(\Ee_i)$, and the single segment that is attached to it has smaller slope than all segments of $\HN(\Ee_j)$. Thus $\HN(\Ff)$ lies below $\HN(\Ee_j)$ and therefore also below $\HN(\Ee)$.
	
	In particular, we see that $(\rk(\Ff),\deg(\Ff))$ lies below $\HN(\Ee)$. But the break points of $\HN(\Ee)$ are of this form too as seen above, and $\HN(\Ee)$ is concave by construction, thus it is indeed the upper concave hull of all points of the given form. This finishes the proof.
\end{proof*}
\begin{prop}\label{prop:Clambdasst}
	Let $\lambda\in \IQ$. Then the full subcategory
	\begin{equation*}
		\Cc_\lambda^\sst=\left\{\Ee\in \Cc\st \Ee \text{ semistable, }\mu(\Ee)\in \{\lambda,\infty\}\right\}
	\end{equation*}
	is abelian and every object in it is  of finite length.
\end{prop}
\begin{proof*}
	From \cref{lem*:MuOnSES} we get that direct sums (and moreover, arbitrary extensions) of objects in $\Cc_\lambda^\sst$ are in $\Cc_\lambda^\sst$ again. Next we construct kernels and cokernels of morphisms $\Ee\morphism \Ee'$. This is trivial if $\Ee=0$ or $\Ee'=0$, so we may assume $\mu(\Ee)=\lambda=\mu(\Ee')$. Let $\Kk$ and $\Kk'$ be as in the proof of \cref{lem:HN}. As was observed there, we have $\mu(\Kk)\leq \lambda$ and $\mu(\Ee/\Kk)\leq \mu(\Kk')\leq \lambda$ (except in the special cases where one of them is zero, but these are easily handled). But then by \cref{lem*:MuOnSES} equality must hold everywhere. In particular, since $\rk(\Ee/\Kk)=\rk(\Kk')$ we must also have $\deg(\Ee/\Kk)=\deg(\Kk')$, hence $\Ee/\Kk\isomorphism \Kk'$ is an isomorphism by assumption~\itememph{2}.
	
	Therefore $\Kk$ and $\Ee/\Kk\cong \Kk'$ are strict subobjects of $\Ee$ and $\Ee'$ of the same slope, hence they are semistable too. So $\Kk,\Kk'\in \Cc_\lambda^\sst$. Another application of \cref{lem*:MuOnSES} shows that $\mu(\Ee'/\Kk')$ must be $\lambda$ or $\Ee'/\Kk'=0$. In the latter case $\Ee'/\Kk'$ is an element of $\Cc_\lambda^\sst$ for trivial reasons. So assume the former is the case and let $\Ff\subseteq \Ee'/\Kk'$ be a strict subobject. Let $\Ff'\subseteq \Ee'$ be the kernel of $\Ee\morphism (\Ee'/\Kk')/\Ff$ (this exists by the argument from the proof of \cref{thm:HN}). Then $\mu(\Ff')\leq \lambda$. But now the short exact sequence\footnote{Here we are veiling a not so trivial detail (and we already did this in the proof of \cref{thm:HN}): that $\Kk'$ is indeed a strict subobject of $\Ff'$. This follows formally from the axioms (that were never given), but that's a bit fiddly.} $0\morphism \Kk'\morphism \Ff'\morphism \Ff\morphism 0$ together with $\mu(\Kk')=\lambda\geq \mu(\Ff')$ implies that $\mu(\Ff)\leq \mu(\Ff')\leq \lambda=\mu(\Ee'/\Kk')$ by \cref{lem*:MuOnSES}. This finally shows that $\Ee'/\Kk'$ is semistable.
	
	Thus, $\Ee\morphism \Ee'$ has a kernel and a cokernel in $\Cc_\lambda^\sst$, and moreover the morphism from its coimage $\Ee/\Kk$ to its image $\Kk'$ is an isomorphism. We conclude that $\Cc_\lambda^\sst$ is abelian. It remains to show that any $\Ee\in \Cc_\lambda^\sst$ has finite length. In fact, we will show that $\Ee$ has length $\rk(\Ee)$. So suppose $\Ee_1\subsetneq \dotsb\subsetneq \Ee_r\subsetneq \Ee$ is a chain of subobjects of length $r>\rk(\Ee)$. By the pidgeonhole principle there must be an $i$ with $\rk(\Ee_i)=\rk(\Ee_{i+1})$. Then $F(\Ee_i)\isomorphism F(\Ee_{i+1})$ must be an isomorphism by assumption~\itememph{1}. Then from assumption~\itememph{2} we get $\deg(\Ee_i)=\deg(\Ee_{i+1})$. Since $\Ee_i$ and $\Ee_{i+1}$ have the same rank, hence the same slope (either $\lambda$ or $\infty$), we get equality and $\Ee_i\subseteq \Ee_{i+1}$ must be an isomorphism, contradicting $\Ee_i\subsetneq \Ee_{i+1}$.
\end{proof*}
\begin{exm}
	Suppose $\Cc$ is the category of vector bundles on $\IP_k^1$. By the Grothen-dieck--Birkhoff theorem, every vector bundle $\Ee$ can be written uniquely as
	\begin{equation*}
		\Ee\cong \bigoplus_{i=1}^r\Oo(d_i)^{\oplus n_i}\,,
	\end{equation*}
	where $d_1>d_2>\dotsb >d_j$ and all $n_i>0$. Then the $j\ordinalth$ piece $\Ee_j$ of the HN-filtration of $\Ee$ is given by $\bigoplus_{i=1}^j\Oo(d_i)^{\oplus n_i}$. Moreover, for $\lambda\in \IZ$ the category $\Cc_\lambda^\sst$ is the full subcategory of vector bundles isomorphic to a finite direct sum of copies of $\Oo(\lambda)$. In particular, $\Cc_\lambda^\sst$ is equivalent to $\cat{Vect}_k$.
\end{exm}
\subsection{\texorpdfstring{$\phi$}{Phi}-Modules and the Vector Bundles \texorpdfstring{$\Oo_{X_\FFC}(\lambda)$}{OX(lambda)}}
\begin{defi}
	Let $\breve{E}=W_{\Oo_E}(\ov{\IF}_q)\localize{\pi}$, where $\ov{\IF}_q$ is the algebraic closure of $\IF_q$ inside $F$. Note that $\breve{E}$, being a localization of a ring of Witt vectors, comes equipped with a natural Frobenius action $\phi$.
\end{defi}
\begin{rem}
	We observe that $\breve{E}$ is a field. In fact, it is the completion of the maximal unramified extension of $E$ (this follows more or less from \cref{prop:FqAlgebrasEquivalence}). Moreover, since $\ov{\IF}_q\subseteq F$, $\breve{E}$ is naturally a subring of $B$.
\end{rem}
\begin{defi}\label{def:phiModule}
	Let $A$ be a ring with an endomorphism $\phi\colon A\morphism A$. A \defemph{$\phi$-module} over $A$ is a pair $(M,\phi_M)$, where $M$ is a finite projective $A$-module and $\phi_M\colon M\isomorphism M$ is $\phi$-semilinear isomorphism. The category of $\phi$-modules is denoted $\phi\cat{\mhyph Mod}_A$.
\end{defi}
\numpar{}\label{par:phiModules}In \cref{def:phiModule}, recall that a $\phi$-semilinear isomorphism is an isomorphism of underlying abelian groups that satisfies $\phi_M(am)=\phi(a)\phi_M(m)$ for all $a\in A$, $m\in M$. If $M$ is even a free $A$-module and $e_1,\dotsc,e_n\in M$ a basis, one can write
\begin{equation*}
	\phi_M(e_i\otimes 1)=\sum_{j=1}^na_{i,j}e_j\,,
\end{equation*}
and we obtain a matrix $a=(a_{i,j})\in \GL_n(A)$. Changing $e_1,\dotsc,e_n$ according to an invertible matrix $g\in \GL_n(A)$ transforms $a$ into $ga\phi(g)^{-1}$ (this operation is called \enquote{$\phi$-conjugation}). Thus, we get a bijection
\begin{equation*}
	\left\{\text{iso.\ classes of free rank $n$ $\phi$-modules}\right\}\isomorphism\GL_n(A)/\phi\text{-conj.}\,.
\end{equation*}
From now on, we consider the category $\Cc=\phi\cat{\mhyph Mod}_{\breve{E}}$, where $\phi$ is the ordinary Frobenius (at least if $E/\IQ_p$ is unramified, this is also known as the category of \defemph{isocrystals}). Since $\breve{E}$ is a field, all $\phi$-modules over $\breve{E}$ are free. Hence the above bijection provides a map
\begin{equation*}
	\deg\colon \left\{\text{iso.\ classes of rank-$1$ $\phi$-modules}\right\}\cong \breve{E}^\times/\phi\text{-conj.}\isomorphism\IZ\,;
\end{equation*}
the isomorphism on the right-hand side is induced by the valuation on $\breve{E}$ and it is an isomorphism because for $a,b\in \breve{E}$, trying to find a $g\in \Oo_{\smash{\breve{E}}\vphantom{E}}^\times$ with $b=ga\phi(g)^{-1}$ leads to a list of polynomial equations in the Teichmüller coefficients of $g$, which always have solutions in the algebraically closed field $\ov{\IF}_q$ as long as $a$ and $b$ have the same valuation. For arbitrary $M\in \Cc$
\begin{equation*}
	\rk M=\dim_{\breve{E}}M\quad \text{and}\quad\deg M=\deg\Bigg(\bigwedge^{\rk M}M\Bigg)
\end{equation*}
Also let $F$ be simply the identity functor on $\Cc$. Then one checks that all conditions from \cref{subsec:HNFormalism} are satisfied, so the HN-formalism is available for $(\Cc,\rk,\deg,\id_{\Cc})$! Moreover, since $\phi\cat{\mhyph Mod}_{\breve{E}}$ is already abelian $F$ is the identity functor, it's easily checked that $(\Cc,\rk,-\deg,\id_{\Cc})$ satisfies the conditions too. Therefore we actually have two HN-structures on $\Cc$! This has interesting consequences. 
\begin{numerate}
	\item Every HN-filtration $0=\Ee_0\subsetneq \dotsb \subsetneq \Ee_r=\Ee$ in $\Cc$ is canonically split, so that there is a canonical isomorphism $\Ee\cong \bigoplus_{i=1}^r\Ee_i/\Ee_{i-1}$. In fact, the splitting is induced by the second filtration $0=\Ee'_0\subsetneq \dotsb \subsetneq \Ee'_r=\Ee$ associated to the second HN-structure. That is, for all $j=1,\dotsc,r$ we have $\Ee_j'=\bigoplus_{i=r-j+1}^r\Ee_i/\Ee_{i-1}$.
	\item If $\Ee$ and $\Ee'$ are semistable of different slopes, then $\Hom_{\Cc}(\Ee,\Ee')=0$. 
\end{numerate}
\begin{warn*}
	Neither \itememph{1} nor \itememph{2} are as easy as the lecture made them sound. For example, \itememph{2} seemingly follows immediately from \cref{lem:HN}, but the major obstacle here is to show that the \emph{semistable objects are the same in both HN-structures}!
	
	What saves our *sses here (and only here, we really need that we are in the particular case where $\Cc=\phi\cat{\mhyph Mod}_{\breve{E}}$!) is the \emph{Dieudonné--Manin decomposition} (see \cref{thm:DieudonneManin} below). It can be checked by hand that this decompositions induces two split filtrations which have the property from \cref{thm:HN} for the respective HN-structures. So all in all, \itememph{1} and \itememph{2} are a consequence of the Dieudonné--Manin decomposition, not the other way around.
\end{warn*}
Let $\lambda=d/r\in \IQ$, where $d$ and $r$ are coprime integers and $r>0$. Let $D(\lambda)$ be the $\phi$-module over $\breve{E}$ whose underlying module is $\breve{E}^{\oplus r}$ and with associated matrix
\begin{equation*}
	\phi_{D(\lambda)}=\begin{pmatrix}
		0 & \cdots & 0 & \pi^d\\
		1 &  & & 0\\
		\vdots & \ddots & & \vdots\\
		0 & \cdots & 1 & 0 \\
	\end{pmatrix}
\end{equation*}

\begin{thm}[Dieudonné--Manin classification]\label{thm:DieudonneManin}
	The category $\Cc=\phi\cat{\mhyph Mod}_{\breve{E}}$ is semisimple and its simple objects are precisely those which are isomorphic to $D(\lambda)$ for $\lambda\in \IQ$ (in particular, every $\phi$-module $M$ has a unique decomposition $M=\bigoplus_{i=1}^rD(\lambda_i)^{\oplus n_i}$). Moreover, the division algebra $\End_\Cc(D(\lambda))$ over $E$ is central (i.e.\  its center is $E$) of invariant $\pm[\lambda]\in \operatorname{Br}E\cong \IQ/\IZ$.
\end{thm}
\begin{proof}[Sketch of a proof]
	Some technical arguments including the HN-formalism, passage to unramified coverings of $E$ (thus replacing $\phi$ by $\phi^h$) and twisting reduces the theorem to its essential part:
	\begin{alphanumerate}
		\item[\itememph{*}] Every semistable $\phi$-module $D$ over $\breve{E}$ of slope $0$ is a direct sum of copies of $D(0)=(\breve{E},\phi)$.
	\end{alphanumerate}
	By inspection, $\Ext_\Cc^1(D(0),D(0))\cong \breve{E}/(\phi-\id)\breve{E}$. But $\Oo_{\breve{E}}/\pi \Oo_{\breve{E}}=\ov{\IF}_q$ is algebraically closed. Hence $\smash{\Oo_{\breve{E}}/(\phi-\id)\Oo_{\breve{E}}}$ vanishes after reduction modulo $\pi$, hence by Nakayama it must vanish all along. Inverting $\pi$, we thus see that $\Ext_\Cc^1(D(0),D(0))=0$, so any self extension of $D(0)$ is split.
	
	Therefore, given an arbitrary $D$ as in \itememph{*}, it suffices to construct a non-zero morphism $D(0)\morphism D$. Indeed, such a morphism is necessarily a monomorphism because $D(0)$ is simple, and using induction on the rank we may assume that its cokernel $D/D(0)$ is already a direct sum of copies of $D(0)$. Then the above extension argument shows that $D$ itself must be such a direct sum. To construct a non-zero morphism $D(0)\morphism D$, write $\phi_D=a$ for some $a\in \GL_n(\breve{E})$. After performing row operations we may assume $a$ is triangular (doing some row operations forces us to do the corresponding \enquote{$\phi$-inverse} column operations to keep the $\phi$-conjugacy property alive; however, these column operations won't stop us from making $a$ upper triangular). Moreover $a_{1,1}\in \Oo_{\smash{\breve{E}}\vphantom{E}}^\times$ because $D$ is semistable of slope $0$, so $\det(\phi_D)$ should have valuation $0$. As $\ov{\IF}_q$ is algebraically closed, we may write $a_{1,1}=\phi(x)/x$ for some $x\in\Oo_{\smash{\breve{E}}\vphantom{E}}^\times$ (as usual, $x$ has to be constructed Teichmüller coefficient-wise). Then we get $D(0)\cong (\breve{E},a_{1,1}\phi)\monomorphism D$, as desired. 
\end{proof}
\begin{defi}\label{def:MysteriousFunctor}
	Recall that $\breve{E}\subseteq B$ canonically. Let $X_\FFC$ denote the Fargues--Fontaine curve as usual. We construct a functor
	\begin{align*}
		\Ee_E(-)=\Ee(-)\colon \phi\cat{\mhyph Mod}_{\breve{E}}&\morphism \cat{QCoh}_{X_\FFC}\\
		(D,\phi_D)&\longmapsto \Bigg(\bigoplus_{d\geq 0}(B\otimes_{\breve{E}}D)^{\phi\otimes \phi_D=\pi^d}\Bigg)^\qcmod\,,
	\end{align*}
	where $(-)^\qcmod$ denotes the graded twiddleization.
\end{defi}
\numpar{Example/Warning}\label{exmwarn:SignSwap}
If $n\in \IZ$, then $D(n)=(\breve{E},\pi^n\phi)$ is sent to the twisting sheaf $\Ee_E(D(n))=\Oo_{X_\FFC}(-n)$. Indeed, its graded components are given by
\begin{equation*}
	(B\otimes_{\breve{E}}\breve{E})^{\phi\otimes \pi^n\phi=\pi^d}=B^{\phi=\pi^{d-n}}\,,
\end{equation*}
hence $\Ee_E(D(n))$ is the quasi-coherent module associated to the shift $P[-n]$, whence we catch a sign swap.
\begin{lem}\label{lem:Xh}
	For $h\geq 1$ let $E_h/E$ be the unique unramified extension of degree $h$ and $X_{\FFC,h}=\Proj\big(\bigoplus_{d\geq 0}B^{\phi^h=\pi^d}\big)$ the corresponding Fargues--Fontaine curve. Let $(D,\phi_D)$ be a $\phi$-module over $\breve{E}$.
	\begin{numerate}
		\item For all $d\geq 0$ we have $E_h\otimes_E(B\otimes_{\breve{E}}D)^{\phi\otimes\phi_D=\pi^d}\cong (B\otimes_{\breve{E}}D)^{\phi^h\otimes \phi_D^h=\pi^{hd}}$.
		\item $X_{\FFC,h}$ is isomorphic to the base change $X_\FFC\otimes_EE_h$ (this works in the ramified case too).
		\item The following diagram commutes:
		\begin{equation*}
			\begin{tikzcd}
			(D,\phi_D)\dar[|->] &[-2.4em] \phi\cat{\mhyph Mod}_{\breve{E}}\dar\rar["\Ee_E(-)"] &[1em] \cat{QCoh}_{X_\FFC}\dar["-\otimes_EE_h"] \\
			(D,\phi_D^h) &[-2.4em] \phi^h\cat{\mhyph Mod}_{\breve{E}} \rar["\Ee_{E_h}(-)"] &[1em] \cat{QCoh}_{X_{\FFC,h}}
			\end{tikzcd}\,.
		\end{equation*}
	\end{numerate}
\end{lem}
\begin{rem}
	\cref{lem:Xh} shows that $\Ee_E(-)$ takes values in vector bundles. Indeed, by \cref{thm:DieudonneManin}, every $M\in\phi\cat{\mhyph Mod}_{\breve{E}}$ is a direct sum of $D(\lambda)$'s, so it suffices to check that every $\Ee_E(D(\lambda))$ is a vector bundle. This can be verified étale-locally. Writing $\lambda=d/r$, we see that $X_{\FFC,r}\morphism X_\FFC$ is an étale covering and by \cref{lem:Xh}\itememph{3}, $\Ee_E(D(\lambda))\otimes_EE_h$ corresponds to
	\begin{equation*}
		\big(D(\lambda),\phi_{D(\lambda)}^r\big)\cong \bigoplus_{i=1}^r(\breve{E},\pi^d\phi)\,,
	\end{equation*}
	which is sent to the vector bundle $\Oo_{X_{\FFC,h}}(-d)^{\oplus r}$ under $\Ee_{E_h}(-)$ by Example/Warning~\cref{exmwarn:SignSwap}. Henceforth we will write $\Oo_{X_\FFC}(\lambda)\coloneqq \Ee_E(D(-\lambda))$. We have just seen that this is a vector bundle of rank $r$.
\end{rem}
\begin{proof}[Proof of \cref{lem:Xh}]
	For \itememph{1}, note that $\IZ/h\IZ\cong \Gal(E_h/E)$ acts $E_h$-semilinearly on $M=(B\otimes_{\breve{E}}D)^{\phi^h\otimes\phi_D^h=\pi^{hd}}$ via $\pi^{-d}\phi\otimes\phi_D$. Moreover, the invariants of this action are $M^{\Gal(E_h/E)}=(B\otimes_{\breve{E}}D)^{\phi\otimes\phi_D=\pi^d}$. Then the claim follows from Hilbert~90 (in the form of Galois descent; see e.g.\ \cite[I~(5.2)]{sga4.5}).
	
	Parts~\itememph{2} and~\itememph{3} are easy consequences of \itememph{1}; for \itememph{2} we use that $\Proj\big(\bigoplus_{d\geq 0}B^{\phi^h=\pi^d}\big)$ coincides with $\Proj \big(\bigoplus_{d\geq 0}B^{\phi^h=\pi^{hd}}\big)$ because that's just how the $\Proj$ construction works.
\end{proof}
A slightly different perspective on the vector bundles $\Oo_{X_\FFC}(\lambda)$ is given by the following lemma.
\begin{lem}\label{lem:OXlambda}
	Let $\lambda=d/r\in \IQ$ and let $f_r\colon X_{\FFC,r}=X_\FFC\otimes_EE_r\morphism X_\FFC$. Then there is a canonical isomorphism
	\begin{equation*}
		\Oo_{X_\FFC}(\lambda)\isomorphism f_{r,*}\Oo_{X_{\FFC,r}}(d)
	\end{equation*}
	In particular, $\Oo_{X_\FFC}(\lambda)$ is a semistable vector bundle of rank $r$ and slope $\lambda$, and a simple object in $\cat{Bun}_{X_\FFC,\lambda}^\sst$.
\end{lem}
\begin{proof*}
	The isomorphism follows easily by pulling back to $X_{\FFC,r}$ and comparing descent datas; details are left as an exercise. To prove the additional assertions, we claim:
	\begin{alphanumerate}
		\item[\itememph{*}] The functors $f_{r,*}$ and $f_r^*$ preserve semistable objects.
	\end{alphanumerate}
	We first observe that $f_r^*$ preserves $\rk$ and scales $\deg$ by $r$, because it is straightforward to check that $\Oo_{X_\FFC}(n)\otimes_EE_r\cong \Oo_{X_{\FFC,r}}(rn)$. Conversely, $f_{r,*}$ scales $\rk$ by $r$ (this is straightforward) and preserves $\deg$ (this follows from the fact that both $f_r^*$ and $f_r^*f_{r,*}\cong (-)^{\oplus r}$ scale $\deg$ by $r$). Now suppose $\Ee'\in \cat{Bun}_{X_{\FFC,r}}$ is semistable and $\Ee\subseteq f_{r,*}\Ee'$ is a strict subobject. Then $f^*\Ee\subseteq f_r^*f_{r,*}\Ee'\cong \Ee'^{\oplus r}$ is a strict subobject and $\Ee'^{\oplus r}$ is semistable, hence
	\begin{equation*}
		r\deg(\Ee)=\deg(f^*\Ee)\leq \deg(\Ee'^{\oplus r})=r\deg(\Ee')\,,
	\end{equation*}
	proving that $f_{r,*}\Ee'$ is semistable as well. In the same way we prove that $f_r^*$ preserves semistable objects. This proves \itememph{*}.
	
	In particular, $\Oo_{X_\FFC}(\lambda)$ is semistable of rank $r$ and slope $d/r=\lambda$. It remains to show that it is simple in $\cat{Bun}_{X_\FFC,\lambda}^\sst$. If $0\neq \Ee\subseteq \Oo_{X_\FFC}(\lambda)$, then $\rk(\Ee)\leq r$. But if $\Ee$ has slope $\lambda$, then equality must hold as $d$ and $r$ are coprime. Since $\cat{Bun}_{X_\FFC,\lambda}^\sst$ is abelian by \cref{prop:Clambdasst}, we see that $\Oo_{X_\FFC}(\lambda)/\Ee$ is a vector bundle again and of rank $0$, hence $\Ee=\Oo_{X_\FFC}(\lambda)$.
\end{proof*}
We are now ready to state our second main theorem: the classification of vector bundles on the Fargues--Fontaine curve. Its proof will occupy most of the remaining lectures.
\begin{mainthm}[Fargues--Fontaine]\label{mainthm:vectorBundles}
	The functor $\Ee(-)$ from \cref{def:MysteriousFunctor} induces a bijection
	\begin{equation*}
		\Ee(-)\colon \left\{\text{iso.\ classes in }\phi\cat{\mhyph Mod}_{\breve{E}}\right\}\isomorphism\left\{\text{iso.\ classes in }\cat{Bun}_{X_\FFC}\right\}\,.
	\end{equation*}
	In particular, every vector bundle $\Ee$ on the Fargues--Fontaine curve $X_\FFC$ has a unique decomposition $\Ee\cong \bigoplus_{i=1}^r\Oo_{X_\FFC}(\lambda_i)^{\oplus n_i}$ with $\lambda_1>\dotsb >\lambda_r$ and all $n_i>0$.
\end{mainthm}
\begin{warn}
	Don't be fooled: $\Ee_E(-)$ will \defemph{not} be an equivalence of categories. Upon closer inspection this can't possibly be true, for $\phi\cat{\mhyph Mod}_{\breve{E}}$ is an abelian category, but $\cat{Bun}_{X_\FFC}$ is not.
\end{warn}
\section{Vector Bundles on the Fargues--Fontaine Curve and \texorpdfstring{$p$}{p}-divisible Groups}
\lecture[$p$-divisible groups and Tate modules. Dieudonné modules. $p$-divisible groups and $p$-adic Hodge theory. The vector bundles $\Ee(G)$ and minuscule modifications.]{2020-01-15}
Today we start working towards \cref{mainthm:vectorBundles}. Along the way will see some nice applications of the Fargues--Fontaine curve to $p$-divisible groups and $p$-adic Hodge theory.
\subsection{A Crash Course on \texorpdfstring{$p$}{p}-divisible Groups}
Let $\cat{FL}/S$ denote the category of finite locally free\footnote{In the lecture we merely talked about \emph{finite flat} group schemes. I believe that whenever people say \enquote{finite flat} in the context of group schemes they actually mean a \enquote{finite locally free}, and this either follows from some additional assumptions such as being proper or having  constant rank (as in \cref{def:pDivGroup}), or these people don't realize that \enquote{finite flat} doesn't imply \enquote{locally free} in non-noetherian situations.} commutative group schemes over a scheme $S$. Via the Yoneda embedding $G\mapsto \Hom_{\cat{Sch}/S}(-,G)$ it becomes a full subcategory of the category $\cat{Ab}((\cat{Sch}/S)_\fppf)$ of sheaves on the big fppf site over $R$. We call a sequence in $\cat{FL}/S$ \defemph{exact} if the corresponding sequence of sheaves is exact. It can be shown that $\cat{FL}/S$, as a full subcategory of $\cat{Ab}((\cat{Sch}/S)_\fppf)$, is stable under extensions, hence an \emph{exact category} in the sense of Quillen. The \emph{strict monomorphisms}, i.e., those that are part of a short exact sequence, are precisely the closed immersions, and the \emph{strict epimorphisms} are precisely the faithfully flat morphisms.

Keeping this in mind, we can now turn to today's central definition.
\begin{defi}\label{def:pDivGroup}
	A \defemph{$p$-divisible group of height $h$} is a collection $G=(G_n,i_n)_{n\in \IN}$ of finite locally free commutative group schemes $G_n$ of rank $p^{nh}$ over $S$, together with closed immersions $i_n\colon G_n\monomorphism G_{n+1}$ for all $n\in \IN$ such that the following sequence is exact:
	\begin{equation*}
		0\morphism G_n\morphism[i_n]G_{n+1}\morphism[p^n]G_{n+1}\,.
	\end{equation*}
	A \defemph{morphism of $p$-divisible groups} $\phi\colon (G_n,i_n)_{n\in \IN}\morphism (G'_n,i'_n)_{n\in \IN}$ is a sequence of group scheme homomorphisms $\phi_n\colon G_n\morphism G_n'$ which are compatible with the respective structure morphisms $i_n$ and $i_n'$ in the sense that the diagram
	\begin{equation*}
		\begin{tikzcd}
			G_{n+1}\rar["\phi_{n+1}"] & G'_{n+1}\\
			G_n\uar[mono, "i_n"]\rar["\phi_n"] & G'_n\uar[mono, "i'_n"']
		\end{tikzcd}
	\end{equation*}
	commutes for all $n\in \IN$.
\end{defi}
\begin{rem}\label{rem:pDivExactSeq}
	We also write $G_n=G[p^n]$ and think of this as the \enquote{$p^n$-torsion of $G$}. It follows from the axioms that for any $n,m$ there exists a short exact sequence
	\begin{equation*}
		0\morphism G_n\xrightarrow{i_{n,m}}G_{n+m}\xrightarrow {j_{n,m}}G_m\morphism 0\,.
	\end{equation*}
	Here $i_{n,m}=i_{n+m-1}\circ i_{n+m-2}\circ\dotsb\circ i_n$ denotes the closed immersion $G_n\monomorphism G_{n+m}$ obtained from the given data. Moreover, $j_{n,m}$ is defined as the dashed arrow in the diagram
	\begin{equation*}
		\begin{tikzcd}
			G_{n+m}\drar["p^n"']\rar["\exists!\ j_{n,m}", dashed] &[1em] G_m\dar[mono]\\
			 &[1em] G_{n+m}
		\end{tikzcd}
	\end{equation*}
	Since the existence of $j_{n,m}$ and the above short exact sequence are not completely obvious, I decided to give it a proof (which was not in the lecture).
\end{rem}
\begin{proof*}
	To show that $p^n\colon G_{n+m}\morphism G_{n+m}$ factors over $G_m$, i.e., the $p^m$-torsion part, it suffices to show that $p^m\circ p^n=p^{n+m}$ is the zero morphism on $G_{n+m}$. But that's clear by definition of $G_{n+m}$ as the $p^{n+m}$-torsion part of $G_{n+m+1}$.
	
	Exactness on the left is now clear by definition of $j_{n,m}$, as $G_n$ is the $p^n$-torsion part of $G_{n+m}$. Moreover, since $i_{n,m}$ is a closed immersion, it has a cokernel $Q$ in $\cat{FL}/S$ as remarked above. We need to show $Q=G_m$. By fundamental results about group schemes, the morphism $G_{m+n}\morphism Q$ is finite locally free of rank $\rk (G_n)=p^{nh}$, hence the rank of $Q$ must be $p^{mh}=\rk (G_m)$. Moreover, $Q$ is the image of $p^{n+m}\colon G_{n+m}\morphism G_{n+m}$ (considered as a morphism of fppf sheaves), hence $Q\morphism G_{n+m}$ factors over $G_m$.
	
	Now $Q\morphism G_m$ is a monomorphism of sheaves, hence a monomorphism of schemes. But it is also proper as both $Q$ and $G_m$ are finite over $S$! As proper monomorphism are closed immersions and $Q$ and $G_m$ have the same rank over $S$, this shows that $Q\morphism G_m$ must be an isomorphism, as claimed.
\end{proof*}
\begin{exm}
	The following are $p$-divisible groups over $S$.
	\begin{numerate}
		\item Let $G_n=(p^{-n}\IZ/\IZ)\times S$ be the constant group scheme with $i_n$ the natural inclusions. This defines a $p$-divisible group called $\IQ_p/\IZ_p$, of height $1$.
		\item Let $S=\Spec R$ and $G_n=\mu_{p^n}=\Spec R[x]/(x^{p^n}-1)$ be the group scheme of $(p^n)\ordinalth$ roots of unity. This defines a $p$-divisible group scheme of height $1$, called $\mu_{p^\infty}$.
		\item Let $A$ be an abelian scheme of dimension $d$ over $S$. Then the $p^n$-torsion $G_n=A[p^n]$ defines a $p$-divisible group $A[p^\infty]$ of height $2d$. Indeed, by a classical result on abelian schemes, the multiplication-by-$N$ morphism $N\colon A\morphism A$ is finite locally free of rank $N^{2d}$ for all $N\neq 0$. Its kernel is the $N$-torsion part $A[N]$ and fits into a pullback diagram
		\begin{equation*}
			\begin{tikzcd}
				A[N]\rar\dar\drar[pullback] & S\dar["1_A"]\\
				A\rar["N"]& A
			\end{tikzcd}\,.
		\end{equation*}
		Taking $N=p^n$, we deduce that $A[p^n]$ is finite locally free of rank $p^{2dn}$, which fits perfectly with \cref{def:pDivGroup}.
	\end{numerate}
\end{exm}
\begin{defi}\label{def:CartierDual}
	For any $p$-divisible group $G$, we obtain a \defemph{dual $p$-divisible group $G^\vee$} as follows: put 
	\begin{equation*}
		(G^\vee)_n\coloneqq (G_n)^\vee=\underline{\Hom}(G_n,\IG_{m,S})\,,
	\end{equation*}
	and for all $n$ the structural morphism $i_n^\vee\colon (G^\vee)_n\monomorphism (G^\vee)_{n+1}$ is induced by $j_{1,n}\colon G_{n+1}\morphism G_n$ from \cref{rem:pDivExactSeq}.
\end{defi}
\begin{rem}\label{rem:CartierDual}
	Perhaps \cref{def:CartierDual} needs some clarifications.
	\begin{numerate}
		\item The scheme $\IG_{m,S}$ is a representing object of the sheaf $T\mapsto \Global(T,\Oo_T^\times)$ on $(\cat{Sch}/S)_\fppf$ (see \cite[\stackstag{022U}]{stacks-project}). It can be shown that for arbitrary $F\in \cat{FL}/S$ the sheaf $\underline{\Hom}(F,\IG_{m,S})$, which denotes the internal $\Hom$ in $\cat{Ab}((\cat{Sch}/S)_\fppf)$, but can also be explicitly described as $T\mapsto \Hom_{\cat{GSch}/T}(F\times_ST,\IG_{m,T})$, is representable by a finite locally free commutative group scheme $F^\vee$, called the \defemph{Cartier dual} of $F$.
		\item The natural evaluation isomorphisms $G_n\isomorphism G_n^{\vee\vee}$ are compatible and define an isomorphism of $p$-divisible groups
		\begin{equation*}
		G\isomorphism G^{\vee\vee}\,.
		\end{equation*}
		Thus, the functor $(-)^\vee\colon G\mapsto G^\vee$ is a (contravariant) auto-equivalence on the category of $p$-divisible groups.
	\end{numerate}
	
\end{rem}
\begin{exm}
	The following pairs of $p$-divisible groups are dual to each other:
	\begin{numerate}
		\item $\IQ_p/\IZ_p$ and $\mu_{p^\infty}$.
		\item $A[p^\infty]$ and $A^\vee[p^\infty]$ for $A$ an abelian scheme over $S$.
	\end{numerate}
\end{exm}
From now on, we restrict our attention to $p$-divisible groups over the base $S=\Spec \Oo_C$, where $C$ is a complete algebraically closed extension of $\IQ_p$ and $\Oo_C$ its ring of integers.
\begin{defi}
	For a $p$-divisible group $G$ over $\Oo_C$, we define its \defemph{Tate module} as
	\begin{equation*}
		T_pG\coloneqq \limit\Big(\dotso\morphism[p]G[p^2](C)\morphism[p]G[p](C)\morphism 1\Big)\,.
	\end{equation*}
\end{defi}
\begin{lem}\label{lem:TpGFree}
	If $G$ is a $p$-divisible group of height $h$ over $\Oo_C$, then $T_pG$ is a free $\IZ_p$-module of rank $h$.
\end{lem}
\begin{proof*}
	We prove by induction on $n$ that the map $p\colon G[p^{n+1}]\morphism G[p^n]$ can be (non-canonically) identified with $(\IZ/p^{n+1}\IZ)^{\oplus h}\morphism (\IZ/p^n\IZ)^{\oplus h}$. This will prove the lemma.
	
	We start with $n=0$. Note that $G[p](C)$, as a set, is in bijection with sections of the base change $G[p]\otimes_{\Oo_C}C\morphism \Spec C$. Since $G[p]\otimes_{\Oo_C}C$ is a flat affine group scheme of finite type over a field of characteristic $0$, it is automatically smooth, hence étale because it is finite. Since $C$ is algebraically closed, it follows that $G\otimes_{\Oo_C}C$ is a disjoint union of $p^h=\rk(G[p])$ copies of $\Spec C$. This shows that $G[p](C)$ has $p^h$ elements, hence it is isomorphic to $(\IZ/p\IZ)^{\oplus h}$ because it must be an $\IF_p$-vector space.
	
	Now assume we know $G[p^n]=(\IZ/p^n\IZ)^{\oplus h}$. By the same argument as above we see that $G[p^{n+1}](C)$ must be a $p^{n+1}$-torsion abelian group of cardinality $p^{(n+1)h}$. Moreover, there is an exact sequence
	\begin{equation*}
		0\morphism G[p](C)\morphism G[p^{n+1}](C)\morphism[p] G[p^n](C)\morphism 0
	\end{equation*}
	(the sequence of fppf sheaves from \cref{rem:pDivExactSeq} is exact on sections over $\Spec C$ because every fppf cover $\{U\morphism \Spec C\}$ admits a section $\Spec C\morphism U$ as $C$ is algebraically closed). Now $G[p^n]=(\IZ/p^n\IZ)^{\oplus h}$ and the classification of finite abelian groups finish the inductive step.
\end{proof*}
\begin{exm}\label{exm:duals}
	\begin{numerate}
		\item We have $T_p(\IQ_p/\IZ_p)=\IZ_p$.
		\item We denote $T_p\mu_{p^\infty}\eqqcolon \IZ_p(1)$ and call this the \defemph{first Tate twist}. This is isomorphic to $\IZ_p$ by \cref{lem:TpGFree}, but only up to \emph{non-canonical} isomorphism. In fact, any such isomorphism depends uniquely on the choice of compatible primitive $(p^n)\ordinalth$ roots $\zeta_{p^n}$.
		\item If $A$ is an abelian scheme over $\Oo_C$, then $T_pA[p^\infty]=T_pA$. This can be canonically identified with the dual of $H_\et^1(A_C,\IZ_p)$ (as a $\IZ_p$-module).
	\end{numerate}
\end{exm}
\begin{defi}
	For an arbitrary $\IZ_p$-module $M$ and $n\in \IZ$, we put 
	\begin{equation*}
		M(n)\coloneqq M\otimes_{\IZ_p}\IZ_p(1)^{\otimes n}
	\end{equation*}
	(where $\IZ_p(-n)\coloneqq \IZ_p(n)^\vee$ by convention) and call this the \defemph{$n\ordinalth$ Tate twist} of $M$.
\end{defi}
\begin{lem}\label{lem:TpGHom}
	Let $G$ be a $p$-divisible group over $\Oo_C$.
	\begin{numerate}
		\item There is a natural $\IZ_p$-linear isomorphism $T_pG\cong\Hom(\IQ_p/\IZ_p,G)$, where $\Hom$ is taken in the category of $p$-divisible groups.
		\item The composition of natural maps
		\begin{equation*}
			T_pG\cong\Hom(\IQ_p/\IZ_p,G)\xrightarrow{(-)^\vee}\Hom(G^\vee,\mu_{p^\infty})\xrightarrow{T_p(-)}\Hom_{\IZ_p}\big(T_pG^\vee,\IZ_p(1)\big)
		\end{equation*}
		defines a perfect $\IZ_p$-bilinear pairing $T_pG\times T_pG^\vee\morphism \IZ_p(1)$.
	\end{numerate}
\end{lem}
\begin{proof*}
	To prove \itememph{1}, we first claim that the natural morphism $G[p^n](\Oo_C)\morphism G[p^n](C)$ is an isomorphism for all $n$. Indeed, write $G[p^n]=\Spec A_n$, where $A_n$ is finite flat over $\Oo_C$. The image of any $\Oo_C$-algebra morphism $A_n\morphism C$ is finite over $\Oo_C$, hence already contained in $\Oo_C$. This proves the claim.
	
	Next, we claim that $\Hom_{\cat{GSch}/\Oo_C}((p^{-n}\IZ/\IZ)\times \Spec \Oo_C,G[p^n])$ is in canonical bijection with $G[p^n](C)$. Indeed, since $G[p^n]$ is a $p^n$-torsion group scheme, every morphism of group schemes $(p^{-n}\IZ/\IZ)\times \Spec \Oo_C\morphism G[p^n]$ is uniquely determined by what it does on the generator $\{p^{-n}\}\times \Spec \Oo_C$. Hence we get a canonical bijection with $G[p^n](\Oo_C)=G[p^n](C)$, as claimed.
	
	Now every morphism $\IQ_p/\IZ_p\morphism G$ of $p$-divisible groups is a sequence of compatible morphisms $\IZ/p^n\IZ\times \Spec \Oo_C\morphism G[p^n]$, hence a sequence of compatible elements of $G[p^n](C)$, and after unraveling what \enquote{compatible} means, \itememph{1} follows.
	
	To prove \itememph{2}, we construct the pairing in a slightly different way (and leave it to the reader to show that both constructions amount to the same). We have
	\begin{equation*}
		G^\vee[p^n](C)\cong\Hom_{\cat{GSch}/C}\big(G[p^n]\otimes_{\Oo_C}C,\IG_{m,C}\big)\cong\Hom\big(G[p^n](C),\IG_m(C)\big)\,.
	\end{equation*}
	The isomorphism on the left follows from the explicit construction in \cref{rem:CartierDual}\itememph{1}. For the right isomorphism, recall that $G[p^n]\otimes_{\Oo_C}C\cong (\IZ/p^n\IZ)^{\oplus h}\times \Spec C$ as shown in the proof* of \cref{lem:TpGFree}. Hence any morphism of group schemes into $\IG_{m,C}$ is uniquely determined by what it does on the $C$-valued points. Finally, observe that $G[p^n](C)$ is a $p^n$-torsion group, hence every morphism into $\IG_m(C)$ lands automatically inside $\mu_{p^n}(C)$. Summarizing,
	\begin{equation*}
		G^\vee[p^n](C)\cong \Hom\big(G[p^n](C),\mu_{p^n}(C)\big)
	\end{equation*}
	Taking limits on both sides, we easily get $T_pG^\vee\cong \Hom_{\IZ_p}(T_pG,\IZ_p(1))$. The same holds if we swap $G$ and $G^\vee$, hence we get indeed a $\IZ_p$-bilinear perfect pairing $T_pG\times T_pG^\vee\morphism \IZ_p(1)$.
\end{proof*}
\subsection{Dieudonné Modules}
An important task in arithmetics is to classify all $p$-divisible groups over a given base $S$. Historically, this was first achieved in the case $S=\Spec k$, where $k$ is a perfect field of characteristic $p>0$.
\begin{thm}[Dieudonné/Cartier, 1960s]\label{thm:DieudonneCartier}
	Let $k$ be a perfect field of characteristic $p>0$. There is an equivalence of categories
	\begin{equation*}
		M\colon \left\{p\text{-divisible groups over }k\right\}\isomorphism \left\{\text{Dieudonné modules over }W(k)\right\}\,.
	\end{equation*}
	Here a \enquote{Dieudonné module} is a finite free $W(k)$-module together with a $\phi$-linear action of an operator $F$, and a $\phi^{-1}$-linear operator $V$, such that $FV=p=VF$.
\end{thm}
\begin{rem}
	\begin{numerate}
		\item Actually, the original formulation of \cref{thm:DieudonneCartier} concerns finite flat group schemes over $k$ (and the version above is an easy consequence), as $p$-divisible have only been defined some five years later.
		\item For the purpose of this lecture we always work with \emph{covariant} Dieudonné modules. They are related to their contravariant counterparts via $M^{\mathrm{co}}(G)=M^{\mathrm{contra}}(G^\vee)$.
	\end{numerate}
\end{rem}
Let $C$ be as before and consider the semi-perfect ring $\Oo_C/p\Oo_C$ (which is to say that the Frobenius on $\Oo_C/p\Oo_C$ is surjective). Recall that $\IA_\inf =W(\Oo_C^\flat)=W((\Oo_C/p\Oo_C)^\flat)$ by \cref{prop:(A/I)b} and we have defined  a map $\IA_\inf \morphism \IA_\cris$ in \cref{subsec:Acris}.
\begin{defi}
	A \defemph{Dieudonné module over $\Oo_C/p\Oo_C$} is a finite free $\IA_\cris$-module $M$ together with operators (\defemph{Frobenius} and \defemph{Verschiebung})
	\begin{equation*}
		F\colon M\otimes_{\IA_\cris,\phi}\IA_\cris\morphism M\,,\quad V\colon M\otimes_{\IA_\cris,\phi^{-1}}\IA_\cris \morphism M
	\end{equation*}
	satisfying $FV=p=VF$.
\end{defi}
\begin{prop}[Grothendieck--Messing, Scholze--Weinstein]
	There exists a fully faithful functor
	\begin{equation*}
		M_\cris\colon \left\{p\text{-divisible groups over $\Oo_C/p\Oo_C$}\right\}\morphism\left\{\text{Dieudonné modules over }\IA_\cris\right\}\,.
	\end{equation*}
	We have $\rk M_\cris(G)=\hoehe G$ and $M_\cris(G^\vee)=M_\cris(G)^\vee$, where the dual on the right-hand side is taken as a dual of $\IA_\cris$-modules.
\end{prop}
\begin{exm}\label{exm:Mcris}
	\begin{numerate}
		\item We have $M_\cris(\IQ_p/\IZ_p)=\IA_\cris$ with $F=p$ and $V=1$.
		\item We have $M_\cris(\mu_{p^\infty})=\IA_\cris$ with $F=1$ and $V=p$.
		\item If $A$ is an abelian scheme over $\Oo_C$, then we have 
		\begin{equation*}
			M_\cris\big(A_{\Oo_C/p\Oo_C}[p^\infty]\big)^\vee=H_\cris^1\big(A_{\Oo_C/p\Oo_C}/\IA_\cris\big)\,,
		\end{equation*}
		(the dual on the left-hand side is taken as an $\IA_\cris$-module) in a way that identifies $F$ with the usual Frobenius $\phi$.
	\end{numerate}
\end{exm}
\subsection{Connections to \texorpdfstring{$p$}{p}-adic Hodge Theory}
\begin{qst}
	Let's take a $p$-divisible group $G$ over $\Oo_C$. Are the Tate module $T_pG$ and the Dieudonné module $M_\cris(G_{\Oo_C/p\Oo_C})$ related in any way? In the special case where $G=A[p^\infty]$, the question is essentially how
	\begin{equation*}
	H_\et^1(A_C,\IZ_p)\quad\text{and}\quad H_\cris^1\big(A_{\Oo_C/p\Oo_C}/\IA_\cris\big)
	\end{equation*}
	are related (see \cref{exm:duals}\itememph{3} and \cref{exm:Mcris}\itememph{3}). This leads straight into the land of \emph{$p$-adic Hodge theory}, which is after all the field of studying  comparisons between different $p$-adic cohomology theories. A partial answer is given by the following theorem.
\end{qst}
\begin{thm}[{\cite[Theorem~14.5(i)]{BMS}}, 2016]\label{thm:BMS}
	Let $\XX$ be a smooth proper formal scheme over $\Oo_C$. Then for all $i\geq 0$ there is an étale-crystalline comparison isomorphism
	\begin{equation*}
		H_\et^i(\XX_C,\IZ_p)\otimes_{\IZ_p}B_\cris\cong H_\cris^i\big(\XX_{\Oo_C/p\Oo_C}/\IA_\cris\big)\otimes_{\IA_\cris}B_\cris\,.
	\end{equation*}
\end{thm}
\begin{rem}\label{rem:periodRings}
	We should leave some remarks on the period rings $B_\cris$ and $B_\dR$.
	\begin{numerate}
		\item Let's first recall their construction. Consider the element $\epsilon=(1,\zeta_p,\zeta_{p^2},\dotsc)\in \Oo_C^\flat\cong \Oo_F$. Then $[\epsilon]$ is an element of $\IA_\inf$, whose logarithm $t=\log{[\epsilon]}$ can be considered both as an element of $B^{\phi=p}$ and of $\IA_\cris$, because the corresponding power series converges both in $B$ (as observed in the proof of \cref{lem:Qp}) and $\IA_\cris$ (basically by construction, as elements of the form $p^n/n!$ were adjoined). As in \cref{subsec:FFCandAcris}, we define
		\begin{equation*}
			B_\cris^+\coloneqq \IA_\cris\localize{p}\quad\text{and}\quad B_\cris\coloneqq B_\cris^+\localize{t}\,.
		\end{equation*}
		Moreover, the element $t\in B^{\phi=p}$ defines a point $\infty_t\in X_\FFC$ on the Fargues--Fontaine curve, with completed local ring
		\begin{equation*}
			B_\dR^+\coloneqq \roof{\Oo}_{X_\FFC,\infty_t}\quad\text{and}\quad B_\dR\coloneqq B_\dR^+\localize{t}\,.
		\end{equation*}
		Finally, we have seen in \cref{lem:BdRx} can also be constructed as the completion of $\IA_\inf$ along the kernel $(\xi_y)$ of Fontaine's map $\theta_y\colon \IA_\inf\morphism \Oo_{C_y}$, where $y\in |Y|$ is mapped to $\infty_t$. Via $C=C_y$ this fits into our situation.
		\item In some sense, the \enquote{first} example of such a comparison isomorphism is given by de Rham's theorem: if $X$ is smooth over $\IC$, then there is an isomorphism 
		\begin{equation*}
			H_\mathrm{sing}^i(X(\IC),\IZ)\otimes_\IZ\IC\cong H_\dR^i(X/\IC)\,.
		\end{equation*}
		Observe that $\IC$ plays the same role here as $B_\cris$ in \cref{thm:BMS}/\cref{thm:crystallineStuff} and $B_\dR$ in \cref{thm:deRhamComp}. The proof of de Rham's theorem relies on the Poincaré lemma, which is about integration of forms. In the algebraic world this leads to the notion of \emph{periods}, hence the name \emph{period rings} for $B_\cris$ and $B_\dR$. In particular, $\IC$ is the corresponding period ring for the singular-de-Rham comparison isomorphism.
	\end{numerate}
\end{rem}
Combining \cref{thm:BMS} with \cref{exm:duals}\itememph{3} and \cref{exm:Mcris}\itememph{3}, we find that there exists a comparison isomorphism between $T_pG$ and $M(G)\coloneqq M_\cris(G_{\Oo_C/p\Oo_C})$ in the special case where $G=A[p^\infty]$. It turns out that this works in fact for arbitrary $G$!
\begin{prop}\label{prop:TpGMGcomparison}
	Let $G$ be a $p$-divisible group over $\Oo_C$.
	\begin{numerate}
		\item There is a a $\phi$-equivariant isomorphism $\beta_G\colon T_pG\otimes_{\IZ_p}B_\cris\isomorphism M(G)\otimes_{\IA_\cris}B_\cris$.
		\item After tensoring with $-\otimes_{\IA_\cris}B_\dR^+$, we get a chain of inclusions
		\begin{equation*}
			T_pG\otimes_{\IZ_p}B_\dR^+\subseteq M(G)\otimes_{\IA_\cris}B_\dR^+\subseteq t^{-1}\left(T_pG\otimes_{\IZ_p}B_\dR^+\right)\,,
		\end{equation*}
		which become isomorphisms after tensoring with $-\otimes_{B_\dR^+}B_\dR$. For future use, the ring in the middle will be denoted $\Xi$.
	\end{numerate}
\end{prop}
\begin{proof}[Sketch of a proof]
	Recall that $T_pG\cong \Hom(\IQ_p/\IZ_p,G)$ by \cref{lem:TpGHom}. This allows us to construct a morphism 
	\begin{equation*}
		\beta_G^+\colon T_pG\otimes_{\IZ_p}\IA_\cris\morphism M(G)
	\end{equation*}
	 as follows: any $\alpha\colon \IQ_p/\IZ_p\morphism G$ induces a map $M(\alpha)\colon \IZ_p= M(\IQ_p/\IZ_p)\morphism M(G)$ (we have to take the equality $\IZ_p=M(\IQ_p)/\IZ_p$ for granted). Now $\beta_G^+$ may be defined via $\beta_G^+(\alpha)=M(\alpha)(1)$ and $\IA_\cris$-extension of scalars.
	 
	 To show that $\beta_G^+$ becomes an isomorphism $\beta_G$ after tensoring with $B_\cris$, we construct a map in the reverse direction. This can be done as follows: swapping $G$ and $G^\vee$, we obtain a map $\beta_{G^\vee}^+\colon T_pG^\vee \otimes_{\IZ_p}\IA_\cris\morphism M(G^\vee)$. Dualizing in the category of $\IA_\cris$-modules yields
	 \begin{equation*}
	 	(\beta_{G^\vee}^+)^\vee\colon M(G)\cong M(G^\vee)^\vee\morphism (T_pG^\vee\otimes_{\IZ_p}\IA_\cris)^\vee\cong (T_pG^\vee)^\vee\otimes_{\IZ_p}\IA_\cris\,.
	 \end{equation*}
	 We will see that $(\beta_{G^\vee}^+)^\vee$ provides the desired map in the reverse direction (Ben points out that this is a common trick in such situations). Observe that 
	 \begin{equation*}
	 	(T_pG^\vee)^\vee\cong \Hom_{\IZ_p}\big(T_pG,\IZ_p(1)\big)=T_pG(-1)
	 \end{equation*}
	 by \cref{lem:TpGHom}\itememph{2}. Moreover, since $\epsilon$ is a generator of the free $\IZ_p$-module $\IZ_p(1)$, we can construct a $B_\cris^+$-linear isomorphism $\IZ_p(1)\otimes_{\IZ_p}B_\cris^+\isomorphism tB_\cris^+$ via $\epsilon^a\otimes x\mapsto \log{[\epsilon^a]}x=tax$. Dualizing once again gives an isomorphism $B_\cris(-1)\cong t^{-1}B_\cris^+$. Thus,
	 \begin{equation*}
	 	(T_pG^\vee)^\vee\otimes_{\IZ_p}B_\cris^+\cong T_pG(-1)\otimes_{\IZ_p}B_\cris^+\cong T_pG\otimes_{\IZ_p}B_\cris^+(-1)\cong t^{-1}\left(T_pG\otimes_{\IZ_p}B_\cris^+\right)\,.
	 \end{equation*}
	 Summarizing, we find out that $\beta_G^+$ and $(\beta_{G^\vee}^+)^\vee$ induce maps
	 \begin{equation*}
	 	T_pG\otimes_{\IZ_p}B_\cris^+\morphism M(G)\otimes_{\IA_\cris}B_\cris^+\morphism t^{-1}\left(T_pG\otimes_{\IZ_p}B_\cris^+\right)\,.
	 \end{equation*}
	 It can be checked that the composition of the above maps is just the natural inclusion $T_pG\otimes_{\IZ_p}B_\cris^+\subseteq t^{-1}(T_pG\otimes_{\IZ_p}B_\cris^+)$ (we have to take this for granted though, since no one told us an explicit construction of $M=M_\cris$). Thus, the first arrow of the above sequence must be an injection. Via a similar argument we find that the same is true for the second arrow. Therefore we get a chain of inclusions
	 \begin{equation*}
	 T_pG\otimes_{\IZ_p}B_\cris^+\subseteq M(G)\otimes_{\IA_\cris}B_\cris^+\subseteq t^{-1}\left(T_pG\otimes_{\IZ_p}B_\cris^+\right)\,.
	 \end{equation*}
	 This immediately implies \itememph{a} since $B_\cris=B_\cris^+\localize{t}$. To deduce \itememph{b}, we can repeat the above arguments with $B_\dR^+$ in place of $B_\cris^+$ (in particular, we get $\IZ_p(1)\otimes_{\IZ_p}B_\dR^+\isomorphism tB_\dR^+$ and then the rest can be carried over) to obtain the same statement for $B_\dR^+$.
\end{proof}
\subsection{The Vector Bundles \texorpdfstring{$\Ee(G)$}{E(G)}}
Today, we restrict ourselves to the case $E=\IQ_p$, $\pi=p$. In this case the Fargues--Fontaine curve can be described as
\begin{equation*}
	X_\FFC=\Proj\Bigg(\bigoplus_{d\geq 0}(B_\cris^+)^{\phi=p^d}\Bigg)
\end{equation*}
by \cref{prop:FFandBcris}. We know that $X_\FFC$ is a Dedekind scheme. Moreover, Fontaine's map $\theta\colon \IA_\inf \morphism \Oo_C$ defines a point $\infty=\infty_t\in X$ with residue field $C$, hence a morphism $i_\infty \colon \Spec C\monomorphism X$. By definition, completion at $\infty$ gives $\Spec B_\dR^+\morphism X$ (compare this to \cref{rem:periodRings}\itememph{1}).
\begin{con}
	Let $G$ be a $p$-divisible group over $\Oo_C/p\Oo_C$. We associate a quasi-coherent sheaf on $X$ via
	\begin{equation*}
		\Ee(G)=\Bigg(\bigoplus_{d\geq 0}\left(M(G)\localize{p}\right)^{F=p^d}\Bigg)^\qcmod\,.
	\end{equation*}
	Here $(-)^\qcmod$ denotes the graded twiddleization, as usual. We will see later that $\Ee(G)$ is a vector bundle of rank $\hoehe G$.
\end{con}
Now let $G$ be a $p$-divisible group over $\Oo_C$. By \cref{prop:TpGMGcomparison}\itememph{1} above, there exists a natural map
\begin{equation*}
	T_pG\otimes_{\IZ_p} \Oo_{X_\FFC}\morphism \Ee(G)
\end{equation*}
which is an isomorphism on the locus $X_\FFC\setminus \{\infty\}$ where $t$ is invertible.
\numpar{Corollary \smash{\Attention}}\label{cor:E(G)SES} \itshape There is a natural short exact sequence of sheaves on $X_\FFC$
	\begin{equation*}
		0\morphism T_pG\otimes_{\IZ_p} \Oo_{X_\FFC}\morphism\Ee(G)\morphism i_{\infty,*}W\morphism 0\,,
	\end{equation*}
	where $W$ is the finite-dimensional $C$-vector space given by the image of $\Xi$ under the map\upshape
	\begin{equation*}
		{\id}\otimes {\theta(-1)}\colon t^{-1}(T_pG\otimes_{\IZ_p}B_\dR^+)\morphism T_pG\otimes_{\IZ_p}C(-1)\,.
	\end{equation*}
\begin{proof}
	First note that $t^{-1}(T_pG\otimes_{\IZ_p}t^{-1}B_\dR^+)\cong T_pG\otimes_{\IZ_p}t^{-1}B_\dR^+$ and $t^{-1}B_\dR^+\cong B_\dR^+(-1)$ as observed in the proof of \cref{prop:TpGMGcomparison}, so ${\id}\otimes {\theta(-1)}$ is indeed a map as claimed.
	
	We have already seen that $\Ff\morphism \Ee(G)$ is an isomorphism away from the vanishing set $\{\infty\}$ of $t$. In particular its cokernel is only supported on $\{\infty\}$. To analyze the behaviour there, we may instead investigate the behaviour on an \enquote{infinitesimal neighbourhood} of $\{\infty\}$, i.e., after tensoring with $B_\dR^+$. That's precisely what \cref{prop:TpGMGcomparison}\itememph{2} is for! Consider the diagram
	\begin{equation*}
		\begin{tikzcd}
		0\rar &T_pG\otimes_{\IZ_p}B_\dR^+\eqar[d]\rar & M(G)\otimes_{\IA_\cris}B_\dR^+\rar\dar[mono] &[1em] W\dar[mono]\rar & 0\\
		0\rar &T_pG\otimes_{\IZ_p}B_\dR^+\rar & t^{-1}(T_pG\otimes_{\IZ_p}B_\dR^+) \rar["\id\otimes{\theta(-1)}"] &[1em] T_pG\otimes_{\IZ_p}C(-1) \rar & 0
		\end{tikzcd}
	\end{equation*}
	The bottom row is exact since $t^{-1}B_\dR^+/B_\dR^+\cong B_\dR^+/tB_\dR^+\cong C\cong C(-1)$ as $C$-vector spaces. The top row is injective on the left and surjective on the right by definition of $W$. Hence the top row is exact too, proving that $\Ff\morphism \Ee(G)$ is injective and has cokernel $i_\infty W$.
\end{proof}
\begin{defi}
	A \defemph{minuscule modification on $X_\FFC$} is a short exact sequence of quasi-coherent sheaves on $X_\FFC$ of the form
	\begin{equation*}
		0\morphism \Ff\morphism \Ff'\morphism i_{\infty,*}W\morphism 0\,,
	\end{equation*}
	where $\Ff$ and $\Ff'$ are vector bundles such that $\Ff$ is trivial, and $W$ is a finite-dimensional $C$-vector space. A \defemph{morphism of minuscule modifications} is a morphism of the corresponding short exact sequences.
\end{defi}
\numpar{}\label{par:functor}
	We can thus form a \defemph{category of minuscule modifications on $X_\FFC$}. In view of Corollary~\cref{cor:E(G)SES} we have defined a functor
	\begin{equation*}
		\left\{\text{$p$-divisible groups over $\Oo_C$}\right\}\morphism \left\{\text{minuscule modifications on $X_\FFC$}\right\}\,.
	\end{equation*}
	Moreover, fixing $\Ff$, we can define a \defemph{category of minuscule modifications of $\Ff$}. For any finite free $\IZ_p$-module $T$, sending a minuscule modification to its cokernel defines a forgetful functor
	\begin{equation*}
	\left\{\text{minuscule modifications of }T\otimes_{\IZ_p}\Oo_{X_\FFC}\right\}\morphism \left\{\text{$C$-subvector spaces }W\subseteq T\otimes_{\IZ_p}C(-1)\right\}\,.
	\end{equation*}
	Surprisingly, it turns out that this is an equivalence of categories! This reduces the problem of classifying $p$-divisible groups over $\Oo_C$ to a simple linear algebra question: determining $C$-subvector spaces of a given $T\otimes_{\IZ_p}C(-1)$!
\begin{thm}[{\cite[Theorem~5.2.1]{ScholzeWeinstein}}, 2012]\label{thm:ScholzeWeinstein}
	The functors from \cref{par:functor} define an equivalence of categories
	\begin{equation*}
		\left\{\text{$p$-divisible groups over $\Oo_C$}\right\}\isomorphism\left\{\begin{tabular}{c}
		pairs $(T,W)$, s.th.\ $T$ is a finite free $\IZ_p$-module\\
		and $W\subseteq T\otimes_{\IZ_p}C(-1)$ a $C$-subvector space
		\end{tabular}\right\}\,.
	\end{equation*}
\end{thm}
	
\subsection{Modifications of Vector Bundles}
Back to $X_\FFC$: we must still show that $\Ee(G)$ is a vector bundle. It is a general fact that $\Ee(G)$ can be reconstructed from the data of $\Ff$ and $\Xi$ by means of the \emph{Beauville--Laszlo theorem}.
\begin{lem}[Beauville--Laszlo]\label{lem:BeuvilleLaszlo}
	Let $A$ be a noetherian ring, $f\in A$ and denote by $\roof{A}\coloneqq \limit_{n\in \IN}A/f^nA$ its $f$-adic completion. Then the functor
	\begin{equation*}
		\cat{Mod}_A\isomorphism \cat{Mod}_{A[f^{-1}]}\times_{\cat{Mod}_{\roof{A}[f^{-1}]}}\cat{Mod}_{\roof{A}}
	\end{equation*}
	is an equivalence of categories.
\end{lem}
\begin{proof*}
	Observe that $\Spec A[f^{-1}]\sqcup \Spec \roof{A}\morphism \Spec A$ is an fpqc cover of $A$. Thus, the assertion follows immediately from faithfully flat descent. In case you are wondering \enquote{How on earth is this a \emph{theorem} that wasn't known until 1995?}, your doubts are legitimate: the \emph{actual} Beauville--Laszlo theorem (see \cite{BeauvilleLaszlo}) is about not necessarily noetherian rings $A$, so that $\roof{A}$ may fail to be flat over $A$.
\end{proof*}
\begin{cor}\label{cor:BeauvilleLaszlo}
	Let $X$ be a Dedekind scheme, $x\in X$ any closed point and put $\roof{X}\coloneqq \Spec \roof{\Oo}_{X,x}\morphism X$. Then
	\begin{equation*}
		\cat{Bun}_X\isomorphism \cat{Bun}_{X\setminus \{x\}}\times_{\cat{Bun}_{\roof{X}\setminus \{x\}}}\cat{Bun}_{\roof{X}}
	\end{equation*}
	is an equivalence of categories. In particular, $\Ee(G)$ is a vector bundle.
\end{cor}
\begin{proof}
	Work affine-locally and use \cref{lem:BeuvilleLaszlo} together fpqc descent of local freeness. Alternatively, apply faithfully flat descent directly to the fpqc cover $(X\setminus \{x\})\sqcup \roof{X}\morphism X$. To see why $\Ee(G)$ is a vector bundle, observe that by \cref{prop:TpGMGcomparison}, $\smash{\Ee(G)|_{X\setminus \{x\}}}$ is a vector bundle and $\Ee(G)\otimes \roof{\Oo}_{X_\FFC,\infty}=\Ee(G)\otimes B_\dR^+$ is a $B_\dR^+$-submodule of the finite free module $t^{-1}(T_pG\otimes_{\IZ_p}B_\dR^+)$, hence finite free itself because the ring $B_\dR^+$ is a DVR (\cref{lem:BdR+DVR}).
\end{proof}
\begin{defi}
	A \defemph{finite free Breuil--Kisin--Fargues module} is a finite free $\IA_\inf$-module $M$ together with a $\IA_\inf$-linear map
	\begin{equation*}
		\phi_M\colon \phi^*M{\localize{\phi(\xi)}}\morphism M{\localize{\phi(\xi)}}\,.
	\end{equation*}
\end{defi}
\begin{thm}[Fargues]
	The following categories are equivalent.
	\begin{numerate}
		\item The category of Breuil--Kisin--Fargues modules.
		\item The category of quadruples $(\Ff,\Ff',\beta,T)$ where $\Ff,\Ff'$ are vector bundles on $X_\FFC$ such that $\Ff$ is trivial, $\beta\colon \Ff|_{X_\FFC\setminus \{\infty\}}\isomorphism \Ff'|_{X_\FFC\setminus \{\infty\}}$ is an isomorphism, and $T\subseteq H^0(X_\FFC,\Ff)$ is a $\IZ_p$-lattice.
		\item The category of pairs $(T,\Xi)$, where $T$ is a finite free $\IZ_p$-module and $\Xi\subseteq T\otimes_{\IZ_p}B_\dR$ is a $B_\dR^+$-lattice.
	\end{numerate}
\end{thm}
\begin{proof}
	Equivalence of \itememph{2} and \itememph{3} can be seen from the Beauville--Laszlo theorem in the form of \cref{cor:BeauvilleLaszlo}. In our situation, $\roof{X}=\Spec B_\dR^+$ and $\roof{X}\setminus \{x\}=\Spec B_\dR^+\setminus \{\infty\}=\Spec B_\dR$. Suppose we're given the data from \itememph{2}. Then only $\Xi$ needs to be constructed. Since $\Ff$ is trivial, we get $H^0(X_\FFC,\Ff)\cong \IQ_p^{\oplus \rk \Ff}$ from \cref{lem:HiOXn} and \cref{lem:PEigenspaces}. Hence $\Ff\cong T\otimes_{\IZ_p}\Oo_{X_\FFC}$. The restriction $\Ff'\otimes B_\dR^+$ is a finite free module over $B_\dR^+$. Since $B_\dR$ is a localization of $B_\dR^+$, we see that $\Ff'\otimes B_\dR^+$ is a $B_\dR^+$-lattice in $\Ff'\otimes B_\dR\cong \Ff\otimes B_\dR\cong T\otimes_{\IZ_p}B_\dR$ (the first isomorphism is induced by $\beta$). Thus we can take $\Xi=\Ff'\otimes B_\dR^+$.
	
	Conversely, assume we are given the data from \itememph{3}. Put $\Ff=T\otimes_{\IZ_p}\Oo_{X_\FFC}$. Then $T$ is a $\IZ_p$-sublattice of $H^0(X_\FFC,\Ff)\cong \IQ_p^{\oplus\rk \Ff}$. Moreover, observe that $\Xi\otimes B_\dR\cong T\otimes_{\IZ_p}B_\dR$ since $\Xi$ is a lattice contained in the right-hand side. Thus, $\Ff|_{X_\FFC\setminus \{\infty\}}$ and $\Xi$ define a vector bundle $\Ff'$ on $X_\FFC$ via \cref{cor:BeauvilleLaszlo}. In particular, the construction yields an isomorphism $\beta\colon \Ff|_{X_\FFC\setminus\{\infty\}}\isomorphism \Ff'|_{X_\FFC\setminus\{\infty\}}$.
	
	This proves that \itememph{2} and \itememph{3} are equivalent. It's much harder to show that \itememph{1} is equivalent to the other two, and we omit the proof.
\end{proof}

\section{Classification of Vector Bundles}
\lecture[This lecture was a big black box: $\blacksquare$.]{2020-01-22}As before, let $X_\FFC$ denote the Fargues--Fontaine curve associated to $E$ and $F$, and let $\breve{E}=W_{\Oo_E}(\ov{\IF}_q)\localize{p}$ denote the completion of the maximal unramified extension of $E$. We won't prove the \cref{mainthm:vectorBundles} today, and not even give a proper sketch. Instead, we will link \cref{mainthm:vectorBundles} to various results in other parts of mathematics and give a rough idea how it follows from these.

By pullback to the étale coverings $X_{\FFC,h}\coloneqq X_\FFC\otimes_EE_h$ for unramified extension $E_h/E$, descent, the HN-formalism, knowledge of $H^\bullet(X_\FFC,\Oo_{X_\FFC}(\lambda)),\dotsc,$ one can reduce the \cref{mainthm:vectorBundles} to the following.
\begin{thm}\label{thm:modifications}
	Let $n>0$ be a positive integer.
	\begin{numerate}
		\item If $0\morphism \Ee\morphism \Oo_{X_\FFC}(1/n)\morphism \Ff\morphism 0$ is an exact sequence in which $\Ff$ is a torsion sheaf of degree one (i.e.\ a skyscraper sheaf supported at a single closed point), then $\Ee\cong \Oo_{X_\FFC}^{\oplus n}$.
		\item If $0\morphism \Ee\morphism \Oo_{X_\FFC}^{\oplus n}\morphism \Ff \morphism 0$ is an exact sequence with $\Ff$ as in \itememph{1}, then there exists an $m\in \{1,\dotsc,n\}$ such that $\Ee\cong \Oo_X^{\oplus \smash{(n-m)}}\oplus \Oo_X(-1/m)$.
	\end{numerate}
\end{thm}
\numpar{}\label{par:PEdecomp}Again, we won't give an actual proof of \cref{thm:modifications}, but a reduction to deep results of Gross--Hopkins and Drinfeld. Fix a closed point $x\in X_\FFC$ and let $C=\kappa(x)$ be its residue field, so that $C$ is an untilt of $F$ in characteristic $0$. Let $i\colon \Spec C\monomorphism X_\FFC$. For an arbitrary sheaf $\Ff$ on $X_\FFC$, we denote $\Ff(x)\coloneqq \Ff\otimes C$ (this is some $C$-vector space). For any vector bundle $\Ee'$ on $X_\FFC$ put
\begin{equation*}
	\Mm_{\Ee'}\coloneqq\left\{\Ee\subseteq \Ee'\st \Ee'/\Ee\cong i_*C\right\}\,.
\end{equation*}
Note that $\Mm_{\Ee'}$ is in canonical bijection with the set of equivalence classes of epimorphisms $\Ee'\epimorphism i_*C$, where two such epimorphisms are considered \emph{equivalent} iff they have the same kernel, or equivalently, iff they only differ by an automorphism of $C$. By the $i^*$-$i_*$ adjunction, this is in canonical bijection with the set of similar equivalence classes of epimorphisms $\Ee'(x)=i^*\Ee\epimorphism C$. The latter, however, characterizes precisely the $C$-valued points of the projectivization $\IP(\Ee'(x))$. The upshot is that we obtain a bijection
\begin{equation*}
	\Mm_{\Ee'}\isomorphism \IP\big(\Ee'(x)\big)(C)
\end{equation*}
sending $\Ee\subseteq \Ee'$ to the epimorphism $\Ee\otimes C\epimorphism\Ee/\Ee'\otimes C$. In particular, we get a decomposition 
\begin{equation*}
	\IP\big(\Ee'(x)\big)(C)=\coprod_{[\Ff]\in \Bun_{X_\FFC}/\cong}\IP\big(\Ee'(x)\big)(C)_{[\Ff]}\,,
\end{equation*}
where the terms on the right-hand side denote the \enquote{loci where $\Ee\cong \Ff$}, i.e., the sets $\left\{\Ee\in \Mm_{\Ee'}\st \Ee\cong \Ff\right\}$.

\numpar{}For simplicity, we only work in the special case $E=\IQ_p$ (in general one needs to replace $p$-divisible groups by \enquote{$\pi$-divisible $\Oo_E$-modules}). Johannes Anschütz pointed out that this special case does not suffice to prove the \cref{mainthm:vectorBundles}, not even in the special case $E=\IQ_p$, because the proof needs \cref{thm:modifications} for all finite unramified extensions $E_h/E$ rather than just for $E$.

Fix a connected $p$-divisible group $H$ of dimension $1$ and height $n$ over $\ov{\IF}_p$. By some result we will not prove, $H$ is unique up to unique isomorphism. We also didn't define what the \enquote{dimension} of a $p$-divisible group is, but the amount of blackboxing in this lecture is high enough that this detail doesn't make a difference any more. By \cref{thm:DieudonneCartier}, $H$ corresponds to some Dieudonné module $M(H)$ over $W(\ov{\IF}_p)$. Upon inverting $p$, the associated Frobenius $F$ becomes a $\phi$-linear isomorphism (with inverse $p^{-1}V$), hence $M(H)\localize{p}=M(H)\otimes\breve{E}$ is an isocrystal over $\smash{\breve{E}}$ in the sense of \cref{def:phiModule}. It can be checked that
\begin{equation*}
	\Ee\big(M(H)\otimes \breve{E}\big)\cong \Oo_{X_\FFC}(1/n)\,.
\end{equation*}
Now Consider the set
\begin{equation*}
	\Mm_{\LT,\eta}^\ad=\left\{\text{iso.\ classes of }(G,\alpha)\st 
	\begin{tabular}{c}
		$G$ is a $p$-divisible group over $\Oo_C$, and\\
		$\alpha\colon G\otimes_{\Oo_C} \Oo_C/p\Oo_C\isomorphism H\otimes_{\ov{\IF}_p}\Oo_C/p\Oo_C$
	\end{tabular}\right\}\,.
\end{equation*}
The notation suggests that this is the set of \enquote{$C$-valued points of the adic generic fibre of some Lubin--Tate space $\Mm_{\LT}^\ad$}, and in fact, this can be made precise. This space comes with the \emph{Gross--Hopkins period morphism}
\begin{equation*}
	\pi_\dR\colon \Mm_{\LT,\eta}^\ad\morphism \IP\big(M(H)\otimes_{W(\ov{\IF}_p)}C\big)(C)
\end{equation*}
(also note that the right-hand side is abstractly isomorphic to $\IP^{n-1}(C)$ since $M(H)\otimes C$ is $n$-dimensional). This morphism can be described as follows: a pair $(G,\alpha)$ is sent to the $C$-valued point of $\IP(M(H)\otimes C)$ given by the surjection 
\begin{equation*}
M(H)\otimes_{W(\ov{\IF}_p)}C\overset{\alpha}{\isomorphism}M(G)\otimes_{\IA_\cris}C\epimorphism \operatorname{Lie}(G)\cong C\,.
\end{equation*}
The $\operatorname{Lie}(G)$ on the right-hand side looks terrifying, but in reality this is just another name (or rather another \emph{construction}) of the space $W$ from Corollary~\cref{cor:E(G)SES}, so no magic is happening. The fact that $\operatorname{Lie}(G)$ is a one-dimensional $C$-vector space is another way of saying that $G$ has dimension $1$, which is true because $H$ has dimension $1$ as well.
\begin{thm}[Gross--Hopkins]\label{thm:GrossHopkins}
	The morphism $\pi_\dR$ is surjective and étale.
\end{thm}
\begin{rem}
	\cref{thm:GrossHopkins} should feel a bit strange on first glance: we would expect that $\IP(M(H)\otimes C)\cong \IP^{n-1}(C)$ has vanishing étale fundamental group, hence every étale covering should be split. However, in the adic world there exist \enquote{infinite étale coverings of $\IP^{n-1}(C)$}.
\end{rem}
\begin{proof}[Sketch of a proof of \cref{thm:modifications}\itememph{1}]
	\cref{thm:GrossHopkins} is all we need to prove the first assertion. Observe that since $\Ee'=\Oo_X(1/n)$, we have $\Ee'(x)=M(G)\otimes_{\IA_\cris} C\cong M(H)\otimes C$. Hence the period morphism $\pi_\dR$ can be written as
	\begin{equation*}
		\pi_\dR\colon \Mm_{\LT,\eta}^\ad\morphism \Mm_{\Ee'}\epimorphism \IP\big(\Ee'(x)\big)(C)\,.
	\end{equation*}
	In particular, every $\Ee\in \Mm_{\Ee'}$ can be written as $\pi_\dR(G,\alpha)$, hence the corresponding sequence is isomorphic to
	\begin{equation*}
		0\morphism T_pG\otimes_{\IZ_p}\Oo_{X_\FFC}\morphism \Ee'\morphism i_*\operatorname{Lie}(G)\morphism 0
	\end{equation*}
	from Corollary~\cref{cor:E(G)SES}, hence $\Ee\cong T_pG\otimes_{\IZ_p}\Oo_{X_\FFC}$ is indeed a trivial vector bundle.
\end{proof}
Conversely, \cref{thm:GrossHopkins} follows from the classification of vector bundles on $X_\FFC$ (and \cref{mainthm:vectorBundles} does have an alternative proof not using $p$-divisible groups, so this is no circular reasoning). 
\begin{proof}[Sketch of a conditional proof of \cref{thm:GrossHopkins}]
	By some magic, the Scholze--Weinstein classification (\cref{thm:ScholzeWeinstein}, which relies on \cref{mainthm:vectorBundles}) implies that the image of $\pi_\dR$ is the \enquote{admissible locus}, i.e., the set of all $\Ee\in \Mm_{\Ee'}$ which are semistable. So suppose we have a sequence 
	\begin{equation*}
		0\morphism \Ee\morphism \Oo_{X_\FFC}(1/n)\morphism i_*C\morphism 0
	\end{equation*}
	in which $\Ee$ is not semistable. We claim:
	\begin{alphanumerate}
		\item[\itememph{*}] $\Ee$ has always degree $0$, hence slope $0$ (and here it is irrelevant whether $\Ee$ is semistable or not).
	\end{alphanumerate}
	The general fact behind \itememph{*} is the following: let $X$ be a Dedekind scheme and $\Ff$, $\Ff'$ vector bundles of rank $r$ on $X$ that fit into a short exact sequence $0\morphism \Ff\morphism\Ff'\morphism \Tt\morphism 0$, in which $\Tt$ is a torsion sheaf. Consider the \defemph{ramification divisor}
	\begin{equation*}
		R=\sum_{x\in X}\length_{\Oo_{X,x}}(\Tt_x)\cdot \{x\}\,.
	\end{equation*}
	Then $\bigwedge^r\Ff'\cong \bigwedge^r\Ff\otimes \Oo_X(R)$. In our situation, $\Tt=i_*C$ is only supported at the chosen point $x\in X_\FFC$, hence $R=\{x\}$ and $\Oo_{X_\FFC}(R)\cong\Oo_{X_\FFC}(1)$. Thus $\deg \Ee=\deg \Oo_{X_\FFC}(1/n)-1=0$, proving \itememph{*}. In case you want a reference for the general fact: I got it from \cite[Lemma~11]{AlternativeSerreDuality}, but there's probably a better reference.
	
	Now if $\Ee$ is not semistable, it has a non-trivial HN-filtration. In particular, there is a $\Oo_X(\lambda)=\Ee_1\subseteq \Ee$ with larger slope and smaller rank, so $\lambda=d/r>0$ and $r<n$. But then $\lambda>1/n$, hence the non-zero morphism $\Oo_X(\lambda)\morphism\Oo_X(1/n)$ contradicts \cref{lem:HN}.
\end{proof}
\numpar{}\label{par:motivation}Our next goal is to prove \cref{thm:modifications}\itememph{2} by a similar trick as in the proof of \itememph{1} above. To get the idea, let's describe the decomposition of $\IP(\Ee'(x))(C)$ according to \cref{par:PEdecomp} in the special case $\Ee'=\Oo_X^{\oplus n}$ we are interested in: if \cref{thm:modifications}\itememph{2} is true, then
\begin{equation}\label{eq:PEdecomp2}
	\IP\big(\Ee'(x)\big)(C)_{[\Ff]}\neq \emptyset\quad\text{only if}\quad \Ff=\Oo_{X_\FFC}^{\oplus(n-m)}\oplus \Oo_{X_\FFC}(-1/m)
\end{equation}
for some $m\in \{1,\dotsc,n\}$. The most interesting case is the case where $\Ff$ is semistable, i.e., $\Ff=\Oo_{X_\FFC}(-1/n)$. Believing \cref{eq:PEdecomp2}, we see that $\Ff=\Oo_{X_\FFC}(-1/n)$ holds iff $H^0(X_\FFC,\Ff)=0$ (here we use $H^0(X_\FFC,\Oo_{X_\FFC}(-1/n))=0$, which follows from \cref{lem:OXlambda} in combination with \cref{lem:PEigenspaces} and \cref{lem:HiOXn}). By the long exact cohomology sequence, the condition $H^0(X_\FFC,\Ff)=0$ is fulfilled iff $\IQ_p^n=H^0(X_\FFC,\Ee')\monomorphism H^0(X_\FFC,i_*C)=C$ is injective.

By some magic, this last condition is equivalent to the condition that the point in $\IP(\Ee'(x))(C)\cong \IP^{n-1}(C)$ defined by $\Ff$ is contained in the complement of the union of all $\IQ_p$-rational hyperplanes of codimension $1$ in $\IP^{n-1}(C)$. This complement will be denoted $\Omega^{n-1}(C)$ and is usually called the \defemph{Drinfeld upper half plane}.
\begin{rem}
	In the case $n=2$ we get $\Omega^1(C)=\IP^1(C)\setminus\IP^1(\IQ_p)$ (since $\IQ_p$-rational hyperplanes of dimension $0$ are $\IQ_p$-rational points, hence assemble into $\IP^1(\IQ_p)$), which is reminiscent of the \enquote{upper half plane} $\IP^1(\IC)\setminus \IP^1(\IR)$. Hence the name.
\end{rem}
\begin{proof}[Sketch of a proof of \cref{thm:modifications}\itememph{2}]
	For all $c\geq 1$ let $\operatorname{Hyp}_c$ denote the set of $\IQ_p$-rational hyperplanes of codimension $c$ in $\IP^{n-1}(C)$, i.e., the set of all $\IQ_p$-rational $Z\subseteq \IP^{n-1}(C)$ such that $Z\cong \IP^{n-1-c}(C)$.
	Then there is a decomposition
	\begin{equation*}
		\IP^{n-1}(C)=\Omega^{n-1}(C)\sqcup\coprod_{Z\in \operatorname{Hyp}_1}\Omega_Z^{n-2}(C)\sqcup\coprod_{Z\in \operatorname{Hyp}_2}\Omega_Z^{n-3}(C)\sqcup\dotsb\,.
	\end{equation*}
	The idea of the proof is to identify the above decomposition with the decomposition of $\IP(\Ee'(x))(C)$ constructed in \cref{par:PEdecomp}, with $\coprod_{Z\in \operatorname{Hyp}_c}\Omega_Z^{n-1-c}(C)$ corresponding to $\IP(\Ee'(x))(C)_{[\Ff]}$ for $\Ff=\Oo_{X_\FFC}^{\oplus c}\oplus \Oo_{X_\FFC}(1/(n-c))$ (hopefully the motivational stuff in \cref{par:motivation} helped to make that point).
	
	Using induction, it is thus enough to show that $\Omega^{n-1}(C)$ is contained in the locus $\IP(\Ee'(x))(C)_{[\Ff]}$ associated to $\Ff=\Oo_{X_\FFC}(-1/n)$. To show this, we pull out a big gun again. Drinfeld constructed a moduli problem for $p$-divisible group with some extra structure, together with a period morphism
	\begin{equation*}
		\pi_\dR\colon \Mm_{\Dr,\eta}^\ad\morphism \IP^{n-1}(C)
	\end{equation*}
	whose image is precisely $\Omega^{n-1}(C)$. One can explicitly check that $\pi_\dR$ factors over the subset
	$\left\{\Ee\in \Mm_{\Ee'}\st \Ee\cong \Oo_{X_\FFC}(-1/n)\right\}\subseteq \Mm_{\Ee'}\cong \IP^{n-1}(C)$. This \enquote{finishes} the proof.
\end{proof}
\numpar{}The adic spaces $\Mm_{\LT,\eta}^\ad$ and $\Mm_{\Dr\eta}^\ad$ are interesting through their  relation with the local \enquote{Langlangs} correspondence (LLC) and the Jacquet--Langlands correspondence (JLC). We can put a level structure on $\Mm_{\LT,\eta}^\ad$ and $\Mm_{\Dr\eta}^\ad$. By a theorem of Faltings--Fargues, the respective \enquote{$\infty$-level Lubin--Tate/Drinfeld spaces} $\Mm_{\LT,\eta,\infty}^\ad$ and $\Mm_{\Dr\eta,\infty}^\ad$ are related by an isomorphism
\begin{equation*}
	\begin{tikzcd}
		\Mm_{\LT,\eta,\infty}^\ad\rar[iso, dashed, "\text{Faltings--Fargues}"'] \dar["\GL_n(\IZ_p)\text{-torsor}"']&[5em] \Mm_{\Dr,\eta,\infty}^\ad\dar["\Oo_D^\times\text{-torsor}"]\\
		\llap{$D^\times\curvearrowright$ }\Mm_{\LT,\eta}^\ad &[5em] \Mm_{\Dr,\eta}^\ad\rlap{ $\curvearrowleft \GL_n(\IQ_p)$}
	\end{tikzcd}\,.
\end{equation*}
Here $D$ is the central division algebra over $\IQ_p$ of invariant $1/n$ ($D$ is unique up to isomorphism), and $D^\times$ its group of units. If moreover $W_{\IQ_p}$ denotes the Weil group of $\IQ_p$, then there is, \enquote{roughly} (more on that in the $12\ordinalth$ lecture), a $W_{\IQ_p}\times \GL_n(\IQ_p)\times D^\times$-equivariant isomorphism
\begin{equation*}
	H_c^{n-1}\big(\Mm_{\LT,\eta,\infty}^\ad\otimes {\ov{\IQ}_p},\ov{\IQ}_\ell\big)\mathrel{\text{\enquote{$=$}}}\bigoplus_\pi\sigma(\pi)\otimes \pi\otimes \rho(\pi)^\vee\,,
\end{equation*}
where the sum on the right-hand side is taken over all super-cuspidal representations $\pi$ of $GL_n(\IQ_p)$. Also $\sigma(\pi)$ denotes the $W_{\IQ_p}$-representation associated to $\pi$ via LLC, and $\rho(\pi)$ denotes the $D^\times$-representation associated to $\pi$ via JLC.

The spaces $\Mm_{\LT,\eta,\infty}^\ad$ and $\Mm_{\Dr\eta,\infty}^\ad$ can be defined via the Fargues--Fontaine curve as
\begin{align*}
	\Mm_{\LT,\eta,\infty}^\ad &\mathrel{\text{\enquote{$=$}}} \left\{\Oo_{X_\FFC}^{\oplus n}\monomorphism \Oo_{X_\FFC}(1/n)\st \text{the cokernel is supported at }\infty\right\}\\
	\Mm_{\Dr,\eta,\infty}^\ad &\mathrel{\text{\enquote{$=$}}} \left\{\Oo_{X_\FFC}(-1/n)\monomorphism \Oo_{X_\FFC}^{\oplus n}\st \text{the cokernel is supported at }\infty\right\}
\end{align*}

The switch from $p$-divisible groups to vector bundles on the Fargues--Fontaine curve has the main advantage that we can form the tensor product of vector bundles. This leads to important generalizations.
\begin{defi}
	Let $G$ be a reductive group over $\IQ_p$ (such as $G=\GL_n,\operatorname{GSp}_{2n},\dotsc$). Then we denote 
	\begin{equation*}
		B(G)\coloneqq G(\breve{\IQ}_p)/\phi\cat{\mhyph conj.}
	\end{equation*}
	where modding out $\phi$-conjugation means $b\sim gb\phi(g)^{-1}$, as in \cref{par:phiModules}.
\end{defi}
\begin{thm}[Fargues]
	Let $G/\IQ_p$ be a reductive group. Then there is a natural isomorphism
	\begin{equation*}
		B(G)\isomorphism H_\et^1(X_\FFC,G)
	\end{equation*}
	given as follows: recall that $H_\et^1(X_\FFC,G)$ parametrizes the isomorphism classes of $G$-torsors on $X_\FFC$. By the Tannakian formalism, the latter are in bijection with isomorphism classes of exact $\otimes$-functors $\omega\colon \Rep_{\IQ_p}G\morphism \cat{Bun}_{X_\FFC}$. Now the above isomorphism sends $[b]\in B(G)$ to the $\otimes$-functor $\Ee_b$ given by
	\begin{equation*}
		\Ee_b\colon (V,\rho)\longmapsto\Ee\left(\breve{\IQ}_p\otimes_{\IQ_p}V,\rho(b)\cdot (\phi\otimes \id_V)\right)
	\end{equation*}
	(note that $(\breve{\IQ}_p\otimes V,\rho(b)\cdot (\phi\otimes \id_V))$ is indeed an isocrystal, so the functor $\Ee(-)$ from \cref{def:MysteriousFunctor} can be applied).
\end{thm}
\begin{defi}
	A \defemph{local Shimura datum} is a triple $(G,[b],\{\mu\})$ consisting of the following data:
	\begin{numerate}
		\item a reductive group $G/\IQ_p$,
		\item a $\phi$-conjugacy class $[b]\in B(G)$, and
		\item a conjugacy class $\{\mu\}$ of minuscule geometric characters $\mu\colon \IG_{m,\ov{\IQ}_p}\morphism G_{\ov{\IQ}_p}$.
	\end{numerate}
\end{defi}
\begin{defi}
	Let $(G,[b],\{\mu\})$ be a local Shimura datum. The associated \defemph{local Shimura variety} is \enquote{defined} as
	\begin{equation*}
		\operatorname{Sh}_{(G,[b],\{\mu\})}\mathrel{\text{\enquote{$\coloneqq$}}}\left\{\alpha\colon \Ee_1|_{X_\FFC\setminus \{\infty\}}\isomorphism \Ee_b|_{X_\FFC\setminus\{\infty\}}\st \begin{tabular}{c}
		the relative position of\\
		$\alpha$ is bounded by $\mu$
		\end{tabular}\right\}\,.
	\end{equation*}
	Here the \enquote{relative position} is an element of $G(B_\dR^+)\cdot\mu(t)\cdot G(B_\dR^+)\subseteq G(B_\dR)$ and the left-hand side comes with an action of $G(\IQ_p)\times J_b(\IQ_p)$, where $J_b(\IQ_p)=\Aut(\Ee_b)$.
\end{defi}
\begin{conj}[Kottwitz, Rappoport, Viehmann]
	Let $\ell\neq p$ be a prime. Then the cohomology groups
	\begin{equation*}
		H_c^\bullet\big({\operatorname{Sh}_{(G,[b],\{\mu\}),\infty}}\otimes {\ov{\IQ}_p},\ov{\IQ}_\ell\big)\,,
	\end{equation*}
	which are equipped with a $G(\IQ_p)\times J_b(\IQ_p)\times W_{E(G,\{\mu\})}$-action (here $E(G,\{\mu\})$ denotes the field of definition of $\{\mu\}$; don't ask) realize the LLC.
\end{conj}

\appendix

\backmatter\KOMAoption{chapterprefix}{false}
\printbibliography
\end{document}